\documentclass[review]{elsarticle}


\usepackage{amssymb}
\usepackage{latexsym}

\usepackage{url}
\usepackage{xcolor}
\definecolor{newcolor}{rgb}{.8,.349,.1}

\usepackage{amsmath}
\usepackage{mathrsfs}
\usepackage{color}
\usepackage{lineno}
\usepackage[pdfencoding=auto]{hyperref}
\hypersetup{colorlinks=true}
\modulolinenumbers[5]
\usepackage{cite}
\usepackage[ruled,vlined]{algorithm2e}
\usepackage{setspace}
\usepackage{etoolbox}
\usepackage{graphicx}
\usepackage[labelformat=simple]{subcaption}
\renewcommand\thesubfigure{(\alph{subfigure})}
\renewcommand\thesubtable{(\alph{subtable})}
\usepackage{multirow}
\usepackage{times}

\AtBeginEnvironment{algorithm}{\setstretch{1.35}}


\newtheorem{assumption}{\bf{Assumption}}
\newcounter{subassumption}[assumption]
\renewcommand{\thesubassumption}{(\textit{\roman{subassumption}})}
\makeatletter
\renewcommand{\p@subassumption}{\theassumption}% Counter prefix.
\makeatother
\newcommand{\subasu}{% Just like \item in a list, but for an asu
  \refstepcounter{subassumption}%
  \thesubassumption~\ignorespaces}
\DeclareMathOperator{\prox}{\mathrm{prox}}
\DeclareMathOperator*{\vect}{vec}

\newcommand{\Npix}{N}
\newcommand{\Ndim}{L}
\newcommand{\Psize}{K}
\newcommand{\Natom}{N_{\mathrm{d}}}
\newcommand{\Npatch}{N_{\mathrm{p}}}


\newcommand{\nd}[1]{\textcolor[rgb]{0.00,0.00,1.00}{#1}}



\journal{ELSEVIER journal}

\begin{document}



\begin{frontmatter}

\title{Coupled dictionary learning for unsupervised change detection between multi-sensor \\remote sensing images\tnoteref{label1}}

\author[n7]{Vinicius {Ferraris}\corref{cor1}}
\ead{firstname.lastname@enseeiht.fr}
\author[n7]{Nicolas {Dobigeon}}
\author[n7]{Yanna {Cavalcanti}}
\author[n7]{Thomas {Oberlin}}
\author[n7]{Marie {Chabert}}
\address[n7]{University of Toulouse, IRIT/INP-ENSEEEIHT, 2 Rue Camichel, 31071 Toulouse, France}

\tnotetext[label1]{Part of this work has been supported by Coordenao de Aperfeioamento de Ensino Superior (CAPES), Brazil, and EU FP7 through the ERANETMED JC-WATER program [MapInvPlnt Project ANR-15-NMED- 0002-02].}



\begin{abstract}
Archetypal scenarios for change detection generally consider two images acquired through sensors of the same modality. However, in some specific cases such as emergency situations, the only images available may be those acquired through sensors with different characteristics. This paper addresses the problem of unsupervisedly detecting changes between two observed images acquired by different sensors. These sensor dissimilarities introduce additional issues in the context of operational change detection that are not addressed by most of classical methods. This paper introduces a novel framework to effectively exploit the available information by modeling the two observed images as a sparse linear combination of atoms belonging to an overcomplete pair of coupled dictionaries learnt from each observed image. As they cover the same geographical location, codes are expected to be globally similar except for possible changes in sparse spatial locations. Thus, the change detection task is envisioned through a dual code estimation which enforces spatial sparsity in the difference between the estimated codes associated with each image. This problem is formulated as an inverse problem which is iteratively solved using an efficient proximal alternating minimization algorithm accounting for nonsmooth and nonconvex functions. The proposed method is applied to real multisensor images with simulated yet realistic and real images. A comparison with state-of-the-art change detection methods evidences the accuracy of the proposed strategy.
\end{abstract}


\end{frontmatter}


\section{Introduction}

Ecosystems exhibit permanent variations at different temporal and spatial scales caused by natural, anthropogenic, or even both factors \citep{coppin_review_2004}. Monitoring spatial variations over a period of time is an important source of knowledge that helps understanding the possible transformations occurring on Earth's surface. Therefore, due to the importance of quantifying these transformations, change detection (CD) has been an ubiquitous issue addressed in the remote sensing and geoscience literature \citep{lu_change_2004}.

Remote sensing CD methods can be first classified with respect to their supervision \citep{bovolo_theoretical_2007}, depending on availability of prior knowledge about the expected changes. More precisely, supervised CD methods require ground reference information about at least one of the observations. Conversely, unsupervised CD can be contextualized as automatic detection of changes without need for any further external knowledge. Each class of CD methods present particular competitive advantages with respect to the others. For instance, supervised CD methods generally achieve better accuracy in overall detection and ease to work with multisensor datasets whereas unsupervised methods are characterized by their flexibility and genericity. Nevertheless, implementing supervised methods require the acquisition of relevant ground information, which is a very challenging and dispendicious task, in terms of human and time resources \citep{bovolo_theoretical_2007}. Relaxing this constraints makes unsupervised methods more suitable in a general CD task.

CD methods can also be categorized with respect to the imagery modalities the method is able to handle. As remote sensing encompasses many different types of imagery modalities (e.g., single- and multi-band optical images, radar, LiDAR), dedicated CD methods have been developed for each one specifically by exploiting its acquisition process and the intrinsic characteristics of the resulting data. Thus, due to differences in the physical meaning and statistical properties of images acquired under different modalities, a general CD method able to handle all modalities is not straightforward to be implemented. For this reason, most of the CD methods focus on a pair of images from one single target modality. In this case, the images are generally compared pixelwisely using the the underlying assumption of same resolutions \citep{singh_review_1989,bovolo_time_2015}. Nevertheless, in some practical scenarios such as, e.g., emergency missions due to natural disasters, when the availability of data and the time are strong constraints, CD methods may have to handle multi-sensor observations of different natures and/or resolutions. This highlights the need for robust and flexible CD techniques able to deal with multi-sensor observations.

The literature about multi-sensor CD, also referred to as multimodal or multi-source CD, is very limited. However,  it has always figured as an important topic since the initial development of CD methods.  Earlier work by Kawamura described the potential of CD between a multi-modal collection of datasets (e.g., photographic, infrared and radar), applied to weather prediction and land surveillance \citep{kawamura_automatic_1971}. In \citet{lu_change_2004}, various methods dedicated to CD between images from different sources of data are grouped as geographical information system-based methods. For instance, \citet{solberg_markov_1996} proposed a supervised classification of multisource satellite images using Markov random fields. The work of \citet{bruzzone_neural-statistical_1999} uses compound classification to detect changes in multisource data. The method uses artificial neural networks to estimate the posterior probability of classes. Moreover, in \citet{inglada_similarity_2002}, the relevance of several similarity measures between multisensor data is studied. These measures are implemented in a CD context \citep{alberga_comparison_2007}. A preprocessing technique based on conditional copula that contributes to better statistically modeling multisensor images was proposed by \citet{mercier_conditional_2008}. Besides, \citet{brunner_earthquake_2010} presented a strategy to assess building damages using a pair of very high resolution (VHR) optical and radar images by geometrically modeling building in both modalities. In \citet{chabert_logistic_2010}, information databases were updated thanks to a logistic regression. More recently, the work of \citet{prendes_new_2015} presented a supervised method to infer changes after learning a manifold defined by pairs of patches extracted from the two images. Although some of these methods present relatively high detection performance, they are in general restrained to a target application. For instance, some methods are only suitable for building damage assessment presenting high level of modeling, but with lower flexibility to other scenarios \citep{brunner_earthquake_2010,chabert_logistic_2010}. Others estimate some metrics from unchanged trained samples, which reduces their applicability within a fully unsupervised context \citep{bruzzone_neural-statistical_1999,prendes_new_2015,mercier_conditional_2008}.

Recently, an unsupervised multi-source CD method based on coupled dictionary learning was addressed by \citet{gong_coupled_2016}. In the proposed methodology, the CD is based on the reconstruction error of patches approximated thanks to estimated coupled dictionary and independent sparse codes. Following the same principle, in \citet{lu_joint_2017}, a semi-supervised method was used to handle multispectral image based on joint dictionary learning. Both methods rely on the rationale that a coupled dictionary estimated from observed images tends to produce stronger reconstruction errors in change regions rather than in unchanged ones. Because of the multi-modality, the problem has not been formulated in the image space, but rather in a latent space formed by the coupled dictionary atoms. However, both methods exhibit some crucial issues that may impair their relative performance. First, the underlying optimization problem is highly nonconvex and no convergence guarantees are ensured, even by using some traditional dictionary learning methods \citep{aharon_k-svd_2006}. Then, the considered CD problem has been split into two distinct steps: dictionary learning and code estimation. Errors between estimations may produce false alarms in final CD even with reliable dictionary estimates. Also, the statistical model of the noise -- inherent to each modality/sensor -- has not been taken into consideration explicitly, which may dramatically impact the CD performance \citep{campbell_introduction_2011}. Finally, they do not consider  overlapping patches and do not explicitly handle the problem of possible differences in spatial resolutions \citep{ferraris_robust_2017,ferraris_detecting_2017}. Differences related to the size of patches and scale of the data may severely contribute to bias the coupling of the dictionaries, restraining their applications and decreasing the detection reliability.

Overcoming these limitations, this paper proposes a similar methodology to learn coupled dictionaries able to conveniently model multi-sensor remote sensing images. Specifically, contrary to the aforementioned methods, the problem is fully formulated without splitting the learning and coding steps. Also, an appropriate statistical model is derived to describe the image from each specific remote sensing modality. Overlapping patches are also considered in the overall estimation process. To couple images with different resolutions, additional scaling matrices inspired by the work by \citet{seichepine_soft_2014} are jointly estimated within the whole process. Finally, as the problem is highly nonconvex, it is iteratively solved based on the proximal alternating linearized minimization (PALM) algorithm \citep{bolte_proximal_2014}, which ensures convergence towards a  critical point for some nonconvex nonsmooth problems.

This manuscript is organized as follows. Generic and well-admitted image models are introduced in Section \ref{sec:models}. Capitalizing on these image models, Section \ref{sec:DL} formulated the CD problem as a coupled dictionary learning. Section \ref{sec:min_alg} proposes an algorithmic solution to minimize the resulting CD-based objective function. Section \ref{sec:experiments} reports experimental results obtained on synthetic images, considering three distinct simulation scenarios. Experiments conducted on real images are presented in Section \ref{subsec:reference_images}. Finally, Section \ref{sec:conclusion} concludes the manuscript.

\section{Image models}
\label{sec:models}

\subsection{Forward model}
\label{subsec:GFM}

Let us consider that the image formation process inherent to all digital remote sensing imagery modalities is modeled as a sequence of operations, denoted $\mathit{T}[\cdot]$, of the original scene into an output image. The output image of a particular sensor is referred to as the observed image and is denoted by $\mathbf{Y} \in \mathbb{R}^{\Ndim \times N}$
  consisting of $N$ voxels $\mathbf{y}_{i} \in \mathbb{R}^{\Ndim}$ stacked in lexicographic order. The voxel dimension $\Ndim$ may represent different quantities depending on the modality of the data. For instance, it stands for the number of spectral bands in the case of multiband optical images \citep{ferraris_robust_2017} or the number of polarization modes in the case of polarimetric synthetic aperture radar (POLSAR) images. The observed image provides a limited representation of the original scene with properties imposed by the image signal processor characterizing the sensor. The original scene cannot be exactly represented because of its continuous nature, but it can be conveniently approximated by a latent (i.e unobserved) image $\mathbf{X} \in \mathbb{R}^{\Ndim \times N}$ related to the observed image following
\begin{equation}
\label{eq:sensortransf}
	\mathbf{Y} = \mathit{T}[\mathbf{X}].
\end{equation}
The sequence of transformations $\mathit{T}[\cdot]$ operated by the sensor over the latent image is often referred to as the \emph{degradation process}. It may represent resolution degradations accounting for the spatial and/or spectral characteristics of the sensor \citep{ferraris_detecting_2017,ferraris_robust_2017}. In this paper, it is specifically attached to the intrinsic noise corruption produced by the sensor associated with a particular modality \citep{sun_alternating_2014}. The latent image $\mathbf{X}$ can be understood, in this context, as a noise-free and unbiased version of the observed image $\mathbf{Y}$ with the same resolution.

More precisely, the transformation $\mathit{T}[\cdot]$ underlies the likelihood function $p(\mathbf{Y}|\mathbf{X})$  which statistically models the observed image $\mathbf{Y}$ conditionally to the latent image $\mathbf{X}$ by taking into account the noise statistics. These statistical noise models mainly depend on the modality of the sensor and rely on some classical distributions, e.g., the Gaussian distribution for optical images or the Gamma distribution for multi-look SAR images. Moreover, as already pointed out by \citet{fevotte_nonnegative_2009} in a different applicative context, for a wide family of distributions, this likelihood function relies on a divergence measure $\mathcal{D}(\cdot|\cdot)$ between the observed and latent images, which finally defines an explicit data-fitting term through a negative-log transformation
\begin{equation}
			- \log p(\mathbf{Y}|\mathbf{X}) = \phi^{-1} \mathcal{D}(\mathbf{Y}|\mathbf{X}) + \theta
\end{equation}
where $\phi$ and $\theta$ are parameters characterizing the distributions. In \ref{ap:dft}, the divergence measures $\mathcal{D}(\cdot|\cdot)$ are derived for two of the most common remote sensing modalities of images, namely optical multiband and SAR images, considered in this work.


\subsection{Latent image sparse model}\label{subsec:sparse_model}

Sparse representations have been an ubiquitous and well-admitted tool to model images in various applicative and task-driven contexts \citep{mairal_sparse_2014}. Indeed, natural images are known to be compressible in a transformed domain, i.e., they can be efficiently represented by a few expansion coefficients acting on basis functions  \citep{mallat_wavelet_2009}. This finding has motivated numerous works on image understanding, compression and denoising \citep{olshausen_sparse_1997,chen_atomic_2001}. In earlier works, this transformed domain, equivalently defined by the associated basis functions, was generally fixed in advance and chosen in agreement with the expected spatial content of the images. Thus, the basis functions belonged to pre-determined families with specific representation abilities, such as cosines, wavelets, contourlets, shearlets, among others  \citep{mallat_wavelet_2009}. More recently, the seminal contribution by Aharon \emph{et al.} proposed a new paradigm by learning an overcomplete dictionary jointly with a sparse code \citep{aharon_k-svd_2006}. This dictionary learning-based approach exploits the key property of self-similarity characterizing the images to provide an adaptive representation. Indeed, it aims at identifying elementary patches that can be linearly and sparsely combined to approximate the observed image patches. In this paper, we propose to resort to this dictionary-based representation to model the latent image $\mathbf{X}$. More precisely, first, the image is decomposed into patches. %
Let $\mathcal{R}_{i} : \mathbb{R}^{\Ndim \times N} \rightarrow \mathbb{R}^{\Psize^{2}\Ndim}$ denote a binary operation modelling the extraction, from the image, of a patch centered at the \textit{i}-th lexicographic pixel such that
\begin{equation}
\label{eq:patch_extraction}
\mathbf{p}_{i} = \mathcal{R}_{i}\mathbf{X}
\end{equation}
where $\mathbf{p}_{i} \in \mathbb{R}^{\Psize^{2}\Ndim}$ represents the $\Psize \times \Psize \times \Ndim$-patch in its vectorized form. The conjugate of the patch-extraction operator\footnote{Note that, despite a slight abuse of notation, the operator $\mathcal{R}$ (resp., $\mathcal{R}^T$) does not stand for a matrix but rather a linear operator acting on the image $\mathbf{X}$ (resp., the patch $\mathbf{p}_{i}$) directly.}, denoted $\mathcal{R}^{T}_{i}$, acts on $\mathbf{p}_{i}$ to produce a zero-padded image composed by the unique patch $\mathbf{p}_{i}$ located at the $i$th spatial position.

Following dictionary-based representation, these patches are assumed to be approximately and independently modeled as sparse combinations of atoms belonging to an overcomplete dictionary $\mathbf{D}  = \left[\mathbf{d}_{1}, \cdots, \mathbf{d}_{\Natom}\right] \in \mathbb{R}^{\Psize^{2}\Ndim \times \Natom}$
\begin{equation}
\label{eq:patchRepresentation}
	\mathbf{p}_{i}|\mathbf{D},\mathbf{a}_{i} \sim \mathcal{N}\left(\mathbf{D}\mathbf{a}_{i},\sigma^2 \mathbf{I}_{\Natom}\right)
\end{equation}
where $\mathbf{a}_{i} \in \mathbb{R}^{\Natom}$ represent the code (coefficients) of the current patch over the dictionary, $\boldsymbol{\Sigma}=\sigma^2 \mathbf{I}_{\Natom}$ is the error covariance matrix and $\Natom$ stands for the number of atoms composing the dictionary, commonly referred to as dictionary size. Besides, $\mathbf{P} \in \mathbb{R}^{\Psize^{2}\Ndim \times \Npatch} = \left[\mathbf{p}_{1}, \cdots, \mathbf{p}_{\Npatch}\right]$ corresponds to the patch matrix which stacks all patches extracted from the latent image and $\mathbf{A} \in \mathbb{R}^{\Natom \times \Npatch} = \left[\mathbf{a}_{1}, \cdots, \mathbf{a}_{\Npatch}\right]$ the code matrix in which each column represents the code for each corresponding column of $\mathbf{P}$. Note that the number of patches is such that $\Npatch \leq N$ and patches may overlap. The overcompletness property of the dictionary, occuring when the number of atoms is greater than the effective dimensionality of the input space, $\Natom \gg \Psize^{2} \Ndim $, allows for the sparsity of the representation \citep{olshausen_sparse_1997}. The overcompletness implies redundancy and non-orthogonality between atoms. This property is not necessary for the decomposition but has been proved to be very useful in some applications like denoising and compression \citep{aharon_k-svd_2006}.


Given the image patch matrix $\mathbf{P}$, dictionary learning aims at recovering the set of atoms $\mathbf{D}$ and the associated code matrix $\mathbf{A}$ and it is generally tackled through a 2-step procedure. First, inferring the code matrix $\mathbf{A}$ associated with the patch matrix $\mathbf{B}$ and the dictionary $\mathbf{D}$ can be formulated as a set of $\Npatch$ sparsity-penalized optimization problems. Sparsity of the code vectors $\mathbf{a}_i=\left[a_{1i},\ldots,a_{\Natom i}\right]^T$ ($i=1,\ldots,\Npatch$) can be promoted by minimizing its $\ell_{0}$-norm. However, since this leads to a non-convex problem \citep{chen_atomic_2001}, it is generally substituted by the corresponding convex relaxation, i.e., an $\ell_1$-norm. Within a probabilistic framework, taking into account the expected non-negativeness of the code, this choice can be formulated by assigning a single-side exponential (i.e., Laplacian) prior distribution to the code components, assumed to be a priori independent
 \begin{equation}
	\label{eq:code_model}
	\mathbf{a}_{i} \sim \prod_{j=1}^{\Natom}\mathcal{L}(a_{ji};\lambda)
\end{equation}
where $\lambda$ is the hyperparameter adjusting the sparsity level over the code.
Conversely, learning the dictionary $\mathbf{D}$ given the code $\mathbf{A}$ can also be formulated as an optimization problem. As the number of solutions for the dictionary learning problem can be extremely large, one common assumption is to constrain the energy of each atom, thereby preventing $\mathbf{D}$ to become arbitrarily large \citep{mairal_online_2009}. Moreover, in the particular context considered in this work, to promote the positivity of the reconstructed patches, the atoms are also constrained to positive values. Thus, each atom will be constrained to the set
\begin{equation}\label{eq:dict_dist}
  \mathcal{S} \triangleq \left\{\mathbf{D} \in \mathbb{R}^{\Psize^{2}\Ndim \times \Natom}_{+} \; \mid \forall j \in\left\{ 1,\ldots,\Natom,\right\} \; \left\|\mathbf{d}_j\right\|^2_{2} = 1\right\}.
\end{equation}
\subsection{Optimization problem}
\label{subsec:sdl}
Adopting a Bayesian probabilistic formulation of the hierarchical image model introduced in Sections \ref{subsec:GFM} and \ref{subsec:sparse_model}, the posterior probability of the unknown variables $\mathbf{X}$, $\mathbf{D}$ and $\mathbf{A}$ can be derived using the probability chain rule \citep{gelman_bayesian_2004}
\begin{equation}
\label{eq:single_bayesian_model}
p(\mathbf{X},\mathbf{D},\mathbf{A}|\mathbf{Y}) \propto p(\mathbf{Y}|\mathbf{X}) p(\mathbf{X}|\mathbf{D},\mathbf{A}) p(\mathbf{D})p(\mathbf{A})
\end{equation}
where $p(\mathbf{Y}|\mathbf{X})$ is the likelihood function relating the observation data to the latent image through the direct model \eqref{eq:sensortransf}, $p(\mathbf{X}|\mathbf{D},\mathbf{A})$ is the dictionary-based prior model of the latent image, $p(\mathbf{D})$ and $p(\mathbf{A})$ are the (hyper-)prior distributions associated with the dictionary and the sparse code. Under a maximum a posteriori (MAP) paradigm, the joint MAP estimator $\left\{\hat{\mathbf{X}}_{\text{MAP}},\hat{\mathbf{D}}_{\text{MAP}},\hat{\mathbf{A}}_{\text{MAP}}\right\}$ can be derived by minimizing the negative log-posterior, leading to the following minimization problem
\begin{equation}
	\label{eq:MAP}
    \left\{\hat{\mathbf{X}}_{\text{MAP}},\hat{\mathbf{D}}_{\text{MAP}},\hat{\mathbf{A}}_{\text{MAP}} \right\} \in \mathop{\rm argmin}\limits_{\mathbf{X},\mathbf{D},\mathbf{A}}   \mathcal{J}\left(\mathbf{X},\mathbf{D},\mathbf{A}\right)
\end{equation}
with
\begin{equation}
\label{eq:objective_single}
\begin{aligned}
  \mathcal{J}\left(\mathbf{X},\mathbf{D},\mathbf{A}\right) &= \mathcal{D}(\mathbf{Y}|\mathbf{X}) \\
  &+\frac{\sigma^{2}}{2}\sum_{i=1}^{\Npatch}\left\|\mathcal{R}_{i}\mathbf{X} - \mathbf{D}\mathbf{a}_{i}\right\|_{\mathrm{F}}^{2} + \\
  & + \lambda \left\|\mathbf{A}\right\|_1 + \iota_{\mathcal{S}}(\mathbf{D})
\end{aligned}
\end{equation}
where $\iota_{\mathcal{S}}$ represent the indicator function on the set $\mathcal{S}$,
 \begin{equation}
	\iota_{\mathcal{S}}(z) = \left\{\begin{matrix}
0 \quad \text{if} \ z \in \mathcal{S}\\
+\infty \quad \text{elsewhere} \\
\end{matrix}\right.
\end{equation}
and $\mathcal{D}(\cdot|\cdot)$ is the data-fitting term associated with the image modality.

This model has been widely advocated in the literature, e.g., for denoising images of various modalities \citep{elad_image_2006,ma_dictionary_2013}. Particularly, in \citet{ma_dictionary_2013}, an additional regularization $\Psi\left(\mathbf{X}\right)$ of the latent image was introduced as the target modalities may present strong fluctuations due to their inherent image formation process, i.e. Poissonian or multiplicative gamma processes. The final objective function \eqref{eq:objective_single} can thus be written as
\begin{equation}
	\label{eq:objective_single_den}
	\begin{aligned}
  \mathcal{J}\left(\mathbf{X},\mathbf{D},\mathbf{A}\right) &=
  \mathcal{D}(\mathbf{Y}|\mathbf{X}) \\
  & + \frac{\sigma^{2}}{2}\sum_{i=1}^{\Npatch}\left\|\mathcal{R}_{{i}}\mathbf{X} - \mathbf{D}\mathbf{a}_{i}\right\|_{\mathrm{F}}^{2} + \Psi\left(\mathbf{X}\right) \\
  &+ \lambda \left\|\mathbf{A}\right\|_1 + \iota_{\mathcal{S}}(\mathbf{D})
	\end{aligned}
\end{equation}
where, for instance, $\Psi\left(\mathbf{X}\right)$ can stand for a total-variation (TV) regularization \citep{ma_dictionary_2013}.

The next section expands the proposed image models to handle a pair of observed images in the specific context of CD.

\section{From change detection to coupled dictionary learning}
\label{sec:DL}

\subsection{Problem statement}
\label{subsec:ps}

Let us consider two co-registered observed images $\mathbf{Y}_{1} \in \mathbb{R}^{\Ndim_{1} \times N_{1}}$ and $\mathbf{Y}_{2} \in \mathbb{R}^{\Ndim_{2} \times N_{2}}$ acquired by two sensors $\mathsf{S}_{1}$ and $\mathsf{S}_{2}$ at times $t_1$ and $t_2$, respectively. Time ordering of acquisition is indifferent, i.e., either $t_2 < t_1$ or $t_2 > t_1$ are possible cases. The problem addressed
in this paper consists in detecting significant changes between these two observed images. This is a challenging task mainly due to the possible dissimilarities in terms of resolution, which prevents any use of classical CD algorithms \citep{singh_review_1989,bovolo_time_2015}, or in term of modality, which makes even flexible CD algorithms inoperative \citep{ferraris_detecting_2017,ferraris_robust_2017}. To alleviate this issue, this work proposes to improve and generalize the CD methods introduced by \citet{seichepine_soft_2014,gong_coupled_2016,lu_joint_2017}. Following the widely admitted forward model described in Section \ref{subsec:GFM} and adopting consistent notations, the observed images $\mathbf{Y}_{1}$ and $\mathbf{Y}_{2}$ can be related to two latent images $\mathbf{X}_{1} \in \mathbb{R}^{\Ndim_{1} \times N_{1}}$ and $\mathbf{X}_{2} \in \mathbb{R}^{\Ndim_{2} \times N_{2}}$
\begin{subequations}
\label{eq:jointobsmodel}
		\begin{align}
			&\mathbf{Y}_{1} = \mathit{T}_{1}[\mathbf{X}_{1}]  \label{eq:jointobsmodel1}\\
			&\mathbf{Y}_{2} = \mathit{T}_{2}[\mathbf{X}_{2}] \label{eq:jointobsmodel2}
		\end{align}
\end{subequations}
where $\mathit{T}_{1}$ and $\mathit{T}_{2}$ denote two degradation operators imposed by the sensors $\mathsf{S}_{1}$ and $\mathsf{S}_{2}$. Note that \eqref{eq:jointobsmodel} is a double instance of the model \eqref{eq:sensortransf}. In particular, in the CD context considered in this work, the two latent images $\mathbf{X}_{1}$ and $\mathbf{X}_{2}$ are supposed to represent the same geographical region provided the observed images have been beforehand co-registered.

Both latent images can be represented thanks to a dedicated dictionary-based decomposition, as stated in Section \ref{subsec:sparse_model}. More precisely, a pair of homologous
patches extracted from each image represents the same geographical spot. Each patch can be reconstructed from a sparse linear combination of atoms of an image-dependent dictionary. In absence of changes between the two observed images, the sparse codes associated with the corresponding latent image are expected to be approximately the same and the two learned dictionaries are coupled \citep{yang_image_2010,yang_coupled_2012,zeyde_single_2010}. This coupling can be understood as the ability of deriving a joint representation for homologous multiple observation in a latent coupled space \citep{gong_coupled_2016}. Akin to the manifold proposed by \citet{prendes_new_2015}, this representation offers the opportunity to analyze images of different modalities in a common dual space. In the case where a pair of homologous patches has been extracted from two images representing the same scene, given perfect estimated coupled dictionaries, each patch should be exactly reconstructed thanks to the same sparse code. In other words, the pair of patches is an element of the latent coupled space.  Nevertheless, in the case where the pair of homologous patches does not represent exactly the same scene, perfect reconstruction cannot be achieved using the same code. This means that the pair of patches does not belong to the coupled space. Using the same code for reconstruction amounts to estimate the point in the coupled space that best approximates the patch pair. Thereby, relaxing this constraint in some possible change locations may provide an accurate reconstruction of both images while
spatially mapping change locations. In the specific context of CD addressed in this work, this finding suggests to evaluate any change between the two observed, or equivalently latent, images by comparing the corresponding codes
\begin{equation}
\label{eq:code_relaxation}
	\Delta\mathbf{A} = \mathbf{A}_{2} - \mathbf{A}_{1}
\end{equation}
where $\Delta\mathbf{A}=\left[\Delta\mathbf{a}_1,\ldots,\Delta\mathbf{a}_{\Npatch} \right] $ and $\Delta\mathbf{a}_{i} \in \mathbb{R}^{\Natom}$ denotes the code change vector associated with the \textit{i}th pixel. Then, to spatially locate the changes, a natural approach consists in monitoring the magnitude of $\Delta\mathbf{A}$, summarized by the code change energy image \citep{bovolo_theoretical_2007,ferraris_robust_2017}
\begin{equation}
\label{eq:change_energy_matrix}
  \mathbf{e} =\left[e_1,\ldots,e_{\Npatch}\right]\in \mathbb{R}^{\Npatch}
\end{equation}
with
\begin{equation*}
  e_i = \left\|\Delta\mathbf{a}_i\right\|_2, \quad i=1,\ldots,\Npatch.
\end{equation*}
When the CD problem in the $i$th pixel is formulated as the binary hypothesis testing
\begin{equation*}
\label{eq:test}
 \left\{
		\begin{array}{rcl}
			\mathcal{H}_{0,i} &:& \text{no change occurs in the $i$th pixel}  \\
			\mathcal{H}_{1,i} &:& \text{a change occurs in the $i$th pixel}
		\end{array}
        \right.
\end{equation*}
a pixel-wise statistical test can be written by thresholding the code change energy pixels
\begin{equation*}
    \label{eq:decision_rule}
  e_i \overset{\mathcal{H}_{1,i}}{\underset{\mathcal{H}_{0,i}}{\gtrless}} \tau.
\end{equation*}
The final binary CD map denoted ${\mathbf{m}} = \left[m_1,\ldots,m_n\right] \in \{0,1\}^N$ can be derived as
\begin{equation*}
	\label{eq:CVArule}
 {m}_i = \left\{\begin{array}{lll}
             1 & \mbox{if } e_i \geq \tau & (\mathcal{H}_{1,i})\\
			 0 & \mbox{otherwise}          & (\mathcal{H}_{0,i}).
				\end{array}\right.
\end{equation*}
As a consequence, to solve the multi-sensor image CD problem, the key issue lies in the joint estimation of the pair of representation codes $\left\{\mathbf{A}_{1},\mathbf{A}_{2}\right\}$ or, equivalently, the joint estimation of one code matrix and the change code matrix, i.e., $\left\{\mathbf{A}_{1},\Delta\mathbf{A}\right\}$, as well as the pair of coupled dictionary $\left\{\mathbf{D}_{1},\mathbf{D}_{2}\right\}$ and consequently the pair of latent images $\left\{\mathbf{X}_{1},\mathbf{X}_{2}\right\}$ from the joint forward model \eqref{eq:jointobsmodel}. Finally, the next paragraph introduces the CD-driven optimization problem to be solved.



\subsection{Coupled dictionary learning for CD}
\label{subsec:cdl}

The single dictionary estimation problem presented on Section \ref{subsec:sdl} can be generalized to take into account the modeling presented in Section \ref{subsec:ps}. Nevertheless, some previous considerations must be carefully handled in order to provide good coupling of both dictionaries.

As the prior information about the dictionaries constrains each atom into the set $\mathcal{S}$ of unitary energy defined by \eqref{eq:dict_dist}, an unbiased estimation of the code change vector would allow a pair of unchanged homologous patches to be reconstructed  with exact the same code, while changed patches would exhibit differences in their code. Obviously, this can only be achieved if the coupled dictionaries represent data with the same dynamics and resolutions. However, when analyzing images of different modalities and/or resolutions, this assumption can be not fulfilled. To alleviate this issue, we propose to resort to the strategy proposed by \citet{seichepine_soft_2014}, by introducing an additional diagonal scaling matrix constrained to the set $\mathcal{C} \triangleq \left\{\mathbf{S} \in \mathbb{R}^{\Natom \times \Natom}_{+} \; \mid \mathbf{S} = \mathrm{diag}(\mathbf{s}),\ \mathbf{s}\succeq 0 \right\}$. This scaling matrix gathers the code energy differences originated from different modalities for each pair of coupled atoms. This is essential to ensure that the sparse codes of the two observed images are directly comparable, following \eqref{eq:code_relaxation}, and then properly estimated. Therefore, considering a pair of homologous patches, their joint representation model derived from \eqref{eq:patchRepresentation} can be written as
\begin{equation}
	\label{eq:coupled_patchRepresentation}
    \begin{aligned}
	&\mathbf{p}_{1{i}} = \mathcal{R}_{1{i}}\mathbf{X}_{1}  \approx \mathbf{D}_{1}\mathbf{S}\mathbf{a}_{1{i}}\\
	&\mathbf{p}_{2{i}} = \mathcal{R}_{2{i}}\mathbf{X}_{2} \approx \mathbf{D}_{2}\mathbf{a}_{2{i}} = \mathbf{D}_{2}\left(\mathbf{a}_{1{i}} + \Delta\mathbf{a}_{i}\right)
	\end{aligned}
\end{equation}
where $\left\{\mathbf{p}_{1{i}},\mathbf{p}_{2{i}}\right\}$
represent the homologous patches pair and $\mathbf{S}$ is the diagonal scaling matrix.

Since the codes $\mathbf{A}_{1}$ and $\mathbf{A}_{2}$ are now element-wisely comparable, a natural choice to enforce coupling between them should be the equality $\mathbf{A}_{1} = \mathbf{A}_{2} = \mathbf{A}$. This has been a classical assumption in various coupled dictionary learning applications \citep{yang_image_2010,zeyde_single_2010,yang_coupled_2012}. Nevertheless, in a CD context, some spatial positions may not contain the same objects. To account for possible changes in some specific locations while most of the patches remain unchanged, as in \citet{ferraris_robust_2017},  the code change energy matrix $\mathbf{e}$ defined by \eqref{eq:change_energy_matrix} is expected to be sparse. As a consequence, the corresponding regularizing function is chosen as the sparsity-inducing
$\ell_1$-norm of the code change energy matrix $\mathbf{e}$ or, equivalently, as the $\ell_{2,1}$-norm of the code change matrix
\begin{equation}
	\label{eq:phi_2}
	\phi_{2}\left(\Delta\mathbf{A}\right) = \left\|\Delta\mathbf{A}\right\|_{2,1} = \sum_{i=1}^{\Npatch} \left\|\Delta \mathbf{a}_i\right\|_2.
\end{equation}
This regularization is a specific instance of the non-overlapping group-lasso penalization \citep{bach_optimization_2011} which has been considered in various applications to promote structured sparsity \citep{wright_sparse_2009,fevotte_nonlinear_2015,ferraris_robust_2017}.

Then, a  hierarchical Bayesian model extending the one derived for a single image \eqref{eq:single_bayesian_model} leads to the posterior distribution of the parameters of interest
\begin{equation}
\begin{aligned}
\label{eq:coupled_bayesian_model}
p&\left(\mathbf{X}_{1},\mathbf{X}_{2},\mathbf{D}_{1},\mathbf{D}_{2},\mathbf{S},\mathbf{A}_{1},\Delta\mathbf{A}|\mathbf{Y}_{1},\mathbf{Y}_{2}\right)  \\ &\propto p(\mathbf{Y}_{1}|\mathbf{X}_{1}) p(\mathbf{Y}_{2}|\mathbf{X}_{2}) \\
&\times p(\mathbf{X}_{1}|\mathbf{D}_{1},\mathbf{S},\mathbf{A}_{1}) p(\mathbf{X}_{2}|\mathbf{D}_{2},\mathbf{A}_{1},\Delta\mathbf{A})\\
&\times p(\mathbf{D}_{1})p(\mathbf{D}_{2})p(\mathbf{S})p(\mathbf{A}_{1})p(\Delta\mathbf{A}).
\end{aligned}
\end{equation}
By incorporating all previously defined prior distributions (or, equivalently, regularizations), the joint MAP estimator $\hat{\boldsymbol{\Theta}}_{\text{MAP}} = \left\{\hat{\mathbf{X}}_{1,{\text{MAP}}},\hat{\mathbf{X}}_{2,{\text{MAP}}},\hat{\mathbf{D}}_{1,{\text{MAP}}},\hat{\mathbf{D}}_{2,{\text{MAP}}},\hat{\mathbf{S}}_{\text{MAP}},\hat{\mathbf{A}}_{1,{\text{MAP}}},\Delta\hat{\mathbf{A}}_{{\text{MAP}}}\right\}$ of the quantities of interest can be obtained by minimizing the negative log-posterior, leading to the following minimization problem
\begin{equation}
	\label{eq:map_coupled}
    \begin{aligned}
    &\hat{\boldsymbol{\Theta}}_{\text{MAP}} \in
    &\mathop{\rm argmin}\limits_{\boldsymbol{\Theta}}   \mathcal{J}\left(\boldsymbol{\Theta}\right)
    \end{aligned}
\end{equation}
with
\begin{equation}
\begin{aligned}
	\label{eq:objective_coupled_den}
  \mathcal{J}\left(\boldsymbol{\Theta}\right) &\triangleq  \mathcal{J}\left(\mathbf{X}_{1},\mathbf{X}_{2},\mathbf{D}_{1},\mathbf{D}_{2},\mathbf{S},\mathbf{A}_{1},\Delta\mathbf{A}\right) \\
  &=\mathcal{D}(\mathbf{Y}_{1}|\mathbf{X}_{1}) + \mathcal{D}(\mathbf{Y}_{2}|\mathbf{X}_{2}) \\
  &+\frac{\sigma_{1}^2}{2}\sum_{i=1}^{\Npatch}\left\|\mathcal{R}_{1{i}}\mathbf{X}_{1} - \mathbf{D}_{1}\mathbf{S}\mathbf{a}_{1{i}}\right\|_{\mathrm{F}}^{2} + \Psi\left(\mathbf{X}_{1}\right) \\
  &+  \frac{\sigma_{2}^2}{2}\sum_{i=1}^{\Npatch}\left\|\mathcal{R}_{2{i}}\mathbf{X}_{2} - \mathbf{D}_{2}\left(\mathbf{a}_{1{i}} + \Delta\mathbf{a}_{i}\right)\right\|_{\mathrm{F}}^{2}  + \Psi\left(\mathbf{X}_{2}\right)\\
  &+ \lambda \left\|\mathbf{A}_{1}\right\|_1 + \lambda \left\|\mathbf{A}_{1}+\Delta\mathbf{A}\right\|_1  + \gamma \left\|\Delta\mathbf{A}\right\|_{2,1}\\
 &+\iota_{\mathcal{S}}(\mathbf{D}_{1}) + \iota_{\mathcal{S}}(\mathbf{D}_{2}) + \iota_{\mathcal{C}}(\mathbf{S}).
\end{aligned}
\end{equation}
The next section describes an iterative algorithm which solves the minimization problem in \eqref{eq:map_coupled}.

\section{Minimization Algorithm}
\label{sec:min_alg}

Given the nature of the optimization problem \eqref{eq:map_coupled}, which is genuinely nonconvex and nonsmooth, the adopted minimization strategy relies on the proximal alternating linearized minimization (PALM) scheme \citep{bolte_proximal_2014}. PALM is an iterative, gradient-based algorithm which generalizes the Gauss-Seidel method. It performs iterative proximal gradient steps with respect to each block of variables from $\boldsymbol{\Theta}$ and ensures convergence to a local critical point $\boldsymbol{\Theta}^{*}$. It has been successfully applied in many matrix factorization cases \citep{bolte_proximal_2014,cavalcanti_unmixing_2017,thouvenin_online_2016}. Now, the goal is to generalize the single factorization to coupled factorization. The resulting CD-driven coupled dictionary learning (CDL) algorithm, whose main steps are described in the following paragraphs, is summarized in Algorithm \ref{algo:PALM_SCDL_Diff}.
\begin{algorithm}
    \DontPrintSemicolon
    \KwData{$\mathbf{Y}$}
    \KwIn{$\mathbf{A}_{1}^{(0)}$, $\Delta\mathbf{A}^{(0)}$, $\mathbf{D}_{1}^{(0)}$, $\mathbf{D}_{2}^{(0)}$, $\mathbf{S}^{(0)}$, $\mathbf{X}_{1}^{(0)}$, $\mathbf{X}_{2}^{(0)}$}
    $k \leftarrow 0$\;
    \Begin{
		\While{stopping criterion not satisfied}{
		\tcp{Code update}
    	\label{algostep:SCDLC_A} $\mathbf{A}^{(k+1)} \leftarrow \text{Update}\left(\mathbf{A}^{(k)}\right)$ \tcp*{cf. (\ref{eq:code_Update})}
      	\label{algostep:SCDLC_dA} $\Delta\mathbf{A}^{(k+1)} \leftarrow \text{Update}\left(\Delta\mathbf{A}^{(k)}\right)$ \tcp*{cf. (\ref{eq:deltacode_Update})}
        \tcp{Dictionary update}
		\label{algostep:SCDLC_Da} $\mathbf{D}_{1}^{(k+1)} \leftarrow \text{Update}\left(\mathbf{D}_{1}^{(k)}\right)$ \tcp*{cf. (\ref{eq:dictionary_Update})}
		\label{algostep:SCDLC_Db} $\mathbf{D}_{2}^{(k+1)} \leftarrow \text{Update}\left(\mathbf{D}_{2}^{(k)}\right)$ \tcp*{cf. (\ref{eq:dictionary_Update})}
    	 \tcp{Scale update}
		\label{algostep:SCDLC_Sa} $\mathbf{S}^{(k+1)} \leftarrow \text{Update}\left(\mathbf{S}^{(k)}\right)$ \tcp*{cf. (\ref{eq:scale_Update})}
        \tcp{Latent image update}
    	\label{algostep:SCDLC_Xa} $\mathbf{X}_{1}^{(k+1)} \leftarrow \text{Update}\left(\mathbf{X}_{1}^{(k)}\right)$\tcp*{cf. (\ref{eq:latent_Update})}
		\label{algostep:SCDLC_Xb} $\mathbf{X}_{2}^{(k+1)} \leftarrow \text{Update}\left(\mathbf{X}_{2}^{(k)}\right)$\tcp*{cf. (\ref{eq:latent_Update})}
		$k \leftarrow k+1$\;
    }
    $\hat{\mathbf{A}}_{1} \leftarrow \mathbf{A}_{1}^{(k+1)}$, $\Delta\hat{\mathbf{A}} \leftarrow \Delta\mathbf{A}^{(k+1)}$,\\
    $\hat{\mathbf{D}}_{1} \leftarrow \mathbf{D}_{1}^{(k+1)}$, $\hat{\mathbf{D}}_{2} \leftarrow \mathbf{D}_{2}^{(k+1)}$,
    \\ $\hat{\mathbf{S}} \leftarrow \mathbf{S}^{(k+1)}$, \\
    $\hat{\mathbf{X}}_{1}  \leftarrow \mathbf{X}_{1} ^{(k+1)}$, $\hat{\mathbf{X}}_{2}  \leftarrow \mathbf{X}_{2} ^{(k+1)}$
		}
    \KwResult{$\hat{\mathbf{A}}_{1}$, $\Delta\hat{\mathbf{A}}$, $\hat{\mathbf{D}}_{1}$, $\hat{\mathbf{D}}_{2}$, $\hat{\mathbf{S}}$, $\hat{\mathbf{X}}_{1}$, $\hat{\mathbf{X}}_{2}$}
    \caption{PALM-CDL \label{algo:PALM_SCDL_Diff}}
    \end{algorithm}

\subsection{PALM implementation}
The PALM algorithm was proposed by \citet{bolte_proximal_2014} for solving a broad class of problems involving the minimization of the sum of finite collections of possibly nonconvex and nonsmooth functions. Particularly, the target optimization function is composed by a coupling function gathering the block of variables, denoted $H(\cdot)$, and regularization functions for each block. Non-convexity constraint is assumed for either coupling or regularization functions. One of the main advantages of the PALM algorithm over classical optimization algorithms is that each bounded sequence generated by PALM converges to a critical point. The rationale of the method can be seen as an alternating minimization approach for the proximal forward-backward algorithm \citep{combettes_signal_2005}. Some assumptions are required in order to solve this problem with all guarantees of convergence (c.f \citep[Assumption~1, Assumption~2]{bolte_proximal_2014}). The most restrictive one \citep[Assumption~2(ii)]{bolte_proximal_2014} requires that the partial gradient of the coupling function $H(\cdot)$ to be globally Lipschitz continuous for each block of variable keeping the remaining ones fixed. Indeed, it is a classical assumption for proximal gradient methods which guarantee a sufficient descent property.

Therefore, given the objective function to be minimized \eqref{eq:objective_coupled_den} and considering the same structure proposed by \citet{bolte_proximal_2014} and the Lipschitz property for linear combinations of functions \citep{eriksson_lipschitz_2004}, let us define the coupling function $H(\Theta)$ as
\begin{multline}
	\label{eq:objective_coupled_term}
  H\left(\boldsymbol{\Theta}\right) \triangleq  H\left(\mathbf{X}_{1},\mathbf{X}_{2},\mathbf{D}_{1},\mathbf{D}_{2},\mathbf{S},\mathbf{A}_{1},\Delta\mathbf{A}\right)  \\
 =\Psi\left(\mathbf{X}_{1}\right)   + \Psi\left(\mathbf{X}_{2}\right) + \frac{\sigma_{1}^2}{2}\sum_{i=1}^{\Npatch}\left\|\mathcal{R}_{1{i}}\mathbf{X}_{1} - \mathbf{D}_{1}\mathbf{S}\mathbf{a}_{1{i}}\right\|_{\mathrm{F}}^{2}\\
  +\frac{\sigma_{2}^2}{2}\sum_{i=1}^{\Npatch}\left\|\mathcal{R}_{2{i}}\mathbf{X}_{2} - \mathbf{D}_{2}\left(\mathbf{a}_{1{i}} + \Delta\mathbf{a}_{i}\right)\right\|_{\mathrm{F}}^{2} +  \lambda \left\|\mathbf{A}_{1}+\Delta\mathbf{A}\right\|_1.
\end{multline}
This coupling function defined accordingly does not fulfill
\citep[Assumption~2(ii)]{bolte_proximal_2014} because some of its terms are nonsmooth, specifically the TV regularizations $\Psi(\cdot)$ and the $\ell_{1}$-norm sparsity promoting regularizations applied to $\mathbf{A}_{2}$. Thus, to ensure such a coupling function is in agreement with the required assumptions, smooth relaxations of $\Psi(\cdot)$ and $\left\|\cdot\right\|_1$ are applied by using the pseudo-Huber function  \citep{fountoulakis_second_2016,jensen_implementation_2012}.

The remaining terms of \eqref{eq:objective_coupled_den} are composed of the regularization functions associated with each variable block. Within the PALM structure, a gradient step applied to the coupling function with respect to a given variable block is followed by proximal step associated with the corresponding regularization functions. As a consequence, those regularization functions must be proximal-like where their proximal mappings or projections must exist and have closed-form solutions. It is important to keep in mind that, even if the convergence is guaranteed for all optimization orderings, it should not vary during iterations. Thus, the updating rules for each optimization variable in Algorithm \ref{algo:PALM_SCDL_Diff} are defined. More details about the proximal operators and projections involved in this section are given in \ref{ap:proj}.

\subsection{Optimization with respect to \texorpdfstring{$\mathbf{A}_{1}$}{A}}

Considering the single block optimization variable $\mathbf{A}_{1}$, and assuming that the remaining variables are fixed, the PALM updating step can be written
\begin{equation}
		\mathbf{A}_{1}^{(k+1)} = \mathrm{prox}^{L_{\mathbf{A}_{1}}}_{\lambda\left\|\cdot\right\|_1 + \geq0}\left(\mathbf{A}_{1}^{(k)} - \frac{1}{L_{\mathbf{A}_{1}}^{(k)} }\nabla_{\mathbf{A}_{1}} H(\boldsymbol{\Theta}) \right)
        \label{eq:code_Update}
\end{equation}
with
\begin{equation}
\begin{aligned}
		\nabla_{\mathbf{A}_{1}} H(\boldsymbol{\Theta}) &= \sigma_{1}^2\mathbf{S}^{T}\mathbf{D}_{1}^{T}\left( \mathbf{D}_{1}\mathbf{S}\mathbf{A}_{1} - \mathbf{P}_{1}\right) \\
&+\sigma_{2}^2\mathbf{D}_{2}^{T}\left( \mathbf{D}_{2}\left(\mathbf{A}_{1} + \Delta\mathbf{A}\right)-\mathbf{P}_{2}\right)  \\
&+ \lambda \frac{\left[\mathbf{A}_{1} + \Delta\mathbf{A}\right]_{i}}{\sqrt{\left[\mathbf{A}_{1} + \Delta\mathbf{A}\right]_{i}^{2} + \epsilon_{\mathbf{A}_{1}}^{2}}}
        \label{eq:nabla_code}
\end{aligned}
\end{equation}
where $[\cdot]_i/[\cdot]_i$ should be understood as a element-wise operation and $L_{\mathbf{A}_{1}}^{(k)}$ is the associated Lipschitz constant
\begin{equation}
		L_{\mathbf{A}_{1}}^{(k)} = \sigma_{1}^2\left\|\mathbf{S}^{T}\mathbf{D}_{1}^{T}\mathbf{D}_{1}\mathbf{S}\right\| +  \sigma_{2}^2\left\|\mathbf{D}_{2}^{T}\mathbf{D}_{2}\right\| + \frac{\lambda}{\epsilon_{\mathbf{A}_{1}}}.
        \label{eq:lip_code}
\end{equation}
Note that $\mathrm{prox}^{L_{\mathbf{A}_{1}}}_{\lambda\left\|\cdot\right\|_1 + \geq0}(\cdot)$ can be simply computed by considering the positive part of the soft-thresholding operator \citep{parikh_proximal_2014}.

\subsection{Optimization with respect to \texorpdfstring{$\Delta\mathbf{A}$}{DA}}

Similarly, considering the single block optimization variable $\Delta\mathbf{A}$ and consistent notations, the PALM update can be derived as
\begin{equation}
		\Delta\mathbf{A}^{(k+1)} = \mathrm{prox}^{L_{\Delta\mathbf{A}}^{(k)} }_{\left\|\cdot\right\|_{2,1}}\left(\Delta\mathbf{A}^{(k)} - \frac{1}{L_{\Delta\mathbf{A}}^{(k)} }\nabla_{\Delta\mathbf{A}} H(\boldsymbol{\Theta}) \right)
        \label{eq:deltacode_Update}
\end{equation}
where
\begin{equation}
\begin{aligned}
		\nabla_{\Delta\mathbf{A}} H(\boldsymbol{\Theta}) &= \sigma_{2}^2\mathbf{D}_{2}^{T}\left( \mathbf{D}_{2}\left(\mathbf{A}_{1} + \Delta\mathbf{A}\right)-\mathbf{P}_{2}\right) \\
&+ \lambda \frac{\left[\mathbf{A}_{1} + \Delta\mathbf{A}\right]_{i}}{\sqrt{\left[\mathbf{A}_{1} + \Delta\mathbf{A}\right]_{i}^{2} + \epsilon_{\mathbf{A}_{1}}^{2}}}
        \label{eq:nabla_deltacode}
\end{aligned}
\end{equation}
and
\begin{equation}
		L_{\Delta\mathbf{A}}^{(k)} = \sigma_{2}^2\left\|\mathbf{D}_{2}^{T}\mathbf{D}_{2}\right\| + \frac{\lambda}{\epsilon_{\mathbf{A}_{1}}}.
        \label{eq:lip_deltacode}
\end{equation}
The proximal operator $\mathrm{prox}^{L_{\Delta\mathbf{A}}^{(k)} }_{\left\|\cdot\right\|_{2,1}}(\cdot)$ can be simply computed as a group soft-thresholding operator \citep{ferraris_robust_2017}, where each group is composed by each column of $\Delta\mathbf{A}$.

\subsection{Optimization with respect to \texorpdfstring{$\mathbf{D}_{\alpha}$}{Da}}

As before, considering the single block optimization variable $\mathbf{D}_{\alpha}$ with $\alpha = \left\{1,2\right\}$, the PALM updating steps can be written as
\begin{equation}
	\label{eq:dictionary_Update}
		\mathbf{D}_{\alpha}^{(k+1)} = \mathcal{P}_{\mathcal{S}}\left(\mathbf{D}_{\alpha}^{(k)} - \frac{1}{L_{\mathbf{D}_{\alpha}}^{(k)} }\nabla_{\mathbf{D}_{\alpha}} H(\boldsymbol{\Theta}) \right)
\end{equation}
where
\begin{equation}
		\nabla_{\mathbf{D}_{\alpha}} H(\boldsymbol{\Theta}) = \sigma^2_{\alpha}\left( \mathbf{D}_{\alpha}\bar{\mathbf{A}}_{\alpha} -\mathbf{P}_{\alpha}\right)\bar{\mathbf{A}}_{\alpha}^{T}
        \label{eq:nabla_dict}
\end{equation}
and
$L_{\mathbf{D}_{\alpha}}^{(k)} $ is the Lipschitz constant
\begin{equation}
		L_{\mathbf{D}_{\alpha}}^{(k)} = \sigma_{\alpha}^{2}\left\|\bar{\mathbf{A}}_{\alpha}\bar{\mathbf{A}}_{\alpha}^{T}\right\|
        \label{eq:lip_dict}
\end{equation}
with $\bar{\mathbf{A}}_{1} = \mathbf{S}\mathbf{A}_{1}$ and $\bar{\mathbf{A}}_{2} = \mathbf{A}_{1} + \Delta\mathbf{A}$. Note that the projection $\mathcal{P}_{\mathcal{S}}(\cdot)$ can be computed as in \citet{mairal_online_2009}, keeping only the values greater than zero.

\subsection{Optimization with respect to \texorpdfstring{$\mathbf{S}$}{S}}
The updating rule of the scaling matrix $\mathbf{S}$ can be written as
\begin{equation}
		\mathbf{S}^{(k+1)} = \mathcal{P}_{\mathcal{C}}\left(\mathbf{S}^{(k)} - \frac{1}{L_{\mathbf{S}^{(k)}}}\nabla_{\mathbf{S}} H(\boldsymbol{\Theta}) \right)
        \label{eq:scale_Update}
\end{equation}
where
\begin{equation}
		\nabla_{\mathbf{S}}H(\boldsymbol{\Theta}) = \sigma_{1}^2\mathbf{D}_{1}^{T}\left( \mathbf{D}_{1}\mathbf{S}\mathbf{A}_{1} - \mathbf{P}_{1}\right)\mathbf{A}_{1}^{T}
        \label{eq:nabla_scale}
\end{equation}
and $L_{\mathbf{S}}^{(k)}$ is the Lipschitz constant related to $\nabla_{\mathbf{S}}f(\boldsymbol{\Theta})$
\begin{equation}
		L_{\mathbf{S}}^{(k)} = \sigma_{1}^2\left\|\mathbf{D}_{1}^{T}\mathbf{D}_{1}\mathbf{A}_{1}\mathbf{A}_{1}^{T}\right\|.
        \label{eq:lip_scale}
\end{equation}
The projection $\mathcal{P}_{\mathcal{C}}(\cdot)$ constrains all diagonal elements of $\mathbf{S}$ to be nonzero.

\subsection{Optimization with respect to \texorpdfstring{$\mathbf{X}_{\alpha}$}{X}}

Finally, the updates of the latent images $\mathbf{X}_{\alpha}$ ($\alpha \in \left\{1,2\right\}$) are achieved as follows
\begin{equation}
 		\mathbf{X}_{\alpha}^{(k+1)} = \mathrm{prox}^{L_{\mathbf{X}_{\mathrm{\alpha}}}^{(k)}}_{\mathcal{D}_{\mathrm{\alpha}}(\mathbf{Y}_{\mathrm{\alpha}}|\cdot)}\left(\mathbf{X}_{\alpha}^{(k)} - \frac{1}{L_{\mathbf{X}_{\alpha}}^{(k)} }\nabla_{\mathbf{X}_{\alpha}} H(\boldsymbol{\Theta}) \right)
       	\label{eq:latent_Update}\\
\end{equation}
with
\begin{equation}
\begin{aligned}
\nabla_{\mathbf{X}_{\alpha}}H(\boldsymbol{\Theta}) &= \sigma_{\alpha}^2\sum_{i=1}^{\Npatch}\mathcal{R}_{\alpha{i}}^{T}\left(\mathcal{R}_{\alpha{i}}\mathbf{X}_{\alpha} - \mathbf{D}_{\alpha}\bar{\mathbf{a}}_{\alpha{i}}\right) \\
       &- \tau_{\alpha}\mathrm{div}\left( \frac{\left[\nabla\mathbf{X}_{1}\right]_{i}}{\sqrt{\left[\nabla\mathbf{X}_{\alpha}\right]_{i}^{2} + \epsilon_{\mathbf{X}_{\alpha}}^{2}}}\right)
        \label{eq:nabla_latent}
        \end{aligned}
\end{equation}
and
\begin{equation}
		L_{\mathbf{X}_{\mathrm{\alpha}}}^{(k)} = \sigma_{\mathrm{\alpha}}^{2}\left\|\sum_{i=1}^{\Npatch}\mathcal{R}_{\alpha i}^{T}\mathcal{R}_{\alpha i}\right\|+ \frac{8\tau_{\mathrm{\alpha}}}{\epsilon_{\mathbf{X}_{\mathrm{\alpha}}}}
        \label{eq:lip_latent}
\end{equation}
and where $\mathrm{div}(\cdot)$  stands for the discrete divergence \citep{chambolle_algorithm_2004}. Note that, $\mathrm{prox}^{L_{\mathbf{X}_{\mathrm{\alpha}}}^{(k)}}_{\mathcal{D}_{\mathrm{\alpha}}(\mathbf{Y}_{\mathrm{\alpha}}|\cdot)}$ represents the proximal mapping for the divergence measure associated with the likelihood function characterizing the modality of the observed image $\mathbf{Y}_{\alpha}$. For the most common remote sensing modalities, e.g., optical and radar, these divergences are well documented and \ref{ap:dft} presents the corresponding proximal operators.

\section{Experiments on synthetic data}
\label{sec:experiments}

\subsection{Simulation framework}

Real datasets for assessing performance of CD algorithms are rarely available. Indeed, this assessment requires couples of images acquired at two different dates, geometrically co-registered, presenting changes and, in our case, they should be also representative of all possible scenarios considered in this paper, i.e., coming from multi-sensor images with possibly different resolutions. In addition, these pairs should be accompanied by a ground truth (i.e., a binary CD mask locating the actual changes) to allow quantitative figures-of-merit to be computed. To alleviate this issue, in the case of multi-band images, a dedicated CD evaluation protocol was proposed by \citet{ferraris_detecting_2017} based on a single high spatial resolution hyperspectral reference image. The experiments conducted in this work follow the same strategy. Two multi-sensor reference images acquired at the same date have been selected as change-free latent images. By conducting simple copy-paste of regions, as in \citet{ferraris_detecting_2017}, changes have been generated in both images as well as their correspondent ground-truth maps. This process allows synthetic yet realistic changes to be incorporated within one of these latent images, w.r.t. a pre-defined binary reference change mask locating the pixels affected by these changes and further used to assess the performance of the CD algorithms. This process is detailed in what follows.

\subsubsection{Reference images}
\label{subsubsec:sim_reference_images}

The reference images $\mathbf{X}^{\mathrm{ref}}_{1}$ and $\mathbf{X}^{\mathrm{ref}}_{2}$ used in this experiment comes from two largely studied open access satellite sensors, namely Sentinel-1 \citep{european_space_agency_sentinel-1_2017} and Sentinel-2 \citep{european_space_agency_sentinel-2_2017} operated by the European Spatial Agency. These images have been acquired over the same geographical area, i.e., the Mud Lake region in Lake Tahoe, at the same date on April 12th 2016. To fulfill the requirements imposed by the considered CD setup, both have been manually geographically and geometrically aligned. Sentinel-2 reference image is a $540 \times 525 \times 3$ image with $10$m spatial resolution and composed of $3$ spectral bands corresponding to visible RGB (Bands $2$ to $4$). On the other hand, Sentinel-1 reference image is a $540 \times 525$ interferometric wide swath high resolution ground range detected multi-looked SAR intensity image with spatial resolution of $10$m according to 5 looks in the range direction.

\subsubsection{Generating the changes}

Using the a procedure similar as the one proposed by \citet{ferraris_detecting_2017}, given the reference images $\mathbf{X}^{\mathrm{ref}}_{\alpha}$ ($\alpha\in\left\{1,2\right\}$), and a previously generated change mask $\mathbf{m}\in \mathbb{R}^{N_\alpha}$, a change image $\mathbf{X}^{\mathrm{ch}}_{\alpha}$ can be generated as
\begin{equation}
\mathbf{X}^{\mathrm{ch}}_{\alpha} = \vartheta\left(\mathbf{X}^{\mathrm{ref}}_{\alpha},\mathbf{m}\right)
\end{equation}
where the change-inducing functions $\vartheta: \mathbb{R}^{\Ndim \times N_\alpha}\times \mathbb{R}^{N_\alpha} \rightarrow \mathbb{R}^{\Ndim \times N_\alpha}$ is defined to simulate realistic changes in some pixels of the reference images. A set of $10$ predefined change masks have been designed according to specific copy-paste change rules similar as the ones introduced by \citet{ferraris_detecting_2017}.

\subsubsection{Generating the observed images}
\label{subsubsec:generating}
The observed images are generated under $3$ distinct scenarios involving $3$ pairs of images of different modalities and resolutions, namely,
\begin{itemize}
  \item Scenario 1 considers two optical images,
  \item Scenario 2 considers two SAR images,
  \item Scenario 3 considers SAR and optical images.
\end{itemize}
Scenarios 1 and 2 are dedicated to images with the same modality. Each test set pair $\left\{\mathbf{X}^{\mathrm{ref}}_{\alpha_1},\mathbf{X}^{\mathrm{ch}}_{\alpha_2}\right\}$ is formed by considering $\left(\alpha_1,\alpha_2\right) = \left(\alpha,\alpha\right)$ with $\alpha=1$ for Scenario 1 and $\alpha=2$ for Scenario 2. Conversely, for Scenario 3 handling multi-sensor images, two test pairs can be formed considering $\alpha_1\neq\alpha_2$, i.e., $\left(\alpha_1,\alpha_2\right) \in \left\{\left(1,2\right),\left(2,1\right)\right\}$.

\subsection{Compared methods}
\label{subsec:compared}

As the number of unsupervised multi-sensor CD methods are extremely reduced, the proposed technique has been compared to the unsupervised fuzzy-based (F) method proposed by \citet{gong_coupled_2016}, that is able to deal with multi-sensor images and to the robust fusion (RF) method proposed by \citet{ferraris_robust_2017} which deals exclusively with multi-band optical images. The fuzzy-based method by \citet{gong_coupled_2016} uses a coupled dictionary learning methodology using a modified K-SVD \citep{aharon_k-svd_2006} with an iterative patch selection procedure to provide only unchanged patches for the coupled dictionary training phase. Then, the sparse code for each observed image is estimated separately from each other allowing to compute the cross-image reconstruction errors. Finally, a local fuzzy C-Means is applied to the mean of the cross-image reconstruction errors in order to separate change and unchanged classes. Equivalently to the proposed one, this method makes no assumption about the joint observation model. On the other hand, the robust fusion method by \citet{ferraris_robust_2017} is based on a more constrained joint observation model, considering that both latent images share the same resolutions and differ only in changed pixels. The final change maps estimated by these two algorithms are denoted as $\hat{\mathbf{m}}_{\mathrm{F}}$ and  $\hat{\mathbf{m}}_{\mathrm{RF}}$, respectively, while the proposed PALM-CDL method provides a change map denoted $\hat{\mathbf{m}}_{\mathrm{CDL}}$.

\subsection{Figures-of-merit}
\label{subsec:figures_of_merit}
The CD performance of these three methods have been assessed by visual inspection of the empirical receiver operating characteristics (ROC) curves, representing the estimated pixel-wise probability of detection ($\mathrm{PD}$) as a function of the probability of false alarm ($\mathrm{PFA}$). Moreover, two quantitative criteria derived from these ROC curves have been computed, namely, i) the area under the curve (AUC), corresponding to the integral of the ROC curve and ii) the distance between the no detection point $(PFA = 1, \mathrm{PD} = 0)$ and the point at the interception of the ROC curve with the diagonal line defined by $\mathrm{PFA} = 1 - \mathrm{PD}$. For both metrics, the greater the criterion, the better the detection.

\newcommand{\subfigwidthROC}{0.65\columnwidth}
\newcommand{\figwidthROC}{1\textwidth}
	\begin{figure*}
    	\centering
        	\begin{subfigure}{\subfigwidthROC}
					\centering
					\includegraphics[width=\figwidthROC]{images/rocS2S2.pdf}
					\caption{}
					\label{fig:rocS2S2}
			\end{subfigure}
			\begin{subfigure}{\subfigwidthROC}
					\centering
				  	\includegraphics[width=\figwidthROC]{images/rocS1S1.pdf}
					\caption{}
					\label{fig:rocS1S1}
			\end{subfigure}
			\begin{subfigure}{\subfigwidthROC}
					\centering
					\includegraphics[width=\figwidthROC]{images/rocS1S2.pdf}
					\caption{}
					\label{fig:rocS1S2}
			\end{subfigure}
			\caption{ROC curves on simulated data for different scenarios:  \protect\subref{fig:rocS2S2} Scenario 1,  \protect\subref{fig:rocS1S1} Scenario 2, \protect\subref{fig:rocS1S2} Scenario 3.}%
            \label{fig:ROC}%
	\end{figure*}


\subsection{Results}

The ROC curves displayed in Fig. \ref{fig:ROC} with corresponding metrics in Table \ref{table:ROCSEN} correspond to the CD results obtained for each specific scenario. These results are discussed below.

\subsubsection{Scenario 1: optical vs. optical}
The ROC curves displayed in Fig. \ref{fig:rocS2S2} with corresponding metrics in Table \ref{table:ROCSEN} (first two rows) correspond to the CD results obtained from a pair of optical observed images. Clearly, the robust fusion achieves the best CD performance. This method has a comparative advantage by exploring the joint model between the optical images, contrary to the fuzzy and proposed methods. Nevertheless, the proposed method achieves very similar performance. More importantly, they provide almost perfect detections even for very low PFA, i.e., for very low energy changes. On the other hand, the fuzzy method, suffers from non detection and false alarm, even when applying the iterative strategy with similar parameter selection approach as in \citet{gong_coupled_2016}. This happens mostly in low energy change regions. The iterative selection is not able to distinguish between low energy and unchanged pixels, which may bias the coupling of dictionaries. Also, the disjoint reconstruction cannot properly deal with low energy changes because coupling is not perfect. In addition, as the methods directly work with the observed images without estimating the latent image, noise can be interpreted as change, thus increasing the false alarm rate.

\subsubsection{Scenario 2: SAR vs. SAR}
As in the previsous case, this dual scenario considers homologous observed SAR images. In this case the ROC is displayed in Fig. \ref{fig:rocS1S1} with corresponding metrics in Table \ref{table:ROCSEN} ($3$rd and $4$th rows). Fig. \ref{fig:rocS1S1} shows that the proposed method offers the highest precision among the compared methods and keeps a high level of detection compared to the Scenario 1. The fuzzy method presents a better accuracy result compared to optical images. One of the reasons is that optical images are generally characterized by richer information, which makes the dictionary coupling more difficult than for two SAR images. At the end, the robust fusion CD method shows a very low detection accuracy as it is not suited to deal with SAR images.

\subsubsection{Scenario 3: optical vs. SAR}
This scenario corresponds to a more difficult problem than the previous one. The physical information extracted in each image cannot be directly related in the observational space, contrary to the previous scenarios. The ROC plot is displayed in Fig. \ref{fig:rocS1S2} with corresponding metrics in Table \ref{table:ROCSEN} (last two rows). As in Scenario 2, Fig. \ref{fig:rocS1S2} shows that the proposed method still offers the highest detection accuracy, while the other methods present a very poor performance. Regarding the fuzzy method, the dictionary and the subsequent sparse code estimations are severely affected by the differences in resolution and dynamics. Even by tuning the algorithmic parameters to increase the weight of the image of lowest dynamics (or lowest resolution), both dictionaries are not properly coupled. Note that, to use the robust fusion method in this challenging scenario, a spectral degradation has been artificially applied to reach both images to the same \emph{spectral} resolution. This has been achieved by considering a band-averaging to finally form a panchromatic image. Resulting detection performance is even poorer than the fuzzy method because it supposes the same physical information between images. Only strong related changes are detected in this case.

\newcommand{\one}[1]{\bf{\textcolor[rgb]{0.00,0.00,1.00}{#1}}}
\newcommand{\two}[1]{\textcolor[rgb]{0.00,0.00,1.00}{#1}}
\setlength{\tabcolsep}{5pt}
\renewcommand{\arraystretch}{1.3}

\begin{table}[h!]
    \caption{Scenarios 1 , 2 \& 3: quantitative detection performance (AUC and distance).}
    \centering
    \begin{tabular}{|c|c|c|c|c|c|}
    \cline{3-5}
    \multicolumn{2}{c|}{} & $\hat{\mathbf{m}}_{\mathrm{CDL}}$ & $\hat{\mathbf{m}}_{\mathrm{F}}$ & $\hat{\mathbf{m}}_{\mathrm{RF}}$ \\
    \hline
		\hline
		\multirow{2}{*}{\rotatebox{00}{Scenario 1}}     & AUC   & $\two{0.9838}$ & $     0.8520$& $\one{0.9946}$\\
                                                     	& Dist. & $\two{0.9677}$ & $	0.7867$& $\one{0.9802}$\\
		\hline
    \multirow{2}{*}{\rotatebox{00}{Scenario 2}}      	& AUC   & $\one{0.9871}$ & $\two{0.9251}$& $0.6819$\\
                                                     	& Dist. & $\one{0.9727}$ & $\two{0.8587}$& $0.6185$\\
    \hline
    \multirow{2}{*}{\rotatebox{00}{Scenario 3}}      	& AUC   & $\one{0.8755}$ & $\two{0.7277}$& $ 0.7227$\\
                                                    	 & Dist. & $\one{0.8097}$ & $\two{0.6758}$& $0.6604$\\
    \hline
    \end{tabular}
  \label{table:ROCSEN}
\end{table}



\section{Experiments on real images}
\label{subsec:reference_images}

Finally, experiments are conducted on real images to emphasize the reliability of the proposed CD method and to illustrate the performance of the proposed algorithmic framework for each specific scenario described in Section \ref{subsubsec:generating}. These experiments consider an additional Sentinel-1 SAR image with the same sensing properties as the one described in Section \ref{subsubsec:sim_reference_images} and acquired on October 28th 2016. Moreover, they consider two multispectral Landsat 8 \citep{united_states_geological_survey_landsat_2017} $180 \times 175$-pixels images with $30$m spatial resolution and composed of the RGB visible bands (Band 2 to 4), acquired over the same region on April 15th 2015 and on September 22th 2015, respectively. Unfortunately, no ground-truth information is available for the chosen dates, as experienced in numerous experimental situations \citep{bovolo_time_2015}. However, this region is characterized by interesting natural meteorological changes occurring along the seasons (e.g., drought of the Mud Lake, snow falls and vegetation growth), which helps to visually infer the major changes between observed images and to assess the relevance of the detected changes. All considered images have been manually geographically and geometrically aligned to fulfill the requirements imposed by the considered CD setup. Each scenario is individually studied considering the same denominations as in Section \ref{subsubsec:generating} and the same compared methods as in Section \ref{subsec:compared}.

\subsection{Scenario 1: optical vs. optical}

In this scenario, two different situations are going to be explored, namely, observed images with the same or different resolutions. The first case considers both Landsat 8 images. Figure \ref{fig:realS2S2_2} depicts the observed images at each date and the change maps estimated by the three compared methods. These change maps have been generated according to \eqref{eq:CVArule} where the threshold has been adjusted such that each method reveals the most important changes, i.e., the drought of the Mud Lake. As expected, the robust fusion method presents better accuracy in detection since it was specifically designed to handle such a scenario. Nevertheless, the proposed method exhibits very similar results. It is worth noting that some of the observed differences are due to the patch decomposition required by the proposed method. The fuzzy method is able to localize the strongest changes, but low energy changes are not detected. The method also suffers from resolution loss due to the size of the patches. Contrary to the proposed method, it does not take the patch overlapping into account, which contributes to decrease the detection accuracy.

\newcommand{\subfwidth}{0.49\columnwidth}
\newcommand{\figsize}{1\columnwidth}

\begin{figure}[h!]
\centering
			\begin{subfigure}{\subfwidth}
					\centering
					\includegraphics[width=\figsize]{images/result_opt_opt/dualCode2/Yt1.pdf}
					\caption{$\mathbf{Y}_{t_1}$}
					\label{fig:s2s2Yt1_2}
			\end{subfigure}
			\begin{subfigure}{\subfwidth}
					\centering
					\includegraphics[width=\figsize]{images/result_opt_opt/dualCode2/Yt2.pdf}
					\caption{$\mathbf{Y}_{t_2}$}
					\label{fig:s2s2Yt2_2}
			\end{subfigure}
			\begin{subfigure}{\subfwidth}
					\centering
					\includegraphics[width=\figsize]{images/result_opt_opt/fuzzy2/cdMAP.pdf}
					\caption{$\hat{\mathbf{m}}_{\mathrm{F}}$}
					\label{fig:s2s2FMAP_2}
			\end{subfigure}
            \begin{subfigure}{\subfwidth}
					\centering
					\includegraphics[width=\figsize]{images/result_opt_opt/rf2/cdMAP.pdf}
					\caption{$\hat{\mathbf{m}}_{\mathrm{RF}}$}
					\label{fig:s2s2RFMAP_2}
			\end{subfigure}
            \begin{subfigure}{\subfwidth}
					\centering
					\includegraphics[width=\figsize]{images/result_opt_opt/dualCode2/cdMAP.pdf}
					\caption{$\hat{\mathbf{m}}_{\mathrm{CDL}}$}
					\label{fig:s2s2DCMAP_2}
			\end{subfigure}
\caption{Scenario 1 (same resolutions): \protect\subref{fig:s2s2Yt1_2}  observed  Landsat 8 MS image $\mathbf{Y}_{t_1}$ acquired on 04/15/2015, \protect\subref{fig:s2s2Yt2_2}  Landsat 8 MS image $\mathbf{Y}_{t_2}$  acquired on 09/22/2015, \protect\subref{fig:s2s2FMAP_2} change map $\hat{\mathbf{m}}_{\mathrm{F}}$ of the fuzzy method, \protect\subref{fig:s2s2RFMAP_2} change map $\hat{\mathbf{m}}_{\mathrm{RF}}$ of the robust fusion method
	 and \protect\subref{fig:s2s2DCMAP_2} change map $\hat{\mathbf{m}}_{\mathrm{CDL}}$ of the proposed method.}%
	\label{fig:realS2S2_2}%
\end{figure}

Under the same scenario (i.e. optical vs. optical), an additional pair of observed images is used to better understand the algorithm behavior when facing to images of the same modality but with different resolutions. The observed image pair is composed of the Sentinel-2 image acquired on April 12th 2016 and the Landsat 8 image acquired in September 22th 2015. Note that the two observed images have the same spectral resolution but different spatial resolutions. Figure \ref{fig:realS2S2_1} depicts the observed images as well as the change maps estimated by the compared methods. Once again it is possible to state the similarity of the results provided by the robust fusion method and the proposed one. It also shows the very poor detection performance of the fuzzy method. This can be explained by the difficulty of coupling due to differences in resolutions.



\begin{figure}[h!]
\centering
			\begin{subfigure}{\subfwidth}
					\centering
					\includegraphics[width=\figsize]{images/result_opt_opt/dualCode/Yt1.pdf}
					\caption{$\mathbf{Y}_{t_1}$}
					\label{fig:s2s2Yt1_1}
			\end{subfigure}
			\begin{subfigure}{\subfwidth}
					\centering
					\includegraphics[width=\figsize]{images/result_opt_opt/dualCode/Yt2.pdf}
					\caption{$\mathbf{Y}_{t_2}$}
					\label{fig:s2s2Yt2_1}
			\end{subfigure}
            \begin{subfigure}{\subfwidth}
					\centering
					\includegraphics[width=\figsize]{images/result_opt_opt/fuzzy/cdMAP.pdf}
					\caption{$\hat{\mathbf{m}}_{\mathrm{F}}$}
					\label{fig:s2s2FMAP_1}
			\end{subfigure}
            \begin{subfigure}{\subfwidth}
					\centering
					\includegraphics[width=\figsize]{images/result_opt_opt/rf/cdMAP.pdf}
					\caption{$\hat{\mathbf{m}}_{\mathrm{RF}}$}
					\label{fig:s2s2RFMAP_1}
			\end{subfigure}
			\begin{subfigure}{\subfwidth}
					\centering
					\includegraphics[width=\figsize]{images/result_opt_opt/dualCode/cdMAP.pdf}
					\caption{$\hat{\mathbf{m}}_{\mathrm{CDL}}$}
					\label{fig:s2s2DCMAP_1}
			\end{subfigure}
\caption{Scenario 1 (different resolutions): \protect\subref{fig:s2s2Yt1_1}  Sentinel-2 MS image $\mathbf{Y}_{t_1}$ acquired on 04/12/2016, \protect\subref{fig:s2s2Yt2_1}  Landsat 8 MS image $\mathbf{Y}_{t_2}$ acquired on 09/22/2015, \protect\subref{fig:s2s2FMAP_1} change map $\hat{\mathbf{m}}_{\mathrm{F}}$ of the fuzzy method, \protect\subref{fig:s2s2RFMAP_1} change map $\hat{\mathbf{m}}_{\mathrm{RF}}$ of the robust fusion method
	 and \protect\subref{fig:s2s2DCMAP_1} change map $\hat{\mathbf{m}}_{\mathrm{CDL}}$ of the proposed method.}%
	\label{fig:realS2S2_1}%
\end{figure}


\subsection{Scenario 2: SAR vs. SAR}

In this scenario, observed SAR images acquired by the same sensor (Sentinel-1) are used to assess the performance of the fuzzy method and the proposed one. The robust fusion method has not been considered due to the poor results obtained on synthetic dataset and reported in Section \ref{sec:experiments}. Figure \ref{fig:realS1S1} presents the observed images at each date and the change maps recovered by the two compared methods. The same strategy of threshold selection as for Scenario 1 has been adopted to reveal the most important changes. As expected, the proposed method presents a higher accuracy in detection than the fuzzy method. Possible reasons that may explain this difference are i) the fuzzy method is unable to handle overlapping patches and ii) the fuzzy method does not exploit appropriate data-fitting terms, in opposite to the proposed one. Besides, as SAR images present strong fluctuations due to their inherent image formation process, the additional TV regularization of the proposed method may contribute to smooth such fluctuations and better couple the dictionaries.

	\begin{figure}[h!]
		\centering
			\begin{subfigure}{\subfwidth}
					\centering
					\includegraphics[width=\figsize]{images/result_sar_sar/dualCode/Yt1.pdf}
					\caption{$\mathbf{Y}_{t_1}$}
					\label{fig:s1s1Yt1}
			\end{subfigure}
			\begin{subfigure}{\subfwidth}
					\centering
					\includegraphics[width=\figsize]{images/result_sar_sar/dualCode/Yt2.pdf}
					\caption{$\mathbf{Y}_{t_2}$}
					\label{fig:s1s1Yt2}
			\end{subfigure}
			\begin{subfigure}{\subfwidth}
					\centering
					\includegraphics[width=\figsize]{images/result_sar_sar/dualCode/cdMAP.pdf}
					\caption{$\hat{\mathbf{m}}_{\mathrm{CDL}}$}
					\label{fig:s1s1DCMAP}
			\end{subfigure}
			\begin{subfigure}{\subfwidth}
					\centering
					\includegraphics[width=\figsize]{images/result_sar_sar/fuzzy/cdMAP.pdf}
					\caption{$\hat{\mathbf{m}}_{\mathrm{F}}$}
					\label{fig:s1s1FMAP}
			\end{subfigure}
\caption{Scenario 2: \protect\subref{fig:s1s1Yt1}  Sentinel-1 SAR image  $\mathbf{Y}_{t_1}$ acquired on 04/12/2016, \protect\subref{fig:s1s1Yt2}  Sentinel-1 SAR image $\mathbf{Y}_{t_2}$ acquired on 10/28/2016, \protect\subref{fig:s1s1FMAP} change map $\hat{\mathbf{m}}_{\mathrm{F}}$ of the fuzzy method and \protect\subref{fig:s1s1DCMAP} change map $\hat{\mathbf{m}}_{\mathrm{CDL}}$ of the proposed method.}%
	\label{fig:realS1S1}%
\end{figure}

\subsection{Scenario 3: optical vs. SAR}

For this scenario, once again, two different situations are addressed: images with the same or different spatial resolutions. The first one considers the Sentinel-2 MS image acquired on April 12th 2016 and the Sentinel-1 SAR image acquired in  October 28th 2016. Figure \ref{fig:realS1S2_2} presents the observed images and the change maps derived from the fuzzy and proposed methods. To derive the change maps, the thresholding strategy is the same as for all previous scenarios. Once again, the proposed method shows better detection accuracy performance than the fuzzy one. It is important to emphasize the similarity of the results achieved in Scenario 3 and Scenario 2 for images acquired at the same date. Note also that this similarity can be observed for the proposed method, which contributes to increase its reliability for CD between multi-sensor images.

\begin{figure}[h!]
\centering
			\begin{subfigure}{\subfwidth}
					\centering
					\includegraphics[width=\figsize]{images/result_sar_opt/dualCode2/Yt1.pdf}
					\caption{$\mathbf{Y}_{t_1}$}
					\label{fig:s1s2Yt1_2}
			\end{subfigure}
			\begin{subfigure}{\subfwidth}
					\centering
					\includegraphics[width=\figsize]{images/result_sar_opt/dualCode2/Yt2.pdf}
					\caption{$\mathbf{Y}_{t_2}$}
					\label{fig:s1s2Yt2_2}
			\end{subfigure}
            \begin{subfigure}{\subfwidth}
					\centering
					\includegraphics[width=\figsize]{images/result_sar_opt/fuzzy2/cdMAP.pdf}
					\caption{$\hat{\mathbf{m}}_{\mathrm{F}}$}
					\label{fig:s1s2FMAP_2}
			\end{subfigure}
			\begin{subfigure}{\subfwidth}
					\centering
					\includegraphics[width=\figsize]{images/result_sar_opt/dualCode2/cdMAP.pdf}
					\caption{$\hat{\mathbf{m}}_{\mathrm{CDL}}$}
					\label{fig:s1s2DCMAP_2}
			\end{subfigure}
\caption{Scenario 3 (same resolution): \protect\subref{fig:s1s2Yt1_2}  Sentinel-2 MS image $\mathbf{Y}_{t_1}$ acquired on 04/12/2016, \protect\subref{fig:s1s2Yt2_2}  Sentinel-1 SAR image $\mathbf{Y}_{t_2}$ acquired on 10/28/2016, \protect\subref{fig:s1s2FMAP_2} change map $\hat{\mathbf{m}}_{\mathrm{F}}$ of the fuzzy method and \protect\subref{fig:s1s2DCMAP_2} change map $\hat{\mathbf{m}}_{\mathrm{CDL}}$ of the proposed method.}%
	\label{fig:realS1S2_2}%
\end{figure}

The second observed image pair consists in a Sentinel-1 SAR image acquired on April 12th 2016 and a Landsat 8 MS image acquired on September 22th 2015. This pair represents the most challenging situation among all presented images, namely differences in both modalities and resolutions. Figure \ref{fig:realS1S2_1} presents the observed images at each date and the recovered change maps. For this last experiment, the proposed method presents better accuracy in detection than the fuzzy one. All differences in all previous situations can be observed in this scenario, culminating in the poor detection performance of the fuzzy method and a reliable change map for the proposed one.

\begin{figure}[h!]
\centering
			\begin{subfigure}{\subfwidth}
					\centering
					\includegraphics[width=\figsize]{images/result_sar_opt/dualCode/Yt1.pdf}
					\caption{$\mathbf{Y}_{t_1}$}
					\label{fig:s1s2Yt1_1}
			\end{subfigure}
			\begin{subfigure}{\subfwidth}
					\centering
					\includegraphics[width=\figsize]{images/result_sar_opt/dualCode/Yt2.pdf}
					\caption{$\mathbf{Y}_{t_2}$}
					\label{fig:s1s2Yt2_1}
			\end{subfigure}
            \begin{subfigure}{\subfwidth}
					\centering
					\includegraphics[width=\figsize]{images/result_sar_opt/fuzzy/cdMAP.pdf}
					\caption{$\hat{\mathbf{m}}_{\mathrm{F}}$}
					\label{fig:s1s2FMAP_1}
			\end{subfigure}
			\begin{subfigure}{\subfwidth}
					\centering
					\includegraphics[width=\figsize]{images/result_sar_opt/dualCode/cdMAP.pdf}
					\caption{$\hat{\mathbf{m}}_{\mathrm{CDL}}$}
					\label{fig:s1s2DCMAP_1}
			\end{subfigure}
\caption{Scenario 3 (different resolutions): \protect\subref{fig:s1s2Yt1_1}  Sentinel-1 SAR image $\mathbf{Y}_{t_1}$ acquired on 04/12/2016, \protect\subref{fig:s1s2Yt2_1}  Landsat 8 MS image $\mathbf{Y}_{t_2}$ acquired on 09/22/2015, \protect\subref{fig:s1s2FMAP_1} change map $\hat{\mathbf{m}}_{\mathrm{F}}$ of the fuzzy method and \protect\subref{fig:s1s2DCMAP_1} change map $\hat{\mathbf{m}}_{\mathrm{CDL}}$ of the proposed method.}%
	\label{fig:realS1S2_1}%
\end{figure}

\section{Conclusion}
\label{sec:conclusion}

This paper proposed an unsupervised multi-sensor change detection technique to handle the most common remote sensing imagery modalities. The technique was based on the definition of a pair of latent images related to the observed images through a direct observation model. These latent images were modelled thanks to a coupled dictionary and sparse codes which provide a common representation of the homologous patches in the latent image pair. The differences between estimated codes were assumed to be spatially sparse, implicitly locating the changes. Inferring these representations, as well as the latent images, was formulated as an inverse problem which was solved with the proximal alternate minimization iterative algorithm dedicated to non-smooth and non-convex functions. Contrary to the methods already proposed in the literature, scaling problems due to differences in resolutions and/or dynamics were solved by introducing a scaling matrix relating coupled atoms. A simulation protocol allowed the performance of the proposed technique in terms of detection and precision to be assessed and compared with the performance of various algorithms. A real dataset collecting images from different multispectral and  SAR sensors at the same region was used to assess the reliability of the proposed method. Results showed that the method outperformed all state-of-the-art comparable methods in multi-sensor scenarios while presenting similar results as methods benefiting from prior knowledge on the scenario modelling.

\begin{appendix}

\section{Data-fitting terms and corresponding proximal operators}
\label{ap:dft}

The data-fitting term $\mathcal{D}(\cdot|\cdot)$ is intimately related to the modality of the target image. This term defines the negative log-likelihood function relating the observed and latent images. Below, the most common data fitting terms and their associated proximal mappings are derived, defined as

\begin{equation}
  \mathrm{prox}^{\eta}_{\mathcal{D}(\mathbf{Y}|\cdot)}\left(\mathbf{U}\right) = \operatornamewithlimits{argmin}_{\mathbf{X}} \mathcal{D}(\mathbf{Y}|\mathbf{X})
  + \frac{\eta}{2} \left\|\mathbf{X}-\mathbf{U}\right\|_{\mathrm{F}}^2.
\end{equation}

\subsection{Multiband optical images}

Multiband optical images represent the most common modality of remotely sensed images. For this modality, the noise model may take into account several different noise sources \citep{deger_sensor_2015}. Nevertheless, it is commonly considered as additive Gaussian, up to some considerations in acquisition, for instance sufficient number of arriving photons. Therefore, the direct model $\mathit{T}_{\mathrm{MO}}[\cdot]$ in \eqref{eq:sensortransf} can be expressed as
\begin{equation}
			\mathbf{Y} = \mathbf{X} + \mathbf{N}
\end{equation}
where the noise matrix $\mathbf{N}$ is assumed to be distributed according to a matrix normal distribution (see, e.g., \citep{ferraris_robust_2017} for more details). Consequently, by assuming the noise components are independent and identically distributed\footnote{Pixelwise independence of the noise is a common assumption while spectral whiteness of the noise can be ensured by applying a whitening transform as pre-processing.} (i.i.d.), the data-fitting term associated with multiband optical images is
\begin{equation}
\mathcal{D}_{\mathrm{MO}}(\mathbf{Y}|\mathbf{X}) = \frac{1}{2}\left\|\mathbf{Y}-\mathbf{X}\right\|_{\mathrm{F}}^{2}.
\end{equation}
An explicit proximal operator associated with this function can be derived as
\begin{equation}
\mathrm{prox}^{\eta}_{\mathcal{D}_{\mathrm{MO}}(\mathbf{Y}|\cdot)}\left(\mathbf{U}\right) = \frac{\mathbf{Y} + \eta\mathbf{U}}{\eta+1}
\end{equation}
\subsection{Multi-look intensity synthetic aperture radar images}

SAR images correspond to the second most common modality of remote sensing images used in many applications. One of the main characteristics of such modality is that it allows to measure the scene in poor weather conditions and also during the night since SAR is an active sensor. Nevertheless, this configuration yields the speckle phenomenon, resulting from random fluctuations of the reflectivity of the backscattered signals. Many studies have been conducted to understand and mitigate the speckle phenomenon. A common approach that helps to decrease the speckle level while increasing the SNR consists in averaging samples of the same pixel acquired over independent observations. This procedure is usually referred to as multi-look processing. According to this strategy, the generation model is considered as a multiplicative perturbation by i.i.d random variables $\mathbf{N} = [n_i,\cdots,n_{N}]$ following a common gamma probability density function in intensity images with unit mean $\mathrm{E}[n_i] = 1$ and variance $\mathrm{var}[n_i] = \frac{1}{r}$ where $r$ is the number of looks. The direct model $\mathit{T}_{\mathrm{SAR}}[\cdot]$ can thus be written as
\begin{equation}
	\mathbf{Y} = \mathbf{X} \odot \mathbf{N}
\end{equation}
where $\odot$ denotes the termwise (i.e., Hadamard) product.

By assuming pixel independence, the data-fitting term for each pixel can be expressed as the sum of Itakura-Saito divergences
\begin{equation}
\mathcal{D}_{\mathrm{SAR}}(\mathbf{Y}|\mathbf{X}) =
 \sum_{i=1}^N \left(\frac{y_i}{x_i} - \log \frac{y_i}{x_i} - 1\right)
\end{equation}
This function has been widely considered for speckle removing \citep{aubert_variational_2008,woo_proximal_2013} and also music analysis \citep{fevotte_nonnegative_2009}. Nevertheless, it usually leads to a challenging non-convex problem which admits more than one global solution. In \citet{sun_alternating_2014}, the associated proximal operator is derived by computing the root of a $3$rd degree-polynomial equation. An alternative consists in considering an approximation by resorting to a log-transform of the data, e.g., leading to an I-divergence \citep{woo_proximal_2013,steidl_removing_2010}. Up to a constant, this divergence can be rewritten equivalently as a Kullback-Leibler divergence which is closely related to Poisson modeling \citep{figueiredo_restoration_2010}
\begin{equation}
\mathcal{D}_{\mathrm{SAR}}(\mathbf{Y}|\mathbf{X}) = \sum_{i=1}^N \left(x_i - y_i \log x_i\right).
\end{equation}
This data-fitting term leads to an explicit proximal operator for the $i$th component given by
\begin{equation}
\mathrm{prox}^{\eta}_{\mathcal{D}_{\mathrm{SAR}}(y_i|\cdot)}\left(u_i\right) = \frac{1}{2}\left(u_i - \frac{1}{\eta} + \sqrt{\left(u_i - \frac{1}{\eta}\right)^2 + \frac{4y_i}{\eta}}\right).
\end{equation}

\section{Usual proximal mappings involved in the parameter updates}
\label{ap:proj}

The projections and proximal operators involved on PALM algorithm \citep{bolte_proximal_2014} and described in Algorithm \ref{algo:PALM_SCDL_Diff} are properly defined as:
\begin{itemize}
\item The proximal map for $\mathbf{A}_{1}$ accounting for the sum $\lambda\left\|\cdot\right\|_1 + \iota_{\geq0}(\cdot)$ is explicitly given by:
\begin{equation}
\mathrm{prox}^{\eta}_{\lambda\left\|\cdot\right\|_1 + \geq0}\left(\mathbf{A}_{1}\right) =   \max\left(|a_{1,{(ji)}}|-\frac{\lambda}{\eta},0\right) \quad \forall (i,j)
\end{equation}
\item The proximal map for $\Delta\mathbf{A}$ accounting for the $\gamma\left\|\cdot\right\|_{2,1}$ is explicitly given by:
\begin{equation}
\hspace{-0.5cm}\mathrm{prox}^{\eta}_{\gamma\left\|\cdot\right\|_{2,1}}\left(\Delta\mathbf{A}\right) =  \begin{cases} \left(1 - \frac{\gamma}{\eta\left\|\Delta\mathbf{a}_i\right\|_{2}}\right)\Delta\mathbf{a}_i & \text{if} \left\|\Delta\mathbf{a}_i\right\|_{2}>\frac{\gamma}{\eta}\\
				0 & \text{otherwise.}
\end{cases}
\end{equation}
\item Projecting $\mathbf{D}$ onto set $\mathcal{S}$ can be computed explicitly based on \citet{thouvenin_modeling_2017,bolte_proximal_2014} which is given by:
\begin{equation}
\mathcal{P}_{\mathcal{S}}\left(\mathbf{D}\right) = \frac{\mathcal{P}_{+}(\mathbf{d}_{i})}{\left\|\mathcal{P}_{+}(\mathbf{d}_{i})\right\|^2_2} \quad \forall i = 1 \cdots \Natom
\end{equation}
with
\begin{equation}
\mathcal{P}_{\mathcal{+}}\left(\mathbf{d}_{i}\right) = \max\left(0,d_{(j,i)}\right) \quad \forall j = 1 \cdots \Ndim
\end{equation}
\item Projecting $\mathbf{S}$ onto set $\mathcal{C}$ is explicitly given by:
\begin{equation}
\mathcal{P}_{\mathcal{C}}\left(\mathbf{S}\right) =  \begin{cases} \max\left(0,s_{(j,i)}\right) & \forall i=j \\
0 & \text{otherwise}
\end{cases}
\end{equation}
\end{itemize}
\end{appendix}
\section*{Acknowledgments}
The authors would like to thank Prof. Jose M. Bioucas-Dias, Universidade de Lisboa, Portugal, for fruitful discussion regarding this work.


\bibliographystyle{elsarticle-harv}
\bibliography{strings_all_ref,publications_clean}
\end{document}



