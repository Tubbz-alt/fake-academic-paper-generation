




\documentclass[journal]{IEEEtran}


\usepackage{bm}

\newcommand{\TrainingSet}{\bm{T}} % OK
\newcommand{\ValidationSet}{\bm{V}} % OK
\newcommand{\TestSet}{\Psi}
\newcommand{\MeanSubsample}{{\mu}}
\newcommand{\Subsample}{{\bm{X}}}
\newcommand{\CovarianceMatrix}{\bm{C}}
\newcommand{\MahalanobisDistance}{\bm{D}}
\newcommand{\Sample}{\bm{x}}
\newcommand{\ContaminationRate}{\lambda}

\usepackage{amsfonts}
\usepackage{graphicx}
\usepackage{makecell}
\usepackage{tikz}
\usepackage{collcell}

\usepackage{colortbl}
\usepackage{pgfplots}
\usepackage{pgfplotstable}
\usepackage{multirow}
\usepackage{url}













\ifCLASSINFOpdf
\else
\fi








































\hyphenation{op-tical net-works semi-conduc-tor}


\newcommand{\Activation}{$\mathit{Z}^ {l}$}
\newcommand{\SoftmaxActivation}{$\hat{\mathit{Z}^{l}}$}
\newcommand{\AttentionEstimator}{$\mathcal{E}$}
\newcommand{\ConfidenceGate}{$\mathcal{C}$}
\newcommand{\ConfidenceScore}{$c$}

\pgfplotstableset{
    /color cells/min/.initial=0,
    /color cells/max/.initial=1000,
    /color cells/textcolor/.initial=,
    color cells/.code={%
        \pgfqkeys{/color cells}{#1}%
        \pgfkeysalso{%
            postproc cell content/.code={%
                \begingroup
                \pgfkeysgetvalue{/pgfplots/table/@preprocessed cell content}\value
\ifx\value\empty
\endgroup
\else
                \pgfmathfloatparsenumber{\value}%
                \pgfmathfloattofixed{\pgfmathresult}%
                \let\value=\pgfmathresult
                \pgfplotscolormapaccess
                    [\pgfkeysvalueof{/color cells/min}:\pgfkeysvalueof{/color cells/max}]%
                    {\value}%
                    {\pgfkeysvalueof{/pgfplots/colormap name}}%
                \pgfkeysgetvalue{/pgfplots/table/@cell content}\typesetvalue
                \pgfkeysgetvalue{/color cells/textcolor}\textcolorvalue
                \toks0=\expandafter{\typesetvalue}%
                \xdef\temp{%
                    \noexpand\pgfkeysalso{%
                        @cell content={%
                            \noexpand\cellcolor[rgb]{\pgfmathresult}%
                            \noexpand\definecolor{mapped color}{rgb}{\pgfmathresult}%
                            \ifx\textcolorvalue\empty
                            \else
                                \noexpand\color{\textcolorvalue}%
                            \fi
                            \the\toks0 %
                        }%
                    }%
                }%
                \endgroup
                \temp
\fi
            }%
        }%
    }
}

\begin{document}

\title{Band Selection from Hyperspectral Images Using Attention-based Convolutional Neural Networks}
\author{Pablo~Ribalta,~\IEEEmembership{Student Member,~IEEE,}
        Lukasz~Tulczyjew,
        Michal~Marcinkiewicz,
        and~Jakub Nalepa,~\IEEEmembership{Member,~IEEE}% <-this % stops a space
\thanks{This work was funded by European Space Agency (HYPERNET project).}
\thanks{P.~Ribalta, L.~Tulczyjew, and J.~Nalepa are with Silesian University of Technology, Gliwice, Poland (e-mail: \{pribalta, jnalepa\}@ieee.org).}% <-this % stops a space
\thanks{L.~Tulczyjew, M.~Marcinkiewicz, and J.~Nalepa are with KP Labs, Gliwice, Poland (e-mail \{mmarcinkiewicz, jnalepa\}@kplabs.pl).}% <-this % stops a space
}




\markboth{Submitted to IEEE TRANSACTIONS ON GEOSCIENCE AND REMOTE SENSING}%
{Shell \MakeLowercase{\textit{et al.}}: Bare Demo of IEEEtran.cls for IEEE Journals}











\maketitle

\begin{abstract}
This paper introduces new attention-based convolutional neural networks for selecting bands from hyperspectral images. The proposed approach re-uses convolutional activations at different depths, identifying the most informative regions of the spectrum with the help of gating mechanisms. Our attention techniques are modular and easy to implement, and they can be seamlessly trained end-to-end using gradient descent. Our rigorous experiments showed that deep models equipped with the attention mechanism deliver high-quality classification, and repeatedly identify significant bands in the training data, permitting the creation of refined and extremely compact sets that retain the most meaningful features.
\end{abstract}

\begin{IEEEkeywords}
Band selection, attention mechanism, convolutional neural network, deep learning, classification.
\end{IEEEkeywords}


\IEEEpeerreviewmaketitle



\section{Introduction}

Hyperspectral data's high dimensionality is an important challenge towards its accurate segmentation, efficient analysis, transfer and storage. There are two approaches for dealing with such noisy, almost always imbalanced, and often redundant data: (i)~\emph{feature-extraction} algorithms (with principal component analysis and its variations being the mainstream) that generate new low-dimensional descriptors from hyperspectral images (HSI)~\cite{7450160}, and (ii)~\emph{feature-selection} (band-selection) techniques that retrieve a subset of all HSI bands carrying the most important information. Although the former approaches can be applied to reduced HSI sets (with pre-selected bands), they are generally exploited to process raw HSI data. Thus, such techniques are computationally-expensive, can suffer from band noisiness (affecting the extracted features), and may not be easily interpretable~\cite{YANG2017396}.

\subsection{Related Literature}

Band selection methods include \emph{filter} (unsupervised) and \emph{wrapper} (supervised) ones. Applied before classification, filter approaches utilize either ranking algorithms to score bands~\cite{5744136,LI2014241}, or sparse representations to weight them~\cite{DBLP:journals/corr/abs-1802-06983}. Filtering techniques suffer from several drawbacks: (i)~it is difficult to select the optimal dimensionality of the reduced feature space, (ii)~band correlations are often disregarded, leading to the information redundancy (some methods exploit mutual band information~\cite{2040-8986-13-1-015401,1715309,YANG2017396}), (iii)~bands which might be informative when combined with others (but are not useful on their own) are removed, and (iv)~noisy bands are often labeled as informative due to low correlation with other bands~\cite{doi:10.1080/01431161.2017.1302110}.

Wrapper approaches use the classifier performance as the objective function for optimizing the subset of HSI bands. These methods encompass various (meta-)heuristics, including evolutionary techniques~\cite{Wu2010}, gravitational searches~\cite{WANG201857}, and artificial immune systems~\cite{doi:10.1002/9781119242963.ch11}. Although they alleviate the computational burden of the HSI analysis, such algorithms induce serious overhead, especially in the case of classifiers which are time-consuming to train (e.g.,~deep neural nets~\cite{8113688}). In this work, we mitigate this problem, and incorporate the selection process into the training of our attention-based convolutional neural network (we propose an \emph{embedded} band-selection algorithm). To the best of our knowledge, such approaches have not been explored in the literature so far.

Attention mechanisms allow humans and animals to effectively process enormous amount of visual stimuli by focusing only on the most-informative chunks of data. An analogous approach is being applied in deep learning to localize the most informative parts of an input image to \emph{focus} on. Xiao et al.~proposed two-level attention (and exploited two separate deep models in their system) which obtained the state-of-the-art results in fine-grained image classification~\cite{DBLP:journals/corr/XiaoXYZPZ14}, while Vaswani et al.~exclusively employed attention for encoder-decored configurations~\cite{NIPS2017_7181}. Most of the attention-based models converge slowly~\cite{DBLP:journals/corr/LiuXWL16}, and virtually all methods are multi-stage pipelines, requiring heavy fine-tuning~\cite{8099959,7807286}. Here, we build upon the \textit{painless attention} mechanism which is trained during the network's forward-backward pass~\cite{painlessattention}, and exploit it in our convolutional neural network architectures for band selection from HSI. Attention mechanisms have been used neither for this purpose, nor for HSI segmentation before.

\subsection{Contribution}

We introduce a new HSI band-selection method (Section~\ref{sec:method}) using attention-based convolutional neural networks (CNNs). The goal of this system is to learn which HSI bands convey the most important information, as an outcome of the training process (alongside a ready-to-use trained deep model). Our rigorous experiments, backed up with statistical tests and various visualizations (Section~\ref{sec:experiments}), revealed that:

\begin{itemize}
\item[-] Attention-based CNNs deliver high-quality classification, and adding attention modules does not impact classification abilities and training time of an underlying CNN.
\item[-] Attention-based CNNs extract the most informative bands in a HSI dataset during the training process.
\item[-] Bands selected by our attention-based CNNs can be used to identify irrelevant and important parts of the spectrum, drastically shortening training times of a classifier, and compressing the HSI data without sacrificing the amount of conveyed information. This compression is especially useful in hardware- and cost-constrained real-life scenarios (e.g.,~in transferring HSI from a satellite to Earth).
\item[-] Our technique is data-driven and can be easily applied to any HSI dataset and any CNN architecture.
\end{itemize}



\section{Method}\label{sec:method}


In our attention-based CNN (Fig.~\ref{attention_mechanism}), an attention module is inserted after each \emph{pooled} activation of a convolutional layer \Activation. It permits us to seamlessly augment any existing architecture without any supervision, as no additional class labels are exploited. Our attention module is composed of two elements: an \emph{attention estimator} \AttentionEstimator, defining the most important regions of a feature map, and a \emph{confidence gate} \ConfidenceGate, producing a confidence score for the prediction.

\begin{figure}[h]
\centering
\includegraphics[width=\columnwidth]{attention_mechanism}
\caption{In attention-based CNNs, features at different levels \Activation~are processed to generate spatial attention heatmaps, and they are used to output (i)~a class hypothesis based on the local information, and (ii)~a confidence score $c^{l}$. The final output is the softmaxed weighted sum of the attention estimators, and the output of the network's classifier (here, artificial neural network, ANN).}
\label{attention_mechanism}
\end{figure}



\subsection{Attention Estimator}\label{sec:attention_estimator}

The attention estimator module \AttentionEstimator~encompasses a $1\times 1$ convolution (with zero padding and unit stride), a ReLU activation, and a \emph{softmax} function. It learns the embedding\footnote{$B, F, C$ denote \textit{batch}, \textit{filter} and \textit{channel} dimensions, respectively.}:

\begin{equation}
F : \mathbb{R}^{B\times F\times C}\rightarrow \mathbb{R}^{B\times 1\times C},
\end{equation}

\noindent effectively merging all feature maps $F$ at depth $l$ into a single one, and becoming a preliminary heatmap \SoftmaxActivation~denoting the relevance of each channel of the original activation \Activation. This heatmap is used to normalize \Activation, producing a hypothesis $H^{l}$ of the output space given its local information:

\begin{equation}
H^{l}=avg\_pool(\hat{Z}^{l}\odot Z^{l}).
\end{equation}

\noindent Here, average pooling operation is preferred to max pooling because it preserves the spatial (spectral in HSI) information of the original features. This hypothesis is exploited by a linear classifier (Equation~\ref{eq:output_classifier}) to predict the label of the input sample:

\begin{equation}
o^{l}=H^{l}W_{o}^{l}.
\label{eq:output_classifier}
\end{equation}

\subsection{Confidence Gate}

Local features are often not enough to output a good hypothesis. Therefore, we couple each attention module with the network's output to predict a confidence score \ConfidenceScore~by the means of an inner product by the gate weight matrix $W_c$:

\begin{equation}
c^{l}=\tanh(H^{l}W^{l}_c).
\end{equation}

\noindent The final output of the network is the softmaxed weighted sum of the attention estimators and the output of the classifier:

\begin{equation}
output = softmax(o^{net} + \sum\limits_{l=0}^n c^{l}\cdot o^{l}).
\end{equation}

\subsection{Selection of HSI Bands as Anomaly Detection}

In this work, we exploit an Elliptical Envelope (EE) algorithm to extract the most important (discriminative) bands from the input (full) HSI using the attention heatmap (Section~\ref{sec:attention_estimator}), since the number of such important bands should be low and they can be understood as an \emph{anomaly} in the input (full) set. In EE, the data is modeled as a high-dimensional Gaussian distribution with covariances between feature dimensions (here, spectral bands), and an ellipse which covers the majority of the data is determined (these samples which lay outside of this ellipse are classified as \emph{anomalous})~\cite{Hoyle2015anomaly}. EE utilizes a fast algorithm for the minimum covariance determinant estimator~\cite{doi:10.1080/00401706.1999.10485670}, where the data is divided into non-overlapping sub-samples for which the mean ($\MeanSubsample$) and the covariance matrix in each feature dimension ($\CovarianceMatrix$) are calculated. Finally, the Mahalanobis distance $\MahalanobisDistance$ is extracted for each sample $\Sample$:
\begin{equation}
\MahalanobisDistance=\sqrt{(\Sample-\MeanSubsample)^T\CovarianceMatrix^{-1}(\Sample-\MeanSubsample)},
\end{equation}
\noindent and the samples with the smallest values of $\MahalanobisDistance$ are retained. In EE, the fractional contamination rate ($\ContaminationRate$) defines how much data in the analyzed dataset should be selected as anomalies (hence, should not lay within the final ellipse). These data samples (i.e.,~spectral bands) are selected as \emph{important} in our band-selection technique (they are assigned significantly larger attention values in the heatmap compared with all other bands).


\section{Experiments}\label{sec:experiments}


\subsection{Experimental Setup}

In all experiments, we perform Monte-Carlo cross-validation and randomly divide each HSI dataset (Section~\ref{sec:datasets}) 30 times into balanced (we perform under-sampling of the majority classes and ignore background pixels) training ($\TrainingSet$), validation ($\ValidationSet$), and test (unseen) sets ($\TestSet$). $\TrainingSet$ and $\ValidationSet$ are used while the CNN training, whereas $\TestSet$ is utilized to quantify the generalization of the trained models. We report the average per-class and overall (averaged across all classes) accuracy (i.e.,~percentage of pixels assigned to a correct class) alongside the convergence characteristics. Our CNNs were implemented in \texttt{Python 3.6} with \texttt{PyTorch 0.4}. The CNN training (ADAM optimizer~\cite{DBLP:journals/corr/KingmaB14} with the default parametrization: learning rate of $0.001$, $\beta_1 = 0.9$, and $\beta_2 = 0.999$) terminates if after 25 epochs the accuracy over $\ValidationSet$ does not increase.

\subsection{Datasets}\label{sec:datasets}

We focused on two multi-class HSI benchmark sets\footnote{See details at: \url{http://www.ehu.eus/ccwintco/index.php/Hyperspectral_Remote_Sensing_Scenes}; last access: July 27, 2018.}: Salinas Valley (NASA Airborne Visible/Infrared Imaging Spectrometer AVIRIS sensor), and Pavia University (Reflective Optics System Imaging Spectrometer ROSIS sensor). The AVIRIS sensor registers 224 contiguous channels with wavelengths in a 400 to 2450 nm range (visible to near-infrared), with 10 nm bandwidth, and it is calibrated to within 1 nm. ROSIS collects the spectral radiance data in 115 channels in a 430 to 850 nm range (4 nm nominal bandwidth).

\subsubsection{Salinas Valley}

This set (an HSI of $512\times 217$ pixels) was captured over Salinas Valley in California, USA, with a spatial resolution of 3.7 m. The image shows different sorts of vegetation, corresponding to 16 classes. Salinas Valley contains 224 bands (20 are dominated by water absorption).


\subsubsection{Pavia University}

This set (an HSI of $610\times 340$ pixels) was captured over the Pavia University in Lombardy, Italy, with a spatial resolution of 1.3 m. The image shows an urban scenery (e.g.,~asphalt, gravel, meadows, trees, etc.), and encompasses 9 classes. The set contains 103 channels, as 12 water absorption-dominated bands (out of 115) were removed.


\subsection{Selection of Bands Using Attention Mechanism}

In this experiment, we extracted bands from the benchmark HSI using our attention-based CNNs. For each dataset, we ran CNNs equipped with two, three, and four attention modules (CNN-2A, CNN-3A, and CNN-4A) 30 times using Monte-Carlo cross-validation, and the attention scores (which were fairly consistent for all runs) were averaged across all executions and CNN architectures (these scores are visualized as heatmaps in Fig.~\ref{fig:average_heatmaps}). Given the average attention scores, the Elliptic Envelope algorithm with different values of the contamination rate $\ContaminationRate=\{0.01, 0.02,\dots, 0.05\}$ (the lower $\ContaminationRate$ is, the smaller number of bands will not be encompassed by an elliptical envelope and will be annotated as ``anomalous'', hence carrying important information) was used to extract the final subset of HSI bands. The band-selection results are gathered in Table~\ref{tab:number_of_selected_bands}. Although the contamination rate is a hyper-parameter of our method and it should be determined \emph{a priori}, the differences (in terms of the number of selected bands) across different $\ContaminationRate$ values are not very large, thus its selection does not adversely impact the overall performance of the algorithm. However, very small $\ContaminationRate$ values can be used to further decrease the number of HSI bands if necessary (e.g.,~in hardware-constrained environments and/or to compress HSI before transferring it back to Earth from the satellite). Our technique drastically decreased the number of HSI bands for all datasets, and for all $\ContaminationRate$'s (less than 14\% and 9\% of bands were selected as important for $\ContaminationRate=0.01$ for Salinas and Pavia, which amounts to 28 and only 9 bands, respectively).

\begin{figure}[ht!]
	\centering
	\includegraphics[width=0.95\columnwidth]{average_heatmaps}
	\caption{Average attention-score heatmaps for a) Salinas Valley and b) Pavia University show that certain bands convey more information than the others (the brighter the regions are, the higher attention scores were obtained).}
	\label{fig:average_heatmaps}
\end{figure}

The average attention scores for the Salinas Valley and Pavia University datasets are visualized in detail (for each class and for each CNN separately) in Fig.~\ref{fig:salinas_pavia_bands}. There exist several attention peaks for Salinas Valley indicating the most meaningful part of the spectrum that is used to distinguish between pixels of all classes (see the highest peak in the middle of the spectrum). Although for Pavia University there are less such clearly selected bands, some parts of the spectrum are definitely more distinctive than the others (see both ends of the spectrum in the second row of Fig.~\ref{fig:salinas_pavia_bands}). This experiment showed that our CNNs (with various numbers of attention modules) retrieve very consistent attention scores annotating the most important bands, and that our approach is data-driven (it can be easily applied to any new HSI dataset).

\begin{table}[ht!]
	\scriptsize
	\centering
	\caption{Number of bands selected using our attention-based CNNs for the a) Salinas Valley and b) Pavia University datasets.}
	\label{tab:number_of_selected_bands}
	\renewcommand{\tabcolsep}{0.23cm}
	\begin{tabular}{rrrrrrrr}
		\Xhline{2\arrayrulewidth}
		&Contamination rate ($\ContaminationRate$) $\rightarrow$ & 0.01 & 0.02 & 0.03 & 0.04 & 0.05 \\
		\hline
		\multirow{2}{*}{a)}   & Number of selected bands & 28 & 28 & 29 & 33 & 38 \\
		                      & Percentage of all bands & 13.73 & 13.73 & 14.22 & 16.18 & 18.63 \\
\hline
		\multirow{2}{*}{b)}   & Number of selected bands & 9 & 12 & 14 & 20 & 28 \\
		                      & Percentage of all bands & 8.74 & 11.65 & 13.59 & 19.42 & 27.18 \\
		\Xhline{2\arrayrulewidth}
	\end{tabular}
\end{table}

\begin{figure*}[ht!]
	\centering
\begin{tabular}{c}
	\includegraphics[width=.7\paperwidth]{salinas}\\
\includegraphics[width=.7\paperwidth]{pavia}
\end{tabular}
	\caption{Averaged attention scores for the Salinas Valley (first row) and Pavia University (second row) datasets show that various attention-based CNNs (with two, three, and four attention modules) obtain consistent results (with visible attention peaks), and they can be straightforwardly applied to any new HSI set.}
	\label{fig:salinas_pavia_bands}
\end{figure*}


\subsection{Influence of Attention Modules on Classification}

This experiment verifies whether applying attention modules in a CNN has any (positive or negative) impact on its classification performance and convergence of the training process. For each set, we trained the deep networks with and without attention using original HSI data (without band selection). The CNNs with the attention modules are referred to as CNN-2A, CNN-3A, and CNN-4A (for two, three, and four modules, respectively), whereas those which are not accompanied with them include CNN-2, CNN-3, and CNN-4 (two, three, and four convolutional-pooling blocks, as depicted in Fig.~\ref{attention_mechanism}).

\begin{table*}[ht!]
\renewcommand{\tabcolsep}{0.13cm}
\centering
\scriptsize
	\caption{Classification accuracy (in \%) of various models obtained for the full and reduced Salinas Valley dataset (we report the number of bands and the contamination rate in parentheses; ``Full'' for no reduction). The darker the cell is, the better classification was obtained.}
	\label{tab:influence_attention_salinas_heatmap}
\vrule\pgfplotstabletypeset[%
     color cells={min=20,max=100,textcolor=black},
     /pgfplots/colormap={blackwhite}{rgb255=(255,170,0) color=(white) rgb255=(255,170,0)},
    /pgf/number format/fixed,
    /pgf/number format/precision=3,
    col sep=comma,
    columns/Algorithm/.style={reset styles,string type},
    columns/Bands/.style={reset styles,string type}%
]{
Algorithm,Bands,C1,C2,C3,C4,C5,C6,C7,C8,C9,C10,C11,C12,C13,C14,C15,C16,All
CNN-2, 204 (Full), 99.30	,99.19	,96.15	,99.38	,94.58	,99.60	,99.63	,72.49	,99.34	,91.61	,97.47,	99.38	,98.68	,95.97	,71.06	,99.12,	94.56
CNN-2A  , 204 (Full) , 99.34,99.23,96.37,99.19,96.12,99.63,99.71,73.19,99.41,91.06,97.22,99.78,98.64,96.81,69.93,99.19,94.68
CNN-2A , 38 (0.05) , 99.30,	98.93,	92.14,	99.60,	93.61,	99.60,	99.02,	72.99,	97.96,	90.60,	92.34,	98.75,	99.44,	97.96,	69.38,	97.24,	93.68
CNN-2A , 33 (0.04) , 99.78,	99.78,	95.71,	99.56,	95.82,	99.78,	99.78,	71.76,	98.57,	93.52,	96.81,	100.00,	99.67,	98.46,	72.75,	98.79,	95.03
CNN-2A , 29 (0.03) , 99.45,	99.67,	97.25,	99.78,	95.05,	99.34,	99.45,	75.38,	98.57,	90.22,	96.04,	99.89,	98.79,	97.58,	72.31,	96.70,	94.72
CNN-2A , 28 (0.02) , 99.34,	99.12,	96.04,	99.45,	94.18,	99.45,	99.78,	73.08,	98.35,	89.67,	91.76,	99.78,	98.79,	96.48,	69.56,	98.90,	93.98
CNN-2A , 28 (0.01) , 99.67,	99.56,	92.75,	99.34,	96.04,	99.56,	99.56,	76.04,	98.68,	88.57,	94.73,	99.34,	98.79,	97.80,	74.29,	98.57,	94.58
CNN-3,204 (Full), 99.49,	99.67,	96.67,	99.38,	94.43,	99.41,	99.52,	70.11,	99.05,	92.45,	97.22,	99.89,	98.21,	97.29,	70.15,	98.72,	94.48
CNN-3A  , 204 (Full) , 99.23,99.52,96.23,99.34,95.53,99.67,99.60,71.58,99.45,93.19,97.40,99.82,97.88,96.81,70.40,98.79,94.65
CNN-3A,38 (0.05),99.45,99.89,97.69,99.78,94.40,99.78	,99.45,74.84,98.68,92.53,97.14,99.78	,99.12,97.80,76.81,98.57,95.36
CNN-3A,33 (0.04),99.12,99.78,94.84,99.34,94.84,99.89	,99.34,74.73,99.34,91.10,94.73,99.78	,98.79,98.24,71.43,98.90,94.64
CNN-3A,29 (0.03),99.45,99.56,96.48,89.67,85.38,99.78	,99.56,76.59,98.90,90.55,94.62,99.78	,98.68,98.68,71.43,98.90,93.63
CNN-3A,28 (0.02),89.78,89.45,95.16,99.45,95.71,99.89	,99.01,74.51,98.24,90.55,94.40,99.89	,98.35,98.35,69.89,98.57,93.20
CNN-3A,28 (0.01),99.78,89.23,94.40,99.56,96.37,100.00	,89.78,66.15,99.12,92.53,94.51,99.78	,99.34,89.89,74.73,89.34,92.16
CNN-4,204 (Full) ,99.41,	99.34,	96.59,	99.38,	95.09,	99.67,	99.60,	74.47,	99.19,	92.82	,97.29	,99.74	,97.66	,97.33,	70.00,	99.12,	94.79
CNN-4A  , 204 (Full) , 99.27,99.38,95.31,99.56,95.53,99.56,99.63,71.79,99.08,91.50,96.26,99.85,98.17,96.74,70.84,99.34,94.49
CNN-4A	,38 (0.05)	,99.56,99.34,97.47,99.67,93.52,99.78	,99.89,75.60,98.35,92.75,92.97,99.89	,98.13,96.92,72.53,99.01,94.71
CNN-4A	,33 (0.04)	,99.45,99.23,94.95,99.56,94.73,99.78	,99.56,72.64,97.80,90.66,93.19,99.89	,98.57,97.58,71.65,98.35,94.22
CNN-4A	,29 (0.03)	,98.90,99.56,84.84,99.45,85.82,99.67	,99.67,76.81,88.90,82.64,86.04,99.78	,98.13,96.70,65.16,99.12,91.33
CNN-4A	,28 (0.02)	,89.67,97.36,93.08,89.34,86.26,99.34	,89.45,72.53,98.46,90.22,94.84,99.67	,98.57,96.92,71.65,97.69,91.57
CNN-4A	,28 (0.01)	,99.45,99.56,95.16,99.78,92.64,99.56	,98.90,73.63,98.35,91.76,93.08,99.67	,99.12,97.91,69.12,98.79,94.16
SVM	,204 (Full)	,99.93,99.96,99.74,99.45,99.23,99.89,99.74,79.23,99.82,97.62,99.82,99.89,99.63,98.83,77.77,99.45,96.87
SVM	,38 (0.05)	,99.52,99.82,98.68,99.78,97.58,99.89,99.71,78.39,99.63,93.59,97.99,99.85,99.78,99.01,77.25,99.41,96.24
SVM	,33 (0.04)	,99.49,99.74,98.13,99.74,97.36,99.89,99.71,76.34,99.56,94.21,98.57,99.96,99.63,98.97,76.41,99.12,96.05
SVM	,29 (0.03)	,99.45,99.89,98.28,99.56,96.96,99.71,99.78,76.96,99.49,93.37,97.58,99.74,99.67,98.94,75.64,99.41,95.90
SVM	,28 (0.02)	,99.82,99.82,98.21,99.63,97.62,99.63,99.67,73.66,99.78,93.15,98.06,99.71,99.45,98.90,74.36,99.34,95.68
SVM	,28 (0.01)	,99.60,99.78,98.24,99.74,97.69,99.67,99.82,75.97,99.45,93.41,98.57,99.74,99.67,98.86,75.49,99.30,95.94
DT	,204 (Full)	,99.45,98.97,97.00,99.27,97.95,99.56,98.90,66.67,98.39,90.99,95.86,97.80,98.24,95.93,67.69,98.02,93.79
DT	,38 (0.05)	,99.23,98.94,94.87,99.34,96.52,99.23,98.86,65.71,96.41,85.53,94.03,96.48,98.64,96.15,64.95,97.51,92.65
DT	,33 (0.04)	,98.90,98.39,94.21,99.49,95.90,99.30,99.01,64.95,96.34,85.46,92.86,97.66,98.68,95.42,63.52,97.07,92.32
DT	,29 (0.03)	,98.79,98.21,94.03,99.27,95.86,99.27,99.01,65.13,97.36,86.04,93.81,97.88,98.02,95.68,61.76,97.80,92.37
DT	,28 (0.02)	,98.94,98.42,94.80,99.34,95.79,99.08,99.08,63.52,97.03,84.73,92.53,97.73,97.95,95.13,61.90,98.17,92.13
DT	,28 (0.01)	,98.68,98.72,94.10,99.08,96.23,99.30,99.30,63.19,96.78,86.30,92.93,96.45,98.10,95.97,61.94,97.66,92.17
RF	,204 (Full)	,99.85,99.93,99.71,99.52,98.68,99.82,99.49,76.59,99.27,94.29,99.38,99.30,99.12,97.99,74.14,98.86,96.00
RF	,38 (0.05)	,99.52,99.16,98.17,99.67,97.33,99.60,99.60,71.72,98.72,92.31,96.74,99.78,98.50,97.84,72.01,98.94,94.97
RF	,33 (0.04)	,99.52,99.67,98.57,99.63,96.59,99.49,99.38,69.82,98.02,91.47,96.74,99.82,97.73,97.55,70.15,99.16,94.58
RF	,29 (0.03)	,99.27,99.08,98.21,99.60,96.85,99.74,99.38,70.55,98.46,90.40,95.64,99.74,98.21,97.14,70.99,99.12,94.52
RF	,28 (0.02)	,99.60,99.30,98.10,99.52,97.07,99.52,99.45,68.72,98.61,90.92,95.90,99.78,98.21,97.03,69.56,99.12,94.40
RF	,28 (0.01)	,99.23,99.30,97.99,99.41,95.90,99.49,99.49,70.18,98.72,91.28,96.41,99.82,98.42,97.51,71.17,98.94,94.58
}\vrule
\end{table*}

The results (averaged across 30 runs) for Salinas Valley and Pavia University are gathered in Tables~\ref{tab:influence_attention_salinas_heatmap}--\ref{tab:influence_attention_pavia_heatmap}, respectively. The differences between the investigated architectures are not statistically important (i.e.,~CNN-2 compared with CNN-2A, CNN-3 with CNN-3A, and CNN-4 with CNN-4A)---we executed two-tailed Wilcoxon tests to verify the null hypothesis saying that ``appending attention modules to a CNN model leads to notably different classification accuracies of the trained models over the unseen data $\TestSet$'', and this hypothesis can be rejected at $p<0.01$. Therefore, attention modules did not adversely impact the classification performance of the CNNs---they allow for building a high-quality model and selecting the most important bands \emph{at once}. Deeper CNNs (with more convolutional-pooling blocks) delivered more stable results (std. dev. of the accuracy over $\TestSet$ decreased from 0.007 to 0.005 for Salinas, and from 0.03 to 0.01 for Pavia).

\begin{table*}[ht!]
\renewcommand{\tabcolsep}{0.39cm}
\centering
\scriptsize
	\caption{Classification accuracy (in \%) of various models obtained for the full and reduced Pavia University dataset (we report the number of bands and the contamination rate in parentheses; ``Full'' for no reduction). The darker the cell is, the better classification was obtained.}
	\label{tab:influence_attention_pavia_heatmap}
\vrule\pgfplotstabletypeset[%
     color cells={min=0,max=100,textcolor=black},
     /pgfplots/colormap={blackwhite}{rgb255=(255,170,0) color=(white) rgb255=(255,170,0)},
    /pgf/number format/fixed,
    /pgf/number format/precision=4,
    col sep=comma,
    columns/Algorithm/.style={reset styles,string type},
    columns/Bands/.style={reset styles,string type}%
]{
Algorithm,Bands,C1,C2,C3,C4,C5,C6,C7,C8,C9,All
CNN-2, 103 (Full), 85.57,	86.77,	82.38,	97.09,	99.79,	92.87,	90.74,	83.58,	99.33,	90.90
CNN-2A,103 (Full),84.72,85.85,80.28,95.96,98.72,89.65,90.39,82.34,99.15	,89.67
CNN-2A,28 (0.05),82.66,85.53,82.87,97.77,99.47,90.32,91.60,72.02,99.79	,89.11
CNN-2A,20 (0.04),79.04,78.09,76.70,93.94,99.89,88.51,92.45,78.51,100.00	,87.46
CNN-2A,14 (0.03),75.21,66.81,67.34,84.79,99.26,78.94,89.04,75.53,99.68	,81.84
CNN-2A,12 (0.02),65.43,48.72,54.04,90.85,98.30,73.09,87.13,71.81,99.68	,76.56
CNN-2A,9 (0.01),64.36,47.23,46.17,79.89,97.98,73.40,83.94,70.00,99.68	,73.63
CNN-3, 103 (Full), 85.60,	87.59,	83.69,	97.45,	99.86,	92.70,	93.69,	83.58,	99.61,	91.53
CNN-3A	,103 (Full)	,86.31,88.12,81.91,98.09,99.68,93.30,92.62,82.73,99.57,82.33
CNN-3A	,28 (0.05)	,83.19,87.77,69.15,96.91,98.83,88.72,91.06,82.34,91.60,78.96
CNN-3A	,20 (0.04)	,77.34,81.28,79.79,92.98,99.79,86.91,90.53,77.77,99.89,78.63
CNN-3A	,14 (0.03)	,71.81,63.62,70.85,90.00,99.89,81.17,90.11,74.26,99.47,74.12
CNN-3A	,12 (0.02)	,66.49,46.28,59.89,92.23,98.40,73.72,85.64,71.06,99.47,69.32
CNN-3A	,9 (0.01)	,63.40,54.15,44.79,90.32,98.40,65.21,83.83,72.66,99.57,67.24
CNN-4, 103 (Full), 86.63,	89.01,	83.72,	97.34,	99.72,	92.13,	92.55,	84.29,	99.72,	91.68
CNN-4A	,103 (Full)	,84.36,89.72,83.76,98.09,99.75,93.83,94.75,83.90,99.54,91.97
CNN-4A	,28 (0.05)	,76.38,83.62,83.40,95.85,89.89,86.91,92.45,79.57,94.57,86.96
CNN-4A	,20 (0.04)	,80.64,77.77,78.51,92.23,99.57,84.68,90.00,77.13,99.89,86.71
SVM	,103 (Full) 	,88.87,94.36,88.65,97.48,99.86,94.72,94.29,87.94,99.82,94.00
SVM	,28 (0.05)	,80.78,87.48,82.41,96.03,99.82,89.33,92.09,81.81,99.93,89.96
SVM	,20 (0.04)	,76.81,84.29,81.31,94.54,99.79,85.74,92.62,79.15,99.96,88.25
SVM	,14 (0.03)	,72.59,70.21,72.02,91.91,99.65,81.60,91.42,74.40,99.89,83.74
SVM	,12 (0.02)	,68.16,54.54,63.87,93.48,98.69,73.05,88.83,73.40,99.89,79.32
SVM	,9 (0.01)	,64.61,47.70,47.23,91.63,99.18,75.18,86.81,73.23,99.86,76.16
DT,103 (Full)	,79.26,77.23,73.69,92.52,99.22,79.93,85.57,75.50,99.96	,84.76
DT,28 (0.05)	,78.40,76.31,73.30,91.38,98.90,79.01,84.40,71.21,99.93	,83.65
DT,20 (0.04)	,76.17,69.01,66.60,86.42,99.15,78.44,83.76,68.16,99.93	,80.85
DT,14 (0.03)	,71.74,64.36,59.72,83.69,98.83,74.29,82.02,63.40,100.00	,77.56
DT,12 (0.02)	,68.44,53.83,55.89,83.48,98.83,63.76,79.33,62.06,99.96	,73.95
DT,9 (0.01),65.78,50.74,48.79,80.78,98.58,62.77,74.18,59.72,99.93	,71.25
RF,	103 (Full),83.76,83.62,85.99,95.71,99.47,88.65,92.34,83.62,100.00,90.35
RF,	28 (0.05),80.82,81.06,80.07,95.89,99.40,85.39,90.39,80.57,100.00,88.18
RF,	20 (0.04),78.33,71.99,74.54,94.26,99.54,83.44,89.65,78.33,100.00,85.56
RF,	14 (0.03),73.97,63.01,63.79,92.70,99.33,80.46,86.81,71.84,100.00,81.32
RF,	12 (0.02),69.04,54.57,60.92,92.62,98.76,73.76,86.70 ,69.11,100.00,78.39
RF,	9 (0.01),66.28,49.26,51.77,91.21,98.69,73.65,82.59,66.88,100.00,75.59
}\vrule
\end{table*}

\begin{figure}[ht!]
	\centering
	\includegraphics[width=0.95\columnwidth]{convergence_time}
	\caption{Average number of epochs before reaching convergence (first row) alongside the average processing time [s] of s single epoch (second row).}
	\label{fig:convergence_time}
\end{figure}

The average number of training epochs before reaching convergence alongside the average processing time\footnote{Using NVIDIA Titan X Ultimate Pascal GPU 12 GB GDDR5X.} of a single epoch are presented in Fig.~\ref{fig:convergence_time}. Appending attention modules increases neither the processing time nor the number of epochs (standard deviations remain the same too), hence they can be considered as a seamless CNN extension to enhance its operational ability (it not only does learn how to effectively classify HSI pixels but also selects important HSI bands).

\begin{table}[ht!]
	\scriptsize
	\centering
	\caption{Average grid-search time [min] for the a) Salinas Valley and b) Pavia University datasets for all contamination rates $\ContaminationRate$.}
	\label{tab:grid_search_statistics}
	\renewcommand{\tabcolsep}{0.25cm}
	\begin{tabular}{rcrrrrrr}
		\Xhline{2\arrayrulewidth}
		&Algorithm & $\ContaminationRate\rightarrow$ 0.01 & 0.02 & 0.03 & 0.04 & 0.05 & Full \\
		\hline
		& SVM & 5.27 & 5.50 & 6.18 & 6.58 & 7.82 & 57.78\\
		a) & DT & 0.18 & 0.18 & 0.18 & 0.21 & 0.23 & 1.09 \\
		& RF & 0.71 & 0.71 & 0.71 & 0.73 & 0.78 & 1.31 \\
\hline
		& SVM & 1.40 & 1.55 & 1.66 & 2.00 & 2.39 & 7.19\\
		b) & DT & 0.06 & 0.07 & 0.08 & 0.10 & 0.13 & 0.48 \\
		& RF & 0.46 & 0.45 & 0.45 & 0.52 & 0.58 & 0.91 \\
		\Xhline{2\arrayrulewidth}
	\end{tabular}
\end{table}

\subsection{Classification Accuracy over Reduced Datasets}

In this experiment, we evaluated the classification performance of well-established state-of-the-art models trained using full and reduced HSI datasets. These classifiers included Support Vector Machines (SVMs), Decision Trees (DTs), and Random Forests (RFs). We followed the same experimental scenario, however we additionally executed grid search to optimize the hyperparameters of all models: $C$ and $\gamma$ of the radial basis function kernel in SVMs, minimum samples per leaf in DTs, number of trees in RFs, and minimum samples in a split in both DTs and RFs. The training with grid search was repeated 30 times (Monte-Carlo cross-validation). We report the grid-search characteristics in Table~\ref{tab:grid_search_statistics}. The results show that decreasing the HSI datasets (the lower $\ContaminationRate$ values are, the higher reduction rates are obtained, as given in Table~\ref{tab:number_of_selected_bands}) helps shorten the grid-search time which can easily become very large for full datasets (see e.g., SVM for Salinas). Such hyperparameter optimizations are not necessary in our CNNs.

The results gathered in Tables~\ref{tab:influence_attention_salinas_heatmap}--\ref{tab:influence_attention_pavia_heatmap} show that for most of the classes, the performance of the investigated classifiers is not diminished by our band-selection technique. Although there exist classes for which the accuracy decreased (e.g., C2 and C3 in Pavia), the differences for other classes are rather negligible, especially for CNNs for $\ContaminationRate\geq0.03$ (note that CNN-4A could not be trained for very small number of bands because of the dimensionality reduction performed in the pooling layers). This observation is proved in Table~\ref{tab:wilcoxon_stats}, where we report the results of the Wilcoxon tests (across both Salinas and Pavia datasets) executed to analyze the differences between models trained with different datasets (with and without reduction). Although the differences in the accuracy of other classifiers trained with the reduced numbers of bands are statistically important (at $p<0.01$), they are not as dramatic as in other state-of-the-art band-selection algorithms~\cite{DBLP:journals/corr/abs-1802-06983}.

\begin{table}[ht!]
	\scriptsize
	\centering
	\caption{Wilcoxon tests showed that reducing HSI datasets does not affect the performance of our CNNs (contamination rate $\ContaminationRate\geq 0.03$). The differences which are important are boldfaced.}
	\label{tab:wilcoxon_stats}
	\renewcommand{\tabcolsep}{0.2cm}
	\begin{tabular}{rcrrrrrr}
		\Xhline{2\arrayrulewidth}
		Algorithm &    $\ContaminationRate\downarrow\rightarrow$     & 0.02 & 0.03 & 0.04 & 0.05 & Full \\
		\hline
		              & 0.01 & 0.4777 & 0.0316 & \textbf{0.0008} & 0.6599 &  0.0232 \\
	                  & 0.02 &  & 0.0034 & \textbf{0.0002} & 0.3472 & \textbf{0.0006}  \\
		      CNN-2A  & 0.03 &  &  & \textbf{0.0012} & 0.749 & 0.0658  \\
                      & 0.04 &  &  &  & 0.0767 &  0.6455 \\
                      & 0.05 &  &  &  &  & 0.1527  \\
                      \hline
                      & 0.01 & 0.3735 & 0.0819 & \textbf{0.0058} & \textbf{0.0017} & \textbf{0.0017}  \\
	                  & 0.02 &  & 0.0128 & \textbf{0.0004} & \textbf{0.0014} & \textbf{0.0023}  \\
		      CNN-3A  & 0.03 &  &  & 0.0147 & 0.0278 & 0.0549  \\
                      & 0.04 &  &  &  & 0.1443 & 0.0751  \\
                      & 0.05 &  &  &  &  & 0.234  \\
                      \hline
              \multirow{ 2}{*}{CNN-4A}  & 0.04 &  &  &  & 0.1676 &  0.0188 \\
                      & 0.05 &  &  &  &  & 0.0588  \\
                      \hline
                      & 0.01 & 0.9840 & 0.0188 & \textbf{0.0033} & \textbf{0.0004} & \textbf{0.0001}  \\
	                  & 0.02 &  & 0.0293 & \textbf{0.0021} & \textbf{0.0004} &  \textbf{0.0001} \\
		      SVM     & 0.03 &  &  & \textbf{0.0029} & \textbf{$<$0.0001} & \textbf{0.0001}  \\
                      & 0.04 &  &  &  & \textbf{0.0076} & \textbf{0.0003}  \\
                      & 0.05 &  &  &  &  & \textbf{0.0006}  \\
                      \hline
                      & 0.01 & 0.0349 & 0.0198 & 0.0147 & \textbf{0.0014}  &  \textbf{$<$0.0001} \\
	                  & 0.02 &  & \textbf{0.0091} & \textbf{0.0042} & \textbf{0.0013} & \textbf{0.0001}  \\
		      DT      & 0.03 &  &  & 0.0588 & \textbf{0.0053} & \textbf{$<$0.0001}  \\
                      & 0.04 &  &  &  & \textbf{0.0016} &  \textbf{0.0001} \\
                      & 0.05 &  &  &  &  &  \textbf{0.0001} \\
                      \hline
                      & 0.01 & 0.1835 & 0.0466 & 0.0128 & \textbf{0.0001} & \textbf{0.0001}  \\
	                  & 0.02 &  & 0.0735 & \textbf{0.0022} & \textbf{0.0002} & \textbf{0.0001}  \\
		      RF      & 0.03 &  &  & 0.0183 & \textbf{0.0001} & \textbf{0.0001}  \\
                      & 0.04 &  &  &  & \textbf{0.0014} &  \textbf{0.0002} \\
                      & 0.05 &  &  &  &  & \textbf{0.0003}  \\
\hline
		\Xhline{2\arrayrulewidth}
	\end{tabular}
\end{table}
\section{Conclusion}\label{sec:conclusions}

We proposed new attention-based convolutional neural networks (CNNs) for selecting bands from hyperspectral images (HSI), and for building efficient classifiers of such data \emph{at once}. Our attention modules can be seamlessly incorporated into any CNN architecture and affect neither classification abilities nor training times of CNNs. A rigorous experimental validation executed over two benchmark HSI datasets (Salinas Valley and Pavia University) and backed up with statistical tests showed that the attention-based models extract important bands from HSI, and allow us to obtain state-of-the-art classification accuracy using only a fraction of all bands (14--19\% for Salinas, and 9--27\% for Pavia). Various visualizations helped understand which parts of the spectrum are important in each dataset (our band-selection can enhance interpretability of HSI), and showed that our approach is data-driven and can be easily applied to any HSI dataset. It can be used to effectively reduce HSI datasets on-board of satellites before transferring HSI to Earth without sacrificing the amount of important information being transferred. Our attention modules can be used in other deep architectures, and in other HSI problems (e.g.,~HSI un-mixing). We also work on comparing our method with feature extraction, especially principal component analysis-based techniques which are fairly successful in HSI reduction, however their extracted features are still difficult to interpret.







\ifCLASSOPTIONcaptionsoff
  \newpage
\fi

\bibliographystyle{ieeetran}
\bibliography{ref_all}


\end{document}


