\documentclass[10pt,twocolumn,letterpaper]{article}

\usepackage{iccv}
\usepackage{times}
\usepackage{epsfig}
\usepackage{graphicx}
\usepackage{amsmath}
\usepackage{amssymb}
 \usepackage{color}
 \usepackage{xcolor}



\usepackage[pagebackref=true,breaklinks=true,letterpaper=true,colorlinks,bookmarks=false]{hyperref}

 \iccvfinalcopy % *** Uncomment this line for the final submission

\def\iccvPaperID{1237} % *** Enter the ICCV Paper ID here
\def\httilde{\mbox{\tt\raisebox{-.5ex}{\symbol{126}}}}
\definecolor{gray}{rgb}{0.1,0.1,0.5}


\ificcvfinal\pagestyle{empty}\fi
\begin{document}


\title{Recurrent Network Models for Human Dynamics}
\author{Katerina Fragkiadaki \quad \quad Sergey Levine \quad \quad Panna Felsen \quad \quad Jitendra Malik\\
University of California, Berkeley\\
Berkeley, CA \\
{\tt\small \{katef,svlevine@eecs,panna@eecs,malik@eecs\}.berkeley.edu}
}

\pdfoutput=1

\maketitle
\footnote{#2}
\newcounter{#1}
\setcounter{#1}{\value{footnote}}
}\newcommand{\footnoterecall}[1]{
}%trajectory\newcommand{\p}{\mathbf{p}}% \newcommand{\P}{\mathbf{P}}%\newcommand{\ntr}{n_{\tr}}%\newcommand{\ntr}{|\mathcal{T}|}\newcommand{\link}{\mathrm{links}}\newcommand{\vol}{\mathrm{vol}}\newcommand{\ncut}{\mathrm{ncut}}\newcommand{\AF}{\mathbb{A}}\newcommand{\BF}{\mathbb{B}}\newcommand{\VF}{\mathbb{V}}\newcommand{\px}{\mathrm{p}}\newcommand{\Edge}{E^D}\newcommand{\nr}{n_R}\newcommand{\xx}{\mathbf{x}}\newcommand{\ww}{\mathbf{w}}\newcommand{\wwp}{\mathbf{w}^{D}}\newcommand{\aex}{a}\newcommand{\ca}{c}\newcommand{\nd}{n_D}% \newcommand{\Wbar}{\mathbf{\hat{W}}}% \newcommand{\Wbar}{\mathbf{\hat{A}}_T}% %\newcommand{\Asteers}{\mathbf{A}_\mathrm{steer}}% % \newcommand{\Asteer}{\mathrm{steer}(\A;h)}% \newcommand{\Asteer}{\W^\mathrm{steer}}% \newcommand{\Asteernoh}{\mathbf{A}_\mathbf{steer}}% \newcommand{\Aco}{\mathbf{A}^{\mathrm{c}}}% %detectlet repulsions% \newcommand{\RD}{\mathbf{R}_D}% \newcommand{\dlsel}{h}% \newcommand{\RDF}{\mathbf{R}_D^\dlsel}\newcommand{\RSPX}{\mathbf{R}_R}% %trajectet affinities% \newcommand{\AT}{\mathbf{A}_T}% % %trajectlet repulsions% \newcommand{\RT}{\mathbf{R}_T}% %set of foreground trajectlets% \newcommand{\Tf}{\mathcal{T}^\mathbf{f}}% %set of foreground detectlets% \newcommand{\Df}{\mathcal{D}^\mathbf{f}}%associations% \newcommand{\C}{\mathbf{C}}% % %trajectlet cluaster to detectlets associations% \newcommand{\Cg}{\mathbf{C}_G}% % %detection compatibility% \newcommand{\D}{\mathbf{D}}\newcommand{\DD}{\mathcal D}%trajectlet cluster% \newcommand{\g}{\mathrm{G}}% % % detection score% \newcommand{\score}{f}\newcommand{\nS}{n_{S}}\newcommand{\den}{\mathrm{\rho}}\newcommand{\spx}{\mathrm{r}}\newcommand{\prt}{\mathrm{d}}\newcommand{\Seg}{\mathbf{S}}% \newcommand{\aff}{\alpha}\newcommand{\aff}{\ww}\newcommand{\affr}{\ww^R}% bounding box% \newcommand{\bbox}{\mathrm{box}}% \newcommand{\T}{\mathcal{T}}\newcommand{\Diag}{\mathrm{Diag}}% \newcommand{\Dcap}{\mathcal{D}}\newcommand{\conf}{\mathbf{c}}\newcommand{\stab}{\mathbf{ncut}}\newcommand{\cmtb}{\mathbf{cmtb}}\newcommand{\pset}{\mathcal{P}}% \newcommand{\todo}{\textcolor{red}{TODO: }}\newcommand{\attr}{\mathrm{att}}% \newcommand{\na}{n(\allX)}% \newcommand{\location}{\mathrm {loc}}% % \newcommand{\Tr}{\mathrm {Tr}}% % %\newcommand{\yf}{\dlsel_{\Dcap}}% \newcommand{\sel}{\mathbf {s}}% % \newcommand{\yf}{h^\sel}% % \newcommand{\yf}{\sel}% % \newcommand{\yf}{y_D}% \newcommand{\ATDh}{\mathbf{A}_T^{\yf}}% % \newcommand{\Ch}{\mathbf{C}^{\yf}}% \newcommand{\Ch}{\mathrm{steer}(\C;h)}% %\newcommand{\RSPXst}{\mathrm{steer}(\A;h)}% \newcommand{\RTh}{\mathbf{R}_T^{\yf}}% % \newcommand{\ATD}{\tilde {\mathbf{A}}_T}\% \newcommand{\ATD}{\mathbf{A}_T}% \newcommand{\Ss}{\mathbf{S}(\PP)}% \newcommand{\RR}{\mathcal{R}}% % \newcommand{\ASPXI}{\mathbf{A}}\newcommand{\V}{\mathbf{V}}\newcommand{\Y}{\mathrm{Y}}\newcommand{\rr}{\mathrm{r}}\newcommand{\pa}{p}\newcommand{\EV}{\mathbf{\Lambda}}\newcommand{\WE}{\mathbf{W}}\newcommand{\K}{K}\newcommand{\ce}{\mathbf{c}}\newcommand{\width}{\mathrm{w}}\newcommand{\height}{\mathrm{h}}\newcommand{\nf}{n_F}\newcommand{\sss}{\tilde{s}}\newcommand{\PS}{\mathbf{P}}\newcommand{\SSp}{\mathcal{S}}\newcommand{\N}{\mathcal{N}}% \newcommand{\DT}{\mathcal{D}}% \newcommand{\DT}{G^D}\newcommand{\DT}{G}% \newcommand{\di}{\mathbf{S}}\newcommand{\di}{\mathbf{d}}% \newcommand{\den}{\mathrm{\delta}}\newcommand{\edge}{\mathrm{e}}\newcommand{\ASPX}{\mathbf{A}_{R}}% \newcommand{\ASPX}{\mathbf{A}}% % \newcommand{\ASPXst}{\mathbf{A}_{R}^{\mathrm{steer}}}% % \newcommand{\ASPXst}{\mathbf{W}_{J}}%  \newcommand{\ASPXst}{\mathbf{A}^{\mathrm{steer}}}\newcommand{\ASPXst}{\mathbf{W}^{\mathrm{steer}}}\newcommand{\PP}{\mathcal{D}}\newcommand{\bo}{t_f}% % \newcommand{\pn}{\textbf{Spec.Part.}}% \newcommand{\pn}{\mathrm{Partition}}% \newcommand{\ptwisemul}{\bullet}% % \newcommand{\ptwisemul}{\bullet}% % \newcommand{\modify}{M}% % \newcommand{\ones}{\mathbf 1}% \newcommand{\suchthat}{\mathrm{s.t.}}% % \newcommand{\e}{e}% \newcommand{\nmultseg}{|\mathrm{Seg}|}% % \newcommand{\pool}{\mathcal P}% \newcommand{\npool}{n(\allX)}% %\newcommand{\npool}{| \mathcal P |}% %\newcommand{\npool}{|\allX|}% \newcommand{\Dr}{\mathbf{D}}% \newcommand{\allX}{\mathbf X}% \newcommand{\Diag}{\mathrm Diag}% \newcommand{\allY}{\mathbf Y}\newcommand{\s}{\mathcal S (\mathcal D)}%  \newcommand{\nd}{n_D}% \newcommand{\footnoteremember}[2]{% \footnote{#2}% \newcounter{#1}% \setcounter{#1}{\value{footnote}}% }% \newcommand{\footnoterecall}[1]{% \footnotemark[\value{#1}]% }
  
  
  
  \begin{abstract}
We propose the Encoder-Recurrent-Decoder (ERD) model for recognition and prediction of human body pose in videos and motion capture. The ERD model is a  recurrent neural network that incorporates nonlinear encoder and decoder networks before and after  recurrent layers. 
We test instantiations of ERD architectures in the tasks of motion capture (mocap) generation, body pose labeling and body pose forecasting in videos. Our model handles mocap training data across multiple subjects and activity domains, and  synthesizes novel motions while avoiding drifting for long periods of time.   
 For human pose labeling, ERD outperforms a per frame body part detector by resolving left-right body part confusions. 
For video pose forecasting, ERD  predicts body joint displacements across a temporal horizon of 400ms and outperforms a first order motion model based on optical flow. 
ERDs extend previous  Long Short Term Memory (LSTM) models  in the literature to jointly learn representations and their dynamics. Our experiments show such representation learning is crucial for both labeling and prediction in space-time. We find this is a distinguishing feature between the spatio-temporal visual domain in comparison to  1D text, speech or  handwriting, where straightforward hard coded representations have shown excellent results when directly combined with recurrent units \cite{DBLP:conf/nips/SutskeverVL14} .   

\end{abstract}%%\begin{abstract}%We propose the Encoder-Recurrent-Decoder (ERD) model for recognition and prediction of human body pose in videos and motion capture. The ERD model is a  recurrent neural network that incorporates nonlinear encoder and decoder networks before and after  recurrent layers. %We test instantiations of ERD architectures in the tasks of motion capture (mocap) generation, body pose labeling and body pose forecasting in videos. Our model handles mocap training data across multiple subjects and activity domains, and  synthesizes novel motions while avoid drifting for long periods of time.   % For human pose labeling, ERD outperforms a per frame body part detector by resolving left-right body part confusions. %For video pose forecasting, ERD  predicts body joint displacements across a temporal horizon of 400ms and outperforms a first order motion model based on optical flow. %ERDs extend previous  Long Short Term Memory (LSTM) models  in the literature to jointly learn representations and their dynamics. Our experiments show such representation learning is crucial for both labeling and prediction in space-time. We find this is a distinguishing feature between the spatio-temporal visual domain in comparison to  1D text, speech or  handwriting, where straightforward hard coded representations have shown excellent results when directly combined with recurrent units.   
  
  

   




 
   \section{$\auto$s for recurrent kinematic tracking and forecasting}%We build a model that  predicts a distribution over plausible poses in the next frame given a human pose sequence so far.  Figure \ref{fig:autornn} illustrates   $\auto$ models for recurrent kinematic tracking and forecasting.    
At  each time step $t$, vector $\data_t$ of a sequence  $\seq=(\data_1, \cdots, \data_T)$ passes through the encoder, the recurrent layers, and the decoder network, producing the output $y_t$. 
In general, we are interested in estimating some function $f(x)$ of the input $x$ at the current time step, or at some time in the future. For example, in the case of motion capture, we are interested in estimating the mocap vector at the next frame. Since both the input and output consists of mocap vectors, $f$ is the identity transformation, and the desired output at step $t$ is $f(x_{t+1})$.
In case of video pose labeling and forecasting, $f(x)$ denotes body joint locations corresponding to the image in the current bounding box $x$. At step $t$, we are interested in estimating either $f(x_t)$ in the case of labeling, or $f(x_{t+H})$ in the case of forecasting, where $H$ is the forecast horizon.

The units in each recurrent layer  implement the Long Short Term Memory functions \cite{Hochreiter:1997:LSM:1246443.1246450}, where writing, resetting, and reading a value from each recurrent hidden unit is explicitly controlled via gating units, as described by Graves \cite{DBLP:journals/corr/Graves13}. Although LSTMs have four times more parameters than regular RNNs, they  facilitate long term storage of task-relevant data. %, as described by Graves \cite{DBLP:journals/corr/Graves13}. 
In Computer Vision, LSTMs have been used  so far for image captioning  \cite{DBLP:journals/corr/VinyalsTBE14} and action classification in videos \cite{DBLP:journals/corr/DonahueHGRVSD14}.   
Omitting the encoder and decoder networks and instead using linear mappings between the input, recurrent state, and output 
caused underfitting on all three of our tasks. 
This can be explained by the complexity of the mocap and video input in comparison to the words or pen stroke 2D  locations considered in prior work \cite{DBLP:journals/corr/Graves13}. For example, word embeddings were not crucial for RNNs to do well in text generation or machine translation, and the standard one hot encoding vocabulary representation also showed excellent results \cite{DBLP:conf/nips/SutskeverVL14}.


Our goal is to predict the mocap vector in the next frame, given a mocap sequence so far. Since the output $y_{t}$ has the same format as  the input $x_{t+1}$, if we can predict $x_{t+1}$, we can ``play'' the motion forward in time to generate a novel mocap sequence by feeding the output at the preceding time step as the input to the current one.

We represent the orientation of each joint by an exponential map in the coordinate frame of its parent, corresponding to 3 degrees of freedom per joint. The global position of the body in the x-y plane and the global orientation about the vertical z axis are predicted relative to the previous frame, since each clip has an arbitrary global position. This is similar to the approach taken in previous work \cite{thr-mhmub-06}. We standardize our input by mean subtraction and division by the standard deviation along each dimension. 



We consider both deterministic and probabilistic predictions. In the deterministic case, the decoder's output $y_t$ is a single mocap vector.  In this case, we train our model by minimizing the Euclidean loss between target and predicted body joint angles. In the probabilistic case, $y_t$  parametrizes  a Gaussian Mixture Model (GMM) over mocap vectors in the next frame.  We then  minimize the GMM negative log-likelihood during training:
\begin{equation}
 \mathcal{L}(\seq) = - \displaystyle\sum_{t=1}^T\mathrm{log}\mathrm{Pr}(x_{t+1}|y_t)
\end{equation}
We use five mixture components and diagonal covariances. The variances are outputs of  exponential layers to ensure  positivity, and the mixture component probabilities are outputs of a softmax layer, similar to \cite{DBLP:journals/corr/Graves13}. During training, we  pad the variances in each iteration by a fixed amount to ensure they do not collapse around the mixture means.
   Weights are initialized randomly. We experimented with initializing the encoder and decoder networks of the mocap $\auto$ from the (first two layers of) encoder and (last two layers of) decoder of a) a ten layer autoencoder trained for dimensionality reduction of mocap  vectors \cite{citeulike:778023}, 
  b) a ``skip'' autoencoder  trained to reconstruct the  mocap vector in few frames in the future given the current one. In both cases, we did not observe improvement over random weight initialization.  We train our $\auto$ model with stochastic gradient descent and backpropagation through time \cite{Williams95gradient-basedlearning} with momentum and gradient clipping at 25, using the publicly available Caffe package \cite{Jia13caffe} and the LSTM layer  implementation from \cite{DBLP:journals/corr/DonahueHGRVSD14}. 
  
At test time, we  run the model forward by feeding the predictions as input to the model in the following time step. Without  denoising, this kind of forward unrolling  suffers from accumulation of small prediction mistakes at each frame, and the model  
quickly falls into unnatural regions of the state space. 
Denoising ensures that corrupted mocap data are shown to the network during training so that it learns to  correct small amounts of drift and stay close to the manifold of natural poses.

In the previous section, we described how the $\auto$ model can be used to synthesize naturalistic human motion by training on  motion capture datasets. 
In this section, we extend this model to identify human poses directly from pixels in a video. We consider a pose labeling task and a pose forecasting task. In the labeling task, given a bounding box sequence depicting a person, we want to estimate body joint locations for the \textit{current} frame, given the sequence so far.
In the forecasting task, we want to estimate body joint locations for a specific future time instance instead.   

Predicting heat maps naturally incorporates uncertainty over  body joint locations, as opposed to predicting body joint pixel coordinates. 
 
Our decoder is a two layer network with fully connected layers interleaved with  rectified linear unit layers.  
 The output of the decoder  is    body joint heat maps over the person bounding box in the current frame for the labeling task, or body joint heat maps at a specified future time instance for the forecasting  task. 

We train both our pose labeler and forecaster $\auto$s under a Euclidean loss between estimated and target heat maps. 
We initialize the weights of the encoder from a   six layer convolutional network trained for per frame body part detection, in which the final CONV6 layer corresponds to the body joint heat maps.  %Such weight initialization is crucial for performance, as shown in the experimental section. 

Empirically, we found it valuable to input to the recurrent layer not the per frame estimated heat maps (CONV6), but rather the preceding CONV5  feature maps.   These feature maps capture rich appearance information, rather than merely  body joint likelihood.  
Rich appearance information  assists the network in discriminating between different actions and pose dynamics without  explicit switching across activity domains, as previous switching dynamical linear systems  or HMMs \cite{prm-lslmh-00}.


We use two  networks on different image scales for our per frame pose detector and $\auto$: one where the output layer  resolution is 6$\times$6 and one that works on double  image size and has output resolution of 12$\times$12. The  heat maps of the coarser scale are upsampled and added to the finer scale to provide the final combined 12$\times$12 heat maps. Multiple scales have shown to be beneficial  for static pose estimation  in \cite{vpsKpsTulsianiM14,DBLP:conf/nips/TompsonJLB14,DBLP:journals/corr/ToshevS13}.
  
  
We test our method on the H3.6M dataset of Ionescu $\ea$\cite{h36m_pami}, which is currently the largest video pose dataset. It consists of 15  activity scenarios, %such as,  smoking, walking, sitting down, eating, giving directions etc. Actions are 
performed by seven  different professional  actors and  recorded from four static cameras. For each activity scenario, subject, and camera viewpoint, there are two video sequences, each  between 3000 and 5000 frames. Each activity scenario features rich gestures, pose variations and interesting subactions  performed by the actors. For example, the walking activity includes holding hands, carrying a heavy load, putting hands in the pockets,  looking around etc. %The actors are particularly animated on purpose. 
The activities are recorded using a Vicon motion capture system that tracks markers on actors' body joints and provides  high quality 3D body joint locations. 
2D body joints locations are obtained by projecting the 3D positions onto the image plane using the  known camera calibration and viewpoint. For all our experiments, we treat subject 5 as the test subject and all others as our training subjects. 

\begin{center}
\includegraphics[trim=0.0in 4in 0in 0.0in, scale=0.28]{figs/mocap_comparisons_extended.pdf}
\end{center}
\caption{ \textbf{Motion synthesis.}  
LSTM-3LR and  CRBMs \cite{thr-mhmub-06} provide  smooth short-term motion completions  (for up to 600msecs), mimicking well novel styles of motion, (e.g., here, walking with upright back).  However,  $\auto$  generates realistic motion for  longer periods of time while LSTM-3LR soon converges to the mean pose and  CRBM  diverges to implausible motion.  NGRAM  has a non-smooth transition from conditioning to generation. Per frame mocap vectors  predicted by GPDM \cite{Wang06gaussianprocess}  look plausible,  but their temporal evolution is far from realistic.  You can watch the corresponding video results  at {\tt https://sites.google.com/site/motionlstm/} }.
\label{fig:crbm}
\end{figure}

We compare our $\auto$ mocap generator with  a) an LSTM recurrent neural network with linear encoder and decoders that has 3 LSTM layers of 1000 units each (architecture found through experimentation to work well), b) Conditional Restricted Boltzmann Machines (CRBMs) of Taylor \textit{et al.}\cite{thr-mhmub-06}, c)  Gaussian Process Dynamic Model (GPDM) of Wang \ea\cite{Wang06gaussianprocess}, and d) a nearest neighbor N-gram model (NGRAM).   For CRBM and GPDM, we used the code made publicly available by the authors. For the nearest neighbor N-gram model, we used a frame window of length $N=6$ and Euclidean distance on 3D angles between the conditioning prefix and our training set,  and copy past the subsequent frames of the best matching training subsequence.  
We applied denoising during training to regularize both the $\auto$ and the LSTM-3LR.  For all models, the mocap frame sequences were subsampled by two. $\auto$, LSTM-3LR and CRBM are trained on multiple activity scenarios (Walking, Eating and Smoking).  GPDM is trained on Walking activity only, because   its  cubic complexity prohibits its training on a large number of  sequences.  
Our comparison focuses  on motion forecasting (prediction) and synthesis, conditioning on motion prefixes of our test subject.  Mocap in-filling and denoising  are nontrivial with our current  model but developing this functionality is an interesting avenue for future work. 

 
We show  qualitative motion synthesis results in Figure \ref{fig:crbm}  and  quantitative motion prediction errors in Table \ref{tab:predictionerror}.  In Figure \ref{fig:crbm},  the conditioning motion prefix from our test subject is shown in green  and the generated motion is shown in blue. 
In Table \ref{tab:predictionerror}, we show Euclidean norm between the synthesized motion  and ground-truth   motion for our test subject  for different temporal horizons past the conditioning motion prefix, the largest being 560msecs, averaged across 8 different prefixes. %Within  small temporal horizons, human motion is  close to deterministic.  
The stochasticity of human motion prevents a metric evaluation for longer temporal horizons, thus all comparisons in previous literature are qualitative.  
LSTM-3LR dominates the short-term motion generation, yet soon converges to the mean pose, as shown in Figure \ref{fig:crbm}. CRBM also provides smooth short term motion completions,  yet quickly drifts to unrealistic motions. $\auto$ provides slightly less smooth completions, yet can generate realistic  motion for long periods of time. For $\auto$, the smallest error was always produced by the most probable GMM sample, which was similar to the output of an ERD trained under a standard Euclidean loss. N-gram model exhibits a sudden change of style during transitioning from the conditioning prefix to the first generated frame, and cannot generate anything outside of the training set. Due to low-dimensional embedding, GPDM cannot adequately handle the breadth of styles in the training data, and produces unrealistic temporal evolution.  


The quantitative and qualitative motion generation results of $\auto$ and LSTM-3LR suggest an interesting trade-off between smoothness of motion completion (interesting motion extrapolations) and stable long-term motion generation. Generating short-term motion that mimics the style of the test subject is possible with LSTM-3LR, yet, since the network has not encountered similar  examples during  training, it is unable to correctly generate motion for longer periods of time. In contrast, $\auto$ gears the  generated motion towards similarly moving training examples. $\auto$  though cannot really extrapolate, but rather interpolate among the training subjects. It does   provides much smoother motion completions than the N-gram baseline. Both setups are interesting and useful in different applications, and in between architectures potentially lie somewhere in between the two ends of that spectrum. Finally, it is surprising that LSTM-3LR outperforms CRBMs given its simplicity during  training and testing, not requiring inference over latent variables. 

\begin{table}[h]{\small
\setlength{\tabcolsep}{2.5pt}
\begin{tabular}{|p{1.52cm}||p{0.75cm}|p{0.75cm}|p{0.75cm}|p{0.75cm}|p{0.75cm}|p{0.75cm}|p{0.75cm}|}
\cline{1-8}
                                             & {80}             & {160}       & {240}       &{320}       & {400} & {480} & {560} \\ \hline 
 ERD                                      & 0.89            & 1.39        & 1.93         & 2.38        & 2.76  &  3.09       &  3.41               \\ \hline 
 LSTM-3LR                           & \textbf{0.41} & \textbf{0.67} & \textbf{1.15} & \textbf{1.50} & \textbf{1.78} & \textbf{2.02} & \textbf{2.26}\\   \hline 
CRBM \cite{thr-mhmub-06} & 0.68              &         1.13      & 1.55             & 2.00             & 2.45              & 2.90              & 3.34\\\hline 
6GRAM &                              1.67                & 2.36              & 2.94              & 3.43             & 3.83             &4.19               &4.53  \\\hline 
GPDM \cite{Wang06gaussianprocess} & 1.76 & 2.5 & 3.04 & 3.52 & 3.92 & 4.28 & 4.61\\   \hline 
\end{tabular}
}
 \caption{ \textbf{Motion prediction error} during 80, 160, 240, 320, 400, 480 and 560 msecs past the conditioning prefix for our test subject during Walking activity.  %We show  Euclidean norm of the 3D angle error averaged across 8 motion completion examples.  
 Quantitative evaluation for longer temporal horizons is not possible due to stochasticity of human motion. }
\label{tab:predictionerror}
\end{table}\begin{figure}[ht]
\begin{center}
\includegraphics[trim=0in 0.3in 0in 1in, scale=0.2]{figs/pretraining.pdf}
\end{center}
\caption{\textbf{Pretraining.}  %We compare training (left) and test (right) loss with initialization of the CNN encoder from a pose estimator versus initialization from random weights. 
Initialization of the CNN encoder with the weights of a body pose detector leads to a much better solution than random weight initialization. For motion generation, we did not observe this  performance gap between pertaining and random initialization, potentially due to much shallower encoder and  low dimensionality of the mocap data.}
\label{fig:pretraining}
\end{figure}%\begin{figure*}[ht]
\begin{center}
\includegraphics[trim=0in 0.2in 0in 0in, scale=0.23]{figs/curves_video_labelling2.pdf}
\end{center}
\caption{\textbf{Video pose labeling in H3.6M.}  Quantitative comparison of  a per frame CNN body part detector  of \cite{vpsKpsTulsianiM14} (PF), dynamic programming for temporal coherence of the body pose sequence in the spirit of \cite{DBLP:conf/iccv/ParkR11,Batra:2012:DMS:2403138.2403140} (VITERBI), and  $\auto$ video pose labeler.  $\auto$  outperforms the per frame detector as well as the dynamic programming baseline.   Oracle curve shows the performance upper-bound  imposed by our grid resolution of 12x12.}
\label{fig:bjtquant}
\end{figure*}\begin{figure*}[ht]
 \begin{center}
 \includegraphics[trim=0.3in 0.8in 0in 0.5in, scale=0.67]{figs/bodyjointtracking_qualit.png}
 \end{center}
 \caption{\textbf{Left-right disambiguation.} $\auto$ corrects  left-right confusions of the per frame CNN detector   by  aggregating   appearance  features (CONV5) across long temporal horizons.}
 \label{fig:leftright}
 \end{figure*}%\vspace{-0.08in}\paragraph{Video pose labeling}
  Given a person bounding  box sequence, we want to label 2D pixel locations of the person's body joint locations. 
  Both occluded and non-occluded body joints are required to be detected correctly: the occluder's appearance often times contains useful information regarding the location of an occluded body joint \cite{DBLP:conf/eccv/DesaiR12}. 
  Further, for transcribing 2D to 3D pose, all body joints are required  \cite{Taylor:2000:RAO:364058.364079}. 
  


We compare our $\auto$ video labeler against two baselines: a per frame CNN pose detector (PF) used as the encoder part of our $\auto$ model, and a dynamic programming approach over multiple body pose hypotheses per frame (VITERBI) similar in spirit to \cite{DBLP:conf/iccv/ParkR11,Batra:2012:DMS:2403138.2403140}. For our VITERBI baseline, we consider for each body joint in each frame all possible grid locations and encode temporal smoothness as the negative exponential of the Euclidean distance between the locations of the same body joint  across consecutive frames.  The intuition behind VITERBI is that temporal smoothness will help rule out isolated, bad pose estimates, by promoting ones that have lower per frame scores, yet are more temporally coherent.   

We evaluate our model and baselines by recording the highest scoring pixel location for each frame and  body joint. We compute the percentage of detected joints within a tolerance radius of a circle centered at the ground-truth body joint locations, for various  tolerance thresholds. We normalize the tolerance radii with the distance between left hip and right shoulder. This is the standard evaluation metric for static image pose labeling \cite{MODEC}. We show pose labeling performance curves  in Figure \ref{fig:bjtquant}. For a video comparison between ERD and the per frame CNN detector, please see the video at { \tt https://sites.google.com/site/motionlstm/}.
discriminatively learning to integrate temporal information  for body joint tracking, instead of employing generic motion smoothness priors.  %Previous pose smoothers do not look at the pixel information during temporal pose integration but rather at the pose output samples, same as our VITERBI baseline. $\auto$'s performance boost stems from correcting left and right confusions of the per frame part detector, as Figure \ref{fig:leftright} qualitatively illustrates.  
Left and right confusion is a major challenge for per frame part detectors, to the extent that certain works measure their performance in image centric coordinates, rather than object centric 

\begin{figure*}[ht]
\begin{center}
\includegraphics[trim=0.0in 1in 0in 0.5in, scale=0.29]{figs/Forecasting2.pdf}
\end{center}
\caption{ \textbf{Video pose forecasting.} Quantitative comparison between the $\ERD$ model, a zero motion (NM), and constant velocity (OF) models. $\ERD$ outperforms the baselines for the lower body limbs, which are frequently occluded and thus their per frame motion is not frequently observed using optical flow. }
\label{fig:forecasting}
\end{figure*}Figure \ref{fig:pretraining} compares ERD training and test losses during finetuning  the encoder from (the first five layers of) our  per frame CNN pose detector, versus training the encoder from scratch (random weights). CNN encoder's  initialization is crucial to reach a good solution.  %This is not the case for our mocap synthesis $\auto$, where random weight  initialization was shown to be equally effective to initializing from mocap autoencoder weights.  We further compare our video labeler in a subset of 200 video sequences of around 50 frames each from the FlicMotion dataset of \cite{DBLP:conf/accv/JainTLB14,MODEC} that we annotated  densely in time with person bounding boxes. We used 170 video sequences for training and 30 for testing. We show performance curves for the upper body joints in Figure \ref{fig:flic}. VITERBI has similar performance as in H3.6M, marginally exceeding the per frame CNN detector. However $\auto$ does much worse since the training set is too small to learn effectively. Finetuning from the model learnt from H3.6M did not help since H3.6M concerns full body motion while FlicMotion captures upper body only.   We did not change the architecture in comparison to the ERD used in H3.6M. It is probable that a smaller recurrent layer and decoder would improve performance preventing overfitting. Large training sets such as in H3.6M allow high capacity discriminative temporal smoothers as our video labelled $\auto$ to outperform generic motion smoothness priors  for human dynamics. 

\begin{figure}[ht]
\begin{center}
\includegraphics[trim=1in 0in 0in 0in, scale=0.2]{figs/flicmotion.pdf}
\end{center}
\caption{\textbf{Video pose labeling in FlicMotion.} 
$\auto$ does not succeed in learning effectively from the small set of 170 videos of about 50 frames each. Large training sets, such as those provided in H3.6M, are necessary for $\auto$ video labeler to outperform generic motion smoothness priors. }
\label{fig:flic}
\end{figure}% \begin{figure}[ht]
 \begin{center}
 \includegraphics[trim=0.0in 0.6in 0in 0in, scale=0.33]{figs/videoposeprediction.pdf}
 \end{center}
 \caption{\textbf{Video pose forecasting}  400ms in the future. \textit{Left:} the prediction of the body part detector 400ms before superimosed on the frame to predict pose for (zero motion model). \textit{MiddleLeft:} Predictions of the $\auto$. The body joints have been moved towards their correct location. \textit{MiddleRight:} The current and 400ms ahead frame superimposed. \textit{Right:} Ground-truth body joint location (discretized in a $N \times N$ heat map grid). In all cases we show the highest scoring heat map grid location. }
 \label{fig:videoprediction}
 \end{figure}%\vspace{-0.08in}\paragraph{Video pose forecasting}
We predict 2D  body joint locations  400ms ahead of the current frame. 
Figure \ref{fig:forecasting} shows pose forecasting performance curves for our $\ERD$ model, a model that assumes zero object and camera motion  (NoMotion-\textit{NM}), and a model that assumes constant optical flow within the prediction horizon (\textit{OF}).  
$\ERD$ carries out more accurate predictions than the zero order and first order motion baselines, as also shown qualitatively in Figure \ref{fig:videoprediction}.  
Optical flow based motion models cannot make reasonable predictions for occluded body joints, since their frame to frame displacements are not observed.  
Further, standard motion models suffer from separation of the observation model (part detector) and temporal aggregation, which $\auto$ combines into a single network.  

\paragraph{Discussion}
Currently, the mocap ERD performs  better on periodic activities (walking, smoking etc) in comparison to non periodic ones (sitting etc.).  Interesting directions for future research is predicting 3D angle differences from frame to frame as opposed to angles directly. Such transformation prediction may generalize better to new subjects, focusing more on  motion rather than  appearance of the skeleton. We are also investigating using large frame volumes as input to our video prediction ERDs with spatio-temporal convolutions in CONV1 as opposed to a single frame LSTM, in order to exploit short temporal horizon more effectively. 
    \section{Conclusion}%\vspace{-0.03in}
We have presented end-to-end  discriminatively trained encoder-recurrent-decoder models for modeling human kinematics in videos and motion capture. $\auto$s   learn the representation for recurrent prediction or labeling, as well as its dynamics, by jointly training encoder recurrent and decoder networks. 
Such expressive models of human dynamics come at a cost of increased need for training examples. In future work, we plan to explore semi-supervised models in this direction, as well learning human dynamics   in multi-person interaction scenarios.   

We would like to thank Jeff Donahue and Philipp Kr\"ahenb\"uhl for useful discussions. 
We  gratefully acknowledge NVIDIA corporation for the donation of K40 GPUs for this research. This research was funded by ONR MURI N000014-10-1-0933.
 




{\small
\bibliographystyle{ieee}
\bibliography{egbib}
}

\end{document}


