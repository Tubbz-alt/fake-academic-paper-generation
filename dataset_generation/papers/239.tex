
%% bare_jrnl.tex
%% V1.4b
%% 2015/08/26
%% by Michael Shell
%% see http://www.michaelshell.org/
%% for current contact information.
%%
%% This is a skeleton file demonstrating the use of IEEEtran.cls
%% (requires IEEEtran.cls version 1.8b or later) with an IEEE
%% journal paper.
%%
%% Support sites:
%% http://www.michaelshell.org/tex/ieeetran/
%% http://www.ctan.org/pkg/ieeetran
%% and
%% http://www.ieee.org/

%%*************************************************************************
%% Legal Notice:
%% This code is offered as-is without any warranty either expressed or
%% implied; without even the implied warranty of MERCHANTABILITY or
%% FITNESS FOR A PARTICULAR PURPOSE! 
%% User assumes all risk.
%% In no event shall the IEEE or any contributor to this code be liable for
%% any damages or losses, including, but not limited to, incidental,
%% consequential, or any other damages, resulting from the use or misuse
%% of any information contained here.
%%
%% All comments are the opinions of their respective authors and are not
%% necessarily endorsed by the IEEE.
%%
%% This work is distributed under the LaTeX Project Public License (LPPL)
%% ( http://www.latex-project.org/ ) version 1.3, and may be freely used,
%% distributed and modified. A copy of the LPPL, version 1.3, is included
%% in the base LaTeX documentation of all distributions of LaTeX released
%% 2003/12/01 or later.
%% Retain all contribution notices and credits.
%% ** Modified files should be clearly indicated as such, including  **
%% ** renaming them and changing author support contact information. **
%%*************************************************************************


% *** Authors should verify (and, if needed, correct) their LaTeX system  ***
% *** with the testflow diagnostic prior to trusting their LaTeX platform ***
% *** with production work. The IEEE's font choices and paper sizes can   ***
% *** trigger bugs that do not appear when using other class files.       ***                          ***
% The testflow support page is at:
% http://www.michaelshell.org/tex/testflow/



\documentclass[journal]{IEEEtran}
\usepackage[pdftex]{graphicx}
\usepackage{cite}
\usepackage{color}
\usepackage[switch]{lineno}
\usepackage{graphicx}
\usepackage{subfigure}
\usepackage{url}
\usepackage{epstopdf}
\usepackage[table,xcdraw]{xcolor}

\usepackage{lineno}
%
% If IEEEtran.cls has not been installed into the LaTeX system files,
% manually specify the path to it like:
% \documentclass[journal]{../sty/IEEEtran}

\usepackage[justification=centering]{caption}
\usepackage{amsmath}
\usepackage{caption}
\usepackage{amsfonts}
\usepackage{booktabs}
\usepackage{multirow}
\usepackage{verbatim}
\usepackage{authblk}
\usepackage{bm}
%\usepackage{balance}




% Some very useful LaTeX packages include:
% (uncomment the ones you want to load)


% *** MISC UTILITY PACKAGES ***
%
%\usepackage{ifpdf}
% Heiko Oberdiek's ifpdf.sty is very useful if you need conditional
% compilation based on whether the output is pdf or dvi.
% usage:
% \ifpdf
%   % pdf code
% \else
%   % dvi code
% \fi
% The latest version of ifpdf.sty can be obtained from:
% http://www.ctan.org/pkg/ifpdf
% Also, note that IEEEtran.cls V1.7 and later provides a builtin
% \ifCLASSINFOpdf conditional that works the same way.
% When switching from latex to pdflatex and vice-versa, the compiler may
% have to be run twice to clear warning/error messages.






% *** CITATION PACKAGES ***
%
%\usepackage{cite}
% cite.sty was written by Donald Arseneau
% V1.6 and later of IEEEtran pre-defines the format of the cite.sty package
% \cite{} output to follow that of the IEEE. Loading the cite package will
% result in citation numbers being automatically sorted and properly
% "compressed/ranged". e.g., [1], [9], [2], [7], [5], [6] without using
% cite.sty will become [1], [2], [5]--[7], [9] using cite.sty. cite.sty's
% \cite will automatically add leading space, if needed. Use cite.sty's
% noadjust option (cite.sty V3.8 and later) if you want to turn this off
% such as if a citation ever needs to be enclosed in parenthesis.
% cite.sty is already installed on most LaTeX systems. Be sure and use
% version 5.0 (2009-03-20) and later if using hyperref.sty.
% The latest version can be obtained at:
% http://www.ctan.org/pkg/cite
% The documentation is contained in the cite.sty file itself.






% *** GRAPHICS RELATED PACKAGES ***
%
\ifCLASSINFOpdf
  %\usepackage[pdftex]{graphicx}
  % declare the path(s) where your graphic files are
  % \graphicspath{{../pdf/}{../jpeg/}}
  % and their extensions so you won't have to specify these with
  % every instance of \includegraphics
  % \DeclareGraphicsExtensions{.pdf,.jpeg,.png}
\else
  % or other class option (dvipsone, dvipdf, if not using dvips). graphicx
  % will default to the driver specified in the system graphics.cfg if no
  % driver is specified.
  % \usepackage[dvips]{graphicx}
  % declare the path(s) where your graphic files are
  % \graphicspath{{../eps/}}
  % and their extensions so you won't have to specify these with
  % every instance of \includegraphics
  % \DeclareGraphicsExtensions{.eps}
\fi
% graphicx was written by David Carlisle and Sebastian Rahtz. It is
% required if you want graphics, photos, etc. graphicx.sty is already
% installed on most LaTeX systems. The latest version and documentation
% can be obtained at: 
% http://www.ctan.org/pkg/graphicx
% Another good source of documentation is "Using Imported Graphics in
% LaTeX2e" by Keith Reckdahl which can be found at:
% http://www.ctan.org/pkg/epslatex
%
% latex, and pdflatex in dvi mode, support graphics in encapsulated
% postscript (.eps) format. pdflatex in pdf mode supports graphics
% in .pdf, .jpeg, .png and .mps (metapost) formats. Users should ensure
% that all non-photo figures use a vector format (.eps, .pdf, .mps) and
% not a bitmapped formats (.jpeg, .png). The IEEE frowns on bitmapped formats
% which can result in "jaggedy"/blurry rendering of lines and letters as
% well as large increases in file sizes.
%
% You can find documentation about the pdfTeX application at:
% http://www.tug.org/applications/pdftex





% *** MATH PACKAGES ***
%
%\usepackage{amsmath}
% A popular package from the American Mathematical Society that provides
% many useful and powerful commands for dealing with mathematics.
%
% Note that the amsmath package sets \interdisplaylinepenalty to 10000
% thus preventing page breaks from occurring within multiline equations. Use:
%\interdisplaylinepenalty=2500
% after loading amsmath to restore such page breaks as IEEEtran.cls normally
% does. amsmath.sty is already installed on most LaTeX systems. The latest
% version and documentation can be obtained at:
% http://www.ctan.org/pkg/amsmath





% *** SPECIALIZED LIST PACKAGES ***
%
%\usepackage{algorithmic}
% algorithmic.sty was written by Peter Williams and Rogerio Brito.
% This package provides an algorithmic environment fo describing algorithms.
% You can use the algorithmic environment in-text or within a figure
% environment to provide for a floating algorithm. Do NOT use the algorithm
% floating environment provided by algorithm.sty (by the same authors) or
% algorithm2e.sty (by Christophe Fiorio) as the IEEE does not use dedicated
% algorithm float types and packages that provide these will not provide
% correct IEEE style captions. The latest version and documentation of
% algorithmic.sty can be obtained at:
% http://www.ctan.org/pkg/algorithms
% Also of interest may be the (relatively newer and more customizable)
% algorithmicx.sty package by Szasz Janos:
% http://www.ctan.org/pkg/algorithmicx




% *** ALIGNMENT PACKAGES ***
%
%\usepackage{array}
% Frank Mittelbach's and David Carlisle's array.sty patches and improves
% the standard LaTeX2e array and tabular environments to provide better
% appearance and additional user controls. As the default LaTeX2e table
% generation code is lacking to the point of almost being broken with
% respect to the quality of the end results, all users are strongly
% advised to use an enhanced (at the very least that provided by array.sty)
% set of table tools. array.sty is already installed on most systems. The
% latest version and documentation can be obtained at:
% http://www.ctan.org/pkg/array


% IEEEtran contains the IEEEeqnarray family of commands that can be used to
% generate multiline equations as well as matrices, tables, etc., of high
% quality.




% *** SUBFIGURE PACKAGES ***
%\ifCLASSOPTIONcompsoc
%  \usepackage[caption=false,font=normalsize,labelfont=sf,textfont=sf]{subfig}
%\else
%  \usepackage[caption=false,font=footnotesize]{subfig}
%\fi
% subfig.sty, written by Steven Douglas Cochran, is the modern replacement
% for subfigure.sty, the latter of which is no longer maintained and is
% incompatible with some LaTeX packages including fixltx2e. However,
% subfig.sty requires and automatically loads Axel Sommerfeldt's caption.sty
% which will override IEEEtran.cls' handling of captions and this will result
% in non-IEEE style figure/table captions. To prevent this problem, be sure
% and invoke subfig.sty's "caption=false" package option (available since
% subfig.sty version 1.3, 2005/06/28) as this is will preserve IEEEtran.cls
% handling of captions.
% Note that the Computer Society format requires a larger sans serif font
% than the serif footnote size font used in traditional IEEE formatting
% and thus the need to invoke different subfig.sty package options depending
% on whether compsoc mode has been enabled.
%
% The latest version and documentation of subfig.sty can be obtained at:
% http://www.ctan.org/pkg/subfig




% *** FLOAT PACKAGES ***
%
%\usepackage{fixltx2e}
% fixltx2e, the successor to the earlier fix2col.sty, was written by
% Frank Mittelbach and David Carlisle. This package corrects a few problems
% in the LaTeX2e kernel, the most notable of which is that in current
% LaTeX2e releases, the ordering of single and double column floats is not
% guaranteed to be preserved. Thus, an unpatched LaTeX2e can allow a
% single column figure to be placed prior to an earlier double column
% figure.
% Be aware that LaTeX2e kernels dated 2015 and later have fixltx2e.sty's
% corrections already built into the system in which case a warning will
% be issued if an attempt is made to load fixltx2e.sty as it is no longer
% needed.
% The latest version and documentation can be found at:
% http://www.ctan.org/pkg/fixltx2e


%\usepackage{stfloats}
% stfloats.sty was written by Sigitas Tolusis. This package gives LaTeX2e
% the ability to do double column floats at the bottom of the page as well
% as the top. (e.g., "\begin{figure*}[!b]" is not normally possible in
% LaTeX2e). It also provides a command:
%\fnbelowfloat
% to enable the placement of footnotes below bottom floats (the standard
% LaTeX2e kernel puts them above bottom floats). This is an invasive package
% which rewrites many portions of the LaTeX2e float routines. It may not work
% with other packages that modify the LaTeX2e float routines. The latest
% version and documentation can be obtained at:
% http://www.ctan.org/pkg/stfloats
% Do not use the stfloats baselinefloat ability as the IEEE does not allow
% \baselineskip to stretch. Authors submitting work to the IEEE should note
% that the IEEE rarely uses double column equations and that authors should try
% to avoid such use. Do not be tempted to use the cuted.sty or midfloat.sty
% packages (also by Sigitas Tolusis) as the IEEE does not format its papers in
% such ways.
% Do not attempt to use stfloats with fixltx2e as they are incompatible.
% Instead, use Morten Hogholm'a dblfloatfix which combines the features
% of both fixltx2e and stfloats:
%
% \usepackage{dblfloatfix}
% The latest version can be found at:
% http://www.ctan.org/pkg/dblfloatfix




%\ifCLASSOPTIONcaptionsoff
%  \usepackage[nomarkers]{endfloat}
% \let\MYoriglatexcaption\caption
% \renewcommand{\caption}[2][\relax]{\MYoriglatexcaption[#2]{#2}}
%\fi
% endfloat.sty was written by James Darrell McCauley, Jeff Goldberg and 
% Axel Sommerfeldt. This package may be useful when used in conjunction with 
% IEEEtran.cls'  captionsoff option. Some IEEE journals/societies require that
% submissions have lists of figures/tables at the end of the paper and that
% figures/tables without any captions are placed on a page by themselves at
% the end of the document. If needed, the draftcls IEEEtran class option or
% \CLASSINPUTbaselinestretch interface can be used to increase the line
% spacing as well. Be sure and use the nomarkers option of endfloat to
% prevent endfloat from "marking" where the figures would have been placed
% in the text. The two hack lines of code above are a slight modification of
% that suggested by in the endfloat docs (section 8.4.1) to ensure that
% the full captions always appear in the list of figures/tables - even if
% the user used the short optional argument of \caption[]{}.
% IEEE papers do not typically make use of \caption[]'s optional argument,
% so this should not be an issue. A similar trick can be used to disable
% captions of packages such as subfig.sty that lack options to turn off
% the subcaptions:
% For subfig.sty:
% \let\MYorigsubfloat\subfloat
% \renewcommand{\subfloat}[2][\relax]{\MYorigsubfloat[]{#2}}
% However, the above trick will not work if both optional arguments of
% the \subfloat command are used. Furthermore, there needs to be a
% description of each subfigure *somewhere* and endfloat does not add
% subfigure captions to its list of figures. Thus, the best approach is to
% avoid the use of subfigure captions (many IEEE journals avoid them anyway)
% and instead reference/explain all the subfigures within the main caption.
% The latest version of endfloat.sty and its documentation can obtained at:
% http://www.ctan.org/pkg/endfloat
%
% The IEEEtran \ifCLASSOPTIONcaptionsoff conditional can also be used
% later in the document, say, to conditionally put the References on a 
% page by themselves.




% *** PDF, URL AND HYPERLINK PACKAGES ***
%
%\usepackage{url}
% url.sty was written by Donald Arseneau. It provides better support for
% handling and breaking URLs. url.sty is already installed on most LaTeX
% systems. The latest version and documentation can be obtained at:
% http://www.ctan.org/pkg/url
% Basically, \url{my_url_here}.




% *** Do not adjust lengths that control margins, column widths, etc. ***
% *** Do not use packages that alter fonts (such as pslatex).         ***
% There should be no need to do such things with IEEEtran.cls V1.6 and later.
% (Unless specifically asked to do so by the journal or conference you plan
% to submit to, of course. )


% correct bad hyphenation here
\hyphenation{op-tical net-works semi-conduc-tor}


\begin{document}
%
% paper title
% Titles are generally capitalized except for words such as a, an, and, as,
% at, but, by, for, in, nor, of, on, or, the, to and up, which are usually
% not capitalized unless they are the first or last word of the title.
% Linebreaks \\ can be used within to get better formatting as desired.
% Do not put math or special symbols in the title.
\title{Cross-Modal Attentional Context Learning for RGB-D Object Detection}
%
%
% author names and IEEE memberships
% note positions of commas and nonbreaking spaces ( ~ ) LaTeX will not break
% a structure at a ~ so this keeps an author's name from being broken across
% two lines.
% use \thanks{} to gain access to the first footnote area
% a separate \thanks must be used for each paragraph as LaTeX2e's \thanks
% was not built to handle multiple paragraphs

\author{
        Guanbin~Li,
        Yukang~Gan,
        Hejun~Wu,       
        Nong~Xiao                
        and~Liang~Lin$^*$
% \thanks{{\color{red}This work was supported in part by the State Key
% Development Program under Grant 2018YFC0830103, in part by the National Natural Science Foundation of China under Grant 61702565 and Grant 61672552, in part by the Science and Technology Planning Project of Guangdong Province under Grant 2017B010116001, in part by the Fundamental Research Funds for the Central Universities under Grant 18lgpy63, and was also supported in part by GD-NSF~(NO.2017A030312006).}}
\thanks{The first two authors contributed equally to this paper. The corresponding author is Liang Lin. G. Li, Y. Gan, H. Wu, N. Xiao and L. Lin are with the School of Data and Computer Science, Sun Yat-sen University, Guangzhou 510006, China.}
}

% \author[1]{YuKang Gan}
% \author[1]{GuanBin Li}
% \author[1]{Nong Xiao}
% \author[1]{HeJun Wu}
% \author[1]{Liang Lin}
%\affil[1]{Sun Yat-Sen University}


%\author{}% <-this % stops a space
%\thanks{M. Shell was with the Department
%of Electrical and Computer Engineering, Georgia Institute of Technology, Atlanta,
%GA, 30332 USA e-mail: (see http://www.michaelshell.org/contact.html).}% <-this % stops a space
%\thanks{J. Doe and J. Doe are with Anonymous University.}% <-this % stops a space
%\thanks{Manuscript received April 19, 2005; revised August 26, 2015.}}

% note the % following the last \IEEEmembership and also \thanks - 
% these prevent an unwanted space from occurring between the last author name
% and the end of the author line. i.e., if you had this:
% 
% \author{....lastname \thanks{...} \thanks{...} }
%                     ^------------^------------^----Do not want these spaces!
%
% a space would be appended to the last name and could cause every name on that
% line to be shifted left slightly. This is one of those "LaTeX things". For
% instance, "\textbf{A} \textbf{B}" will typeset as "A B" not "AB". To get
% "AB" then you have to do: "\textbf{A}\textbf{B}"
% \thanks is no different in this regard, so shield the last } of each \thanks
% that ends a line with a % and do not let a space in before the next \thanks.
% Spaces after \IEEEmembership other than the last one are OK (and needed) as
% you are supposed to have spaces between the names. For what it is worth,
% this is a minor point as most people would not even notice if the said evil
% space somehow managed to creep in.



% The paper headers
%\markboth{Journal of \LaTeX\ Class Files,~Vol.~14, No.~8, August~2015}%
%{Shell \MakeLowercase{\textit{et al.}}: Bare Demo of IEEEtran.cls for IEEE Journals}
% The only time the second header will appear is for the odd numbered pages
% after the title page when using the twoside option.
% 
% *** Note that you probably will NOT want to include the author's ***
% *** name in the headers of peer review papers.                   ***
% You can use \ifCLASSOPTIONpeerreview for conditional compilation here if
% you desire.




% If you want to put a publisher's ID mark on the page you can do it like
% this:
%\IEEEpubid{0000--0000/00\$00.00~\copyright~2015 IEEE}
% Remember, if you use this you must call \IEEEpubidadjcol in the second
% column for its text to clear the IEEEpubid mark.



% use for special paper notices
%\IEEEspecialpapernotice{(Invited Paper)}




% make the title area
\maketitle

% As a general rule, do not put math, special symbols or citations
% in the abstract or keywords.
\begin{abstract}
Recognizing objects from simultaneously sensed photometric (RGB) and depth channels is a fundamental yet practical problem in many machine vision applications such as robot grasping and autonomous driving. In this paper, we address this problem by developing a Cross-Modal Attentional Context (CMAC) learning framework, which enables the full exploitation of the context information from both RGB and depth data. Compared to existing RGB-D object detection frameworks, our approach has several appealing properties. First, it consists of an attention-based global context model for exploiting adaptive contextual information and incorporating this information into a region-based CNN~(e.g., Fast RCNN) framework to achieve improved object detection performance. Second, our CMAC framework further contains a fine-grained object part attention module to harness multiple discriminative object parts inside each possible object region for superior local feature representation. While greatly improving the accuracy of RGB-D object detection, the effective cross-modal information fusion as well as attentional context modeling in our proposed model  provide an interpretable visualization scheme. Experimental results demonstrate that the proposed method significantly improves upon the state of the art on all public benchmarks. 
\end{abstract}

% Note that keywords are not normally used for peerreview papers.
\begin{IEEEkeywords}
RGB-D Object Detection, Attentional Context Modeling, Cross Modal Feature, Convolutional Neural Network.  
\end{IEEEkeywords}


% For peer review papers, you can put extra information on the cover
% page as needed:
% \ifCLASSOPTIONpeerreview
% \begin{center} \bfseries EDICS Category: 3-BBND \end{center}
% \fi
%
% For peerreview papers, this IEEEtran command inserts a page break and
% creates the second title. It will be ignored for other modes.
\IEEEpeerreviewmaketitle

\begin{figure*}
  \centering
  %% insert PDF file testpdf.pdf
  \includegraphics[width=0.9\textwidth]{./figure1.pdf}
  \caption{\label{figure:1} Example visualization results for global context and object part attention generated by our proposed CMAC model. For global context, information from relevant regions (the highlighted regions) of the object proposals is obtained through a recurrent attentional model. For local context, multiple parallel spatial transformers are utilized to exploit information from the discriminative parts (green rectangles) of the object proposals. Red rectangles indicate the object proposals.}
\end{figure*}


\section{Introduction}
% The very first letter is a 2 line initial drop letter followed
% by the rest of the first word in caps.
% 
% form to use if the first word consists of a single letter:
% \IEEEPARstart{A}{demo} file is ....
% 
% form to use if you need the single drop letter followed by
% normal text (unknown if ever used by the IEEE):
% \IEEEPARstart{A}{}demo file is ....
% 
% Some journals put the first two words in caps:
% \IEEEPARstart{T}{his demo} file is ....
% 
% Here we have the typical use of a "T" for an initial drop letter
% and "HIS" in caps to complete the first word.
\IEEEPARstart{R}GB-D object detection attempts to localize and classify objects within an image with depth information. It is one of the core technologies in the field of robotics application and can be beneficial to many intelligent tasks, including pose estimation \cite{hinterstoisser2012model, wang2016human}, content-based image retrieval \cite{wu2014hierarchical} and robot task planning \cite{schuster2015generating}. In recent years, the successful application of deep convolutional neural networks has pushed this research into a new phase and achieved very good results.

Most CNN-based RGB-D object detection frameworks are extended from RCNN-based object detectors \cite{girshick2014rich, fast-rcnn, ren2015faster} for RGB images. R-CNN-Depth \cite{gupta2014learning} is the first deep learning framework for RGB-D object detection that extends the R-CNN system~\cite{girshick2014rich} to take advantage of depth information by incorporating two parallel network streams for both RGB and depth modalities. %one that extends object detector from RGB images to RGB-D images and has achieved state-of-the-art performance. The R-CNN-Depth model generalizes the R-CNN system~\cite{girshick2014rich} to leverage depth information, in which a two-stream CNN was applied to generate feature from RGB-D images. 
 This two-stream pipeline later became the basis for many visual perception tasks in RGB-D images \cite{eitel2015multimodal, gupta2016cross, schwarz2015rgb, li2016lstm}. In this framework, the features from the RGB and depth modalities are computed independently and concatenated after applying fully connected layers for final proposal classification. However, this pipeline has its own limitations: (1)~Independent feature computation and simple feature concatenation ignore the correlation between the two modalities. (2)~Only information inside the object proposal is used for object classification, which neglects the auxiliary role of context information outside the bounding box in object classification.

In this paper, we propose a Cross Modal Attentional Context (CMAC) learning framework for RGB-D object detection that incorporates the consistency and complementary information between two diverse modalities~(RGB and depth), as well as an attentional model for global context mining and discriminative object part discovery. To exploit the correlation between RGB and depth modalities, the CMAC model employs a cross-modal feature fusion component to fuse the features extracted from the output feature maps of the two fully convolutional networks~(with different input sources). Instead of directly applying fused features to classification and object location refinement, our proposed CMAC model further learns attentional context and explores discriminative object parts based on the fused features. We believe that both the attentional global context and the discriminative parts attended inside each possible object region~(object proposal) are crucial for accurate RGB-D object detection. 

To capture the global context, our model employs a recurrent attention model that consists of multiple stacked Long Short-Term Memory (LSTM) units. The recurrent neural network is optimized to infer relevant regions for each given region proposal. As shown in Figure~\ref{figure:1}, the regions that are considered helpful for classification of the object proposal are highlighted. As can be seen, our proposed CMAC model can identify an adaptive global context for different object proposals~(\textit{i.e.,} the regions of the keyboard, parts of the table around the target monitor as well as the other monitor are highlighted when the input region proposal contains a monitor. When the input region proposal contains a chair, the regions including parts of the table and other chairs are assigned higher weights in the final classification.). Moreover, inspired by the fact that 
humans tend to quickly capture distinguishable parts for more accurate object classification judgment when observing objects with occluded regions, we propose to further incorporate a fine-grained object part attention module in our network framework. Considering the flexible attention mechanism and the excellent spatial manipulation ability of Spatial Transform Networks (STNs), we adopt multiple STNs in parallel to examine the discriminative parts located inside a specific object proposal for capturing local context. As illustrated in Figure~\ref{figure:1}, the CMAC model is able to successfully locate the most discriminative location that can differentiate an object's category~(\textit{i.e.,} the main screen and the base of the monitors, as well as the back and legs of the chairs). Acquiring such fine-grained object parts provides enhanced feature representations for region proposals.

In summary, the main contributions of the proposed CMAC model can be listed as follows:



 %the spatial transformer network (STN) is a exible attention mechanism which can be integrated with LSTM for recurrently identifying discriminative object parts purely with backpropagation.


 %Such global attentional context can assist the recognition of object proposals. Furthermore, this scheme of perceiving relevant regions of a specific object can also be applied to other fields, such as relationship discovery \cite{liang2017deep}, object affordance \cite{zhu2014reasoning} and object attributes analysis \cite{farhadi2009describing}.

%{\color{red} When the target object is partially occluded by other objects, human can still locate and classify the target object correctly. This ability comes from that human can identify the most discriminative parts of the target object and acquired accurate perception based on these information. Due to the flexible attention mechanism and the excellent spatial manipulation ability of STN, we adopt multiple Spatial Transform Networks (STNs) in parallel to look into the discriminative parts located inside a specific object proposal for capturing local context.} As shown in Figure~\ref{figure:1}, CMAC model is able to successfully locate the most discriminative location that can differentiate object's category~(\textit{i.e.} the main screen and the base of monitors, the back and the legs of chairs). Acquiring such fine-grained object parts provides enhanced feature representation for region proposals.

%The main contributions of the proposed CMAC model can be summarized as follows:
\begin{itemize}
\item[-] We propose a novel Cross Modal Attentional Context~(CMAC) deep learning framework that effectively incorporates the correlated information between different modalities and successfully identifies useful contextual information both locally and globally for RGB-D object detection.
\item[-] An attention-based global context module, based on an LSTM network, is utilized to recurrently generate contextual information from a global view for each object proposal.
\item[-] Multiple spatial transform networks are adopted in parallel to localize discriminative object parts for accurate object recognition. 
\item[-] Extensive experiments on the SUNRGBD and NYUv2 datasets well demonstrate the effectiveness of the proposed CMAC model, which outperforms the state-of-the-art method \cite{gupta2016cross} by 3.7\% and 3.2\%, respectively, in terms of mAP. 
\end{itemize}

\begin{figure*}
  \centering
  %% insert PDF file testpdf.pdf
  \includegraphics[width=0.9\textwidth]{./framework.pdf}
  \caption{\label{fig:framework} The network architecture of our proposed cross-modal attentional context~(CMAC) learning framework. The input consists of one RGB image and one HHA image (geocentric encoding of the depth image). Our network framework is composed of four components: convolutional feature extraction, cross-modal feature fusion, attention-based global context modeling and fine-grained object part attention.}
\end{figure*}

\section{Related Work}
\subsection{Object Detection in RGB-D Images}
Object detection in RGB-D images has attracted increased attention because of the rapid development of affordable depth sensors and their diverse application scenarios. Many successful algorithms have been proposed to effectively exploit information from RGB-D data. \cite{bo2011depth} and \cite{lai2011large} took advantage of hand-designed features such as SIFT and multiple shape features in the depth channel for RGB-D object recognition. Schwarz \textit{et al}. \cite{schwarz2015rgb} utilized two-stream CNNs pre-trained on ImageNet to extract features from RGB-D images. While most work mainly focuses on the RGB modality, some recent work has been dedicated to improving the object detection performance by taking depth information into consideration. Gupta \textit{et al}.~\cite{gupta2014learning} proposed a geocentric embedding to convert each single-channel depth map into a three-channel depth image~(HHA image), in which they encoded each pixel with three channels of information, \textit{i.e.,} the height above the ground, the horizontal disparity and the angle with respect to gravity. They also introduced a generalized method for the R-CNN detector that can be applied to RGB-D images; they used large CNNs pre-trained on RGB images to extract features from HHA data. To learn rich representations for the depth modality, \cite{gupta2016cross} transferred supervisions from labeled RGB images to unlabeled depth images. In this paper, we follow \cite{gupta2014learning} and encode depth information into HHA images for improved feature learning and take the model in~\cite{gupta2016cross} as our compared baseline model.

Another core issue of RGB-D object detection is how to merge the features from different sources. Existing fusion strategies can be divided into two streams: (1) Early fusion \cite{blum2012learned, bo2011object, bo2011depth}, in which the depth channel is being treated as an extra channel to RGB images and is concatenated with the RGB channels for feature extraction. (2) Late fusion~\cite{eitel2015multimodal, gupta2014learning, spinello2012leveraging, gupta2016cross}, where features are separately learned for each modality and are concatenated at later stages for object classification. Our model is similar to the late fusion approach, but instead of directly concatenating features for classification, we apply the attention model to the fused features to learn a better global context and discriminative object parts to achieve more accurate object recognition.

\subsection{Context Information in Object Detection}
Context information has been applied in many methods to enhance the performance of object detection~\cite{carbonetto2004statistical,li2016visual, divvala2009empirical, heitz2008learning, hoiem2005geometric, li2018contrast,torralba2010using}. For instance, \cite{torralba2010using} exploited context from information about the entire scene  for  object detection and localization. \cite{heitz2008learning} explored contextual relationships between regions in an unsupervised manner, where objects are detected using a discriminative approach. Spatial support and geographic information are used as context clues in \cite{divvala2009empirical}. Context models have also been applied to deep-learning-based object detectors. \cite{li2017multi} proposed a group recursive learning approach to refine object proposals by incorporating semantic and spatial layout correlations of surrounding proposals. Chu et al.~\cite{chu2016deep} formulated a fully connected conditional random field (CRF) to incorporate the local appearance and the contextual information in terms of relationships among objects and the global scene based on contextual features generated by a convolutional neural network. Inside-Outside net (ION) \cite{bell2016inside} introduced spatial recurrent networks (RNNs) to integrate the contextual information outside the region of interest while utilizing skip pooling to extract fine-grained information from multiple low-level convolutional layers. Although our proposed model also explores global contextual information through recurrent networks, it explicitly learn to attend the most relevant regions of the object proposal by generating a weight map for each proposal. The weight map can well reveal the contextual region that corresponds to the final classification result. One the other hand, instead of directly extracting local features from the whole object bounding box, our model can achieve better object feature representations by recurrently discovering the most discriminative object parts inside the object proposal and performing part-level feature fusion.

\subsection{Recurrent Attention Models}
Recurrent attentional models have been widely incorporated in deep-learning-based computer vision tasks~\cite{bahdanau2014neural, li2016attentive, mnih2014recurrent, xu2015show} to achieve better performance. Bahdanau et al.~\cite{bahdanau2014neural} introduced recurrent attention to neural machine translation, which allows the model to adaptively attend to the most relevant part of a sentence.~\cite{mnih2014recurrent} adopted visual attention to dynamically select a sequence of regions and only processed the selected regions for efficient computation. A recent work in~\cite{xu2015show} used an LSTM-based attention model to learn a description of static images. More recently, an attention mechanism has also been applied to vision tasks for videos. For instance, \cite{yao2015describing} extended an attention model for video description and employed a temporal attention mechanism to model the dynamic temporal structure of videos. \cite{sharma2015action} optimized the attention model to attend to the relevant parts within a single frame and attached higher importance to them while performing action recognition.

The  work that is most relevant  to our proposed method is the attentive context proposed in~\cite{li2016attentive}, which also incorporated a recurrent attention model to exploit global contextual information. However, the attention model used in \cite{li2016attentive} generated a static attentive location map for all object proposals. Instead of utilizing a fixed attentive context, our model generates an attentional context feature adaptive to the input region proposals. Furthermore, we employ a fine-grained object part attention module to harness multiple discriminative object parts inside each object proposal for achieving a superior local feature representation. Experimental analysis in Sec.~\ref{sec:ablation} demonstrates that our method is more robust to background and inter-class noise.

\section{Framework}\label{sec:framework}
An overview of our framework is illustrated in Fig.~\ref{fig:framework}. Our RGB-D object detection system, which is based on cross modal attentional context learning, is composed of four components, including fully convolutional networks based feature extraction, cross-modal feature fusion, attention-based global context modeling and fine-grained object part attention. We term this network Cross-Modal Attentional Context~(CMAC) network. Specifically, given an RGB-D image, we first employ Multiscale Combinatorial Grouping~(MCG)~\cite{mcg} to generate a number of object proposals from RGB information and encode the original depth value to the three-channel HHA representation, as proposed in~\cite{gupta2014learning}. Following the benchmark object detection framework of Fast R-CNN~\cite{fast-rcnn}, our CMAC model takes as input an RGB image, an HHA image and corresponding object proposals to generate class labels as well as a refined bounding box for each object proposal. 

As shown in Fig.~\ref{fig:framework}, the feature extraction module is built on two separate fully convolutional sub-networks, including the VGG16 model~\cite{VGG} for RGB modality and AlexNet model~\cite{alexnet} for depth modality. The output of the last convolutional layer is being treated as our initial feature for object detection, therein including $D$ convolutional maps. The two fully convolutional sub-networks take as input the RGB image and the HHA image to generate the corresponding feature cube. Region-of-Interest (RoI) pooling operations are performed on the two feature cubes to obtain both global~(whole image) and local features~(object proposal) of the two modalities before being fed to a cross-modal feature fusion module. Moreover, both the fused global feature and the fused local feature are fed to a global context modeling module to obtain an attentional global context feature for the corresponding object proposal, while the fused local feature itself is also treated as an input for the fine-grained object part attention, which generates an embedded local feature. Finally, the concatenation of the global context feature and the embedded local feature are employed for final object detection, while local feature embedding is applied for further bounding box regression. %In the following subsections, we explain the feature fusion module as well as the two attentional context modeling sub-networks in more details.

\subsection{Cross-Modal Feature Fusion}
It has been widely verified that the RGB modality and depth modality are complementary, the combination of which can help to boost the RGB-D object detection performance~\cite{gupta2014learning,gupta2016cross}. In this paper, we exploit the features extracted from the two modalities for both global context modeling and local proposal feature description. Specifically, we design a simple yet effective sub-network to fuse features extracted from both modalities. For each object proposal, we extract a fixed-size feature representation using ROI pooling \cite{fast-rcnn} in both modalities, denoted as \textit{$\bm{F}_{l\_rgb}$} and \textit{$\bm{F}_{l\_depth}$}. We also apply a pooling operation to the output feature map of the last convolution layer of the two fully convolutional networks to generate fixed-size feature cubes, denoted as~\textit{$\bm{F}_{g\_rgb}$} and \textit{$\bm{F}_{g\_depth}$}, respectively. The feature fusion between RGB and depth modality can be represented by

\begin{equation}
  \textit{$\bm{F}_{l\_fused}$} = \textit{concat}(\textit{$\bm{F}_{l\_rgb}$},        \textit{$\bm{F}_{l\_depth}$})
\end{equation}

\begin{equation}
  \textit{$\bm{F}_{g\_fused}$} = \textit{concat}(\textit{$\bm{F}_{g\_rgb}$},        \textit{$\bm{F}_{g\_depth}$})
\end{equation} 

where \textit{$\bm{F}_{l\_fused}$} and \textit{$\bm{F}_{g\_fused}$} are the global context feature and local object proposal feature after fusion, respectively, and \textit{concat($\cdot$)} indicates the concatenation operation of feature representations along the channel axis.

In contrast to \cite{gupta2014learning, gupta2016cross, song2016deep}, which apply two independent CNNs to separately extract features from both modalities and directly perform simple concatenation for final classification, our cross-modal feature fusion operation is treated as a feature generation step for further global context modeling and local feature embedding before final classification. In the experiment section, we verify that our proposed cross-modal feature representation can help to produce more effective local and global context information, greatly improving the performance of the final classification.

\subsection{Attention-based Global Context Modeling}\label{sssec:global}
It is well known that contextual representation is crucial for accurate visual recognition~\cite{bell2016inside, li2015visual, li2016attentive, mottaghi2014role,li2017context,li2017instance}. Instead of directly obtaining fixed context information to assist in object detection~\cite{li2016attentive,mottaghi2014role}, we focus on exploiting adaptive context information for each object proposal. Specifically, we design a soft attention model  based on multi-layered RNNs with  LSTM units to spatially weight the features and generate an adaptive global context feature for each object proposal. Average pooling and max pooling operations over the feature map of the whole image can be considered as special cases of our method.

The attentional context model takes as input the concatenation of the global feature cube and that of the local feature cube before being fed to a $1 \times 1$ convolutional layer for feature embedding. The dimensions of the embedded global and local feature are denoted as $K\times K\times D$ (20$\times$20$\times 512$ in our experiments) and $S\times S\times D$ (7$\times$7$\times 512$ in our experiments), respectively. Based on these embedded feature cubes, the RNN model learns an attentional map of size \textit{K $\times$ K} to determine the effectiveness of the contextual region that may be beneficial to the object detection. 

Inspired by the LSTM-based soft attention model proposed in~\cite{sharma2015action}, we apply an LSTM network to generate a contextual attention map at every time step conditioned on the previous hidden state, the globally embedded feature vector as well as the local feature. Specifically, at each time-step $t$, we extract $K^2$ D-dimensional global feature vectors as well as $S^2$ local object proposal feature vectors. As in~\cite{sharma2015action}, we refer to these feature vectors as global feature slices and local feature slices, respectively, denoted as

\begin{table} \centering \renewcommand\arraystretch{1.2}
	\captionsetup{font={small}}
	\caption{\label{table:6} Detection results from different methods on SUNRGBD and NYUv2. AC-CNN* indicates our implementation of the RGB-D version of AC-CNN \cite{li2016attentive}. G and L  denote our proposed model incorporated with a single LSTM module~(G) or STN module~(L), respectively. (w/o fusion) and (w/ fusion) denote without and with multi-modal context fusion, respectively.}
	\begin{tabular}{c|cc|c|c}
	\toprule
	\multirow{2}{*}{\textbf{Method}} &
	\multirow{2}{*}{\textbf{G}} &
	\multirow{2}{*}{\textbf{L}} &
	\multicolumn{2}{c}{\textbf{mAP}} \\

	\cline{4-5}
		& & & \small{SUNRGBD} & \small{NYUv2} \\

	\midrule
	ST(baseline) \cite{gupta2016cross} &&& 43.8 & 49.1 \\
	AC-CNN* \cite{li2016attentive} &$\surd$ &$\surd$ & 45.4 & 50.2 \\
	
	\midrule
	
	\multirow{3}*{Ours (w/o fusion)} & &$\surd$ & 46.3 & 50.9 \\
	&$\surd$ & & 46.2 & 51.3 \\
	&$\surd$ &$\surd$ & 46.9 & 51.9 \\
	\midrule
	Ours (w/ fusion) &$\surd$ &$\surd$ & \textbf{47.5} & \textbf{52.3} \\
	
	\bottomrule
	\end{tabular}
\end{table}

\begin{table} \centering \renewcommand\arraystretch{1.2}
	\captionsetup{font={small}}
	\caption{\label{table:8} Comparison of exploiting global context using different methods on SUNRGBD and NYUv2}
	\begin{tabular}{c|c|c}
	\toprule
	\multirow{2}{*}{\textbf{Method}} &
	\multicolumn{2}{c}{\textbf{mAP}} \\

	\cline{2-3}
		& SUNRGBD & NYUv2 \\

	\midrule
	Average Pooling &44.3 & 49.4\\
	Fixed Attentive Context \cite{li2016attentive} & 44.8 & 49.7 \\
	Adaptive Attentive Context (Ours) & 46.2 & 51.3 \\
	
	\bottomrule
	\end{tabular}
\end{table}


%% add formulation
\begin{equation}
\begin{cases}
G_t = \left [ G_{t,1},...,G_{t,K^2} \right ] & G_{t,i} \in \mathbb{R}^D\\ 
\cr
L_t = \left [ L_{t,1},...,L_{t,S^2} \right ] & L_{t,i} \in \mathbb{R}^D 
\end{cases}
\end{equation}


Each vertical column of $G_t$ and $L_t$ denotes the feature representation~(receptive field) in the input image. We follow the implementation of the LSTM network in~\cite{hochreiter1997long}, which is formulated as

\begin{equation}
\label{equ:3}
\left( 
  \begin{array}{lr}
  \bm{i}_{t} \\ 
  \bm{f}_{t} \\
  \bm{o}_{t} \\
  \bm{g}_{t}
  \end{array} 
\right) \\ 
= \\
\left( 
  \begin{array}{lr}
  \sigma \\ 
  \sigma \\
  \sigma \\
  tanh
  \end{array} 
\right) \\
\\textit{\large{T}} \\
\\left( 
  \begin{array}{lr}
  \bm{h}_{t-1} \\ 
  \bm{x}_{t} \\
  \bm{z}_{t} \\
  \end{array} 
\right),
\end{equation}
\begin{equation}
\label{equ:4}
  \bm{c}_{t} = \bm{f}_{t} \odot \bm{c}_{t-1} + \bm{i} \odot \bm{g}_{t}
\end{equation}
\begin{equation}
\label{equ:5}
  \bm{h}_{t} = \bm{o}_{t} \odot tanh(\bm{c}_{t})
\end{equation}

where $\bm{i}_{t}$, $\bm{f}_{t}$, $\bm{c}_{t}$, $\bm{o}_{t}$, and $\bm{h}_{t}$ are the input gate, forget gate, cell state, output gate and hidden state of the LSTM, respectively; $\bm{x}_{t}$ is the global context feature vector input to the LSTM at time step \textit{t}; the vector $\bm{z} \in \mathbb{R}^{D}$ is the local feature embedding of the object proposal with the global average pooling operation; $\textit{T} \in \mathbb{R}^{(2D + d) \times 4d}$ denotes a simple affine transformation with trainable parameters, where $d$ is the dimensionality of $i_t$, $f_t$, $c_t$ and $h_t$; and $\sigma$ and $\odot$  denote the logistic sigmoid activation and element-wise multiplication, respectively.

At each time step $t$, our LSTM model learns to predict a weight map $\alpha_{t+1}$ of size $K\times K$, where its value corresponds to the spatial attention that should be paid when performing proposal classification. The weight map \textit{$\alpha_{i}$} is computed by a multilayer perception $\phi$ conditioned on the previous hidden state \textit{$h_{t-1}$}. The spatial weight of $\alpha_{t}$ at location $i$ can thus be computed as follows: 

\begin{equation}
  e_{ti} = \phi(\bm{h}_{t-1}) 
\end{equation}

\begin{equation}{\alpha}_{ti} = \frac{exp(e_{ti})}{\sum_{k=1}^{K \times K}exp(e_{tk})}
\end{equation}
Based on the weight map, the global context feature vector $x$ at time step $t$ is computed as an average of the feature slices weighted according to $\alpha_{t}$, formulated as
\begin{equation}
  \bm{x}_{t} = \sum_{i=1}^{K^{2}}{{\alpha}_{t,i}\bm{F}_{g\_fused, i}}
\end{equation}
where $F_{g\_fused, i}$ is the $i^{th}$ global feature slice. Because the relevant regions are given higher weights, the global feature $\bm{x}_{t}$ will be dominated by features from these regions and hence provide more useful contextual information for more accurate object detection.

During the initialization stage, we follow the same strategy proposed in~\cite{show2015tell} for faster convergence. Specifically, we initialize the cell state $\bm{c}_t$ and the hidden state $\bm{h}_t$ of the LSTM network as 
\begin{equation}
  \bm{c}_0 = \textit{f}_{\textit{init, c}} \left(\frac{1}{K^2}\sum_{i=1}^{K^2}\bm{F}_{g\_fused, i} \right)
\end{equation}

\begin{equation}
  \bm{h}_0 = \textit{f}_{\textit{init, h}} \left(\frac{1}{K^2}\sum_{i=1}^{K^2}\bm{F}_{g\_fused, i} \right)
\end{equation}
where $\textit{f}_{\textit{init, c}}$ and $\textit{f}_{\textit{init, h}}$ are two multi-layer perceptions. The two initial values are applied to infer the initial weights $\alpha_1$ for the initialization of the global context feature $\bm{x}_1$.

As shown in Fig.~\ref{fig:framework}, the output of our LSTM model is a D-dimensional global context feature, which is further fed to two fully connected layers to produce the final feature representation, denoted as $F_G$.

\begin{figure}
  \centering
  %% insert PDF file testpdf.pdf
  \includegraphics[width=0.4\textwidth]{./stn_fig.pdf}
  \caption{\label{fig:stn} Illustration of the STN module. The STN module takes the feature of the object proposal as input and attends to the most discriminative parts. The feature from these parts will subsequently serve as an enhanced local feature in object classification and bounding box regression.}
\end{figure}

\subsection{Fine-grained object part attention}
\label{sssec:local}
Because the local salient parts inside a specific object proposal play an important guiding role in judging the classification of an object (especially for partially occluded objects), we further propose to employ multiple STNs~\cite{jaderberg2015spatial} in parallel to infer discriminative object parts for each object proposal. The spatial transformer is a differential module that learns to spatially transform the input feature maps $U$ to the output feature maps $V$. A spatial transformer is applied in the following three steps. First, a localization network is employed to predict the affine transformation matrix $A_{\theta}$ to be applied to the input feature map. Second, $A_{\theta}$ is being applied to create a sampling grid in $U$ by the grid generator. Finally, a sampler is adopted to produce the output maps sampled from the regions of input maps at the sampling grid. As shown in Figure \ref{fig:stn}, we train each transformer to automatically attend to discriminative object parts inside an object proposal. During training, we fix the scaling factor to 0.5 and only accept scaling and translating in each spatial transformer. Thus, $A_{\theta}$ is given by
\begin{equation}
A_{\theta} = \\
\left[
	\begin{array}{ccc}
	0.5 & 0 & t_x \\
	0 & 0.5 & t_y \\
	\end{array}	
\right]
\end{equation}

where $\theta = [t_x, t_y]$ are the translation parameters that are predicted based on the localization network.

Taking the local context feature map $\bm{F}_{l\_fused} \in \mathbb{R}^{D \times S \times S}$ as input, each transformer in our object part attention module transforms and samples the input map to the output map $\bm{q} \in \mathbb{R}^{D \times S \times S}$. After normalization, the outputs of each transformer are concatenated with the local context feature to form a mid-level feature representation for an object proposal, defined as

\begin{equation}
	\bm{F}_{mid} = concat(\bm{F}_{l\_fused}, \ \bm{q}_i, \ ..., \ \bm{q}_N).
\end{equation}

where $\bm{q}_i$ is the output of the $i_{th}$ transformer and $N$ is the number of spatial transformers.

As shown in Figure \ref{fig:framework}, we use a $1 \times 1$ convolution layer after re-scaling to reduce the dimensions of $\bm{F}_{mid}$ from $S \times S \times (N \times D)$ to $S \times S \times D$, which is then fed to two fully connected layers to infer the final feature representation for the object proposal, denoted as $\bm{F}_L$.

\begin{figure*}
\centering
  %% insert PDF file testpdf.pdf
  \includegraphics[width=0.9\textwidth]{./attention_map.pdf}
  \caption{Illustration of the attentional weight maps generated by the attention-based global context modeling module. The top rows are the input images and region proposals. The middle and bottom rows are  the attentional weight maps generated by our model without context fusion and those with context fusion, respectively. The bottom two rows show that our model can perceive the most relevant regions to the given object proposal and that more useful regions can be acquired through context fusion. A detailed discussion can be found in section \ref{sec:visualization}.}
\label{fig:detection maps}
\end{figure*}


\subsection{Training Objective}
Denote $p = (p_0,...,p_L)$ as the predicted discrete probability distribution (per ROI) over $C+1$ categories and $t^*$ as the predicted bounding-box regression offsets. Given the obtained local and global context features ~$\bm{F}_L$ and~$\bm{F}_G$, $p$ and $t^*$ can be computed as follows:

\begin{equation}
	p = \textit{Softmax} \left( \textit{$f_{cls}$} \left( concat(\bm{F}_L, \bm{F}_G) \right) \right)
\end{equation}

\begin{equation}
	t^* = \textit{$f_{loc}$} \left( \bm{F}_L \right)
\end{equation}

where $Softmax(\cdot)$ indicates the softmax operation and $f_{cls}$ and $f_{loc}$ are two fully connected layers with $C+1$ units and $4 \times C$ units, respectively.

Note that we only incorporate local contextual information for bounding-box regression. Finally, we minimize an objective function following the multi-task loss given in Fast-RCNN \cite{fast-rcnn}, which is defined as

\begin{equation}
	L(p, u, t^u, v) = L_{cls}(p, u) + [u \geq 1] L_{loc}(t^u, v)
\end{equation}

where $u$ is the ground-truth label, $v$ is the regression target, $L_{cls}$ is the log loss for ground-truth class $u$, and $L_{loc}$ is the smooth $L_1$ loss proposed in \cite{fast-rcnn}. $[u \geq 1]$ evaluates to 1 when $u \geq 1$ and 0 otherwise. By convention, the background class is labeled as $u = 0$.

\begin{table*} \addtolength{\tabcolsep}{-2pt} \centering \renewcommand\arraystretch{1.5}
	\caption{\label{table:1} Detection results on SUNRGBD. AC-CNN* indicates our implementation of the RGB-D version of AC-CNN \cite{li2016attentive}. (w/o fusion) and (w/ fusion) denote without and with multi-modal context fusion, respectively.}
	\begin{tabular}{c|c|ccccccccccccccccccc}
	\toprule
	\textbf{\small{Method}}
	 	&\textbf{\small{mAP}}
	 	&\textbf{\tiny{bathtub}}
		&\textbf{\tiny{bed}}&\textbf{\tiny{bookshelf}}
		&\textbf{\tiny{box}}&\textbf{\tiny{chair}}
		&\textbf{\tiny{counter}}&\textbf{\tiny{desk}}
		&\textbf{\tiny{door}}&\textbf{\tiny{dresser}}
		&\textbf{\tiny{garbage\_bin}}&\textbf{\tiny{lamp}} 					&\textbf{\tiny{monitor}}&\textbf{\tiny{night\_stand}}
		&\textbf{\tiny{pillow}}&\textbf{\tiny{sink}}
		&\textbf{\tiny{sofa}}&\textbf{\tiny{table}}
		&\textbf{\tiny{tv}}&\textbf{\tiny{toilet}} \\
	\midrule
	
	\small{RGB-D RCNN \cite{gupta2014learning}}
		&\small{35.2}
		&\tiny{49.6}&\tiny{76.0}
		&\tiny{35.0}&\tiny{5.8}
		&\tiny{41.2}&\tiny{8.1}
		&\tiny{16.6}&\tiny{4.2}
		&\tiny{31.4}&\tiny{46.8}
		&\tiny{22.0}&\tiny{10.8}
		&\tiny{37.2}&\tiny{16.5}
		&\tiny{41.9}&\tiny{42.2}
		&\tiny{43.0}&\tiny{32.9}
		&\tiny{69.8} \\
		
	%result of baseline
	\small{ST(baseline) \cite{gupta2016cross}}
		&\small{43.8}
		&\tiny{65.3}&\tiny{83.0}
		&\tiny{54.4}&\tiny{14.4}
		&\tiny{46.9}&\tiny{14.6}
		&\tiny{23.9}&\tiny{15.3}
		&\tiny{41.3}&\tiny{51.0}
		&\tiny{32.1}&\tiny{36.8}
		&\tiny{46.6}&\tiny{23.4}
		&\tiny{43.9}&\tiny{61.3}
		&\tiny{48.7}&\tiny{50.5}
		&\tiny{79.4} \\
		
	% result of AC-CNN
	\small{AC-CNN*}
		&\small{45.4}
		&\tiny{65.8}&\tiny{83.3}
		&\tiny{56.2}&\tiny{16.4}
		&\tiny{47.5}&\tiny{16.0}
		&\tiny{24.9}&\tiny{16.6}
		&\tiny{42.7}&\tiny{53.4}
		&\tiny{33.8}&\tiny{39.5}
		&\tiny{47.1}&\tiny{25.2}
		&\tiny{45.3}&\tiny{61.9}
		&\tiny{49.0}&\tiny{54.1}
		&\tiny{84.2} \\
	\midrule
	
	\small{Ours (w/o fusion)}
		&\small{46.9}
		&\tiny{68.2}&\tiny{85.7}
		&\tiny{56.0}&\tiny{17.3}
		&\tiny{49.8}&\tiny{17.1}
		&\tiny{25.2}&\tiny{16.9}
		&\tiny{43.5}&\tiny{54.2}
		&\tiny{35.5}&\tiny{40.7}
		&\tiny{49.4}&\tiny{26.1}
		&\tiny{\textbf{46.6}}&\tiny{66.3}
		&\tiny{52.0}&\tiny{56.5}
		&\tiny{84.3} \\
	\small{Ours (w/ fusion)}
		&\small{\textbf{47.5}}
		&\tiny{\textbf{69.0}}&\tiny{\textbf{86.1}}
		&\tiny{\textbf{57.9}}&\tiny{\textbf{18.2}}
		&\tiny{\textbf{50.3}}&\tiny{\textbf{17.4}}
		&\tiny{\textbf{26.8}}&\tiny{\textbf{17.3}}
		&\tiny{\textbf{44.4}}&\tiny{\textbf{54.4}}
		&\tiny{\textbf{35.6}}&\tiny{\textbf{40.5}}
		&\tiny{\textbf{49.8}}&\tiny{\textbf{26.7}}
		&\tiny{46.6}&\tiny{\textbf{67.2}}
		&\tiny{\textbf{52.9}}&\tiny{\textbf{56.7}}
		&\tiny{\textbf{84.9}} \\
	\bottomrule
	\end{tabular}
\end{table*}

\begin{table*} \addtolength{\tabcolsep}{-2pt} \centering  \renewcommand\arraystretch{1.5}
	\caption{\label{table:2} Detection results on NYUv2. AC-CNN* indicates our implementation of the RGB-D version of AC-CNN \cite{li2016attentive}. (w/o fusion) and (w/ fusion) denote without and with multi-modal context fusion, respectively.}
	\begin{tabular}{c|c|ccccccccccccccccccc}
	\toprule
	\textbf{\small{Method}}
	 	&\textbf{\small{mAP}}
	 	&\textbf{\tiny{bathtub}}
		&\textbf{\tiny{bed}}&\textbf{\tiny{bookshelf}}
		&\textbf{\tiny{box}}&\textbf{\tiny{chair}}
		&\textbf{\tiny{counter}}&\textbf{\tiny{desk}}
		&\textbf{\tiny{door}}&\textbf{\tiny{dresser}}
		&\textbf{\tiny{garbage\_bin}}&\textbf{\tiny{lamp}} 					&\textbf{\tiny{monitor}}&\textbf{\tiny{night\_stand}}
		&\textbf{\tiny{pillow}}&\textbf{\tiny{sink}}
		&\textbf{\tiny{sofa}}&\textbf{\tiny{table}}
		&\textbf{\tiny{tv}}&\textbf{\tiny{toilet}} \\
	\midrule
	
	\small{RGB-D RCNN \cite{gupta2014learning}}
		&\small{32.5}
		&\tiny{22.9}&\tiny{66.5}
		&\tiny{21.8}&\tiny{3.0}
		&\tiny{40.8}&\tiny{37.6}
		&\tiny{10.2}&\tiny{20.5}
		&\tiny{26.2}&\tiny{37.6}
		&\tiny{29.3}&\tiny{43.4}
		&\tiny{39.5}&\tiny{37.4}
		&\tiny{24.2}&\tiny{42.8}
		&\tiny{24.3}&\tiny{37.2}
		&\tiny{53.0} \\
		
	%result of baseline on NYUv2
	\small{ST(baseline) \cite{gupta2016cross}}
		&\small{49.1}
		&\tiny{50.6}&\tiny{81.0}
		&\tiny{52.6}&\tiny{5.4}
		&\tiny{53.0}&\tiny{56.1}
		&\tiny{21.0}&\tiny{34.6}
		&\tiny{57.9}&\tiny{46.2}
		&\tiny{42.5}&\tiny{62.9}
		&\tiny{54.7}&\tiny{49.1}
		&\tiny{50.0}&\tiny{65.9}
		&\tiny{31.9}&\tiny{50.1}
		&\tiny{68.0} \\
		
	% result of AC-CNN on NYUv2
	\small{AC-CNN*}
		&\small{50.2}
		&\tiny{52.2}&\tiny{82.4}
		&\tiny{52.5}&\tiny{8.6}
		&\tiny{54.8}&\tiny{57.3}
		&\tiny{22.7}&\tiny{34.1}
		&\tiny{58.1}&\tiny{46.5}
		&\tiny{42.9}&\tiny{63.6}
		&\tiny{55.2}&\tiny{49.7}
		&\tiny{51.4}&\tiny{66.8}
		&\tiny{33.5}&\tiny{51.8}
		&\tiny{70.4} \\
	
	\midrule	
	\small{Ours (w/o fusion)}
		&\small{51.9}
		&\tiny{55.2}&\tiny{83.4}
		&\tiny{\textbf{54.2}}&\tiny{9.4}
		&\tiny{55.1}&\tiny{58.5}
		&\tiny{24.0}&\tiny{35.9}
		&\tiny{58.3}&\tiny{46.6}
		&\tiny{44.8}&\tiny{65.7}
		&\tiny{57.0}&\tiny{\textbf{52.7}}
		&\tiny{53.6}&\tiny{68.4}
		&\tiny{\textbf{35.3}}&\tiny{54.8}
		&\tiny{73.9} \\
		
	\small{Ours (w/ fusion)}
		&\small{\textbf{52.3}}
		&\tiny{\textbf{55.6}}&\tiny{\textbf{83.9}}
		&\tiny{54.0}&\tiny{\textbf{9.8}}
		&\tiny{\textbf{55.4}}&\tiny{\textbf{59.2}}
		&\tiny{\textbf{24.1}}&\tiny{\textbf{36.3}}
		&\tiny{\textbf{58.5}}&\tiny{\textbf{47.2}}
		&\tiny{\textbf{45.0}}&\tiny{\textbf{65.8}}
		&\tiny{\textbf{57.6}}&\tiny{52.7}
		&\tiny{\textbf{53.8}}&\tiny{\textbf{69.1}}
		&\tiny{35.0}&\tiny{\textbf{56.9}}
		&\tiny{\textbf{74.7}} \\
	\bottomrule
	\end{tabular}
\end{table*}

\section{Experimental Results}

\subsection{Experimental Settings}

\textbf{Datasets and Evaluation Metrics:} We evaluate our model on two RGB-D datasets: SUNRGBD \cite{song2015sun} and NYUv2 \cite{silberman2012indoor}. The SUNRGBD and NYUv2 datasets contain 10335 and 1449 RGB-D images, respectively, and are divided into \textit{train} and \textit{test} subsets. We adopt \textit{Average Precision} (AP) and \textit{mean of Average Precision} (mAP) following the PASCAL challenge protocols as our evaluation metrics.

\textbf{Implementation Details:} In our experiments, we implement our model based on Fast R-CNN~\cite{fast-rcnn}, an open-source framework for traditional RGB object detection built on the Caffe platform~\cite{jia2014caffe}. We utilize the network architecture from Gupta\textit{et al}~\cite{gupta2016cross} as our basic CNN network structure for convolutional feature map extraction. All the newly added fully connected and convolutional layers are randomly initialized with a zero-mean Gaussian distribution with standard deviations of 0.01 and 0.001. The recurrent attention model consists of 4 stacked LSTM units with shared parameters. All the parameters of the LSTM units are initialized based on the xavier algorithm~\cite{glorot2010understanding}.

\begin{table} \centering \renewcommand\arraystretch{1.2}
	\caption{\label{table:4} Comparison of different LSTM settings utilized in the attention-based global context sub-module. The experiments are conducted on SUNRGBD. (2 $\times$ LSTM) denotes that there are 2 stacked LSTM units in the global contextualized sub-network.}
	\begin{tabular}{c|c}
	\toprule
	\textbf{LSTM Settings}&\textbf{mAP} \\
	\midrule
	Ours (2 $\times$ LSTM)  & 45.4 \\
	Ours (3 $\times$ LSTM)  & 46.0 \\	
	Ours (4 $\times$ LSTM)  & \textbf{46.2} \\
	Ours (5 $\times$ LSTM)  & \textbf{46.2} \\
	\bottomrule
	\end{tabular}
\end{table}

We apply Stochastic Gradient Decent (SGD) to fine tune our model. Each SGD mini-batch is composed of 128 randomly sampled object proposals from 2 randomly chosen images. In each mini-batch, we select 25\% of the ROIs as foreground from object proposals that have intersection over union (IoU) overlap with a ground-truth bounding box of at least 0.5. The remaining ROIs are sampled from object proposals that have a maximum IoU with ground truth in the interval $[0.1, 0.5)$ and act as background with ground truth label $u = 0$. During training, images are horizontally flipped with a probability of 0.5 for data augmentation, and no other augmentation is used. We run SGD for approximately 10 epochs on the training set to fine tune the network parameters. The momentum is set to 0.9, and the learning rate is initialized to 0.001 and decreased by 10 every 4 epochs. It takes approximately 1.5 days to train our model on a single NVIDIA GeForce GTX TITAN X GPU with 12 GB of memory. 

It costs approximately 10 GB of GPU memory to train our model. The average training time for each iteration is approximately 1.23 seconds. However, the testing process is particularly efficient and takes approximately 0.58 seconds (excluding object proposal extraction) to process one image. 


\subsection{Performance Comparisons}
\textbf{RGB-D Datasets}: We compare our proposed method against recent state-of-the-art RGB-D object detection methods, including rich image and depth feature-based RGB-D object detection~\cite{gupta2014learning} and the supervision-transfer-based model~\cite{gupta2016cross}. Moreover, to better validate the superiority of the attention-based global context and fine-grained object part attention on RGB-D datasets, we also implement an RGB-D version (denoted as AC-CNN*) of the AC-CNN model proposed in~\cite{li2016attentive} for comparison. AC-CNN follows a similar idea to our proposed method but incorporates fixed global and local attentive contexts to assist in improving the object detection performance. In the implementation, we apply the Fast RCNN \cite{fast-rcnn} framework based on AlexNet \cite{alexnet} to the depth modality for proposal classification and bounding-box position regression. The final results are obtained by averaging the results from the RGB modality and depth modality. For fair comparison, we also apply the same depth modality processing as in AC-CNN* to our model; we call this custom model RGB-D detection without cross-modal fusion~(denoted as w/o fusion).

Table \ref{table:1} and Table \ref{table:2} illustrate the object detection results of our model, AC-CNN*, and the other two state-of-the-art RGB-D object detection models on the SUNRGBD and NYUv2 datasets. As shown in the table, our proposed method obtains state-of-the-art mAP scores of 47.5\% and 52.3\% on SUNRGBD and NYUv2, which outperforms the ST model \cite{gupta2016cross} by 3.7\% and 3.2\%, respectively. The improvements validate the effectiveness of our model in RGB-D object detection by incorporating the proposed attention-based global context and fine-grained attentional object parts learned from the fused cross-modal context.  Furthermore, our model~(Ours (w/o fusion)) gains 1.5\% and 1.7\% improvements in mAP scores over AC-CNN* on the  SUNRGBD and NYUv2 datasets,  respectively, and achieve better detection results on most of the categories.

\textbf{RGB Dataset}: To compare our model with the AC-CNN model~\cite{li2016attentive} in a more equitable way, we remove the depth modality from our model and perform an extra evaluation on PASCAL VOC 2007, which contains 9963 RGB images. Specifically, we implement a variant of our model (denoted as Ours*) that performs global context modeling and object part attention only on the RGB modality without incorporating information from the depth modality. As shown in Table~\ref{table:3}. Our model outperforms the baseline FRCN \cite{fast-rcnn} and AC-CNN \cite{li2016attentive} by 3.6\% and 1.2\% in terms of mAP scores, respectively. The improvement on the RGB dataset as well as the favorable results achieved for RGB-D object detection well demonstrate the superiority of the proposed attention-based global context and fine-grained object part attention over the fixed global context and multi-scale local context proposed in \cite{li2016attentive}. Table \ref{table:voc_2012} provides the comparisons of the proposed method with several state-of-the-art methods~\cite{ravishankar2008multi, liu2016ssd, shen2017dsod, bell2016inside, dai2016r} on PASCAL VOC 2012. It can be observed that our model obtains an mAP score of 76.7\%, which outperforms the baseline model by 2.9\%. Our model also achieves competitive results compared with the state-of-the-art methods, which validates the effectiveness of the proposed method.

\begin{table} \centering \renewcommand\arraystretch{1.5}
	\caption{\label{table:5} Comparison of different STN settings utilized in fine-grained object part attention sub-module. The experiments are conducted on SUNRGBD. (2 $\times$ STN) indicates that there are 2 parallel spatial transformers in the local contextualize sub-network.}
	\begin{tabular}{c|c}
	\toprule
	\textbf{STN Settings}&\textbf{mAP} \\
	\midrule
	%result of our model
	Ours (1 $\times$ STN) & 45.7 \\
	Ours (2 $\times$ STN) & \textbf{46.3} \\
	Ours (3 $\times$ STN) & 46.0 \\
	\bottomrule
	\end{tabular}
\end{table}

\subsection{Ablation Studies}
\label{sec:ablation}
In this subsection, we show the effectiveness and necessity of each component in our proposed model and also demonstrate the effectiveness of the network design. %The experiments are conducted on SUNRGBD and NYUv2 based on the Fast R-CNN detection framework.

\begin{figure*}
\centering
  %% insert PDF file testpdf.pdf
  \includegraphics[width=0.9\textwidth]{./detection_results.pdf}
	\caption{Comparison of detection results produced by ST
		\cite{gupta2016cross} (top row), AC-CNN 
		\cite{li2016attentive} (middle row) and our model (bottom row). 	
		The red and green rectangles indicate the ground-truth bounding box and the predicted results, respectively.}
\label{fig:detection results}
\end{figure*}

\begin{table*} \addtolength{\tabcolsep}{-3pt} \renewcommand\arraystretch{1.5}
	\caption{\label{table:3} Detection results on VOC 2007. Ours* denotes a variant of our model in which we incorporate only RGB information for object detection}
	\begin{tabular}{c|c|cccccccccccccccccccc}
	\toprule
	\textbf{\small{Method}}
	 	&\textbf{\small{mAP}}
	 	&\textbf{\scriptsize{aero}}&\textbf{\scriptsize{bike}}
	 	&\textbf{\scriptsize{bird}}&\textbf{\scriptsize{boat}}
		&\textbf{\scriptsize{bottle}}&\textbf{\scriptsize{bus}}
		&\textbf{\scriptsize{car}}&\textbf{\scriptsize{cat}}
		&\textbf{\scriptsize{chair}}&\textbf{\scriptsize{cow}}
		&\textbf{\scriptsize{table}}&\textbf{\scriptsize{dog}}
		&\textbf{\scriptsize{horse}}&\textbf{\scriptsize{mbike}}
		&\textbf{\scriptsize{person}}&\textbf{\scriptsize{plant}}
		&\textbf{\scriptsize{sheep}}&\textbf{\scriptsize{sofa}}
		&\textbf{\scriptsize{train}}&\textbf{\scriptsize{tv}} \\
	\midrule
	
	\small{FRCN(Baseline) \cite{fast-rcnn}}
		&\small{70.0}
		&\scriptsize{77.0}&\scriptsize{78.1}
		&\scriptsize{69.3}&\scriptsize{59.4}
		&\scriptsize{38.3}&\scriptsize{81.6}
		&\scriptsize{78.6}&\scriptsize{86.7}
		&\scriptsize{42.8}&\scriptsize{78.8}
		&\scriptsize{68.9}&\scriptsize{84.7}
		&\scriptsize{82.0}&\scriptsize{76.6}
		&\scriptsize{69.9}&\scriptsize{31.8}
		&\scriptsize{70.1}&\scriptsize{74.8}
		&\scriptsize{80.4}&\scriptsize{70.4} \\
	
	\small{AC-CNN \cite{li2016attentive}}
		&\small{72.4}
		&\scriptsize{79.1}&\scriptsize{79.2}
		&\scriptsize{71.9}&\scriptsize{\textbf{61.0}}
		&\scriptsize{43.2}&\scriptsize{83.0}
		&\scriptsize{81.4}&\scriptsize{87.7}
		&\scriptsize{50.0}&\scriptsize{82.1}
		&\scriptsize{73.6}&\scriptsize{83.4}
		&\scriptsize{84.2}&\scriptsize{77.5}
		&\scriptsize{72.0}&\scriptsize{35.8}
		&\scriptsize{71.9}&\scriptsize{74.7}
		&\scriptsize{\textbf{85.8}}&\scriptsize{71.0} \\
	
	\small{Ours*}
		&\small{\textbf{73.6}}
		&\scriptsize{\textbf{81.0}}&\scriptsize{\textbf{80.2}}
		&\scriptsize{\textbf{72.4}}&\scriptsize{60.5}
		&\scriptsize{\textbf{45.3}}&\scriptsize{\textbf{84.1}}
		&\scriptsize{\textbf{82.8}}&\scriptsize{\textbf{88.0}}
		&\scriptsize{\textbf{51.6}}&\scriptsize{\textbf{82.5}}
		&\scriptsize{\textbf{74.8}}&\scriptsize{\textbf{85.7}}
		&\scriptsize{\textbf{84.9}}&\scriptsize{\textbf{79.6}}
		&\scriptsize{\textbf{72.2}}&\scriptsize{\textbf{36.9}}
		&\scriptsize{\textbf{72.1}}&\scriptsize{\textbf{76.8}}
		&\scriptsize{85.5}&\scriptsize{\textbf{74.3}} \\
			
	\bottomrule
	\end{tabular}
\end{table*}

\textbf{Contribution of Each Component in CMAC model}: As described in Section~\ref{sec:framework}, our proposed CMAC model
consists of three newly added sub-networks on the top of deep feature representation, including cross-modal feature fusion, attention-based global context modeling and fine-grained  object part attention, which are employed to incorporate the strong correlation between different modalities and capture the global and local contextual information, respectively. We investigate the contributions of each component by gradually applying each sub-network to the object detection.  Table~\ref{table:6} shows that 2.5\% and 1.8\% improvements in mAP scores over the baseline model are obtained using only fine-grained object part attention. Similar improvements of 2.4\% and 2.2\% on SUNRGBD and NYUv2 can be observed when only incorporating attention-based global context modeling. The better performance achieved by exploiting both global context features and discriminative object parts evidences the complementarity of the two sub-networks. Furthermore, incorporating cross-modal feature fusion into our detection framework brings an extra performance increase of 0.6\% and 0.4\% on SUNRGBD and NYUv2, respectively. The above experimental results and analysis well demonstrate the effectiveness of  each component in our proposed CMAC framework.

\textbf{Comparison of Diverse Global Context Modeling}: To validate the effectiveness of our attention-based global context, which is generated based on a recurrent model, we compare our model with two variants: the global average pooling method in which the global contextual information is produced by applying the average pooling operation to the extracted feature map, and AC-CNN, which utilizes an attention-based recurrent model to generate the fixed global context. We conduct experiments on the SUNRGBD dataset, and the results are listed in Table~\ref{table:8}. No local context is used during these experiments. It can be observed that our model outperforms the global averaging pooling method and AC-CNN by 1.9\% and 1.5\%, respectively. Simply averaging the features of all regions may introduce both background and inter-class noise, which may deteriorate the object detection performance. Although background noise can be overcome by AC-CNN, which generates a fixed attention map for global context feature extraction and benefits the proposal classification, AC-CNN still suffers from a decreased performance caused by inter-class noise~(e.g., regions that are beneficial for desk classification might provide noisy information to garbage\_bin classification). Note that our attention map for global context weighting is generated according to the diverse contents of each  ROI feature and can be optimized to attend to the most effective regions related to the input content. The results shown in Table~\ref{table:8} verify that our model performs better in mitigating both background and inter-class noise by incorporating global context and thus greatly enhances the accuracy of object detection.

\textbf{Effectiveness of LSTM Settings}: In our proposed CMAC model, we have employed a recurrent model to exploit the attentional global context, in which multiple stacked LSTM units are utilized to generate the attentional weight map in an iterative manner. To investigate the effectiveness of different LSTM settings, we implement several variants, whereby the recurrent model is constructed with different numbers (2 to 5) of LSTM units. The experimental results are listed in Table~\ref{table:4}. As shown in the table, the mAP metric increases by 0.6\% and 0.8\% when the number of stacked LSTM units is increased from $2$ to $3$ and $4$, respectively. When this number reaches or exceeds $5$, no significant performance boosts are achieved, indicating that our model can obtain better context information through recurrent iterations and will converge quickly. We believe that good performance can be obtained in complicated images through more recurrent iterations. 

\textbf{Effectiveness of STN Settings}: In the proposed method, we adopt several parallel multiple transform networks (STNs) to attend to discriminative object parts inside an object proposal. To investigate the most effective STN setting, we implement several variants whereby the fine-grained object parts are inferred from different numbers (2 to 4) of spatial transformers. As shown in Table~\ref{table:5}, the detection performance increases from 43.8\% (baseline) to 45.7\% and 46.3\% with $1$ and $2$ spatial transformers, respectively, which indicates that STNs are able to mine discriminative object parts to enhance the local feature representation. However, increasing the number of spatial transformers does not always bring about a better performance. We observe a 3\% decrease in mAP when increasing the number of spatial transformers from 2 to 3, indicating that the STNs may start to enroll confusing object parts after most of the discriminative parts have been detected.

\begin{table}[h] \centering
	\caption{\label{table:voc_2012}  PASCAL VOC 2012 test detection results. \textbf{07+12+S:} 07 trainval + 12 trainval + segmentation labels, \textbf{07++12:} 07
trainval + 07 test + 12 trainval}
	\begin{tabular}{c|c|c}
	\toprule
	\textbf{Method}&\textbf{data}&\textbf{mAP} \\
	\midrule
	Faster-Rcnn~\cite{ren2015faster} (Baseline) & 07++12  & 73.8 \\
	Multi-stage~\cite{li2017multi}     & 07++12  & 74.9 \\
	SSD300~\cite{liu2016ssd}                    & 07++12  & 75.8 \\
	DOSD~\cite{shen2017dsod}                    & 07++12  & 76.3 \\
	ION~\cite{bell2016inside}                   & 07+12+S & 76.4 \\
	R-FCN multi-scale~\cite{dai2016r}           & 07++12  & \textbf{77.6} \\  
	\midrule
	ours                   & 07++12  & 76.7 \\
	\bottomrule
	\end{tabular}
\end{table}

\subsection{Visualization} 
\label{sec:visualization}
In this subsection, we present some visual comparisons of the RGB-D object detection results as well as some visual effects of the attentional weight maps generated by our global context modeling component. Figure~\ref{fig:detection results} shows some detection results of the ST~\cite{gupta2016cross} model, the AC-CNN \cite{li2016attentive} model and our model. It can be observed that our model performs best in detecting small and occluded objects~(e.g., monitor, box, garbage\_bin and the occluded chair). Furthermore, as shown in the third column, our proposed method is also more robust to appearance-similar instances because of the fusion of the geometry context~(e.g., the pillow with similar texture to the bed). Figure~\ref{fig:detection maps} demonstrates the attentional weight maps generated by our model without~(middle row) and with~(bottom row) context fusion. Obviously, our attentional model is able to perceive regions most relevant to the specific object proposal,~\textit{i.e.,} a lamp is likely to be placed on top of a night stand near a bed, and a night stand is also likely to be placed on the floor near a bed and often co-occurs with a lamp. Moreover, our model obtains more accurate attentional weight maps by fusing information from both RGB and depth modalities since the depth image can provide geometric information. For example, our model is capable of attending to the chairs near the target chair, as they share similar geometric structures. The last column in Fig.~\ref{fig:detection maps} shows that our model will attend to the background regions when the proposal does not contain objects, which helps in making correct classifications.


\section{Conclusion}
In this paper, we have introduced an approach to effectively learn the cross-modal attentive context for RGBD object detection. In our model, the contextual representations from different sources (\textit{i.e.,} RGB and depth modalities) are fused in the cross-modal feature fusion module. Based on the fused local and global feature, a recurrent attention model including several stacked LSTM units is employed to capture a global context that is closely related to the object proposal. Furthermore, our model adopts several parallel spatial transformers, which learn to attend to discriminative parts inside each object proposal, to generate the enhanced local context information. Extensive experiments and state-of-the-art detection results on SUNRGBD and NYUv2 well demonstrate the effectiveness of our model in exploiting contextual information.




% if have a single appendix:
%\appendix[Proof of the Zonklar Equations]
% or
%\appendix  % for no appendix heading
% do not use \section anymore after \appendix, only \section*
% is possibly needed

% use appendices with more than one appendix
% then use \section to start each appendix
% you must declare a \section before using any
% \subsection or using \label (\appendices by itself
% starts a section numbered zero.)
%


%\appendices
%\section{Proof of the First Zonklar Equation}
%Appendix one text goes here.
%
%% you can choose not to have a title for an appendix
%% if you want by leaving the argument blank
%\section{}
%Appendix two text goes here.
%
%
%% use section* for acknowledgment
%\section*{Acknowledgment}
%
%
%The authors would like to thank...


% Can use something like this to put references on a page
% by themselves when using endfloat and the captionsoff option.

\ifCLASSOPTIONcaptionsoff
  \newpage
\fi



% trigger a \newpage just before the given reference
% number - used to balance the columns on the last page
% adjust value as needed - may need to be readjusted if
% the document is modified later
%\IEEEtriggeratref{8}
% The "triggered" command can be changed if desired:
%\IEEEtriggercmd{\enlargethispage{-5in}}

% references section

% can use a bibliography generated by BibTeX as a .bbl file
% BibTeX documentation can be easily obtained at:
% http://mirror.ctan.org/biblio/bibtex/contrib/doc/
% The IEEEtran BibTeX style support page is at:
% http://www.michaelshell.org/tex/ieeetran/bibtex/

{\small
\bibliographystyle{IEEEtran}
%\balance
\bibliography{bare_jrnl}
}



% \begin{IEEEbiography}[{\includegraphics[width=1in,height=1.25in,clip,keepaspectratio]{./guanbin.jpg}}]{Guanbin Li} is currently a research associate professor in School of Data and Computer Science, Sun Yat-sen University. He received his PhD degree from the University of Hong Kong in 2016. He was a recipient of Hong Kong Postgraduate Fellowship. His current research interests include computer vision, image processing, and deep learning. He has authorized and co-authorized on more than 20 papers in top-tier academic journals and conferences. He serves as an area chair for the conference of VISAPP. He has been serving as a reviewer for  numerous academic journals and conferences such as TPAMI, TIP, TMM, TC, CVPR2018 and IJCAI2018.

% \end{IEEEbiography}

% \begin{IEEEbiography}[{\includegraphics[width=1in,height=1.25in,clip,keepaspectratio]{./yukang.jpeg}}]{YuKang Gan} received the B.E. degree from the School of Data and Computer Science, Sun Yat-sen University, Guangzhou, China, in 2015. He is currently pursing the M.Sc. degree in computer science with the School of Data and Computer Science. His research interest includes computer vision and machine learning.
% \end{IEEEbiography}

% \begin{IEEEbiography}[{\includegraphics[width=1in,height=1.25in,clip,keepaspectratio]{./hejun.png}}]{Hejun Wu} received his Ph.D. degree in Computer Science and Engineering from Hong Kong University of Science and Technology in 2008. He is currently an Associate Professor with the School of Data and Computer Science, Sun Yat-sen University, Guangzhou, China. His main research interests include Distributed Computing and Machine Learning. He is a member of the IEEE and ACM.
% \end{IEEEbiography}


% \begin{IEEEbiography}[{\includegraphics[width=1in,height=1.25in,clip,keepaspectratio]{./nongxiao.pdf}}]{Nong Xiao} received the BS and PhD degrees in computer science from the College of Computer at National University of Defense Technology (NUDT) in China, in 1990 and 1996, respectively. He is currently a professor in the State Key Laboratory of High Performance Computting at NUDT, China. His current research interests include large-scale storage system, network computing, and computer architecture. He has more than 130 publications to his credit in journals and international conferences including IEEE TSC, IEEE TMM, JPDC, JCST, HPCA, ICCAD, MIDDLEWARE, MSST, IPDPS, CLUSTER, SYSTOR and MASCOTS. He is a member of the IEEE and ACM.
% \end{IEEEbiography}





% \begin{IEEEbiography}[{\includegraphics[width=1in,height=1.25in,clip,keepaspectratio]{./LiangLin.eps}}]{Liang Lin}
% (M'09, SM'15) is the Executive R$\&$D Director of SenseTime Group Limited and a full Professor of Sun Yat-sen University. He is the Excellent Young Scientist of the National Natural Science Foundation of China. From 2008 to 2010, he was a Post-Doctoral Fellow at University of California, Los Angeles. From 2014 to 2015, as a senior visiting scholar, he was with The Hong Kong Polytechnic University and The Chinese University of Hong Kong. He currently leads the SenseTime R$\&$D teams to develop cutting-edges and deliverable solutions on computer vision, data analysis and mining, and intelligent robotic systems. He has authorized and co-authorized on more than 100 papers in top-tier academic journals and conferences. He has been serving as an associate editor of IEEE Trans. Human-Machine Systems, The Visual Computer and Neurocomputing. He served as Area/Session Chairs for numerous conferences such as ICME, ACCV, ICMR. He was the recipient of Best Paper Runners-Up Award in ACM NPAR 2010, Google Faculty Award in 2012, Best Paper Diamond Award in IEEE ICME 2017, and Hong Kong Scholars Award in 2014. He is a Fellow of IET.

% \end{IEEEbiography}



%
% <OR> manually copy in the resultant .bbl file
% set second argument of \begin to the number of references
% (used to reserve space for the reference number labels box)
%\begin{thebibliography}{1}
%
%\bibitem{IEEEhowto:kopka}
%H.~Kopka and P.~W. Daly, \emph{A Guide to \LaTeX}, 3rd~ed.\hskip 1em plus
%  0.5em minus 0.4em\relax Harlow, England: Addison-Wesley, 1999.
%
%\end{thebibliography}

% biography section
% 
% If you have an EPS/PDF photo (graphicx package needed) extra braces are
% needed around the contents of the optional argument to biography to prevent
% the LaTeX parser from getting confused when it sees the complicated
% \includegraphics command within an optional argument. (You could create
% your own custom macro containing the \includegraphics command to make things
% simpler here.)
%\begin{IEEEbiography}[{\includegraphics[width=1in,height=1.25in,clip,keepaspectratio]{mshell}}]{Michael Shell}
% or if you just want to reserve a space for a photo:

% \begin{IEEEbiography}{Michael Shell}
% Biography text here.
% \end{IEEEbiography}

% if you will not have a photo at all:
% \begin{IEEEbiographynophoto}{John Doe}
% Biography text here.
% \end{IEEEbiographynophoto}

% insert where needed to balance the two columns on the last page with
% biographies
%\newpage

% \begin{IEEEbiographynophoto}{Jane Doe}
% Biography text here.
% \end{IEEEbiographynophoto}

% You can push biographies down or up by placing
% a \vfill before or after them. The appropriate
% use of \vfill depends on what kind of text is
% on the last page and whether or not the columns
% are being equalized.

%\vfill

% Can be used to pull up biographies so that the bottom of the last one
% is flush with the other column.
%\enlargethispage{-5in}



% that's all folks
\end{document}


