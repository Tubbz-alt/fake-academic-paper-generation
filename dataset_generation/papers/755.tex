\documentclass[conference]{IEEEtran}
\usepackage{times}

% numbers option provides compact numerical references in the text. 
\usepackage[numbers]{natbib}
\usepackage{multicol}
\usepackage[bookmarks=true]{hyperref}

\usepackage{graphicx}
\usepackage{amsmath}
\usepackage{amssymb}

\usepackage{subfigure}
\usepackage{xcolor}
\usepackage{footnote}
\usepackage{makecell}
\usepackage{multirow, booktabs}
\graphicspath{{./figures/}{./figures/draw/}}

\newcommand\red[1]{{\color{black}#1}}
%\newcommand\red[1]{{\color{red}#1}}
\newcommand\blue[1]{{\color{blue}#1}}
\newcommand\green[1]{{\color{green}#1}}
\newcommand{\junk}[1]{}
\newcommand{\iancomment}[1]{\red{IAN: #1}}
\newcommand{\method}[1]{Deep-6DPose}
\def\R{{\mathbb R}}
\def\r{{\mathbf R}}

\newcommand\norm[1]{\left\lVert#1\right\rVert}


 
\pdfinfo{
   /Author (Homer Simpson)
   /Title  (Robots: Our new overlords)
   /CreationDate (D:20101201120000)
   /Subject (Robots)
   /Keywords (Robots;Overlords)
}

\begin{document}

% paper title
%\title{Pose R-CNN: Recovering 6D Object Pose from a Single RGB Image}
%\title{6DPoseNet: Recovering 6D Object Pose \\from a Single RGB Image}
%\title{Pose R-CNN: Recovering 6D Object Pose \\from a Single RGB Image}
\title{\method{}: Recovering 6D Object Pose \\from a Single RGB Image}
% You will get a Paper-ID when submitting a pdf file to the conference system
\author{Thanh-Toan Do, Ming Cai, Trung Pham, Ian Reid\\
The University of Adelaide, Australia}

%\author{\authorblockN{Michael Shell}
%\authorblockA{School of Electrical and\\Computer Engineering\\
%Georgia Institute of Technology\\
%Atlanta, Georgia 30332--0250\\
%Email: mshell@ece.gatech.edu}
%\and
%\authorblockN{Homer Simpson}
%\authorblockA{Twentieth Century Fox\\
%Springfield, USA\\
%Email: homer@thesimpsons.com}
%\and
%\authorblockN{James Kirk\\ and Montgomery Scott}
%\authorblockA{Starfleet Academy\\
%San Francisco, California 96678-2391\\
%Telephone: (800) 555--1212\\
%Fax: (888) 555--1212}}


% avoiding spaces at the end of the author lines is not a problem with
% conference papers because we don't use \thanks or \IEEEmembership


% for over three affiliations, or if they all won't fit within the width
% of the page, use this alternative format:
% 
%\author{\authorblockN{Michael Shell\authorrefmark{1},
%Homer Simpson\authorrefmark{2},
%James Kirk\authorrefmark{3}, 
%Montgomery Scott\authorrefmark{3} and
%Eldon Tyrell\authorrefmark{4}}
%\authorblockA{\authorrefmark{1}School of Electrical and Computer Engineering\\
%Georgia Institute of Technology,
%Atlanta, Georgia 30332--0250\\ Email: mshell@ece.gatech.edu}
%\authorblockA{\authorrefmark{2}Twentieth Century Fox, Springfield, USA\\
%Email: homer@thesimpsons.com}
%\authorblockA{\authorrefmark{3}Starfleet Academy, San Francisco, California 96678-2391\\
%Telephone: (800) 555--1212, Fax: (888) 555--1212}
%\authorblockA{\authorrefmark{4}Tyrell Inc., 123 Replicant Street, Los Angeles, California 90210--4321}}


\maketitle

%\begin{abstract}
%The abstract goes here.
%\end{abstract}
\begin{abstract}
Detecting objects and their 6D poses from only RGB images is an important task for many robotic applications. While deep learning methods have made significant progress in visual object detection and segmentation, the object pose estimation task is still challenging. In this paper, we introduce an end-to-end deep learning framework, named \textit{\method{}}, that jointly detects, segments, and most importantly recovers 6D poses of object instances from a single RGB image. In particular, we extend the recent state-of-the-art instance segmentation network \emph{Mask R-CNN} with a novel pose estimation branch to directly regress 6D object poses without any post-refinements. Our key technical contribution is the decoupling of pose parameters into translation and rotation so that the rotation can be regressed via a Lie algebra representation. The resulting pose regression loss is differential and unconstrained, making the training tractable. The experiments on two standard pose benchmarking datasets show that our proposed approach compares favorably with the state-of-the-art RGB-based multi-stage pose estimation methods. Importantly, due to the end-to-end architecture, \textit{\method{}} is considerably faster than competing multi-stage methods, offers an inference speed of 10 fps that is well suited for robotic applications. %The source code and trained models will be made available. 
 
%Recovering 6D pose and segmenting of objects play an important role in robot manipulation problem. %That problem is, however, challenging due to the object variations and the scene complexity.  
%%Recovering 6D pose of objects is important for the robots to interacts with object in the real world. %That
%In this paper, we propose \textit{\method{}}, an end-to-end deep learning approach for joint 6D pose estimation, recognition, and segmentation of object instances from a single RGB image. 
%%In this paper, we propose \textit{\method{}}, an end-to-end deep learning approach for joint 6D pose estimation, and segmentation of object instances from a single RGB image. 
%Depart from the backbone of the recent instance segmentation network \textit{Mask R-CNN}, we propose a novel head branch for regressing the 6D object pose in parallel %with the existing branches for bounding box recognition and segmentation. 
%with the branches for the bounding box segmentation and recognition. 
%Unlike recent CNN-based techniques for 6D pose estimation which are not end-to-end or require post-refinement steps to refine the predicted coarse poses. The proposed \textit{\method{}} is a single shot approach which is end-to-end trainable and can directly produce accurate poses without any posterior refinement. 
% The experiments on two benchmark datasets show that proposed approach compares favorably with the state-of-the-art RGB-based pose estimation methods. Furthermore, the end-to-end  architecture of \textit{\method{}} allows the inference at the speed of 10 fps, which is faster than competing methods. This makes \textit{\method{}} well suited to robotic applications. The source code and trained models will be made available. 
\end{abstract}
\IEEEpeerreviewmaketitle

%\input{abstract}

\section{Introduction}
Detecting objects and their 6D poses (3D location and orientation) is an important task for many robotic applications including object manipulations (e.g., pick and place), parts assembly, to name a few. In clutter environments, objects must be first detected before their poses can be estimated. Accurate object segmentation is also important, especially when objects are occluded. Our goal is to develop a deep learning framework that jointly detects, segments,  and most importantly recovers 6D poses of object instances from a single RGB image. While the first two tasks are getting more and more mature thanks to the power of deep learning, the 6D pose estimation problem remains a challenging problem.

%We are interested in the problem of estimating 6D pose of object instances from a single RGB image. A 6D pose describes the transformation from the local coordinate system of the object to the camera coordinate system, %in which the origin of the local coordinate system of an object is at the object center.%in which the origin of the object local coordinate system is at the object center.%Object pose estimation has wide range applications in robotics, e.g., robot manipulation, robot learning from demonstration. %%%For example, estimating the orientation and translation from the objects to the robot is important for robot manipulation. It is also useful for human-robot interaction problems such as robot learning from demonstration. %Furthermore, to make robots interact with objects, in addition to estimating the 6D poses of objects, correctly segmenting the objects from the scene is also important, especially when the objects are occluded. Knowing the segmentations allows robot to interact (e.g., grasping, pushing) with the visible parts of objects. Hence, it is necessary to develop an approach which not only can estimate object poses but also can segment objects. %In order to robots interact with object, in addition to estimating the 6D pose of object, correctly segmenting the object is also important, especially when the objects are occluded. Knowing the segmentation allows robot to correctly interact (e.g., grasping, pushing) with the visible parts of objects. Hence it is necessary to develop an approach which not only can estimate object poses but also can segment objects. %For example, estimating the orientation and translation from the objects to the robot is important for robot manipulation.  

Traditional object pose estimation methods are mainly based on the matching of hand-crafted local features (e.g. SIFT~\cite{SIFT_Lowe}). However, these local feature matching based approaches are only suitable for richly textured objects. For poorly textured objects, template-based matching~\cite{ACCV12,DBLP:conf/eccv/TejaniTKK14,DBLP:conf/iccv/Rios-CabreraT13} or dense feature learning approaches are usually used~\cite{ECCV14,ICCV15,CVPR17,CVPR17_2}. However, the template-based methods are usually sensitive to illuminations and occlusions.  
The feature learning approaches~\cite{ECCV14,ICCV15,CVPR17,CVPR17_2} have shown better performances against template-based methods, but they suffer from several disadvantages, i.e., 
they require a time-consuming multi-stage processing for learning dense features, generating coarse pose hypotheses, and refining the coarse poses. 

%In addition, only local information is used when generating pose hypotheses. Furthermore, those works are mainly designed for RGBD inputs, i.e., the depth information is required for both generating and refining pose hypotheses. %Different from previous RGBD-based 6D pose estimation methods~\cite{ECCV14,ICCV15,CVPR17}, in this work, we develop a deep neural network which can recover the 6D pose of object instances in a single RGB image only. Furthermore, instead of relying on pixel-wise dense prediction and pose hypothesis generation approaches~\cite{ECCV14,ICCV15,CVPR17}, our method utilizes on the recent advance region-based CNN~\cite{Fast-RCNN,Faster-RCNN}, in which we directly regress the 6D pose for each object instance in a single forward pass. 

With the rising of deep learning, especially Convolutional Neural Networks (CNN), the object classification~\cite{krizhevsky2012imagenet}, object detection~\cite{Fast-RCNN,Faster-RCNN}, and recently object instance segmentation~\cite{Mask-RCNN,affnet} tasks have achieved remarkable  improvements. However, the application of CNN to 6D object pose estimation problem is still limited. 
%There are few works which use CNN for 6D camera pose estimation problem~\cite{DBLP:conf/iccv/KendallGC15,kendall2017posenet}. The camera pose estimation problem, however, may be easier than the object pose estimation problem. Different from the camera pose estimation task which can consider whole image as an object, the object pose estimation task  requires to detect, segment, and recover the pose for every object instance in the image. 
Recently, there are few works~\cite{SSD-6D,BB8,posecnn} which apply deep learning for 6D object pose estimation. These methods, however, are not end-to-end or only estimate a coarse object poses. They require further post-refinements to improve the accuracy, which linearly increases the running time, w.r.t. the number of detected objects. 

\junk{
The key idea for recent large improvements in object detection and object instance segmentation problems~\cite{Faster-RCNN,Mask-RCNN} is the development of a CNN architecture, i.e., Region Proposal Network (RPN)~\cite{Faster-RCNN}. RPN is actually a CNN which is trained to produce multiple object (bounding boxes) proposals in an image at different shapes and sizes.  
Faster R-CNN~\cite{Faster-RCNN} further refines and classify bounding boxes produced by RPN using additional fully connected layers. %It also simultaneously classifies bounding box labels in a single forward pass. 
The recent work Mask R-CNN~\cite{Mask-RCNN} goes beyond Faster-RCNN, i.e., it performs  binary segmentation in each bounding box produced by RPN. %Despite its simple design, Mask R-CNN achieves state-of-the-art results for the instance segmentation task. %Note that in both Faster R-CNN and Mask R-CNN, the training and testing of the network is in an end-to-end fashion, e.g., in Mask R-CNN, the network simultaneously localizes, classifies, and segments object instances. 
}

Recently, Mask R-CNN~\cite{Mask-RCNN} achieves state-of-the-art results in the instance segmentation problem. The key component of Mask R-CNN is a Region Proposal Network (RPN)~\cite{Faster-RCNN}, which predicts multiple object (bounding boxes) proposals in an image at different shapes and sizes. 
Mask R-CNN further segments instances inside bounding boxes produced from RPN by using additional convolutional layers. 

Inspired by the impressive results of Mask R-CNN for object instance segmentation, 
%Inspired by the impressive results of Faster R-CNN and Mask R-CNN for object detection and object instance segmentation, %which are two key components in a 6D object pose estimation task, 
we are motivated to find the answer for the question that, \textit{can we exploit the merits of RPN to not only segment but also recover the poses of object instances in a single RGB image, in an end-to-end fashion?}
To this end, we design a network which simultaneously detects\footnote{The detection means the prediction of both bounding boxes and class labels.}, segments, and also recovers 6D poses of object instances from a single RGB image. In particular, we propose a \method{} network, which goes beyond Mask R-CNN by adding a novel branch for regressing the poses for the object instances inside bounding boxes produced by RPN. The proposed pose branch is parallel with the detection and segmentation branches. 

Our main contribution is a novel object pose regressor, where the network regresses translation and rotation parameters seperately. Cares must be taken when regressing 3D rotation matrices as not all $3\times3$ matrices are valid rotation matrices. To work around, we resort to the Lie algebra associated with the $SO(3)$ Lie group for our 3D rotation representation. Compared to other representations such as quaternion or orthonormal matrix, Lie algebra is an optimal choice as it is less parammeters and unconstrained, thus making the training process easier. Although the Lie algebra representation has been widely used in geometry-based robot vision problems~\cite{DBLP:conf/iros/Agrawal06,DBLP:conf/icra/RosGSPL13}, to our best knowledge, this is the first work which successfully uses the Lie algebra in a CNN for regressing 6D object poses.

%Different from recent deep learning-based 6D pose estimation methods which are not end-to-end trainable~\cite{BB8} or only predict a rough pose followed by a pose refinement step~\cite{SSD-6D,BB8,posecnn}, the proposed \method{} is a single deep CNN architecture. It takes a RGB image as input and directly outputs 6D object poses without a pose post-refinement. Additionally, our system also returns segmentation masks of object instances. The experimental results show that \method{} is competitive or outperforms similar benchmarks on standard datasets. %Furthermore, \method{} is simple to train and test. Its end-to-end architect allows to process the inference at the speed of $10 fps$, which is comparable or several times faster than the state-of-the methods.

Different from recent deep learning-based 6D pose estimation methods which are not end-to-end trainable~\cite{BB8} or only predict a rough pose followed by a pose refinement step~\cite{SSD-6D,BB8,posecnn}, the proposed \method{} is a single deep learning architecture. It takes a RGB image as input and directly outputs 6D object poses without any pose post-refinements. Additionally, our system also returns segmentation masks of object instances. The experimental results show that \method{} is competitive or outperforms the state-of-the-art methods on standard datasets. Furthermore, \method{} is simple and elegant, allows the inference at the speed of $\textrm{10 fps}$, which is several times faster than many existing methods.

The remainder of this paper is organized as follows. Section~\ref{sec:related} presents related works. Section~\ref{sec:method} details the proposed \method{}. Section~\ref{sec:exp} evaluates and compares \method{} to the state-of-the-art 6D object pose estimation methods. Section~\ref{sec:conl} concludes the paper. 

\section{Related Work}\label{sec:related}
In this section, we first review the 6D object pose estimation methods.  We then brief the main design of the recent methods which are based on RPN for object detection and segmentation.

\textbf{Classical approaches.} The topic of pose estimation has great attention in the past few years. For objects with rich of texture, sparse feature matching approaches have been shown good accuracy~\cite{DBLP:conf/clor/GordonL06,SIFT_Lowe,DBLP:conf/icra/MartinezCS10}. Recently, researchers have put more focus on poor texture or texture-less objects. The most traditional approaches for poor texture objects are to use object templates~\cite{ACCV12,DBLP:conf/eccv/TejaniTKK14,DBLP:conf/iccv/Rios-CabreraT13}. The most notable work belonging to this category is LINEMOD~\cite{ACCV12} which is based on stable gradient and normal features. However, LINEMOD is designed to work with RGBD images. Furthermore, template-based approaches are sensitive to the lighting and occlusion. 

\textbf{Feature learning approach.} Recent 6D pose estimation researches have relied on feature learning for dealing with insufficient texture objects~\cite{ECCV14,ICCV15,CVPR17,CVPR17_2}. In~\cite{ECCV14,ICCV15}, the authors show that the dense feature learning approach outperforms matching approach. The basic  design of \cite{ECCV14,ICCV15,CVPR17,CVPR17_2} is a time-consuming multi-stage pipeline, i.e., a random forest is used for jointly learning the object categories for pixels %(known as object labels) 
and the coordinates of pixels w.r.t. object coordinate systems (known as object coordinates). A set of pose hypotheses is generated by using the outputs of the forest and the depth channel of the input image. %This hypothesis generation stage only uses local information,  i.e., only three or four pixels are used to generate a hypothesis. 
An energy function is defined on the generated pose hypotheses to select hypotheses. The selected pose hypotheses are further refined to obtain the final pose. Note that the pipelines in those works heavily depend on the depth channel. The depth information is required in both pose hypothesis generation and refinement. The work~\cite{CVPR16} also follows a multi-stage approach as~\cite{ECCV14,ICCV15} but is designed to work with RGB inputs. %In~\cite{CVPR16}, the authors use auto-context random forest to improve the prediction of the object labels and the object coordinates. 
In order to deal with the missing depth information, the distribution of object coordinates is approximated as a mixture model when generating pose hypotheses. 

\junk{
The disadvantage of feature learning approaches~\cite{ECCV14,ICCV15,CVPR16,CVPR17_2} is that the generation of pose hypotheses uses only local information, i.e., only three or four pixels are used to generate a hypothesis. As result, this may generate bad hypotheses because it does not consider a global context over the whole object. Furthermore, by requiring multiple processing steps, those approaches are are time-consuming, making them unsuitable for real-time applications. 
}%the learning approaches in those works are time-consuming, making them  unsuitable for real-time applications. %In contrast to the multi-state approach of aforementioned works, we directly recover object pose in a single shot from a single RGB image. In addition, instead of generating pose hypotheses by using only local information, we rely on global information, i.e., whole object, to regress the pose. \textbf{CNN-based approach.} In recent years, CNN has been applied for 6D pose problem, firstly for camera pose~\cite{DBLP:conf/iccv/KendallGC15,kendall2017posenet}, and recently for object pose~\cite{ICCV15,CVPR17_2,SSD-6D,BB8,posecnn,yolo-6D}. 
%In~\cite{DBLP:conf/iccv/KendallGC15}, the authors train a CNN to directly regress 6D camera pose from a single RGB image. The work is then extended by taking into account the geometric loss during training~\cite{kendall2017posenet}. 

In~\cite{DBLP:conf/iccv/KendallGC15,kendall2017posenet}, the authors train CNNs to directly regress 6D camera pose from a single RGB image. The camera pose estimation task is arguably easier than the object pose estimation task, because to estimate object pose, it also requires accurate detection and classification of the object, while these steps are not typically needed for camera pose. 

In~\cite{ICCV15}, the authors use a CNN in their object pose estimation system. However,  %in that work, 
the CNN is only used as a probabilistic model to learn to compare the learned information (produced by a random forest) and the rendered image. The CNN outputs an energy value for selecting pose hypotheses which are further refined. %The selected pose hypotheses are further refined for getting the final pose.  
The work in~\cite{CVPR17_2} improves over~\cite{ICCV15} by using a CNN and reinforcement learning for joint selecting and refining pose hypotheses. 
%In contrast to~\cite{ICCV15} and~\cite{CVPR17_2}, we use CNN as a regressor which directly regresses  object poses from a single RGB input. 

In SSD-6D~\cite{SSD-6D}, the authors extend SSD detection framework~\cite{SSD} to 3D detection and 3D rotation estimation. The authors decompose 3D rotation space into discrete viewpoints and in-plane rotations. They then treat the rotation estimation as a classification problem. %The estimated bounding box and rotations are used together with the known 3D object model to infer the 3D translation. 
However, to get the good results, it is required to manually find an appropriate sampling for the rotation space. Furthermore, the approach SSD-6D does not directly output the translation, i.e., to estimate  the translation, for each object, an offline stage is required to precomputes bounding boxes w.r.t. all possible sampled rotations. This precomputed information is used together with the estimated bounding box and rotation to estimate the 3D translation. 
In the recent technical report~\cite{posecnn}, the authors propose a network, dubbed PoseCNN, which jointly segments objects and estimates the rotation and the distance of segmented objects to camera. However, by relying on a semantic segmentation approach (which is a FCN~\cite{FCN}) to localize objects, it may be difficult for PoseCNN to deal with input image which contains multiple instances of an object. 
Both SSD-6D and PoseCNN also require further pose refinement steps to improve the accuracy. 

In BB8~\cite{BB8}, the authors propose a cascade of multiple CNNs for object pose estimation task. A segmentation network is firstly applied to the input image to localize objects. Another CNN is then used to predict 2D projections of the corners of the 3D bounding boxes around objects. The 6D pose is estimated for the correspondences between the projected 2D coordinates and the 3D ground control points of 
bounding box corners using a PnP algorithm. Finally, a CNN per object is trained to refine the pose. By using multiple separated CNNs, BB8 is not end-to-end and is time-consuming for the inference. %Furthermore, by using a semantic segmentation network to localize objects, it may be difficult for BB8 to deal with input image which contains multiple instances of an object. 
Similar to~\cite{BB8}, in the recent technical report~\cite{yolo-6D}, the authors extend YOLO object detection network~\cite{yolo9000} to predict 2D projections of the corners of the 3D bounding boxes around objects. Given the projected 2D coordinates and the 3D ground control points of bounding box corners, a PnP algorithm is further used to estimate the 6D object pose.  

\textbf{RPN-based detection and segmentation.} One of key components in the recent successful object detection method  Faster R-CNN~\cite{Faster-RCNN} and instance segmentation method Mask R-CNN~\cite{Mask-RCNN} is the Region Proposal Network --- RPN. The core idea of RPN is to dense sampling the whole input image by many overlap bounding boxes at different shapes and sizes. The network is trained to produce multiple object proposals (also known as Region of Interest --- RoI). This design of RPN allows to smoothly search over different scales of feature maps. 
\junk{
For each RoI, a fixed-size small feature map (e.g., $7\times7$) is pooled from the image feature map using the RoIPool layer~\cite{RCNN} or RoIAlign layer~\cite{Mask-RCNN}. These layers work by dividing the RoI into a regular grid and then max-pooling the feature map values in each grid cell. In Faster R-CNN, the outputs of the RoIPool layer are used to refine the RoI coordinates and to classify the RoI label. In Mask-RCNN, the outputs of the RoIAlign layer are used not only for refining and recognizing the RoI but also for segmenting the object inside the RoI. %Although the design of Mask R-CNN is quite simple and straightforward development of Faster R-CNN, it achieves state-of-the-art results on instance segmentation problem. This motivates us to rely on and extend Mask R-CNN for 6D object pose estimation problem. 
}
Faster R-CNN~\cite{Faster-RCNN} further refines and classifies RoIs with additional fully connected layers, %while Mask R-CNN~\cite{Mask-RCNN} goes beyond Fast R-CNN by further segmenting instances inside RoIs with additional convolutional layers. 
while Mask R-CNN~\cite{Mask-RCNN} further improves over Fast R-CNN by segmenting instances inside RoIs with additional convolutional layers. 

In this paper, we go beyond Mask RCNN. In particular, depart from the backbone of Mask R-CNN, we propose a novel head branch which takes RoIs from RPN as inputs to regress the 6D object poses and is parallel with the existing branches. This results a novel end-to-end architecture which is not only detecting, segmenting but also directly recovering the 6D poses of object instances from a single RGB image. 

%Although the design of Faster R-CNN and Mask R-CNN are quite simple, they achieve state-of-the-art results on object detection and instance segmentation. This motivates us to go beyond Mask R-CNN, i.e., we exploit the merit of RPN to recover the 6D poses of object instances in a single RGB image. 

\section{Method}\label{sec:method}%Our goal is to simultaneously detect the 2D positions, class labels, segmentation masks, and the 6D poses of object instances in the input image. 
Our goal is to simultaneously detect, segment, and estimate the 6D poses of object instances in the input image. 
Mask R-CNN performs well for the first two tasks, except the 6D pose estimation. In order to achieve a complete system, we propose a novel branch which takes RoIs from RPN as inputs and outputs the 6D poses of the instances inside the RoIs. %\method{} is thus conceptually simple. 
Although the concept is simple, the additional 6D pose is distinct from the other branches. It requires an effective way to represent the 6D pose and a careful design of the loss function. 
In this paper, we represent a pose by a 4-dimensional vector, in which the first three elements represent the Lie algebra associated with the rotation matrix of the pose; the last element represents the $z$ component of the translation vector of the pose. Given the predicted $z$ component and the predicted bounding box from the box regression branch, we use projective property to recover the full translation vector. The architecture of \method{} is shown in Figure~\ref{fig:overview}.

\subsection{\method{}}\begin{figure*}[!t] 
\centering   				
\includegraphics[scale=0.4]{figure-1_edit.pdf}    
    \caption{An overview of \method{} framework. {From left to right:} The input to \method{} is a RGB image. A deep CNN backbone (i.e., VGG) is used to extract features over the whole image. The RPN is attached on the last convolutional layer of VGG (i.e., $conv5\_3$) and outputs RoIs. For each RoI, the corresponding features from the feature map $conv5\_3$ are extracted and pooled into a fixed size $7\times 7$. The pooled features are used as inputs for $4$ head branches. For the box regression and classification heads, we follow Mask-RCNN~\cite{Mask-RCNN}. The segmentation head is \textit{adapted} from~\cite{Mask-RCNN}, i.e., four $3\times3$ consecutive convolutional layers (denoted as `$\times4$') are used. The ReLu is used after each convolutional layer. A deconvolutional layer is used to upsample the feature map to $28\times28$ which is the segmentation mask. The proposed pose head consists of four fully connected layers. The ReLu is used after each of the first three fully connected layers. The last fully connected layer outputs four numbers which represent for the pose. As shown on the right image, the network outputs the detected instances (with classes, i.e., Shampoo), the predicted segmentation masks (different object instances are shown with different colors) and the predicted 6D poses for detected instances (shown with 3D boxes). %For clear visualization, we do not show the detected bounding boxes.
}
    \label{fig:overview} 
\end{figure*}

Let us first briefly recap Mask R-CNN~\cite{Mask-RCNN}. Mask R-CNN consists of two main components. The first component is a RPN~\cite{Faster-RCNN} which produces candidate RoIs. The second component extracts features from each candidate RoI using RoIAlign layer~\cite{Mask-RCNN} and performs the classification, the bounding box regression, and the segmentation. We refer readers to~\cite{Mask-RCNN} for details of Mask-RCNN. 

\method{} also consists of two main components. The first component is also a RPN. In the second component, in \textit{parallel} to the existing branches of Mask R-CNN, \method{} also outputs a 6D pose for the objects inside RoIs. 

\paragraph{Pose representation}
An important task when designing the pose branch is the representation space of the output poses. We learn the translation in the Euclidean space. In stead of predicting full translation vector, our network is trained to regress the $z$ component only. The reason is that when projecting a 3D object model into a 2D image, two translation vectors with the same $z$ and the different $x$ and $y$ components may produce two objects which have very similar appearance and scale in 2D image (at different positions in the image -- in the extreme case of parallel projection, there is no difference at all). This causes difficulty for the network to predict the $x$ and $y$ components by using only appearance information as input. However, the object size and the scale of its textures in a 2D image provide strong cues about the $z$-coordinate. 
%, for example, a translation with large $z$ will produce a small object, while a translation with a small $z$ will produce a large object in 2D image. 
This projective property allows the network to learn the $z$ component of the translation using the 2D object appearance only. Given the $z$ component, it is used together with predicted bounding box, which is outputted by the bounding box regression branch, to fully recover the translation. 
%we can use predicted bounding box, which is outputted by the bounding box regression branch, to fully recover the translation. 
The detail of this recovering process is presented in the following sections.  

Representing the rotation part of the pose is more complicated than the translation part. Euler angles are intuitive due to the explicit meaning of parameters. %However, they are not injective, causing them difficult to learn as a uni-modal scalar regression task. 
However, the Euler angles wrap around at 2$\pi$ radians, i.e., having multiple values representing the same angle. This causes difficulty in learning a uni-modal scalar regression task.  
Furthermore, the Euler angles-based representation suffers from the well-studied problem of gimbal lock~\cite{pose}. Another alternative, the use of $3\times3$ orthonormal matrix is over-parametrised, and creates the problem of enforcing the orthogonality constraint when training the network through back-propagation. A final common representation is the unit length 4-dimensional quaternion. 
One of the downsides of quaternion representation is its norm should be unit. This constraint may harm the optimization~\cite{DBLP:conf/iccv/KendallGC15}. 
 
In this work, we use the Lie algebra $so(3)$ associated with the Lie group $SO(3)$ (which is space of 3D rotation matrices) as our rotation representation. 
The Lie algebra $so(3)$  is known as the tangent space at the identity element of the Lie group $SO(3)$. 
We choose the Lie algebra $so(3)$ to represent the rotation because an arbitrary element of $so(3)$ admits a skew-symmetric matrix representation parameterized by a vector in $\R^3$ which is continuous and smooth.
%The Lie algebra $so(3)$ which is the space of skew-symmetric matrices is preferred because a skew-symmetric matrix can be parameterized as a vector in $\R^3$ which is continuous and smooth. 
This means that the network needs to regress only three scalar numbers for a rotation, without any constraints. 
To our best knowledge, this paper is the first one which uses Lie algebra for representing rotations in training a deep network for 6D object pose estimation task. 

During training, we map the groundtruths of rotation matrices to their associated elements in $so(3)$ by the closed form Rodrigues logarithm mapping~\cite{log-exp-map}. %\red{The detail of the logarithm mapping is presented in the supplementary material.} %\begin{equation}%logarith map%\end{equation}
The mapped values are used as regression targets when learning to predict the rotation. 

In summary, the pose branch is trained to regress a 4-dimensional vector, in which the first three elements represent rotation part and the last element represents the $z$ component of the  translation part of the pose. 

\paragraph{Multi-task loss function}
In order to train the network, we define a multi-task loss to jointly train the bounding
box class, the bounding box position, the segmentation, and the pose of the object inside the box. Formally, the loss function is defined as follows
\begin{equation}
 L=\alpha_1L_{cls} + \alpha_2L_{box} + \alpha_3L_{mask} + \alpha_4L_{pose} 
 \label{eq:allloss}
\end{equation}
  The classification loss $L_{cls}$, the bounding box regression loss $L_{box}$, and the segmentation loss $L_{mask}$ are defined similar as~\cite{Mask-RCNN}, which are \textit{softmax} loss, \textit{smooth L1} loss, and \textit{binary cross entropy} loss, respectively. The $\alpha_1, \alpha_2, \alpha_3, \alpha_4$  coefficients are scale factors to control the important of each loss during training. 

The pose branch outputs 4 numbers for each RoI, which represents the Lie algebra for the rotation and $z$ component of the translation. It is worth noting that in our design, the output of pose branch is class-agnostic, %\footnote{\red{We also tested the architecture of the pose branch with the class-specific design (i.e., with a $4C$-dimensional output vector, where $C$ is the number of classes), but it only gives minor improvements, while increasing the inference time.}}.  
but the class-specific counterpart (i.e., with a $4C$-dimensional output vector in which $C$ is the number of classes) is also applicable. 
% \iancomment{Could do with elaboration -- is it really the case that you can regress the pose accurately without the class info? What if instead of 4C outputs you had a class conditioned output?}
The pose regression loss $L_{pose}$ is defined as follows
\begin{equation}
L_{pose} = \norm{r-\hat{r}}_p + \beta\norm{t_z - \hat{t}_z}_p
\label{eq:poseloss}
\end{equation}
where $r$ and $\hat{r}$ are two 3-dimensional vectors representing the regressed rotation and the groundtruth rotation, respectively; $t_z$ and $\hat{t}_z$ are two scalars representing the regressed $z$ and the groundtruth $z$ of the translation; $p$ is a distance norm; $\beta$ is a scale factor to control the rotation and translation regression errors. %\iancomment{What does the euclcidean loss applied to the lie algebra mean? Is there an argument for a dot-product or similar?}\paragraph{Network architecture}
Figure~\ref{fig:overview} shows the schematic overview of \method{}. We differentiate two parts of the network, i.e., the backbone and the head branches. The backbone is used to extract features over the whole image and is shared between head branches. There are four head branches corresponding to the four different tasks, i.e., the bounding box regression, the bounding box classification, the segmentation, and the 6D pose estimation for the object inside the box. 
For the backbone, we follow Faster R-CNN~\cite{Faster-RCNN} which uses VGG~\cite{SimonyanZ14} together with a RPN attached on the last convolutional layer of VGG (i.e., $conv5\_3$). For each output RoI of RPN, a fixed-size $7\times7$ feature map is pooled from the $conv5\_3$ feature map using the RoIAlign layer~\cite{Mask-RCNN}. This pooled feature map is used as input for head branches. For the network heads of the bounding box regression and classification, we closely follow the Mask R-CNN~\cite{Mask-RCNN}. 
For the segmentation head, we adapt from Mask R-CNN. In our design, four $3\times3$ consecutive convolutional layers (denoted as `$\times4$' in Figure~\ref{fig:overview}) are used after the pooled feature map. The ReLu is used after each convolutional layer. A deconvolutional layer is used to upsample the feature map to $28\times28$ which is the segmentation mask. It is worth noting that for segmentation head, we use the class-agnostic design, i.e., this branch outputs a single mask, regardless of class. We empirically found that this design reduces the model complexity and the inference time, while it is nearly effective as the class-specific design. This observation is consistent with the observation in Mask R-CNN~\cite{Mask-RCNN}. 
%As noted in Mask R-CNN~\cite{Mask-RCNN}, this design reduces the model complexity and the inference time, while it is nearly effective as the class-specific design. We refer readers to~\cite{Mask-RCNN} for further details on those heads. 

In order to adapt the shared features to the specific pose estimation task, the pose head branch consists of a sequence of 4 fully connected layers in which the number of outputs are $4096 \to 4096 \to 384 \to 4$.  The ReLU is used after each fully layer, except for the last layer. This is because the regressing targets (i.e., the groundtruths) contain both negative and positive values. We note that our pose head has a simple structure. More complex design may have potential improvements. 

\subsection{Training and inference}\paragraph{Training}
We implement \method{} using Caffe deep learning library~\cite{DBLP:conf/mm/JiaSDKLGGD14}. The input to our network is a RGB image with the size $480 \times 640$. 
The RPN outputs RoIs at different sizes and shapes. We use 5 scales and 3 aspect ratios, resulting 15 anchors in the RPN. The 5 scales are $16 \times 16$, $32\times32$, $64 \times 64$, $128\times128$ and $256 \times 256$; the 3 aspect ratios are $2:1$, $1:1$, $1:2$. This design allows the network to detect small objects. 

The $\alpha_1, \alpha_2, \alpha_3$, and $\alpha_4$ in (\ref{eq:allloss}) are empirically set to 1, 1, 2, 2, respectively. The values of $\beta$ in (\ref{eq:poseloss}) is empirically set to 1.5. 
An important choice for the pose loss (\ref{eq:poseloss}) is the regression norm $p$. Typically, deep learning models use $p=1$ or $p=2$. With the datasets used in this work, we found that $p=1$ give better results and hence is used in our experiments. 

We train the network in an end-to-end manner using stochastic gradient descent with $0.9$ momentum and $0.0005$ weight decay. The network is trained on a Titan X GPU for $350k$ iterations. Each mini batch has $1$ image. %\iancomment{How many objects per image on average?} 
The learning rate is set to $0.001$ for the first $150k$ iterations and then decreased by 10 for the remaining iterations. The top $2000$ RoIs from RPN (with a ratio of 1:3 of positive to negative) are subsequently used for computing the multi-task loss. A RoI is considered positive if it has an intersection over union (IoU) with a groundtruth box of at least 0.5 and negative otherwise. The losses $L_{mask}$ and $L_{pose}$ are defined for only positive RoIs. %The source code and trained models will be made available. \paragraph{Inference}
At the test phase, we run a forward pass on the input image.  The top $1,000$ RoIs produced by the RPN are selected and fed into the box regression and classification branches, followed by non-maximum suppression~\cite{Fast-RCNN}. Based on the outputs of the classification branch, we select the outputted boxes from the regression branch that have classification scores higher than a certain threshold (i.e., $0.9$) as the detection results. The segmentation branch and the pose branch are then applied on the detected boxes, which output segmentation masks and the 6D poses for the objects inside the boxes. 

\paragraph{From the 4-dimensional regressed pose to the full 6D pose}
Given the predicted Lie algebra, i.e., the first three elements of the predicted 4-dimensional vector from pose branch, we use the exponential Rodrigues mapping~\cite{log-exp-map} to map it to the corresponding rotation matrix.  
%\red{The detail of the exponential mapping is presented in the supplementary material.}%\begin{equation}%exponential map%\end{equation}
In order to recover the full translation, we rely on the predicted $z$ component ($t_z$ -- the last element of the 4-dimensional predicted vector) and the predicted bounding box coordinates to compute two missing components $t_x$ and $t_y$. We assume that the bounding box center (in 2D image) is the projected point of the 3D object center (the origin of the object coordinate system). Under this assumption, using the 3D-2D projection formulation, we compute $t_x$ and $t_y$ as follows
\begin{equation}
t_x = \frac{(u_0 - c_x)t_z}{f_x}
\end{equation}\begin{equation}
t_y = \frac{(v_0 - c_x)t_z}{f_y}
\end{equation}
where $u_0$, $v_0$ are the bounding box center in 2D image, and the matrix $[f_x, 0, c_x; 0, f_y, c_y; 0, 0, 1]$ is the known intrinsic camera calibration matrix.


\section{Experiments}\label{sec:exp}
We evaluate \method{} on two widely used datasets, i.e., the single object pose dataset LINEMOD provided by Hinterstoisser et al.~\cite{ACCV12} and the multiple object instance pose dataset provided by Tejani et al.~\cite{DBLP:conf/eccv/TejaniTKK14}.  
%Firstly, we evaluate our system on the single object pose estimation dataset LINEMOD provided by Hinterstoisser et al.~\cite{ACCV12}. Then we evaluate our system on the dataset with multiple object instances provided Tejani et al.~\cite{DBLP:conf/eccv/TejaniTKK14}. 
We also compare \method{} to the state-of-the-art methods for 6D object pose estimation from RGB images~\cite{CVPR16,BB8,SSD-6D}.

\junk{When evaluating on the dataset of Hinterstoisser et al.~\cite{ACCV12}, we mainly target to compare our work with the recent state-of-the-art 6D pose estimation method from Brachmann et al.~\cite{CVPR16} because that work also estimates 6D pose from a single RGB input~\cite{CVPR16}. For reference purposes, we also mention the results of LINE2D~\cite{LINE2D}, which is a template-based approach\footnote{LINE2D~\cite{LINE2D} is originally proposed for object detection. It is then extended for 6D pose estimation by~\cite{CVPR16}.}. It is because the method~\cite{CVPR16} has already improves over LINE2D~\cite{LINE2D}.}\textbf{Metric:}
Different 6D pose measures have been proposed in the past. In order to evaluate the recovered poses, we use the \textit{standard} metrics used in~\cite{CVPR16,BB8}. To measure pose error in 2D, we project the 3D object model into the image using the groundtruth pose and the estimated pose. The estimated pose is accepted if the IoU between two project boxes is higher than 0.5. This metric is called as \textit{2D-pose} metric. To measure the pose error in 3D, the $5cm5^\circ$ and \textit{$\textrm{ADD}$} metrics is used. 
In $5cm5^\circ$ metric,  an estimated pose is accepted if it is within $5cm$ translational error and $5^\circ$ angular error of the ground truth pose.
In $\textrm{ADD}$ metric, an estimated pose is accepted if the average distance between transformed model point clouds by the groundtruth pose and the estimated pose is smaller than $10\%$ of the object's diameter. We also provide the $F1$ score of the detection and segmentation results. A detection / segmentation is accepted if its IoU with the groundtruth box / segmentation mask is higher than a threshold. We report results with the widely used thresholds, i.e., $0.5$ and $0.9$~\cite{Faster-RCNN,Mask-RCNN}.
\subsection{Single object pose estimation}\label{exp:hoi}\begin{table*}[!t]
\vspace{0.5cm}
   \centering
   \footnotesize
    \begin{center}
    \begin{tabular}{c| c c c c c c c c c c c c c c} 
&Ape &Bvise &Cam &Can &Cat &Driller &Duck &Box &Glue &Holep &Iron &Lamp &Phone &Average\\  
\Xhline{0.75pt}
	&\multicolumn{14}{c}{\textbf{IoU 0.5}} \\ 
Detection &99.8 &100 &99.7 &100 &99.5 &100 &99.8 &99.5 &99.2 &99.0 &100 &99.8 &100 &99.7\\ 
Segmentation &99.5 &99.8 &99.7 &100 &99.1 &100 &99.4 &99.5 &99.0 &98.6 &99.2 &99.4 &99.7 &99.4
\\  \hline
%Detection (\method{})    &96.8 &97.0 &96.7 &97.2 &96.5 &96.8 &96.8 &95.5 &96.2 &96.0 &97.0 &96.8 &97.3 &00.0\\ 
%Segmentation(\method{}) &96.5 &96.8 &95.7 &95.1 &94.2 &95.0 &94.4 &94.2 &95.0 &95.6 &96.2 &95.4 &95.7 &00.0\\ \hline
%Detection (SSD-6D\cite{SSD-6D})    &76.3 &97.1 &92.2 &93.1 &89.3 &97.8 &80.0 &93.6 &76.3 &71.6 &98.2 &93.0 &92.4 &00.0 \\ \hline
%Detection (Kehl\cite{Wadim_eccv16}) &98.1 &94.8 &93.4 &82.6 &98.1 &96.5 &97.9 &100 &74.1 &97.9 &91.0 &98.2 &84.9 &00.0 \\ \hline

	&\multicolumn{14}{c}{\textbf{IoU 0.9}} \\
%Detection 	 &71.7 &80.0 &59.3 &79.2 &82.8 &73.4 &68.5 &84.2 &39.1 &64.7 &83.0 &79.3 &59.2 &71.1\\ 
%Segmentation &70.0 &63.3 &58.8 &68.3 &69.7 &67.8 &65.7 &83.9 &37.4 &62.1 &82.7 &76.0 &56.9 &66.3\\ \\ \hline
%Detection 	 &80.4 &85.2 &88.7 &89.6 &89.3 &83.5 &83.5 &90.2 &80.1 &91.2 &91.5 &88.3 &88.6 &00.0\\ 
%Segmentation &77.6 &57.0 &84.5 &62.3 &52.0 &72.6 &78.2 &86.9 &72.5 &83.6 &88.3 &82.0 &81.8 &00.0\\ \hline
Detection 	 &85.4 &91.7 &93.3 &93.6 &89.3 &87.5 &86.3 &94.2 &81.1 &93.2 &92.5 &91.3 &90.8 &90.0\\ 
Segmentation &80.6 &57.0 &91.4 &62.5 &52.1 &74.6 &81.2 &91.9 &73.3 &84.6 &90.3 &85.0 &84.6 &77.6\\ \hline
    \end{tabular}
    \end{center}
    \vspace{-0.2cm}
    \caption{F1 score for 2D detection and segmentation of \method{} on LINEMOD dataset~\cite{ACCV12} for single object.} 
    \label{tab:2D_det_seg}
\end{table*}\begin{table*}[!t]
   \centering
   \footnotesize
   \begin{center}
    \begin{tabular}{c| c c c c c c c c c c c c c c} 
&Ape &Bvise &Cam &Can &Cat &Driller &Duck &Box &Glue &Holep &Iron &Lamp &Phone &Average\\  
\Xhline{1pt}
	&\multicolumn{14}{c}{\textbf{2D-pose metric}}\\ 
\method{} &99.8 &100 &99.7 &100 &99.2 &100 &99.8 &99.0 &97.1 &98.0 &99.7 &99.8 &99.1 &99.3\\ 
Brachmann\cite{CVPR16} &98.2 &97.9 &96.9 &97.9 &98.0 &98.6 &97.4 &98.4 &96.6 &95.2 &99.2 &97.1 &96.0 &97.5\\
SSD-6D\cite{SSD-6D} &- &- &- &- &- &- &- &- &- &- &- &- &- &99.4\\

%LINE2D\cite{LINE2D} &- &- &- &- &- &- &- &- &- &- &- &- &- &86.5\\
   	\hline
   	   	   	&\multicolumn{14}{c}{$\mathbf{5cm5^\circ}$ \textbf{metric}} \\
\method{} &57.8 &72.9 &75.6 &70.1 &70.3 &72.9 &67.1 &68.4 &64.6 &70.4 &60.7 &70.9 &69.7 &68.5\\ 
Brachmann\cite{CVPR16} &34.4 &40.6 &30.5 &48.4 &34.6 &54.5 &22.0 &57.1 &23.6 &47.3 &58.7 &49.3 &26.8 &40.6\\
BB8\cite{BB8} &80.2 &81.5 &60.0 &76.8 &79.9 &69.6 &53.2 &81.3 &54.0 &73.1 &61.1 &67.5 &58.6 &69.0\\
   	\hline   	
   	   	&\multicolumn{14}{c}{$\mathbf{ADD}$ \textbf{metric}} \\
\method{} &38.8 &71.2 &52.5 &86.1 &66.2 &82.3 &32.5 &79.4 &63.7 &56.4 &65.1 &89.4 &65.0 &65.2\\ 
Brachmann\cite{CVPR16}  &33.2 &64.8 &38.4 &62.9 &42.7 &61.9 &30.2 &49.9 &31.2 &52.8 &80.0 &67.0 &38.1 &50.2\\
BB8\cite{BB8} &40.4 &91.8 &55.7 &64.1 &62.6 &74.4 &44.3 &57.8 &41.2 &67.2 &84.7 &76.5 &54.0 &62.7\\
SSD-6D\cite{SSD-6D} &- &- &- &- &- &- &- &- &- &- &- &- &- &76.3\\
   	\hline   	
    \end{tabular}
    \end{center}
   \vspace{-0.2cm}
    \caption{Pose estimation accuracy on the LINEMOD dataset~\cite{ACCV12} for single object.}
    \label{tab:pose} 
\end{table*}\begin{figure*}[!t]
%vspace{-0.2cm}
\vspace{0.1cm}
\centering
\subfigure{
       \includegraphics[scale=0.17]{Hinter/05color00013.jpg}
}
\subfigure{
       %\includegraphics[scale=0.17]{Hinter/05color00013_m.png} 
       \includegraphics[scale=0.17]{Hinter/05color00013_m.jpg} 
}
\subfigure{
%       \includegraphics[scale=0.17]{Hinter/05color00013_p2.png} 
       \includegraphics[scale=0.17]{Hinter/05color00013_p2.jpg} 
}
\\
%\vspace{-0.2cm}
\vspace{0.1cm}
\subfigure{
       \includegraphics[scale=0.17]{Hinter/06color00050.jpg}
}
\subfigure{
       %\includegraphics[scale=0.17]{Hinter/06color00050_m.png} 
       \includegraphics[scale=0.17]{Hinter/06color00050_m.jpg} 
}
\subfigure{
%       \includegraphics[scale=0.17]{Hinter/06color00050_p2.png} 
       \includegraphics[scale=0.17]{Hinter/06color00050_p2.jpg} 
}
\\
%\vspace{-0.2cm}
\vspace{0.1cm}
\subfigure{
       \includegraphics[scale=0.17]{Hinter/08color00002.jpg} 
}
\subfigure{
%       \includegraphics[scale=0.17]{Hinter/08color00002_m.png} 
       \includegraphics[scale=0.17]{Hinter/08color00002_m.jpg} 
}
\subfigure{
%       \includegraphics[scale=0.17]{Hinter/08color00002_p2.png} 
       \includegraphics[scale=0.17]{Hinter/08color00002_p2.jpg} 
}
\caption[]{Qualitative results for single object pose estimation on the LINEMOD dataset~\cite{ACCV12}. From left to right: (i) original images, (ii) the predicted 2D bounding boxes, classes, and segmentations, (iii) 6D poses in which the green boxes are the groundtruth poses and the red boxes are the predicted poses. Best view in color.}
\label{fig:singleobj}
\vspace{-0.3cm}
\end{figure*}
In~\cite{ACCV12}, the authors publish LINEMOD, a RGBD dataset,  which has become a \textit{de facto} standard benchmark for 6D pose estimation. The dataset contains poorly textured objects in a cluttered scene. We only use the RGB images to evaluate our method. The dataset contains 15 object sequences. For fair comparison to~\cite{CVPR16,BB8,SSD-6D}, we evaluate our method on the $13$ object sequences for which the 3D models are available. 
The images in each object sequence contain multiple objects, however, only one object is annotated with the groundtruth class label, bounding box, and 6D pose. The camera intrinsic matrix is also provided with the dataset. Using the given groundtruth 6D poses, the object models, and the camera matrix, we are able to compute the groundtruth segmentation mask for the annotated objects. 
%Each object sequence (set of images has the same class label) contains about $1,200$ images. 
We follow the evaluation protocol in~\cite{CVPR16,BB8} which uses RGB images from the object sequences for training and testing. %For each object sequence, we randomly split $50\%$ of the images for training and validation. The remaining images serve as the test set. 
For each object sequence, %we randomly select $15\%$ of the images for training. We also further randomly select $15\%$ of the images for validation. The remaining images serve as the test set. %%\red{we randomly select $30\%$ of the images for training and validation (i.e., a half for training and a half for validation). The remaining images serve as the test set.}  \red{we randomly select $30\%$ of the images for training and validation. The remaining images serve as the test set.}   
A complication when using this dataset for training arises because  
%not most of the objects are not annotated in most images (only one per sequence, even though multiple objects are present).%
not all objects in each image are annotated, i.e., only one object is annotated per sequence, even though multiple objects are present. This is problematic for training a detection/segmentation network such as~\cite{Faster-RCNN,Mask-RCNN} because the training may be confused, e.g. slow or fail to converge, if an object is annotated as foreground in some images and as background other images. 
Hence, we preprocess the training images as follows. For each object sequence, we use the RefineNet~\cite{Lin:2017:RefineNet}, a state-of-the-art semantic segmentation algorithm, to train a semantic segmentation model. The trained model is applied on all training images in other sequences. The predicted masks in other sequences are then filtered out, so that the appearance of the objects without annotated information does not hinder the training. 

\junk{
\paragraph{Evaluation protocol} 
In the evaluation protocol of~\cite{CVPR16}, the authors assume that for each testing image, the object sequence which the testing image belongs to is known. Hence they can select the learned features corresponding to the known object class for computing the pose. It helps to improve the accuracy. 
We do not require the object sequence to be known in our evaluation. From the detection results which have already filtered by a global threshold, i.e. 0.9, for each object class, we keep only one detection with the highest classification score. This allows us to access the 2D detection and segmentation performance. We count a 2D detection / segmentation to be correct if its IoU with the groundtruth box / segmentation mask is higher than a threshold. We report the detection and segmentation results with IoU thresholds 0.5 and 0.9. In order to evaluate the recovered poses, we follow the metrics used in~\cite{CVPR16}. To measure pose error in 2D, we project the 3D object model into the image using the groundtruth pose and the estimated pose. The estimated pose is accepted if the IoU between two project boxes is higher than 0.5. This metric is called as \textit{2D pose}. To measure the pose error in 3D, the $5cm5^\circ$ metric is used. An estimated pose is accepted if it is within $5cm$ translational error and $5^\circ$ angular error of the ground truth pose. \junk{The angular distance between two rotation matrices $P$ and $Q$ is computed as follows 
\begin{equation}
\theta = \arccos \frac{tr(R)-1}{2}
\end{equation}
where $R=PQ^T$.}
}\textbf{Results:}
Table~\ref{tab:2D_det_seg} presents the 2D detection and segmentation results of \method{}. At an IoU 0.5, the results show that both detection and segmentation achieve nearly perfect scores for all object categories. This reconfirms the effective design of Faster R-CNN~\cite{Faster-RCNN} and Mask R-CNN~\cite{Mask-RCNN} for object detection and instance segmentation. 
%We note that there is a very small gap between the detection and segmentation. %This indicates that the network is able to close the gap between the object detection task and the more challenging instance segmentation task. 
When increasing the IoU to 0.9, both detection and segmentation accuracy  significantly decrease. The more decreasing is observed for the segmentation, i.e., the dropping around \red{10\%} and \red{22\%} for detection and segmentation, respectively. %We find the most loss comes from small objects, especially the Glue sequence. 

We put our interest on the pose estimation results, which is the main focus of this work. 
Table~\ref{tab:pose} presents the comparative pose estimation accuracy between \method{} and the state-of-the-art works of Brachmann et al.~\cite{CVPR16}, BB8~\cite{BB8}, SSD-6D~\cite{SSD-6D} which also use RGB images as inputs to predict the poses. Under \textit{2D-pose} metric, \method{} is comparable to SSD-6D, while outperforms over \cite{CVPR16} around \red{2\%}.
% although their method have already achieved an impressive accuracy, \method{} still improves their performance by around {2\%}. 
Under $5cm5^\circ$ metric, \method{} is slightly lower than BB8, while it significantly outperforms~\cite{CVPR16}, i.e., around \red{28\%}. Under $\textrm{ADD}$ metric, \method{} outperforms BB8 \red{2.5\%}. %Furthermore, we found that the $5cm5^\circ$ metric is less sensitive than $\textrm{ADD}$ metric, e.g., the standard deviations of the accuracy of \method{} are  4.7 and 16.5 for $5cm5^\circ$ and $\textrm{ADD}$, respectively. This is understandable because the $\textrm{ADD}$ metric strongly depends on object's diameters which are vary for LINEMOD dataset. 
The results of \method{} are also more stable than BB8~\cite{BB8}, e.g., under $5cm5^\circ$ metric, the standard deviations in the accuracy of \method{} and BB8~\cite{BB8} are \red{5.0} and \red{10.6}, respectively. 
The Table~\ref{tab:pose} also shows that both \method{} and BB8~\cite{BB8} are worse than SSD-6D. However, it is worth noting that SSD-6D~\cite{SSD-6D} does not use images from the object sequences for training. 
The authors~\cite{SSD-6D} perform a discrete sampling over \textit{whole} rotation space and use the known 3D object models to generate synthetic images used for training. By this way, the training data of SSD-6D is able to cover more rotation space than~\cite{CVPR16}, BB8~\cite{BB8}, and~\method{}. 
Furthermore, SSD-6D also further uses an ICP-based refinement to improve the accuracy. Different from SSD-6D, \method{} directly outputs the pose without any post-processing. 
Figure~\ref{fig:singleobj} shows some qualitative results of \method{} for single object pose estimation on the LINEMOD dataset. %More qualitative results are provided in the supplementary material. %The authors~\cite{SSD-6D} use the given object models to discrete sampling whole rotation space and use the generated synthetic images for training. By this way, the training data of SSD-6D is able to cover more rotation space than~\cite{CVPR16}, BB8~\cite{BB8}, and~\method{}. %We find that the improvements of \method{} are mainly at small objects such as Glue, Duck, Cat. On the other hand,~\cite{CVPR16} outperforms \method{} at some larger objects such as Iron, Hole-puncher, Driller. \subsection{Timing}
When testing on the LINEMODE dataset~\cite{ACCV12}, Brachman {{et al.}}~\cite{CVPR16} reports a running time around $0.45s$ per image. \method{} is several times faster than their method, i.e., its end-to-end architecture allows the inference at around $0.1s$ per image on a Titan X GPU. SSD-6D~\cite{SSD-6D} and BB8~\cite{BB8} report the running time around $0.1s$ and $0.3s$ per image, respectively. \method{} is comparable to SSD-6D in the inference speed, while it is around 3 times faster than BB8. It is worth noting that due to using post-refinement, the inference time of SSD-6D and BB8 may be increased when the input image contains multiple object instances. We note that although \method{} is fast, its parameters are still not optimized for the speed.  %For most of the parameters of the backbone, we follow Mask R-CNN~\cite{Mask-RCNN}.
 As shown in the recent study~\cite{DBLP:journals/corr/HuangRSZKFFWSG016}, better trade-off between speed and accuracy may be achieved by carefully selecting parameters, e.g., varying the number of proposals after RPN, image sizes, which is beyond scope of the current work. 





\bibliographystyle{plainnat}
\bibliography{ref}
\end{document}


