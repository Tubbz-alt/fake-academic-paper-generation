\documentclass[a4paper]{article}

\usepackage{INTERSPEECH2018}
\usepackage{amsmath,graphicx,color}
\usepackage{array}
\usepackage{longtable}
\usepackage{cite,amssymb}
%\usepackage[font=scriptsize]{caption}
\usepackage{subcaption}
\usepackage{authblk}
\usepackage{url}


\newcommand*{\RR}[1]{\textcolor{purple}{#1}}
\newcommand*{\JE}[1]{\textcolor{magenta}{#1}}
\newcommand*{\SL}[1]{\textcolor{red}{#1}}

\title{Transfer Learning for Improving Speech Emotion Classification Accuracy}
\name{Siddique Latif$^{1,3}$, Rajib Rana$^2$, Shahzad Younis$^3$, Junaid Qadir$^1$, Julien Epps$^4$}
%The maximum number of authors in the author list is twenty. If the number of contributing authors is more than twenty, they should be listed in a footnote or in acknowledgement section, as appropriate.
\address{
  $^1$Information Technology University (ITU)-Punjab, Pakistan\\
  $^2$University of Southern Queensland, Australia\\
$^3$National University of Sciences and Technology (NUST), Pakistan\\
$^4$The University of New South Wales, Sydney, Australia}
\email{siddique.latif@itu.edu.pk, rajib.rana@usq.edu.au, muhammad.shahzad@seecs.edu.pk, junaid.qadir@itu.edu.pk, j.epps@unsw.edu.au}

\begin{document}

\maketitle
% 
\begin{abstract}
 The majority of existing speech emotion recognition research focuses on automatic emotion detection using training and testing data from same corpus collected under the same conditions. The performance of such systems has been shown to drop significantly in cross-corpus and cross-language scenarios. To address the problem, this paper exploits a transfer learning technique to improve the performance of speech emotion recognition systems that is novel in cross-language and cross-corpus scenarios. Evaluations on five different corpora in three different languages show that Deep Belief Networks (DBNs) offer better accuracy than previous approaches on cross-corpus emotion recognition, relative to a Sparse Autoencoder and SVM baseline system. Results also suggest that using a large number of languages for training and using a small fraction of the target data in training can significantly boost accuracy compared with baseline also for the corpus with limited training examples. 
\end{abstract}
\noindent\textbf{Index Terms}: cross-corpus, speech, emotion recognition, Deep Belief Networks

\section{Introduction}
\label{sec:intro}


In recent years, speech emotion recognition has received increasing interest. Speech emotion recognition focuses on using linguistic and acoustic attributes as input features and machine learning models as classifiers to classify the emotions of the speaker \cite{batliner2011automatic}. These systems achieve promising results when training and testing are performed from the same corpus. However, for real applications, such systems have been demonstrated not to perform well when speech utterances from different languages and different age groups, in quite different conditions, are combined \cite{schuller2012synthesized}. 
%\JE{reference missing} 

At present, various emotional corpora exist, but they are dissimilar in terms of the spoken language, type of emotion (i.e., naturalistic, elicited, or acted) and labelling scheme (i.e., dimensional or categorical) \cite{schuller2010cross}. There are more than 5,000 spoken languages around the world, but only 389 languages account for 94\% of the world's population\footnote{\url{https://www.ethnologue.com/statistics}}. Even for 389 languages, very few adequate resources (speech corpus) are available for language and speech processing research. This means that research in language and speech analysis must confront the problem of data scarcity for many languages. This imbalance, variation,  diversity, and dynamics in speech and language databases means that it is almost impossible to learn a model from a single corpus and then expect it to be effective in practice in general. 
  
In automatic speech emotion recognition, most studies focus on a single corpus at a time, without considering the performance of model in cross-language and cross-corpus scenarios.
%\JE{linguistic difference ?} 
However, ever since transfer learning has been applied to cross-domain classification and pattern recognition problems, interest in applying it to cross-corpus emotion recognition has bee growing. Transfer learning focuses on adapting knowledge from available auxiliary resources to transfer this learning to a target domain, where a very few or even no labelled data is available \cite{pan2010survey,lu2015transfer}. 

Deep neural network (DNN) based transfer learning has recently improved image classification by using a very large dataset as source domain and small data as a target domain \cite{sawada2015transfer}. 
Inspired by this success, deep learning based transfer learning has recently been used for speech analysis. However, the existing research has focused on basic DNNs. The impact of using models like Deep Belief Networks (DBNs), which have strong generalisation power and are therefore suitable for cross-corpus emotion recognition, has not been thoroughly explored. A few studies have explored DBNs for speech emotion recognition (e.g., \cite{le2013emotion,rana2016emotion}) and numerous studies focus on DBNs for features extraction \cite{xia2017multi,schmidt2011learning,huang2014research} from speech signal. However, transfer learning using DBNs is very rare. Furthermore, how to maximise the transfer learning performance for cross-corpus/cross-language emotion recognition still needs to be explored further.

% For example, the performance of sound event classification is improved by training DNNs on very large speech data \cite{lim2016cross}. In \cite{richardson2015deep}, authors used a single DNNs for language and speaker recognition with a large gain on performance by training the model on speech recognition data. 

 

In this study, we address the above challenges. We investigate DBNs for transfer learning over five widely-used emotional speech databases. By using the experimental results from various scenarios, we indicated how a large gain in accuracy comparable to baseline can be achieved using transfer learning technique for cross-corpus emotion recognition.




% and demonstrate its improved performance over DNN models been used in the literature. Using various experimental scenarios we also demonstrate how transfer learning can be used to obtain a large gain in accuracy for cross-corpus emotion recognition.
% \JE{there are already transfer learning results in the literature. I suggest focusing on what is different, special and effective about DBNs}
% that how  and performed pair-wise experiments using comparatively large dataset as training domain and other small datasets as a target domain to explore the feasibility of transfer learning using Deep Belief Networks (DBNs).  We also show that the recognition rate comparable to baseline results can be achieved by using a small fraction of target data with training examples. 
% We believe that our exploratory research will help researchers gather valuable knowledge about using cross-corpus and cross-language speech emotion recognition as a transfer learning technique and apply it in their relevant area of research.

% Rest of the paper is organized as follow. In next section, we present the related work on cross-corpus emotion recognition. In Section \ref{ES}, we present experimental setup adopted in this work. In Section \ref{sec:experimentation}, we discuss different scenarios and obtained results. Section \ref{Sec: Insigts} discuss key findings in this study followed by conclusions in Section \ref{sec: con}. 

\section{Related Work}
\label{sec:back}

Although cross-language and cross-corpus speech emotion recognition is an interesting problem, relatively few studies have addressed this topic. Existing studies have mostly studied the preliminary feasibility of cross-corpus learning and pointed to the need for further in-depth research. For example, Schuller et al. \cite{schuller2010cross} used six different corpora to analyse cross-corpora emotion recognition using support vector machines (SVM) and highlighted the limitations of current systems for cross-corpus emotion recognition. Eyben et al. \cite{eyben2010cross} used four corpora to evaluate some pilot experiments on cross-corpus emotion recognition while using SVM. They used three datasets for training and a fourth for testing, and showed that the cross-corpus emotion recognition is feasible. To explore the universal cues of emotions across languages, Xia et al. \cite{xiao2016speech} investigated cross-language emotion recognition for Mandarin vs. Western languages (i.e., German, and Danish). The authors focused on gender-specific speech emotion recognition and achieved the classification rates higher than the chance level but less than baseline accuracy.  Albornoz et al. \cite{albornoz2017emotion} developed an ensemble SVM for emotion detection with a focus on emotion recognition in unseen languages. 



 Deep learning techniques have been widely used for transfer learning in speech recognition but only basic DNN models have been utilised so far. Lim et al. \cite{lim2016cross} proposed cross-acoustic transfer learning framework by using DNNs. The authors trained a model on a large data of speech and use it for sound event classification. After a series of experiments, the results showed that the cross-acoustic transfer learning can significantly enhance the sound event classification rate. In \cite{richardson2015deep}, authors used a single DNN for speaker and language recognition with a large gain on performance by training the model on speech recognition data. 
 These studies exploited the models that have good learning abilities so that the learned features are transferable to enable model adaptation regarding the target domain. 
 
 
% In this paper we use Deep Belief Networks (DBNs) for transfer learning  speech emotion, which is a rather complex model.
%\JE{I find the above analysis doesn't really say anything much about either cross-corpus AER or transfer learning. I suggest a more detailed and critical analysis}

In this paper, we use Deep Belief Networks (DBNs) for transfer learning  speech emotion. The key reason for employing DBN is its power of generalisation, which is not present in most conventional DNN models \cite{lee2010unsupervised}. Because, the building block of DBNs (i.e., RBMs) are universal approximators and very powerful to approximate any distribution \cite{le2008representational}.
% \JE{reference(s) missing. Also why does DBN provide better generalisation ? Why might it work well in this particular application ?}
Intuitively, for cross-corpus and cross-language emotion recognition, the generalisation power of a model is crucial. In addition, DBN can learn more powerful and effective discriminative long-range of features \cite{hinton2006reducing} that have been shown to help in speech-related problems \cite{deng2010binary}.
% \JE{More background on DBN is needed !! - OK, I see it further down now}

% based transfer learning has recently been used for speech analysis and recognition. However, the existing research has focused on basic DNNs, but the impact of using complex DNN models suitable for cross-corpus/cross-language setting is yet to be explored.  %We use DBNs and demonstrate that it performs better than a popular DNN model - autoencoders. %\textcolor{red}{Most of the previous studies used DBN directly or indirectly for robust feature extraction for emotion recognition with in the same data.  Our work investigated the performance of DBN for transfer learning based cross-corpus emotion recognition.} 
%  Examples of some existing research using deep learning are presented next.


Apart from DNNs, researchers have also used interesting deep architectures for transfer learning. In \cite{gideon2017progressive}, the authors focused on using Progressive Neural Networks to transfer knowledge for three paralinguistic tasks, i.e., emotion, speaker, and gender detection. Progressive Networks are useful for conducting multitasking in a network, however, we focus on a single task of emotion recognition as speaker and gender recognition are not the focus of this paper. Zong et al. \cite{zong2016cross} proposed a domain-adaptive least-squares regression (DaLSR) model for cross-corpus speech emotion recognition. They used three datasets for the evaluations and found that DaLSR can achieved better results than other models like SVM. They did not focus on achieving results higher than the baseline accuracy. Similarly, Deng et al. \cite{deng2013sparse} used sparse autoencoders (AE) for feature transfer learning in speech emotion recognition.  They used six standard databases and a single-layer sparse AE and train this model on class-specific instances from the target domain, then apply this representation to the source domain for reconstruction of those data. This experimental approach improves the performance of the model as compared with independent learning from every source domain. 


\section{Experimental Setup}
\label{ES}
\subsection{Speech Databases}
\label{sec:datasets}


\begin{table*}[ht]
\centering
\scriptsize
\caption{Corpora information and the mapping of class labels onto Negative/Positive valence.}
\begin{tabular}{|m{1.2cm}|m{1.1cm}|m{1cm}|m{1.2cm}|m{4.15cm}|m{3.28cm}|m{1.2cm}|}
\hline
\textbf{Corpus}
&\textbf{Language}
&\textbf{Age}
&\textbf{Utterances}
&\textbf{Negative Valance}
&\textbf{Positive Valance}
&\textbf{References}
\\ \hline
\begin{tabular}[c]{@{}l@{}}FAU-AIBO\end{tabular}
&\begin{tabular}[c]{@{}l@{}}German\end{tabular}
&\begin{tabular}[c]{@{}l@{}}Children\end{tabular}
&\begin{tabular}[c]{@{}l@{}}18216\end{tabular}
&\begin{tabular}[c]{@{}l@{}}Angry, Touchy, Emphatic,
Reprimanding\end{tabular}
&\begin{tabular}[c]{@{}l@{}}Motherese, Joyful, Neutral,
Rest\end{tabular}
&\begin{tabular}[c]{@{}l@{}}\cite{schuller2009interspeech}\end{tabular}
\\\hline
\begin{tabular}[c]{@{}l@{}}IEMOCAP\end{tabular}
&\begin{tabular}[c]{@{}l@{}}English\end{tabular}
&\begin{tabular}[c]{@{}l@{}}Adults\end{tabular}
&\begin{tabular}[c]{@{}l@{}}5531\end{tabular}
&\begin{tabular}[c]{@{}l@{}}Angry, Sadness\end{tabular}
&\begin{tabular}[c]{@{}l@{}}Neutral, Happy, Excited\end{tabular}
&\begin{tabular}[c]{@{}l@{}}\cite{busso2008iemocap}\end{tabular}
\\\hline
 \begin{tabular}[c]{@{}l@{}}EMO-DB\end{tabular}
&\begin{tabular}[c]{@{}l@{}}German\end{tabular}
&\begin{tabular}[c]{@{}l@{}}Adults\end{tabular}
&\begin{tabular}[c]{@{}l@{}}494\end{tabular}
&\begin{tabular}[c]{@{}l@{}}Anger, Sadness, Fear, Disgust, Boredom\end{tabular} 
&\begin{tabular}[c]{@{}l@{}}Neutral, Happiness\end{tabular}
&\begin{tabular}[c]{@{}l@{}}\cite{burkhardt2005database}\end{tabular}
\\ \hline
 \begin{tabular}[c]{@{}l@{}}SAVEE\end{tabular}
&\begin{tabular}[c]{@{}l@{}}English\end{tabular}
&\begin{tabular}[c]{@{}l@{}}Adults\end{tabular}
&\begin{tabular}[c]{@{}l@{}}480\end{tabular}
&\begin{tabular}[c]{@{}l@{}}Anger, Sadness, Fear, Disgust \end{tabular}
&\begin{tabular}[c]{@{}l@{}}Neutral, Happiness, Surprise\end{tabular}
&\begin{tabular}[c]{@{}l@{}}\cite{jackson2014surrey}\end{tabular}
\\ \hline 
 \begin{tabular}[c]{@{}l@{}}EMOVO\end{tabular}
&\begin{tabular}[c]{@{}l@{}}Italian\end{tabular}
&\begin{tabular}[c]{@{}l@{}}Adults\end{tabular}
&\begin{tabular}[c]{@{}l@{}}588\end{tabular}
&\begin{tabular}[c]{@{}l@{}}Anger, Sadness, Fear, Disgust\end{tabular}
&\begin{tabular}[c]{@{}l@{}}Neutral, Joy, Surprise\end{tabular}
&\begin{tabular}[c]{@{}l@{}}\cite{costantini2014emovo}\end{tabular}
\\ \hline

\end{tabular}

\centering
\label{table: MAP}
\end{table*}

To investigate the performance of DBN for cross-corpora and cross-language emotion recognition, we selected five publicly available and highly popular corpora which have maximum diversity in languages. These databases are annotated differently, therefore, one of the only consistent ways to investigate transfer learning is by considering the binary positive/negative valence classification problem. We adopt the binary valence mapping per emotion category from \cite{deng2013sparse,eyben2016geneva,schuller2010cross}.  
The names of the datasets used in our experiment and the categorical mappings to binary valence classes are provided in Table \ref{table: MAP}. These databases were chosen to span a variety of languages. % used by the researchers.

\subsection{Speech Features}

In this study, we use eGeMAPS feature set, which is a widely used reference feature set for speech emotion recognition studies \cite{gideon2017progressive}. 
The feature set includes Low-Level Descriptor (LLD) features of the speech signal which are described most relevant to emotions by Paralinguistic studies \cite{eyben2016geneva}. The eGeMAPS feature set contains 88 features including frequency, energy, spectral, cepstral, and dynamic information. The overall components are the arithmetic mean and coefficient of variation of 18 LLDs, 6 temporal features, 4 statistics over the unvoiced segments, 8 functionals applied to loudness and pitch, and 26 additional dynamic and cepstral components.
%\JE{not sure if you even need this much explanation - the feature set is very well known}

\subsection{Deep Belief Networks}
% We have used Deep Belief Networks (DBNs) for emotion recognition from speech. 
DBNs are very popular deep architectures that consist of the stack of Restricted Boltzmann Machines (RBMs) to make a powerful probabilistic generative model by using layer-wise training in a greedy manner. RBM is an undirected stochastic neural network consisting of a visible layer, a hidden layer, and a bias unit. Each visible unit of the visible layer is fully connected to hidden units in the hidden layer, and the bias is connected to all the visible units and the hidden units. There is no connection between visible to visible and between hidden to hidden units. RBMs can also be used as classifiers. They are trained on the joint distribution of input data and corresponding labels, then the label is assigned to the new input which has the highest probability under the model. The joint distribution of between visible layer ($v$) and hidden layer ($h$) is given by~\cite{hinton2006fast}: 

\begin{equation}
    P(v,h)= \frac{1}{Z}\exp({-E(v,h)})
\end{equation}

where Z represents the normalisation constant and $E(v, h)$ is an energy
function which is defined as:

\begin{equation}
    E(v,h)=  -\sum_{i=1}^{D}\sum_{j=1}^{k}W_{ij}v_{i}h_{j}-\sum_{i=1}^{D}b_{i}v_{i}-\sum_{j=1}^{k}a_{j}h_{j}
\end{equation}
%\JE{K?}
where $v_{i}$ and $h_{i}$ are the binary states of visible and hidden units. $W_{ij}$ represents the weights of connections between hidden and visible nodes. The conditional probabilities for the visible and hidden units are given by the following equations, where $g$ is the sigmoid function: $g(x)= \frac{1}{1+e^{-x}}$.

\begin{equation}
    P(v_{i}=1|h)= g\big(b_{i}^{v}+ \sum_{j}h_{j}W_{ij}\big)
\end{equation}

\begin{equation}
    P(h_{j}=1|v)= g\big(b_{j}^{h}+ \sum_{i}v_{i}W_{ij}\big)
\end{equation}




An RBM is pre-trained for the maximisation of data log-likelihood $log P (v)$. The stack of generatively pre-trained RBMs constitutes a powerful DBN that can be discriminatively fine-tuned to improve performance. Weight initialisation with pre-training can help the network to avoid poor local minima and give better discriminative results when compared with a neural network initialised by small random weights \cite{erhan2010does}. In this work, we also use layer-by-layer pre-training for DBN. The description of DBNs and their training methodologies can be reviewed in \cite{hinton2002training,hinton2006fast}.

During experimental work, a DBN with three RBM layers was selected, where the first two RBMs have 1000 hidden unit each, and the third RBM have 2000 hidden units with learning rate of $10^{-3}$ and 500 epochs. This configuration was obtained using cross validation experiments on  validation data. %\JE{on what data?}.
The other network parameters were chosen by following the setup in \cite{rana2016emotion,keyvanrad2014brief}. 
% We used layer-by-layer pre-training in DBN, and back-propagation technique through the whole network to fine-tune the weights to maximise classification accuracy. 


%\section{Experimentation}
\section{Results}
\label{sec:experimentation}


In this section, we explore various scenarios for cross-corpus and cross-language speech emotion recognition and conduct experiments to test the scenarios.  %\RR{This should go in the next section $->$}  

\subsection{Within Corpus Scheme}
In order to obtain the baseline comparison results, we compare the performance of DBN with a popular approach of using sparse autoencoder (AE) with SVM for feature transfer learning in speech emotion recognition\cite{deng2013sparse}. This preliminary experiment enables us to set maximum achievable baseline accuracy when both systems are trained and tested using the data of same corpus. % in Figure \ref{fig:base}. 
For baseline experiments, 75\% of randomly selected data is used for training and remaining 25\% unseen data is used for testing. 
%\sout{We also compare the performance of DBN with sparse autoencoder (AE) for feature learning and SVM for classification.}
Figure \ref{fig:base} shows the comparison results, where DBN  outperforms sparse AE for all databases. %\RR{Can you say here why you chose autoencoder? Also tell people that you will be use SVM with autoencoder.}  


% \RR{We perhaps do not need to show the baseline accuracies in a separate table. We can include them in other tables.} \textcolor{red}{Dear Sir, by adding extra column in language test the width of table increase and it will take full page. To avoid this I have included a picture. In section V we are comparing baseline with other results so I think we can avoid here.}
% \RR{You can span the table over the whole page if needed.}
% %\sout{The obtained accuracies are compared in table \ref{table: DBN} for all the databases. Following are different schemes that we performed for investigation of cross-corpus/language emotion recognition.}

\begin{figure}[!ht]
\centering
\centerline{\includegraphics[width=.42\textwidth]{baseline.eps}}
\caption{Comparison of baseline accuracy using DBN and sparse AE on different databases.}
\label{fig:base}
\end{figure}
\subsection{Language Tests}
In this experiment, we use one language dataset for training and the remaining datasets for testing. For brevity, we just use FAU-AIBO (German) and IEMOCAP (English) datasets for training. %\textcolor{red}{ Dear Sir  cross corpus with similar language also give good trend that I have highlighted in discussion section. I think we should have this \RR{$->$ OK} }.
In order to evaluate the model on IEMOCAP, we used two sessions out of five with two-fold cross validation because overall data is large. The other databases are small comparative to IEMOCAP, therefore, we used them completely. Figure \ref{fig:Lang} shows the recognition rate achieved in these experiments and its comparison with previous techniques using sparse autoencoder and SVM (sparse AE+SVM) for cross-corpus transfer learning. When the IEMOCAP database was used for training the DBN, we performed pairwise testing using OHM and MONT separately for FAU-AIBO. It can be noted from Figure \ref{fig:Lang} that DBN outperforms sparse AE for all scenarios. Beyond this point, the accuracy of sparse AE is not given, as we observe that DBNs consistently outperform sparse AE.


\begin{figure*}[!ht]%
%\hspace*{\fill}%
\centering
\begin{subfigure}{0.4\linewidth}
\includegraphics[width=\linewidth]{Lang1.eps}%
\captionsetup{justification=centering}
\caption{}%
\label{Lang1}%
\end{subfigure}%\hfill%
\begin{subfigure}{0.4\linewidth}
\includegraphics[width=\linewidth]{Lang2.eps}%
\captionsetup{justification=centering}
\caption{} %
\label{lang2}%
\end{subfigure}%
\caption{Comparison of language tests using DBN and sparse AE. Figure \ref{Lang1} represents the recognition rate using IEMOCAP (English) for training and other databases for testing whereas \ref{lang2} shows the recognition rate using FAU-AIBO (German) for training and other databases for testing.}
\label{fig:Lang}
\end{figure*}

\begin{figure*}[!ht]%
%\hspace*{\fill}%
\centering
\begin{subfigure}{0.4\linewidth}
\includegraphics[width=\linewidth]{percent.eps}%
\captionsetup{justification=centering}
\caption{}%
\label{Per1}%
\end{subfigure}%\hfill%
\begin{subfigure}{0.4\linewidth}
\includegraphics[width=\linewidth]{percent1.eps}%
\captionsetup{justification=centering}
\caption{} %
\label{Per2}%
\end{subfigure}%
\caption{Impact of using a percentage of target date with training data. Where \ref{Per1} shows the training with IEMOCAP and \ref{Per2} is when training is performed using FAU-AIBO.}
\label{Perentage}
\end{figure*}


\subsection{Percentage of Target Data}

In this experiment, we vary the percentage (10\% to 80\%) of the target dataset for the training of the model. The training was performed using IEMOCAP and FAU-AIBO separately and EMOVO, EMO-DB and SAVEE were used for testing. The results are shown in Figure \ref{Perentage}. The straight horizontal lines in the figure show the baseline recognition rate for the respective corpora. These results show that the recognition rate significantly improves (than baseline) by including target domain data with the training data.


\subsection{Multi-language Training}

In this experiment, we use multiple languages jointly for training to observe whether this improves the performance of using languages individually for training. We use both FAU-AIBO and IEMOCAP for training and remaining for testing. We also evaluate the model within the corpora. For IEMOCAP, we used three sessions (plus FAU-AIBO) for training  and testing was performed using the remaining two sessions with two-fold cross validation. Similarly, for FAU-AIBO, a two-fold cross-validation was used, i.e., training on OHM (plus IEMOCAP) and evaluating on MONT and the inverse.

%\RR{Can you merge table 2 3 and 4 into a plot and group by language and just say the performance of DBN for cross corpus is significantly better than AE for all cases and do not need to present the results for AE anymore. This will allow readers to compare the performances side by side. If the plot does not fit the column span it over the whole page.} \textcolor{red}{working on it}

Further, we also performed training using a leave-one-data-out scheme. For FAU-AIBO, we have performed evaluation by using OHM and MONT independently taking the average results. In the case of IEMOCAP, we used two sessions (with two-fold cross validation) to evaluate the model. This performs better than baseline and two-language training as shown in Figure \ref{fig:G2}. %\RR{What is baseline here - DBN?} 

\begin{figure}[!ht]
\centering
\captionsetup{justification=centering}
\centerline{\includegraphics[width=.42\textwidth]{COMP.eps}}
\caption{Comparison of baseline results and transfer learning using FAU-AIBO+IEMOCAP and leave-one-language-out scheme.}
\label{fig:G2}
\end{figure}






%\subsection{Leave-one-database Out}
% Here we test the impact of the size of the training database. We use both 

\section{Discussion}
\label{Sec: Insigts}
From the experiments, Leave-one-Out seems to be standing out in-terms of obtaining the highest accuracy. This essentially means that training the model using a large range of languages would help learn many intrinsic features from each languages, which can essentially help to achieve high accuracy in an unknown language - even higher than when the same language is used for training and testing (baseline). The performance of the Leave-one-out (see Figure~\ref{fig:G2}) on EMOVO database is a prime example of this. Both German and English languages have two datasets each, i.e., in a Leave-one-Out scheme there will be at least one of these language in the training set. But for EMOVO there will be a situation that emotions in the Italian language are predicted simply based on emotions in German and English language. 
% Although this is an useful finding, it warrants further investigation to pinpoint the underlying reasons for this improved performance. 
%\JE{Did anyone else ever do leave one out experiments ? What did they find ?}

Another interesting aspect we learned from the experiments that including a fraction of the target data into training can help improve the performance and help achieve better results than baseline. %\JE{this is not an insight; we could have guessed it without running the experiment} 
Based on our experiments, augmenting other databases with around 20\% of data (around 90 utterances in case of EMO-DB) from the target database can help achieve better than the baseline accuracy. However, this is worse while using FAU-AIBO for training. Interestingly, IEMOCAP performs well on EMO-DB that is in the German language as compared to FAU-AIBO that is also in German. We note that FAU-AIBO consists of children speech whereas EMO-DB database contains adult speech.
 
% This warrants further investigation if this is due to age difference as FAU-AIBO consists of children speech whereas EMO-DB database contains adult speech. 

The performance of DBN in the language test results in Figure~\ref{fig:Lang} using both IEMOCAP (English) and FAU-AIBO (German) on target datasets is poor than the baseline. The drop in accuracy is not only for the target dataset with a different language but also for target data having similar language. From this experiment, we learned that the different studio conditions, age and language differences, and type of emotional corpus cause drop in the performance of the model. This problem can be addressed by previous two findings, i.e., either by training the model with the data of multiple languages or by including a small portion of data target domain with training data. 

%Even more interestingly, the language test results in Figure~\ref{fig:Lang} show that the performance of both IEMOCAP (English) and FAU-AIBO (German) on target datasets is poor than the baseline results. both SAVEE (English) and EMO-DB (German) are almost the same. This raise the question if the age factor as discussed before be only applicable while using a fraction of data from the target domain. %\JE{avoid pure speculation like this}

% Finally, based on our experimental results, deep Learning model selection is also important to achieve high accuracy. We observe from  Figure~\ref{fig:base} and Figure~\ref{fig:G2} that networks with generalisation power like Deep Belief Networks are more preferable than conventional discriminative networks like sparse auto encoders. 
%More experiments need to be conducted to generalise this findings. 



\section{Conclusions}
\label{sec: con}
%We conclude that cross-corpus and cross-language speech emotion recognition is definitely a very useful transfer learning technique. \JE{previous papers have shown this} Information from multiple languages or even different datasets in the same language can be leveraged to achieve high accuracy in even an unknown language. \JE{previous papers have shown this} For practical applications, this would be very helpful to build a robust speech emotion recognition system using data from multiple languages. Also, this would be equally useful for emotion recognition in languages with very limited or no datasets. \JE{previous papers have shown this}


In this paper, we investigated the performance of DBNs for transfer learning based cross-corpus and cross-language speech emotion recognition. In order to evaluate the feature transference across different corpora, we performed comprehensive experiments and found that DBNs outperformed sparse autoencoders due to its increased feature learning abilities. Also,  DBNs can learn from many training languages and improve the baseline accuracy even also when a small fraction of target data is included in the model while training it with a single corpus. For practical applications, these findings would be very helpful to build a robust speech emotion recognition system using data from multiple languages. Also, this would be equally useful for emotion recognition in languages with very limited or no datasets. 





%\bibliographystyle{IEEEtran}

%\bibliography{mybib}

\begin{thebibliography}{10}


\bibitem{batliner2011automatic}
A.~Batliner, B.~Schuller, D.~Seppi, S.~Steidl, L.~Devillers, L.~Vidrascu,
  T.~Vogt, V.~Aharonson, and N.~Amir, ``The automatic recognition of emotions
  in speech,'' in \emph{Emotion-Oriented Systems}.\hskip 1em plus 0.5em minus
  0.4em\relax Springer, 2011, pp. 71--99.

\bibitem{schuller2012synthesized}
B.~Schuller, Z.~Zhang, F.~Weninger, and F.~Burkhardt, ``Synthesized speech for
  model training in cross-corpus recognition of human emotion,''
  \emph{International Journal of Speech Technology}, vol.~15, no.~3, pp.
  313--323, 2012.

\bibitem{schuller2010cross}
B.~Schuller, B.~Vlasenko, F.~Eyben, M.~Wollmer, A.~Stuhlsatz, A.~Wendemuth, and
  G.~Rigoll, ``Cross-corpus acoustic emotion recognition: Variances and
  strategies,'' \emph{IEEE Transactions on Affective Computing}, vol.~1, no.~2,
  pp. 119--131, 2010.

\bibitem{pan2010survey}
S.~J. Pan and Q.~Yang, ``A survey on transfer learning,'' \emph{IEEE
  Transactions on knowledge and data engineering}, vol.~22, no.~10, pp.
  1345--1359, 2010.

\bibitem{lu2015transfer}
J.~Lu, V.~Behbood, P.~Hao, H.~Zuo, S.~Xue, and G.~Zhang, ``Transfer learning
  using computational intelligence: a survey,'' \emph{Knowledge-Based Systems},
  vol.~80, pp. 14--23, 2015.

\bibitem{sawada2015transfer}
Y.~Sawada and K.~Kozuka, ``Transfer learning method using multi-prediction deep
  boltzmann machines for a small scale dataset,'' in \emph{Machine Vision
  Applications (MVA), 2015 14th IAPR International Conference on}.\hskip 1em
  plus 0.5em minus 0.4em\relax IEEE, 2015, pp. 110--113.

\bibitem{le2013emotion}
D.~Le and E.~M. Provost, ``Emotion recognition from spontaneous speech using
  hidden markov models with deep belief networks,'' in \emph{Automatic Speech
  Recognition and Understanding (ASRU), 2013 IEEE Workshop on}.\hskip 1em plus
  0.5em minus 0.4em\relax IEEE, 2013, pp. 216--221.

\bibitem{rana2016emotion}
R.~Rana, ``Emotion classification from noisy speech-a deep learning approach,''
  \emph{arXiv preprint arXiv:1603.05901}, 2016.

\bibitem{xia2017multi}
R.~Xia and Y.~Liu, ``A multi-task learning framework for emotion recognition
  using 2d continuous space,'' \emph{IEEE Transactions on Affective Computing},
  vol.~8, no.~1, pp. 3--14, 2017.

\bibitem{schmidt2011learning}
E.~M. Schmidt and Y.~E. Kim, ``Learning emotion-based acoustic features with
  deep belief networks,'' in \emph{Applications of Signal Processing to Audio
  and Acoustics (WASPAA), 2011 IEEE Workshop on}.\hskip 1em plus 0.5em minus
  0.4em\relax IEEE, 2011, pp. 65--68.

\bibitem{huang2014research}
C.~Huang, W.~Gong, W.~Fu, and D.~Feng, ``A research of speech emotion
  recognition based on deep belief network and svm,'' \emph{Mathematical
  Problems in Engineering}, vol. 2014, 2014.

\bibitem{eyben2010cross}
F.~Eyben, A.~Batliner, B.~Schuller, D.~Seppi, and S.~Steidl, ``Cross-corpus
  classification of realistic emotions-some pilot experiments,'' in \emph{Proc.
  LREC workshop on Emotion Corpora, Valettea, Malta}, 2010, pp. 77--82.

\bibitem{xiao2016speech}
Z.~Xiao, D.~Wu, X.~Zhang, and Z.~Tao, ``Speech emotion recognition cross
  language families: Mandarin vs. western languages,'' in \emph{Progress in
  Informatics and Computing (PIC), 2016 International Conference on}.\hskip 1em
  plus 0.5em minus 0.4em\relax IEEE, 2016, pp. 253--257.

\bibitem{albornoz2017emotion}
E.~M. Albornoz and D.~H. Milone, ``Emotion recognition in never-seen languages
  using a novel ensemble method with emotion profiles,'' \emph{IEEE
  Transactions on Affective Computing}, vol.~8, no.~1, pp. 43--53, 2017.

\bibitem{lim2016cross}
H.~Lim, M.~J. Kim, and H.~Kim, ``Cross-acoustic transfer learning for sound
  event classification,'' in \emph{Acoustics, Speech and Signal Processing
  (ICASSP), 2016 IEEE International Conference on}.\hskip 1em plus 0.5em minus
  0.4em\relax IEEE, 2016, pp. 2504--2508.

\bibitem{richardson2015deep}
F.~Richardson, D.~Reynolds, and N.~Dehak, ``Deep neural network approaches to
  speaker and language recognition,'' \emph{IEEE Signal Processing Letters},
  vol.~22, no.~10, pp. 1671--1675, 2015.

\bibitem{lee2010unsupervised}
H.~Lee, \emph{Unsupervised feature learning via sparse hierarchical
  representations}.\hskip 1em plus 0.5em minus 0.4em\relax Stanford University,
  2010.

\bibitem{le2008representational}
N.~Le~Roux and Y.~Bengio, ``Representational power of restricted boltzmann
  machines and deep belief networks,'' \emph{Neural computation}, vol.~20,
  no.~6, pp. 1631--1649, 2008.

\bibitem{hinton2006reducing}
G.~E. Hinton and R.~R. Salakhutdinov, ``Reducing the dimensionality of data
  with neural networks,'' \emph{science}, vol. 313, no. 5786, pp. 504--507,
  2006.

\bibitem{deng2010binary}
L.~Deng, M.~L. Seltzer, D.~Yu, A.~Acero, A.-r. Mohamed, and G.~Hinton, ``Binary
  coding of speech spectrograms using a deep auto-encoder,'' in \emph{Eleventh
  Annual Conference of the International Speech Communication Association},
  2010.

\bibitem{gideon2017progressive}
J.~Gideon, S.~Khorram, Z.~Aldeneh, D.~Dimitriadis, and E.~M. Provost,
  ``Progressive neural networks for transfer learning in emotion recognition,''
  \emph{arXiv preprint arXiv:1706.03256}, 2017.

\bibitem{zong2016cross}
Y.~Zong, W.~Zheng, T.~Zhang, and X.~Huang, ``Cross-corpus speech emotion
  recognition based on domain-adaptive least-squares regression,'' \emph{IEEE
  Signal Processing Letters}, vol.~23, no.~5, pp. 585--589, 2016.

\bibitem{deng2013sparse}
J.~Deng, Z.~Zhang, E.~Marchi, and B.~Schuller, ``Sparse autoencoder-based
  feature transfer learning for speech emotion recognition,'' in
  \emph{Affective Computing and Intelligent Interaction (ACII), 2013 Humaine
  Association Conference on}.\hskip 1em plus 0.5em minus 0.4em\relax IEEE,
  2013, pp. 511--516.

\bibitem{schuller2009interspeech}
B.~Schuller, S.~Steidl, and A.~Batliner, ``The interspeech 2009 emotion
  challenge,'' in \emph{Tenth Annual Conference of the International Speech
  Communication Association}, 2009.

\bibitem{busso2008iemocap}
C.~Busso, M.~Bulut, C.-C. Lee, A.~Kazemzadeh, E.~Mower, S.~Kim, J.~N. Chang,
  S.~Lee, and S.~S. Narayanan, ``Iemocap: Interactive emotional dyadic motion
  capture database,'' \emph{Language resources and evaluation}, vol.~42, no.~4,
  p. 335, 2008.

\bibitem{burkhardt2005database}
F.~Burkhardt, A.~Paeschke, M.~Rolfes, W.~F. Sendlmeier, and B.~Weiss, ``A
  database of german emotional speech.'' in \emph{Interspeech}, vol.~5, 2005,
  pp. 1517--1520.

\bibitem{jackson2014surrey}
P.~Jackson and S.~Haq, ``Surrey audio-visual expressed emotion(savee)
  database,'' \emph{University of Surrey: Guildford, UK}, 2014.

\bibitem{costantini2014emovo}
G.~Costantini, I.~Iaderola, A.~Paoloni, and M.~Todisco, ``Emovo corpus: an
  italian emotional speech database.'' in \emph{LREC}, 2014, pp. 3501--3504.

\bibitem{eyben2016geneva}
F.~Eyben, K.~R. Scherer, B.~W. Schuller, J.~Sundberg, E.~Andr{\'e}, C.~Busso,
  L.~Y. Devillers, J.~Epps, P.~Laukka, S.~S. Narayanan \emph{et~al.}, ``The
  geneva minimalistic acoustic parameter set (gemaps) for voice research and
  affective computing,'' \emph{IEEE Transactions on Affective Computing},
  vol.~7, no.~2, pp. 190--202, 2016.

\bibitem{hinton2006fast}
G.~E. Hinton, S.~Osindero, and Y.-W. Teh, ``A fast learning algorithm for deep
  belief nets,'' \emph{Neural computation}, vol.~18, no.~7, pp. 1527--1554,
  2006.

\bibitem{erhan2010does}
D.~Erhan, Y.~Bengio, A.~Courville, P.-A. Manzagol, P.~Vincent, and S.~Bengio,
  ``Why does unsupervised pre-training help deep learning?'' \emph{Journal of
  Machine Learning Research}, vol.~11, no. Feb, pp. 625--660, 2010.

\bibitem{hinton2002training}
G.~E. Hinton, ``Training products of experts by minimizing contrastive
  divergence,'' \emph{Neural computation}, vol.~14, no.~8, pp. 1771--1800,
  2002.

\bibitem{keyvanrad2014brief}
M.~A. Keyvanrad and M.~M. Homayounpour, ``A brief survey on deep belief
  networks and introducing a new object oriented toolbox (deebnet),''
  \emph{arXiv preprint arXiv:1408.3264}, 2014.

\end{thebibliography}

\end{document}
