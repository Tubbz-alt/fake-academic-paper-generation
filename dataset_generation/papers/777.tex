
%% bare_jrnl.tex
%% V1.4a
%% 2014/09/17
%% by Michael Shell
%% see http://www.michaelshell.org/
%% for current contact information.
%%
%% This is a skeleton file demonstrating the use of IEEEtran.cls
%% (requires IEEEtran.cls version 1.8a or later) with an IEEE
%% journal paper.
%%
%% Support sites:
%% http://www.michaelshell.org/tex/ieeetran/
%% http://www.ctan.org/tex-archive/macros/latex/contrib/IEEEtran/
%% and
%% http://www.ieee.org/

%%*************************************************************************
%% Legal Notice:
%% This code is offered as-is without any warranty either expressed or
%% implied; without even the implied warranty of MERCHANTABILITY or
%% FITNESS FOR A PARTICULAR PURPOSE! 
%% User assumes all risk.
%% In no event shall IEEE or any contributor to this code be liable for
%% any damages or losses, including, but not limited to, incidental,
%% consequential, or any other damages, resulting from the use or misuse
%% of any information contained here.
%%
%% All comments are the opinions of their respective authors and are not
%% necessarily endorsed by the IEEE.
%%
%% This work is distributed under the LaTeX Project Public License (LPPL)
%% ( http://www.latex-project.org/ ) version 1.3, and may be freely used,
%% distributed and modified. A copy of the LPPL, version 1.3, is included
%% in the base LaTeX documentation of all distributions of LaTeX released
%% 2003/12/01 or later.
%% Retain all contribution notices and credits.
%% ** Modified files should be clearly indicated as such, including  **
%% ** renaming them and changing author support contact information. **
%%
%% File list of work: IEEEtran.cls, IEEEtran_HOWTO.pdf, bare_adv.tex,
%%                    bare_conf.tex, bare_jrnl.tex, bare_conf_compsoc.tex,
%%                    bare_jrnl_compsoc.tex, bare_jrnl_transmag.tex
%%*************************************************************************


% *** Authors should verify (and, if needed, correct) their LaTeX system  ***
% *** with the testflow diagnostic prior to trusting their LaTeX platform ***
% *** with production work. IEEE's font choices and paper sizes can       ***
% *** trigger bugs that do not appear when using other class files.       ***                          ***
% The testflow support page is at:
% http://www.michaelshell.org/tex/testflow/



%\documentclass[journal]{IEEEtran}
%\documentclass[11pt,draftcls,onecolumn]{IEEEtran} %for draft
\documentclass[10pt,twocolumn,twoside]{IEEEtran} %for camera-ready
%
% If IEEEtran.cls has not been installed into the LaTeX system files,
% manually specify the path to it like:
% \documentclass[journal]{../sty/IEEEtran}

\usepackage{times}
\usepackage{epsfig}
\usepackage{graphicx}
\graphicspath{{figures/}}
\usepackage{amsmath}
\usepackage{amssymb}

\usepackage{algorithm}
\usepackage{algorithmic}
\renewcommand{\algorithmicrequire}{\textbf{Input:}}
\renewcommand{\algorithmicensure}{\textbf{Output:}}
\usepackage{multirow}
\usepackage{array}
\usepackage{rotating}

\usepackage{xcolor}
\usepackage{url}
\usepackage{float}
\usepackage{subfigure}
%\usepackage{tabularx}

%\usepackage{array}
%\newcolumntype{C}{>{\centering\arraybackslash}p{5.3em}}

%\usepackage{refcheck}

\usepackage[american]{babel}

%\hyphenation{SR}
%\hyphenation{PSNR}

\newcommand{\HCZ}{{\textbf{[HCZ] }}}\newcommand{\ZW}{{\textbf{[Atlas] }}}\newcommand{\M}{{\mathcal{M}}}%\newcommand{\sm}[1]{\mbox{\footnotesize #1}}%\newcommand{\stimes}{{\mkern-2mu\times\mkern-2mu}}%\newcommand{\sminus}{{\mkern-2mu-\mkern-2mu}}%\newcommand{\scdots}{{\mkern-2mu\cdots\mkern-2mu}}%\newcommand{\seqv}{{\mkern-2mu=\mkern-2mu}}%\newcommand{\sless}{{\mkern-2mu<\mkern-2mu}}%\newcommand{\sgt}{{\mkern-2mu>\mkern-2mu}}%\newcommand{\sm}[1]{\mbox{\footnotesize #1}}%\newcommand{\stimes}{{\times}}%\newcommand{\sminus}{{-}}%\newcommand{\scdots}{{\cdots}}%\newcommand{\seqv}{{=}}%\newcommand{\sless}{{<}}%\newcommand{\sgt}{{>}}\renewcommand{\algorithmicrequire}{\textbf{Input:}}\renewcommand{\algorithmicensure}{\textbf{Output:}}

%\usepackage[pagebackref=true,breaklinks=true,letterpaper=true,colorlinks,bookmarks=false,driverfallback=dvipdfm]{hyperref}
% Some very useful LaTeX packages include:
% (uncomment the ones you want to load)


% *** MISC UTILITY PACKAGES ***
%
%\usepackage{ifpdf}
% Heiko Oberdiek's ifpdf.sty is very useful if you need conditional
% compilation based on whether the output is pdf or dvi.
% usage:
% \ifpdf
%   % pdf code
% \else
%   % dvi code
% \fi
% The latest version of ifpdf.sty can be obtained from:
% http://www.ctan.org/tex-archive/macros/latex/contrib/oberdiek/
% Also, note that IEEEtran.cls V1.7 and later provides a builtin
% \ifCLASSINFOpdf conditional that works the same way.
% When switching from latex to pdflatex and vice-versa, the compiler may
% have to be run twice to clear warning/error messages.






% *** CITATION PACKAGES ***
%
%\usepackage{cite}
% cite.sty was written by Donald Arseneau
% V1.6 and later of IEEEtran pre-defines the format of the cite.sty package
% \cite{} output to follow that of IEEE. Loading the cite package will
% result in citation numbers being automatically sorted and properly
% "compressed/ranged". e.g., [1], [9], [2], [7], [5], [6] without using
% cite.sty will become [1], [2], [5]--[7], [9] using cite.sty. cite.sty's
% \cite will automatically add leading space, if needed. Use cite.sty's
% noadjust option (cite.sty V3.8 and later) if you want to turn this off
% such as if a citation ever needs to be enclosed in parenthesis.
% cite.sty is already installed on most LaTeX systems. Be sure and use
% version 5.0 (2009-03-20) and later if using hyperref.sty.
% The latest version can be obtained at:
% http://www.ctan.org/tex-archive/macros/latex/contrib/cite/
% The documentation is contained in the cite.sty file itself.






% *** GRAPHICS RELATED PACKAGES ***
%
\ifCLASSINFOpdf
  % \usepackage[pdftex]{graphicx}
  % declare the path(s) where your graphic files are
  % \graphicspath{{../pdf/}{../jpeg/}}
  % and their extensions so you won't have to specify these with
  % every instance of \includegraphics
  % \DeclareGraphicsExtensions{.pdf,.jpeg,.png}
\else
  % or other class option (dvipsone, dvipdf, if not using dvips). graphicx
  % will default to the driver specified in the system graphics.cfg if no
  % driver is specified.
  % \usepackage[dvips]{graphicx}
  % declare the path(s) where your graphic files are
  % \graphicspath{{../eps/}}
  % and their extensions so you won't have to specify these with
  % every instance of \includegraphics
  % \DeclareGraphicsExtensions{.eps}
\fi
% graphicx was written by David Carlisle and Sebastian Rahtz. It is
% required if you want graphics, photos, etc. graphicx.sty is already
% installed on most LaTeX systems. The latest version and documentation
% can be obtained at: 
% http://www.ctan.org/tex-archive/macros/latex/required/graphics/
% Another good source of documentation is "Using Imported Graphics in
% LaTeX2e" by Keith Reckdahl which can be found at:
% http://www.ctan.org/tex-archive/info/epslatex/
%
% latex, and pdflatex in dvi mode, support graphics in encapsulated
% postscript (.eps) format. pdflatex in pdf mode supports graphics
% in .pdf, .jpeg, .png and .mps (metapost) formats. Users should ensure
% that all non-photo figures use a vector format (.eps, .pdf, .mps) and
% not a bitmapped formats (.jpeg, .png). IEEE frowns on bitmapped formats
% which can result in "jaggedy"/blurry rendering of lines and letters as
% well as large increases in file sizes.
%
% You can find documentation about the pdfTeX application at:
% http://www.tug.org/applications/pdftex





% *** MATH PACKAGES ***
%
%\usepackage[cmex10]{amsmath}
% A popular package from the American Mathematical Society that provides
% many useful and powerful commands for dealing with mathematics. If using
% it, be sure to load this package with the cmex10 option to ensure that
% only type 1 fonts will utilized at all point sizes. Without this option,
% it is possible that some math symbols, particularly those within
% footnotes, will be rendered in bitmap form which will result in a
% document that can not be IEEE Xplore compliant!
%
% Also, note that the amsmath package sets \interdisplaylinepenalty to 10000
% thus preventing page breaks from occurring within multiline equations. Use:
%\interdisplaylinepenalty=2500
% after loading amsmath to restore such page breaks as IEEEtran.cls normally
% does. amsmath.sty is already installed on most LaTeX systems. The latest
% version and documentation can be obtained at:
% http://www.ctan.org/tex-archive/macros/latex/required/amslatex/math/





% *** SPECIALIZED LIST PACKAGES ***
%
%\usepackage{algorithmic}
% algorithmic.sty was written by Peter Williams and Rogerio Brito.
% This package provides an algorithmic environment fo describing algorithms.
% You can use the algorithmic environment in-text or within a figure
% environment to provide for a floating algorithm. Do NOT use the algorithm
% floating environment provided by algorithm.sty (by the same authors) or
% algorithm2e.sty (by Christophe Fiorio) as IEEE does not use dedicated
% algorithm float types and packages that provide these will not provide
% correct IEEE style captions. The latest version and documentation of
% algorithmic.sty can be obtained at:
% http://www.ctan.org/tex-archive/macros/latex/contrib/algorithms/
% There is also a support site at:
% http://algorithms.berlios.de/index.html
% Also of interest may be the (relatively newer and more customizable)
% algorithmicx.sty package by Szasz Janos:
% http://www.ctan.org/tex-archive/macros/latex/contrib/algorithmicx/




% *** ALIGNMENT PACKAGES ***
%
%\usepackage{array}
% Frank Mittelbach's and David Carlisle's array.sty patches and improves
% the standard LaTeX2e array and tabular environments to provide better
% appearance and additional user controls. As the default LaTeX2e table
% generation code is lacking to the point of almost being broken with
% respect to the quality of the end results, all users are strongly
% advised to use an enhanced (at the very least that provided by array.sty)
% set of table tools. array.sty is already installed on most systems. The
% latest version and documentation can be obtained at:
% http://www.ctan.org/tex-archive/macros/latex/required/tools/


% IEEEtran contains the IEEEeqnarray family of commands that can be used to
% generate multiline equations as well as matrices, tables, etc., of high
% quality.




% *** SUBFIGURE PACKAGES ***
%\ifCLASSOPTIONcompsoc
%  \usepackage[caption=false,font=normalsize,labelfont=sf,textfont=sf]{subfig}
%\else
%  \usepackage[caption=false,font=footnotesize]{subfig}
%\fi
% subfig.sty, written by Steven Douglas Cochran, is the modern replacement
% for subfigure.sty, the latter of which is no longer maintained and is
% incompatible with some LaTeX packages including fixltx2e. However,
% subfig.sty requires and automatically loads Axel Sommerfeldt's caption.sty
% which will override IEEEtran.cls' handling of captions and this will result
% in non-IEEE style figure/table captions. To prevent this problem, be sure
% and invoke subfig.sty's "caption=false" package option (available since
% subfig.sty version 1.3, 2005/06/28) as this is will preserve IEEEtran.cls
% handling of captions.
% Note that the Computer Society format requires a larger sans serif font
% than the serif footnote size font used in traditional IEEE formatting
% and thus the need to invoke different subfig.sty package options depending
% on whether compsoc mode has been enabled.
%
% The latest version and documentation of subfig.sty can be obtained at:
% http://www.ctan.org/tex-archive/macros/latex/contrib/subfig/




% *** FLOAT PACKAGES ***
%
%\usepackage{fixltx2e}
% fixltx2e, the successor to the earlier fix2col.sty, was written by
% Frank Mittelbach and David Carlisle. This package corrects a few problems
% in the LaTeX2e kernel, the most notable of which is that in current
% LaTeX2e releases, the ordering of single and double column floats is not
% guaranteed to be preserved. Thus, an unpatched LaTeX2e can allow a
% single column figure to be placed prior to an earlier double column
% figure. The latest version and documentation can be found at:
% http://www.ctan.org/tex-archive/macros/latex/base/


%\usepackage{stfloats}
% stfloats.sty was written by Sigitas Tolusis. This package gives LaTeX2e
% the ability to do double column floats at the bottom of the page as well
% as the top. (e.g., "\begin{figure*}[!b]" is not normally possible in
% LaTeX2e). It also provides a command:
%\fnbelowfloat
% to enable the placement of footnotes below bottom floats (the standard
% LaTeX2e kernel puts them above bottom floats). This is an invasive package
% which rewrites many portions of the LaTeX2e float routines. It may not work
% with other packages that modify the LaTeX2e float routines. The latest
% version and documentation can be obtained at:
% http://www.ctan.org/tex-archive/macros/latex/contrib/sttools/
% Do not use the stfloats baselinefloat ability as IEEE does not allow
% \baselineskip to stretch. Authors submitting work to the IEEE should note
% that IEEE rarely uses double column equations and that authors should try
% to avoid such use. Do not be tempted to use the cuted.sty or midfloat.sty
% packages (also by Sigitas Tolusis) as IEEE does not format its papers in
% such ways.
% Do not attempt to use stfloats with fixltx2e as they are incompatible.
% Instead, use Morten Hogholm'a dblfloatfix which combines the features
% of both fixltx2e and stfloats:
%
% \usepackage{dblfloatfix}
% The latest version can be found at:
% http://www.ctan.org/tex-archive/macros/latex/contrib/dblfloatfix/




%\ifCLASSOPTIONcaptionsoff
%  \usepackage[nomarkers]{endfloat}
% \let\MYoriglatexcaption\caption
% \renewcommand{\caption}[2][\relax]{\MYoriglatexcaption[#2]{#2}}
%\fi
% endfloat.sty was written by James Darrell McCauley, Jeff Goldberg and 
% Axel Sommerfeldt. This package may be useful when used in conjunction with 
% IEEEtran.cls'  captionsoff option. Some IEEE journals/societies require that
% submissions have lists of figures/tables at the end of the paper and that
% figures/tables without any captions are placed on a page by themselves at
% the end of the document. If needed, the draftcls IEEEtran class option or
% \CLASSINPUTbaselinestretch interface can be used to increase the line
% spacing as well. Be sure and use the nomarkers option of endfloat to
% prevent endfloat from "marking" where the figures would have been placed
% in the text. The two hack lines of code above are a slight modification of
% that suggested by in the endfloat docs (section 8.4.1) to ensure that
% the full captions always appear in the list of figures/tables - even if
% the user used the short optional argument of \caption[]{}.
% IEEE papers do not typically make use of \caption[]'s optional argument,
% so this should not be an issue. A similar trick can be used to disable
% captions of packages such as subfig.sty that lack options to turn off
% the subcaptions:
% For subfig.sty:
% \let\MYorigsubfloat\subfloat
% \renewcommand{\subfloat}[2][\relax]{\MYorigsubfloat[]{#2}}
% However, the above trick will not work if both optional arguments of
% the \subfloat command are used. Furthermore, there needs to be a
% description of each subfigure *somewhere* and endfloat does not add
% subfigure captions to its list of figures. Thus, the best approach is to
% avoid the use of subfigure captions (many IEEE journals avoid them anyway)
% and instead reference/explain all the subfigures within the main caption.
% The latest version of endfloat.sty and its documentation can obtained at:
% http://www.ctan.org/tex-archive/macros/latex/contrib/endfloat/
%
% The IEEEtran \ifCLASSOPTIONcaptionsoff conditional can also be used
% later in the document, say, to conditionally put the References on a 
% page by themselves.




% *** PDF, URL AND HYPERLINK PACKAGES ***
%
%\usepackage{url}
% url.sty was written by Donald Arseneau. It provides better support for
% handling and breaking URLs. url.sty is already installed on most LaTeX
% systems. The latest version and documentation can be obtained at:
% http://www.ctan.org/tex-archive/macros/latex/contrib/url/
% Basically, \url{my_url_here}.




% *** Do not adjust lengths that control margins, column widths, etc. ***
% *** Do not use packages that alter fonts (such as pslatex).         ***
% There should be no need to do such things with IEEEtran.cls V1.6 and later.
% (Unless specifically asked to do so by the journal or conference you plan
% to submit to, of course. )


% correct bad hyphenation here
%\hyphenation{op-tical net-works semi-conduc-tor}


\begin{document}
%
% paper title
% Titles are generally capitalized except for words such as a, an, and, as,
% at, but, by, for, in, nor, of, on, or, the, to and up, which are usually
% not capitalized unless they are the first or last word of the title.
% Linebreaks \\ can be used within to get better formatting as desired.
% Do not put math or special symbols in the title.

%\title{Enhance Deep Visual Recognition under Adverse Conditions by Leveraging Robust Pre-training}
\title{Enhance Visual Recognition under Adverse Conditions via Deep Networks}
%
%
% author names and IEEE memberships
% note positions of commas and nonbreaking spaces ( ~ ) LaTeX will not break
% a structure at a ~ so this keeps an author's name from being broken across
% two lines.
% use \thanks{} to gain access to the first footnote area
% a separate \thanks must be used for each paragraph as LaTeX2e's \thanks
% was not built to handle multiple paragraphs
%

\author{Ding~Liu,~\IEEEmembership{Student Member,~IEEE,}
        Bowen~Cheng,
        Zhangyang~Wang,~\IEEEmembership{Member,~IEEE,}
        Haichao~Zhang,~\IEEEmembership{Member,~IEEE,}
        and~Thomas~S.~Huang,~\IEEEmembership{Life~Fellow,~IEEE}% <-this % stops a space
\thanks{The first two authors contributed equally to this work.}
%\thanks{* denotes equal contribution to this work.}
\thanks{This work was supported in part by US Army Research Office grant W911NF-15-1-0317.}
\thanks{D. Liu, B. Cheng and T. S. Huang are with the Department
of Electrical and Computer Engineering and Beckman Institute, Univerisity of Illinois at Urbana-Champaign, Urbana,
IL, 61801 USA e-mail: (dingliu2@illinois.edu; bcheng9@illinois.edu; t-huang1@illinois.edu).}% <-this % stops a space
%\thanks{B. Wen is with the Department
%	of Electrical and Computer Engineering and Coordinated Science Laboratory, Univerisity of Illinois at Urbana-Champaign, Urbana,
%	IL, 61801 USA e-mail: (bwen3@illinois.edu).}
%\thanks{Z. Wang is with Adobe Systems Inc., San Jose, CA 95110 USA (e-mail:zhawang@adobe.com).}
%\thanks{X. Liu is with Facebook Inc., San Francisco, CA 94025 USA (e-mail:	xmliu@fb.com).}
\thanks{Z. Wang is with the Department of Computer Science and Engineering, Texas A\&M University, TX 77843 USA (e-mail: atlaswang@tamu.edu).}
\thanks{H. Zhang is with Baidu Research, Sunnyvale, CA 94089 USA (e-mail: hczhang1@gmail.com).}
%\thanks{J. Doe and J. Doe are with Anonymous University.}% <-this % stops a space
%\thanks{Manuscript received April 19, 2005; revised September 17, 2014.}
}

% note the % following the last \IEEEmembership and also \thanks - 
% these prevent an unwanted space from occurring between the last author name
% and the end of the author line. i.e., if you had this:
% 
% \author{....lastname \thanks{...} \thanks{...} }
%                     ^------------^------------^----Do not want these spaces!
%
% a space would be appended to the last name and could cause every name on that
% line to be shifted left slightly. This is one of those "LaTeX things". For
% instance, "\textbf{A} \textbf{B}" will typeset as "A B" not "AB". To get
% "AB" then you have to do: "\textbf{A}\textbf{B}"
% \thanks is no different in this regard, so shield the last } of each \thanks
% that ends a line with a % and do not let a space in before the next \thanks.
% Spaces after \IEEEmembership other than the last one are OK (and needed) as
% you are supposed to have spaces between the names. For what it is worth,
% this is a minor point as most people would not even notice if the said evil
% space somehow managed to creep in.



% The paper headers
%\markboth{Journal of \LaTeX\ Class Files,~Vol.~13, No.~9, September~2014}%
%{Shell \MakeLowercase{\textit{et al.}}: Bare Demo of IEEEtran.cls for Journals}
% The only time the second header will appear is for the odd numbered pages
% after the title page when using the twoside option.
% 
% *** Note that you probably will NOT want to include the author's ***
% *** name in the headers of peer review papers.                   ***
% You can use \ifCLASSOPTIONpeerreview for conditional compilation here if
% you desire.




% If you want to put a publisher's ID mark on the page you can do it like
% this:
%\IEEEpubid{0000--0000/00\$00.00~\copyright~2014 IEEE}
% Remember, if you use this you must call \IEEEpubidadjcol in the second
% column for its text to clear the IEEEpubid mark.



% use for special paper notices
%\IEEEspecialpapernotice{(Invited Paper)}




% make the title area
\maketitle

% As a general rule, do not put math, special symbols or citations
% in the abstract or keywords.
\begin{abstract}

%Visual recognition under adverse conditions is an extremely challenging task with great practical values, due to the ubiquitous existence of quality distortions during image acquisition, transmission, or storage.
Visual recognition under adverse conditions is a very important and challenging problem of high practical value, due to the ubiquitous existence of quality distortions during image acquisition, transmission, or storage.
While deep neural networks have been extensively exploited in the techniques of low-quality image restoration and high-quality image recognition tasks respectively, few studies have been done on the important problem of recognition from very low-quality images. 
This paper proposes a deep learning based framework for improving the performance of image and video recognition models under adverse conditions, using robust adverse pre-training or its aggressive variant. 
%This paper proposes a systematic framework for improving deep learning-based image and video recognition models under adverse conditions, using robust adverse pre-training or its aggressive variant.
The robust adverse pre-training algorithms leverage the power of pre-training and generalizes conventional unsupervised pre-training and data augmentation methods. 
We further develop a transfer learning approach to cope with real-world datasets of unknown adverse conditions. 
The proposed framework is comprehensively evaluated on a number of image and video recognition benchmarks, and obtains significant performance improvements under various single or mixed adverse conditions. 
Our visualization and analysis further add to the explainability of results.



\end{abstract}

% Note that keywords are not normally used for peerreview papers.
\begin{IEEEkeywords}
deep learning, neural network, image recognition.
\end{IEEEkeywords}






% For peer review papers, you can put extra information on the cover
% page as needed:
% \ifCLASSOPTIONpeerreview
% \begin{center} \bfseries EDICS Category: 3-BBND \end{center}
% \fi
%
% For peerreview papers, this IEEEtran command inserts a page break and
% creates the second title. It will be ignored for other modes.
\IEEEpeerreviewmaketitle

% used for double column version
\section{Introduction}\label{sec:intro}%Adversity leads to prosperity

While the visual recognition research has made tremendous progress in recent years, most models are trained, applied, and evaluated on high-quality (HQ) visual data, such as the LFW \cite{LFW} and ImageNet \cite{Alex} benchmarks. 
However, in many emerging applications such as autonomous driving, intelligent video surveillance and robotics, the performances of visual sensing and analytics can be seriously endangered by different adverse conditions \cite{de2014face} in complex unconstrained scenarios, such as limited resolution, noise, occlusion and motion blur. 
For example, video surveillance systems have to rely on cameras of limited definitions, due to the prohibitive costs of installing high-definition cameras everywhere, leading to the practical need to recognize faces reliably from very low-resolution images \cite{VFR}. 
Other quality factors, such as occlusion and motion blur, are also known as critical concerns for commercial face recognition systems. 
As similar problems are ubiquitous for recognition tasks in the wild, it becomes highly desirable to investigate and improve the robustness of visual recognition systems to low-quality (LQ) image data.

\begin{figure}[t]
	\centering
%	\vspace{-5mm}
	\begin{minipage}{0.47\textwidth}
		\centering{
			\includegraphics[width=\textwidth]{intro.png}
	}\end{minipage}
	\caption{The original high-quality image from the MSRA-CFW dataset in (a), and (b) - (j) list various low-quality images generated from (a), that are all correctly recognized by our proposed models: (b) downsampled by a factor of 4; (c) 50\% salt \& pepper noise; (d) Gaussian noise (${\rm std} = 25$); (e) Gaussian blur (${\rm std} = 5$); (f)-(h) random synthetic occlusions; (i) downsampled by 4 followed by adding Gaussian noise (${\rm std} = 25$); (j) downsampled by 4 followed by adding Gaussian blur (${\rm std} = 5$).}
	\label{fig:intro}
	%	\vspace{-1mm}
\end{figure}

Unfortunately, exiting studies demonstrate that most state-of-the-art models appear fragile when applied on low-quality data.
The literature \cite{DVS12, ab14} has confirmed the significant effects of quality factors such as low-resolution, contrast, brightness, sharpness, focus, and illumination on commercial face recognition systems. 
The recent work \cite{karahan2016image} revealed that common degradations can even dramatically lower face recognition accuracy of the latest deep learning based face recognition models \cite{Alex, parkhi2015deep, szegedy2015going}. 
In particular, blur, noise, and %periocular region 
occlusion cause the most significant performance deterioration.
Besides face recognition, the low-quality data is also found to adversely affect other recognition applications, such as hand-written digit recognition \cite{basu2015learning} and style recognition \cite{vlrr}. 

 
This paper targets this important but less explored problem of visual recognition under adverse conditions. 
We study how and to what extent such adverse visual conditions can be coped with, aiming to improve the robustness of visual recognition systems on low-quality data. 
We carry out a comprehensive study on improving deep learning models for both image and video recognition tasks. 
We generalize conventional unsupervised pre-training and data augmentation methods, and propose the \textit{robust adverse pre-training} algorithms. 
The algorithms are generally applicable to various adverse conditions, and are jointly optimized with the target task. 
Figure~\ref{fig:intro}~(b)-(j) depict a series of heavily corrupted, low-quality images.  
They are all correctly recognized by our proposed models, though challenging even for human to recognize.

%Our previous work \cite{vlrr} introduced an approach of boosting image recognition performance via pre-training deep neural networks and weight sharing in the low resolution case, which can be considered as a preliminary work towards this direction. %The present work is built upon it and adds to it in significant ways. 
The major technical innovations are summarized in three aspects:
\begin{itemize}
	\item We present a framework for visual recognition under adverse conditions, that improves deep learning based models via robust pre-training and its aggressive variant. The framework is extensively evaluated on various datasets, settings and tasks. Our visualization and analysis further add to the explainability of results.
	\item We extend the framework to video recognition, and discuss how the temporal fusion strategy should be adjusted under different adverse conditions. 
	%Beyond object recognition, we also examine the task of video facial expression recognition. 
	\item We develop a transfer learning approach for real-world datasets of unknown adverse conditions without synthetic LQ-HQ pairs directly available. We empirically demonstrates that our approach also improves the recognition on the original benchmark dataset. 
	%with no explicitly-known degradation process. 
\end{itemize}%Another related scenario is that the acquired visual data have to be actively down-scaled for subsequent processing, due to resource constraints such as bandwidth and processing speed, which frequently occurs in mobile and wearable computer vision applications. 

In the following, we will first review related work in Section~\ref{sec:related}.
Our proposed robust adverse pre-training algorithm and its variant, as well as the corresponding image based experiments are introduced in Section ~\ref{sec:image}.
Video based experiments are reported with implementation details in Section~\ref{sec:video}.
The transfer learning approach for dealing with real-world datasets is described in Section~\ref{sec:unknown}.
Finally, conclusions and discussions are provided in Section~\ref{sec:conc}.
\section{Related Work}\label{sec:related}\subsection{Visual Recognition under Adverse Conditions}

In a real-world visual recognition problem, there is indeed no absolute boundary between LQ and HQ images. Yet as commonly observed, while some mild degradations may have negligible impact on the recognition performance, the impact will turn much notable once the level of adverse conditions passes some empirical threshold. The object and scene recognition literature reported a significant performance drop when the image resolution was decreased below $32 \times 32$ pixels \cite{80tiny}. In \cite{VFR}, the authors found the face recognition performance to be notably deteriorated when face regions became smaller than $16 \times 16$ pixels. \cite{karahan2016image} reported a rapid decline of face recognition accuracies, with Gaussian noise of standard deviation (std) between $10$ and $20$. \cite{DVS12, ab14} revealed more impacts of contrast, brightness, sharpness, and out-of-focus on image based face recognition. 

To resolve that, the conventional approach first resorts to image restoration and then feeds the restored image into a classifier \cite{fergus2006removing,yang2010image,liu2017robust}. Such a straightforward approach yields the sub-optimal performance: the artifacts introduced by the reconstruction process will undermine the final recognition. \cite{VFR, cvpr08} incorporated class-specific features in the restoration as a prior. \cite{zhang2011close} presented a joint image restoration and recognition method, based on the assumption that the degraded image, if correctly restored, will also have a good identifiability.
A similar approach was adopted for jointly dealing with image dehazing and object detection in \cite{li2017aod}.
Those ``close-the-loop'' ideas achieved superior performance over the traditional two-stage pipelines. 

Compared to single image object recognition, the impact of adverse conditions on video recognition is as profound and significant, with many attentions paid to tasks such as video face recognition and tracking \cite{stasiak2009face}, license plate recognition \cite{chen2007license}, and facial expression recognition \cite{tian2004evaluation}. \cite{shan2005recognizing} introduced robust hand-crafted features to low-resolution and head motion blur. \cite{arandjelovic2007manifold} combined a shape-illumination manifold framework with implicit super-resolution. \cite{herrmann2016low} adapted a residual neural network trained with synthetic LQ samples, which are generated by a controlled corruption process such as adding motion blur or compression artifacts. 

\subsection{Deep Networks under Adverse Conditions}

Convolutional neural networks (CNNs) have gained explosive popularity in recent years for visual recognition tasks \cite{Alex, karpathy2014large}. However, their robustness to adverse conditions remain unsatisfactory \cite{karahan2016image}. Deep networks were shown to be susceptible to adversarial samples \cite{goodfellow2014explaining}, generated by introducing carefully chosen perturbations to the input. 
%Although such ``worst case'' adverse condition is unlikely to occur in practice, 
Besides that,
the common  adverse conditions, stemming from artifacts during image acquisition, transmission, or storage, still easily mislead deep networks in practice \cite{dodge2016understanding}. \cite{karahan2016image} confirmed the fragility of the state-of-the-art deep face recognition models \cite{Alex, parkhi2015deep, szegedy2015going}, to various adverse conditions, in particular blur, noise, and periocular region occlusion. 
Besides face recognition, the adverse conditions are also found to negatively affect other recognition tasks, such as hand-written digit recognition \cite{basu2015learning} and style recognition \cite{vlrr}. 

While data augmentation has become a standard tool \cite{Alex}, the primary goal is to artificially  increase the training data volume and improve the model generalization. The augmentation methods are moderate in practice, by adding small noise or pixel translations, etc. The learned model is then to be applied on clean HQ images for testing. Those methods are thus not dedicated to handling specific types of severe degradation.
%heavy adverse conditions. 

Unsupervised pre-training \cite{erhan2009difficulty} also effectively regularizes the training process, especially when labeled data is insufficient. Classical pre-training methods reconstruct the input data from itself \cite{erhan2009difficulty} or its slightly transformed versions \cite{masci2011stacked}. 
%Our preliminary work \cite{vlrr} described the closest prior work to this paper, by 
The recent work \cite{vlrr} described an approach of
pre-training a deep network model for image recognition under the low-resolution case. 
However, it neither considered any other type of adverse conditions or mixed degradations\footnote{The  solutions to low-resolution cases cannot be straightforwardly extended to other adverse conditions. For example, we tried Model III of \cite{vlrr} in salt \& pepper noise and occlusion cases, finding the performance to be hurt sometimes.}, nor took into account any video based problem setting. Most crucially, \cite{vlrr} required pairs of synthetic training samples before and after degradation. While the degradation process is unknown in real-world data, the applicability of the proposed algorithm is severely limited.


\section{Image Based Visual Recognition under Single or Mixed Adverse Conditions}\label{sec:image}\subsection{Problem Statement}

We start by introducing single image based visual recognition models in this section, and extend to the video recognition models later. 
We define the visual recognition model $\M$ that predicts the category labels $\{l_i\}_{i=1}^N$ from the images $\{\mathbf{y}_i\}_{i=1}^N$. Due to the adverse conditions, $\{\mathbf{y}_i\}_{i=1}^N$ can be viewed as low-quality (LQ) images, degraded from high-quality (HQ) ground truth images $\{\mathbf{x}_i\}_{i=1}^N$. For now, we treat the original training datasets as HQ images $\{\mathbf{x}_i\}$, and generate LQ images $\{\mathbf{y}_i\}$ using synthetic degradation. In testing, our model operates with only LQ inputs.
%we assume that the $\{\mathbf{x}_i, \mathbf{y}_i\}_{i=1}^N$ pairs are available as the training dataset. We by default treat the original training datasets as HQ images $\{\mathbf{x}_i\}$ and perform various degradation to generate LQ images $\{\mathbf{y}_i\}$, as detailed in the sequel. 

We define a CNN based image recognition model $\M$ with $d$ layers. The first $d_1$ layers are convolutional, while the remaining $d - d_1$ layers are fully connected. The $i$-th convolutional layer, denoted as $conv_i$ ($i = 1, \cdots, d_1$), contains $n_i$ filters of size $c_i \times c_i$, with default stride size 1 and zero-padding. The $j$-th fully connected (fc) layer, denoted as $fc_j$ ($j =  1, \cdots, d - d_1$), has $m_j$ nodes. We use ReLU activation and apply dropout with a rate of 0.5 to fully connected layers. Cross-entropy loss is adopted for classification, while mean square error (MSE) is used for reconstruction. 

%not limited to deep learning. we tried in dictionary learning and also help\subsection{Robust Adverse Pre-training of Sub-models}

Building a classifier $\M$ directly on $\{\mathbf{y}_i\}$ is usually not robust due to the severe information loss  caused by adverse conditions. Training $\M$ over \{$\{\mathbf{x}_i\}$, $\{l_i$\}\} also does not perform well when tested on $\{\mathbf{y}_i\}$ due to the domain mismatch \cite{vlrr, VFR}. Our main intuition is to regularize and enhance the feature extraction from $\{\mathbf{y}_i\}$, via injecting auxiliary information from $\{\mathbf{x}_i\}$. With the help of $\{\mathbf{x}_i\}$, the model better discriminates the true signal from the severe corruption, and learns more robust filters from low-quality inputs. The entire $\M$ can be well adapted for the mapping from $\{\mathbf{y}_i\}$ to $\{l_i$\} by a joint optimization step followed.

%Pre-training has shown effective to enhance the model robustness to data variations \cite{erhan2009difficulty}. 
To pre-train $\M$, we first define the sub-model $\M_s$ with $k$ layers. Its first $k_p$ layers are configured the same as the first $k_p$ layers from $\M$. The last $k - k_p$ layers reconstruct the input image from the output feature maps of the $k_p$-th layer. 
We generate $\{\mathbf{y}_i\}$ from $\{\mathbf{x}_i\}$, based on a degradation process parameterized by the \textit{adverse factor}$\alpha$\footnote{Here the \textit{adverse factor} is defined in a broad sense. It can be the downsampling factor for low-resolution, the proportion of image for noise corruption, the degree of blur and so on.}, in order to meet the adverse conditions in testing. 
We then train $\M_s$ to reconstruct $\{\mathbf{x}_i\}$ from $\{\mathbf{y}_i\}$. 
We empirically find that pre-training only a part of convolutional layers (i.e., $k_p \leq d_1$) maintains a good balance between the feature extraction and the discrimination ability, with the best performance. After $\M_s$ is trained, its first $k_p$ layers are exported to initialize the first $k_p$ layers of $\M$. $\M$ is then jointly tuned for the recognition task over $\{\mathbf{y}_i, \{l_i\}$. The algorithm, termed as \textit{Robust Adverse Pre-training} (\textbf{RAP}), is outlined in Algorithm \ref{pretrain}.

\begin{algorithm}[h]
	%	\caption{Training $\M$ using robust adverse pre-training}	
	\caption{Robust adverse pre-training}
	\label{pretrain}
	\begin{algorithmic}[1]
		\REQUIRE Configuration of $\M$; $\{\mathbf{x}_i\}$ and $\{l_i\}$, $i$ = $1, 2,..., N$; the choice of $k$; the adverse factor $\alpha$.
		\STATE Generate $\{\mathbf{y}_i\}$ from $\{\mathbf{x}_i\}$, based on a degradation process parameterized by $\alpha$
		
		\STATE Construct the $k$-layer sub-model $\M_s$. Its first $k_p$ layers are configured identically to those of $\M$.
		
		\STATE Train $\M_s$ to reconstruct $\{\mathbf{x}_i\}$ from $\{\mathbf{y}_i\}$, under MSE.
		
		\STATE Export the first $k_p$ layers from $\M_s$ to initialize  the first $k_p$ layers of $\M$, where $k_p < k$.
		
		\STATE Tune $\M$ over \{$\{\mathbf{y}_i\}$, $\{l_i$\}\}, under the cross-entropy loss.
		
		\ENSURE $\M$.
	\end{algorithmic}
\end{algorithm}\subsection{Aggressively Robust Adverse Pre-training}
Different from testing when only LQ data is available, we have the flexibility to synthesize LQ images for training at our will. 
While the RAP algorithm  trains $\M$ and $\M_s$ under the same adverse condition, 
%which is met in testing. %Straightforward as this idea looks like, 
we continue to explore when the $\M_s$ pre-training and $\M$ joint-tuning are performed under different levels of adverse conditions. 
This is motivated by the denoising autoencoders~\cite{vincent2010stacked}, where the pre-training was conducted by noisy data and the subsequent classification model was learned with clean data. Our conjecture is that pre-training $\M_s$ in severer degradation can actually help $\M_s$ capture more robust feature mappings. 
This leads to the \textit{Aggressively Robust Adverse Pre-training} (\textbf{ARAP}), a variant of RAP, outlined in Algorithm \ref{dpretrain}. We assume the degradation process of $\{\mathbf{y}_i\}$ to be identical to the target testing data, while $\{\mathbf{z}_i\}$ is a more heavily degraded set independently generated from $\{\mathbf{x}_i\}$. 
The larger adverse factor indicates the severer degradation, 
and thus in this case the adverse factor $\beta$ for generating $\{\mathbf{z}_i\}$ is larger than $\alpha$ for $\{\mathbf{y}_i\}$. 
RAP can be a special case of ARAP where $\alpha$ and $\beta$ coincide. 

%As a potential alternative, we tried a weighted function of cross-entropy and MSE, and trained $\M$ from end to end under this hybrid loss. Its performance was shown to hinge on the choice of the weight, which was difficult to decide as the two loss values were often on distinct magnitude levels. Adjusting learning rates also appeared to be critical to ensure its proper converge. In comparison, RAP/ARAP are less sensitive to hyper-parameters and more efficient to train.\begin{algorithm}[ht]
	\caption{Aggressively robust adverse pre-training}
	\label{dpretrain}
	\begin{algorithmic}[1]
		\REQUIRE Configuration of $\M$; $\{\mathbf{x}_i\}$ and $\{l_i\}$, $i$ = $1, ..., N$; the choice of $k$; two adverse factors $\alpha$ and $\beta$ ($\beta > \alpha$).
		
		\STATE Generate $\{\mathbf{y}_i\}$, $\{\mathbf{z}_i\}$ from $\{\mathbf{x}_i\}$, based on two degradation processes parameterized by $\alpha$ and $\beta$, respectively.
		
		\STATE Construct the sub-model $\M_s$ same as in Algorithm 1.
		
		\STATE Train $\M_s$ to reconstruct $\{\mathbf{x}_i\}$ from $\{\mathbf{z}_i\}$, under MSE.
		
		\STATE Export the first $k_p$ layers from $\M_s$ to initialize the first $k_p$ layers of $\M$, where $k_p < k$.
		
		\STATE Tune $\M$ over \{$\{\mathbf{y}_i\}$, $\{l_i$\}\}, under the cross-entropy loss.
		
		\ENSURE $\M$.
	\end{algorithmic}
\end{algorithm}\subsection{Experiments on Benchmarks}\subsubsection{Object Recognition on the CIFAR-10 Dataset}\begin{table*}
	%\small
	\fontsize{10pt}{12pt}\selectfont
	\caption{The top-1 and top-5 classification accuracy (\%) on the CIFAR-10 dataset, where LQ images are generated by downsampling the original images with a factor of $\alpha$ = 2.}
	\begin{center}
		\begin{tabular}{c|c|c|c|c|c|c|c|c}
			\hline
			& HQ & LQ-2 & RAP-2-non-joint & RAP-2 & ARAP-2-4 & ARAP-2-8 & ARAP-2-12 & ARAP-2-16 \\ 
			\hline
			\hline
			Top-1  & 67.43 & 60.79 & 46.89 & 62.12 & 62.80 & \textbf{63.31} & 62.91 & 62.56 \\
			%\hline
			Top-5  & 96.61 & 95.32 & 90.77 & 95.10 & 95.52 & \textbf{95.80} & 95.34 & 95.10 \\ 
			\hline
			%& LQ-2-joint-4 & LQ-2-joint-8 & LQ-2-joint-12 & LQ-2-joint-16 \\ \hline
			%Top-1 & 62.80\% & 63.31\% & 62.91\% & 62.56\% \\\hline
			%Top-5 & 95.52\% & 95.80\% & 95.34\% & 95.10\% \\ \hline
		\end{tabular}
	\end{center}
	
	\label{CIFARlr2}
	%\vspace{-1em}
\end{table*}\begin{table}
	%\footnotesize
	\fontsize{10pt}{12pt}\selectfont
	\caption{The top-1 and top-5 classification accuracy (\%) on the CIFAR-10 dataset, where LQ images are generated by adding $\alpha$ = 50\% salt \& pepper noise.}
	\begin{center}
		%\begin{tabular}{|@{\hskip 1mm}c@{\hskip 1mm}|@{\hskip 1mm}c@{\hskip 1mm}|@{\hskip 1mm}c@{\hskip 1mm}|@{\hskip 1mm}c@{\hskip 1mm}|@{\hskip 1mm}c@{\hskip 1mm}|}
		\begin{tabular}{@{\hskip 1mm}c@{\hskip 1mm}|@{\hskip 1mm}c@{\hskip 1mm}|@{\hskip 1mm}c@{\hskip 1mm}|@{\hskip 1mm}c@{\hskip 1mm}|@{\hskip 1mm}c@{\hskip 1mm}}
			\hline
			& HQ & LQ-50\% & RAP-50\%-no-joint & RAP-50\% \\ \hline
			\hline
			Top-1  & 67.43 & 33.46 & 38.64 & \textbf{50.32} \\ %\hline
			Top-5  & 96.61 & 83.22 & 86.86 & \textbf{92.03} \\ \hline
		\end{tabular}
	\end{center}
	
	\label{CIFARsalt}
	%\vspace{-1em}
\end{table}\begin{table*}[t]
	%\footnotesize
	\fontsize{10pt}{12pt}\selectfont
	\caption{The top-1 and top-5 classification accuracy (\%) on the CIFAR-10 dataset, where LQ images are generated by blurring original images (HQ), with Gaussian kernel of std $\alpha$ = 2.}
	\begin{center}
		\begin{tabular}{c|c|c|c|c|c|c|c}
			\hline
			% & HQ & LQ-2 & RAP-2-non-joint \\ \hline
			%Top-1 & 67.43\% & 52.62\% & 39.80\%  \\ \hline
			%Top-5 & 96.61\% & 92.70\% & 87.34\%  \\ \hline
			& HQ & LQ-2 & RAP-2-non-joint & RAP-2 & RAP-2-5 & RAP-2-8 & RAP-2-9 \\ \hline
			\hline
			Top-1  & 67.43 & 52.62 & 39.80 & 54.73 & 54.77 & \textbf{55.67} & 54.35 \\ 
			Top-5  & 96.61 & 92.70 & 87.34 & 93.24 & 93.50 & \textbf{93.52} & 93.15 \\ 
			\hline
		\end{tabular}
		%\begin{tabular}{|c|c|c|c|}
		%\hline
		%RAP-2 & ARAP-2-5 & ARAP-2-8 & ARAP-2-9 \\ \hline
		% 54.73\% & 54.77\% & \textbf{55.67\%} & 54.35\% \\ \hline
		% 93.24\% & 93.50\% & \textbf{93.52\%} & 93.15\% \\ \hline
		%\end{tabular}
	\end{center}
	
	\label{CIFARblur2}
	%\vspace{-0.5em}
\end{table*}

In order to validate our algorithm, we first conduct object recognition on
the CIFAR-10 dataset \cite{cifar}, which consists of  60,000 color images of $32 \times 32$ pixels from 10 classes (we convert all to grayscale ones). Each class has 5,000 training images and 1,000 test images. We generate LQ images as per each specific type of adverse conditions, where the adverse factors $\alpha$ or $\beta$ become concrete degradation hyper-parameters such as downsampling factor, noise level, or blur kernel. We perform no other data augmentation beyond generating LQ images. 
%mean subtraction, normalization, or any 

We choose $\M$ with $d = 4$, with $d_1 = 3$ convolutional layers, followed by $d - d_1 = 1$ fully connected layer with $m_1$ always equaling the number of classes. 
Unless otherwise stated, we set $\M_s$ as a fully convolutional network with the empirical values $k=3, k_p = 2$, which work well in all experiments. The default configuration of convolutional layers are: $n_1$ = 64, $c_1$ = 9; $n_2$ = 32, $c_2$ = 5; $n_3$ = 20, $c_3$ = 5. We first train $\M_s$ with learning rate 0.0001, and then jointly tune $\M$ with a learning rate 0.001 for the first $k_p$ layers and 0.01 for the rest $d - k_p$ layers. Both learning rates are reduced by a factor of 10 every 5,000 iterations. 
%We carefully tuned our architecture, ranging from simple auto-encoders to fully-convolutional networks, with the purpose to ensure all methods to reach their best performance. We strived to confirm that the comparison results were not tied to any specific model configuration. Once accepted, we will release all architectures that we tried, the training codes and their results. \paragraph{Low-Resolution}%\textbf{Low Resolution} 
We generate LQ (low-resolution) images $\{\mathbf{y}_i\}$ by following the process in \cite{dong2014learning,liu2016robust}: first downsampling the HQ (high-resolution) images $\{\mathbf{x}_i\}$ by a factor of $\alpha$, then upsampling back to the original size with bicubic interpolation. 
We use the same process for all the following experiments of low-resolution degradation, unless otherwise stated.
We compare the following approaches:
%\vspace{-0.5em}\begin{itemize}
	%\setlength\itemsep{-0.35em}
	\item \textbf{HQ:} $\M$ is trained and tested on $\{\{\mathbf{x}_i\}, \{l_i\}\}$. 
	\item \textbf{LQ-$\alpha$:} $\M$ is trained and tested on $\{\{\mathbf{y}_i\}, \{l_i\}\}$. 
	\item \textbf{RAP-$\alpha$-non-joint:} $\M_s$ is pre-trained using the Step 3 of Algorithm \ref{pretrain} on $\{\{\mathbf{y}_i\},\{\mathbf{x}_i\}\}$. The remaining $d-k_p$ layers of $\M$ are then trained on $\{\{\mathbf{y}_i\}, \{l_i\}\}$, with the first $k_p$ pre-trained layers fixed. It is identical to RAP except for no jointly tuning $\M$. 
	\item \textbf{RAP-$\alpha$:}  $\M$ is trained using RAP (Algorithm \ref{pretrain}).
	\item \textbf{ARAP-$\alpha$-$\beta$:}  $\M$ is trained using ARAP (Algorithm \ref{dpretrain}), where $\beta$ is a larger downsamping factor than $\alpha$.
	%Generate $\{\mathbf{y}_i\}$ and $\{\mathbf{z}_i\}$ with downsampling factors $\alpha$ and $\beta$ respectively ($\beta \ge \alpha$), and train $\M$ using Algorithm \ref{dpretrain}. LQ-$\alpha$-$\beta$-joint is reduced to LQ-$\alpha$-joint when $\alpha$ = $\beta$.
\end{itemize}%\vspace{-0.5em}
The evaluation of $\M$s is all performed on the testing set of LQ images (except for the HQ baseline), downsampled by the factor $\alpha$. The first two baselines aim to examine how much the adverse condition affects the performance. 
%We also tried to blend HQ and LQ images as the new training set and train $\M$ from end to end on the new set\cite{vlrr}, but found it only marginally superior to LQ-$\alpha$. 

Table \ref{CIFARlr2} displays the results at $\alpha = 2$, which is a challenging problem of recognizing objects from   images of $16\times16$ pixels. Such an adverse condition dramatically affects the performance, by dropping the top-1 accuracy for nearly 7\%, after comparing LQ-2 with HQ. It might be unexpected that the performance of RAP-2-non-joint is much inferior to that of LQ-2.  As observed in this and many following experiments, the reconstruction based pre-training step, if not jointly optimized for the recognition step, often hurts the performance rather than does any help. By adding the joint tuning step, RAP-2 gains a 1.33\% advantage over LQ-2 in the top-1 accuracy, which is owning to the $\M_s$ pre-training that involves auxiliary yet beneficial information from HQ data. 
%It outperformed LQ-2-non-joint dramatically, justifying the necessity of end-to-end tuning of $\M$. 

It is noteworthy that all four ARAP methods ($\beta = 4, 8, 12, 16$) show superior results over RAP-2. ARAP-2-8 achieves the best accuracy of 63.31\% (top-1) and 95.80\% (top-5). The observation confirms our conjecture that more robust feature extractions could be achieved by purposely pre-training $\M_s$ in severer degradation ($\beta > \alpha$). As $\beta$ grows with $\alpha$ fixed at 2, the performance of ARAP first improves and then drops, with the peak at $\beta = 8$. That is also explainable, since if $\{\mathbf{z}_i\}$ are too much degraded, little information is left for training $\M_s$. 

%\hline% & HQ & LQ-2 & RAP-2-non-joint & RAP-2 & RAP-2-5 & RAP-2-8 & RAP-2-9 \\ \hline%Top-1 & 67.43\% & 52.62\% & 39.80\% & 54.73\% & 54.77\% & \textbf{55.67\%} & 54.35\% \\ \hline%Top-5 & 96.61\% & 92.70\% & 87.34\% & 93.24\% & 93.50\% & \textbf{93.52\%} & 93.15\% \\ \hline%\hline%\begin{table}%\footnotesize%\begin{center}%\begin{tabular}{|c|c|c|c|c|}%\hline% & HQ & LQ & RAP-non-joint & RAP\\ \hline%Top-1 & 67.43\% & 47.89\% & 37.38\% & \textbf{49.01\%} \\ \hline%Top-5 & 96.61\% & 90.58\% & 85.95\% & \textbf{91.51\%} \\ \hline%\end{tabular}%\end{center}%\caption{The top-1 and top-5 classification accuracy on the CIFAR-10 dataset, where LQ images are generated by first downsampling original images (HQ) by a factor of 2 and then adding Gaussian noise with std 25.}%\label{CIFARgausslr}%\vspace{-1em}%\end{table}\paragraph{Noise}%\textbf{Noise} 
Since adding moderate Gaussian noise has been standard for data augmentation, we focus on the more destructive salt \& pepper noise. The LQ images $\{\mathbf{y}_i\}$ are generated by randomly choosing $\alpha = 50\%$ pixels in each HQ image $\mathbf{x}_i$ to be replaced with either 0 or 255. We compare HQ, LQ-$\alpha$, RAP-$\alpha$-non-joint, and RAP-$\alpha$, all of which are similarly defined as in the low-resolution case. We tried RAP-$\alpha$-$\beta$, but did not get much performance improvement over RAP-$\alpha$ as we did for low-resolution. In Table \ref{CIFARsalt}, the severe information loss by 50\% salt \& pepper noise is reflected on the 34\% top-1 accuracy drop from HQ to LQ-50\%. After only pre-training the first few layers, there is a 5.18\% increase in the top-1 accuracy, obtained by RAP-50\%-non-joint. RAP-50\% achieves the closest accuracy to the HQ baseline, and outperforms RAP-50\%-non-joint by 11.68\% and 5.17\%, in terms of top-1 and top-5 accuracy, respectively. Those results re-confirm the necessity of both per-training and end-to-end tuning for RAP.

\begin{table}
	%\footnotesize
	\fontsize{10pt}{12pt}\selectfont
	\caption{The top-1 and top-5 face identification accuracy (\%) on the MSRA-CFW dataset, where LQ images are generated by downsampling original images by a factor of $\alpha$ = 4.}
	\begin{center}
		\begin{tabular}{@{\hskip 1mm}c@{\hskip 1mm}|@{\hskip 1mm}c@{\hskip 1mm}|@{\hskip 1mm}c@{\hskip 1mm}|@{\hskip 1mm}c@{\hskip 1mm}|@{\hskip 1mm}c@{\hskip 1mm}|@{\hskip 1mm}c@{\hskip 1mm}}
			\hline
			& HQ & LQ-4 & RAP-4-non-joint & RAP-4 & ARAP-4-6\\ \hline
			\hline
			Top-1  & 57.25 & 50.79 & 50.50 & \textbf{54.23} & 54.10\\ 
			Top-5  & 76.89 & 72.81 & 72.88 & 74.06 & \textbf{74.97}\\ \hline
		\end{tabular}
	\end{center}
	
	\label{msralr}
	%\vspace{-0.5em}
\end{table}\begin{table}
	%\footnotesize
	\fontsize{10pt}{12pt}\selectfont
	\caption{The top-1 and top-5 face identification accuracy (\%) on the MSRA-CFW dataset, where LQ images are generated by adding $\alpha$ = 50\% salt \& pepper noise.}
	\begin{center}
		%\begin{tabular}{@{\hskip 1mm}c@{\hskip 1mm}|@{\hskip 1mm}c@{\hskip 1mm}|@{\hskip 1mm}c@{\hskip 1mm}|@{\hskip 1mm}c@{\hskip 1mm}|@{\hskip 1mm}c@{\hskip 1mm}}
		\begin{tabular}{c|c|c|c|c}
			\hline
			& HQ & LQ-50\% & RAP-50\% & RAP-50\% \\ \hline
			\hline
			Top-1  & 57.25 & 14.75 & 26.20 & \textbf{49.86} \\ 
			Top-5  & 76.89 & 36.28 & 51.59 & \textbf{72.14} \\ \hline
		\end{tabular}
	\end{center}
	
	\label{msrasalt}
	%\vspace{-0.5em}
\end{table}\begin{table}
	%\footnotesize
	\fontsize{10pt}{12pt}\selectfont
	\caption{The top-1 and top-5 face identification accuracy (\%) on the MSRA-CFW dataset, where LQ images are generated by blurring the original images (HQ), with Gaussian kernel of std $\alpha$ = 5.}
	\begin{center}
		\begin{tabular}{@{\hskip 1mm}c@{\hskip 1mm}|@{\hskip 1mm}c@{\hskip 1mm}|@{\hskip 1mm}c@{\hskip 1mm}|@{\hskip 1mm}c@{\hskip 1mm}|@{\hskip 1mm}c@{\hskip 1mm}|@{\hskip 1mm}c@{\hskip 1mm}}
			\hline
			& HQ & LQ-5 & RAP-5-non-joint & RAP-5 & ARAP-5-8 \\ \hline
			\hline
			Top-1  & 57.25 & 49.96 & 45.66 & \textbf{52.19} & 51.94\\ 
			Top-5  & 76.89 & 72.51 & 69.08 & 73.73 & \textbf{73.88}\\ \hline
		\end{tabular}
	\end{center}
	
	\label{msrablur}
	%\vspace{-0.5em}
\end{table}\paragraph{Blur}%\textbf{Blur}
Images commonly suffer from various types of blurs, such as simple Gaussian blur, motion blur, out-of-focus blur, or their complex combinations \cite{zhang2011close}. We focus on the Gaussian blur, while similar strategies can be naturally extended to other types. The LQ images $\{\mathbf{y}_i\}$ are generated by convolving the HR images $\{\mathbf{x}_i\}$ with a Gaussian kernel with std $\alpha = 2$, and the fixed kernel size of $9\times9$ pixels. We compare HQ, LQ-$\alpha$, RAP-$\alpha$-non-joint, RAP-$\alpha$, and ARAP-$\alpha$-$\beta$ ($\beta$ denotes a larger std than $\alpha$), all similarly defined. 

Table \ref{CIFARblur2} demonstrates similar findings as the low-resolution case. The non-adapted restoration in RAP-$\alpha$-non-joint only leaves it worse than LQ-$\alpha$. RAP-$\alpha$ gains 1.21\% over LQ-$\alpha$ in top-1 accuracy. Two out of three ARAP methods ($\beta = 5, 8$) yield greatly improved results than RAP-$\alpha$, while $\beta = 9$ is only marginally inferior. Using Algorithm \ref{dpretrain}, $\M_s$ trained with heavier blurs 
tends to produce more discriminative features, when applied to LQ data with lighter blurs, which benefits recognition tasks.
%tends to reconstruct overly enhanced edges and sharper textures, when applied to LQ data with lighter blurs. While not visually favorable, they may supply helpful details for recognition.%\textbf{Mixed Adverse conditions}%In real-world applications, an adverse condition hardly appears by itself. To this end, we examine if the proposed algorithms remain effective under a mixture of multiple adverse conditions. We conduct a simple simulation  on CIFAR-10: the LQ images $\{\mathbf{y}_i\}$ are generated by first downsampling HQ images $\{\mathbf{x}_i\}$ by a factor of $2$, then adding Gaussian noise with std of $25$. Table \ref{CIFARgausslr} compares \textbf{HQ}, \textbf{LQ},  \textbf{RAP-non-joint}, and \textbf{RAP}, and the results are consistent with the previous ones, confirming that the proposed algorithms seamlessly generalize when mixed adverse conditions are present.%\begin{table}
	%\footnotesize
	\fontsize{10pt}{12pt}\selectfont
	\caption{The top-1 and top-5 accuracy (\%) on MSRA-CFW, where LQ images are generated with random synthetic occlusions.}
	\begin{center}
		\begin{tabular}{c|c|c|c|c}
			\hline
			& HQ & LQ-$\alpha$ & RAP-$\alpha$-no-joint & RAP-$\alpha$ \\ \hline
			\hline
			Top-1  & 59.41 & 32.62 & 34.91 & \textbf{43.96} \\ 
			Top-5  & 78.11 & 56.32 & 60.16 & \textbf{67.20} \\ \hline
		\end{tabular}
	\end{center}
	
	\label{msrahole}
	%\vspace{-0.5em}
\end{table}\begin{table}
	%\footnotesize
	\fontsize{10pt}{12pt}\selectfont
	\caption{The top-1 and top-5 accuracy (\%) on MSRA-CFW, where LQ images are generated by first downsampling original images by $\alpha$ = 2 and then adding Gaussian noise with std 25.}
	\begin{center}
		\begin{tabular}{@{\hskip 1mm}c@{\hskip 1mm}|@{\hskip 1mm}c@{\hskip 1mm}|@{\hskip 1mm}c@{\hskip 1mm}|@{\hskip 1mm}c@{\hskip 1mm}|@{\hskip 1mm}c@{\hskip 1mm}|@{\hskip 1mm}c@{\hskip 1mm}}
			\hline
			& HQ & LQ-2 & RAP-2-non-joint & RAP-2 & ARAP-2-4 \\ \hline
			\hline
			Top-1  & 57.25 & 45.57 & 44.30 & 48.63 & \textbf{50.34}\\ 
			Top-5  & 76.89 & 69.82 & 68.00 & 71.89 & \textbf{73.76}\\ \hline
		\end{tabular}
	\end{center}
	
	\label{msragausslr}
	%\vspace{-0.5em}
\end{table}\begin{table}
	%\footnotesize
	\fontsize{10pt}{12pt}\selectfont
	\caption{The top-1 and top-5 accuracy (\%) on MSRA-CFW, where LQ images are generated by first downsampling original images by $\alpha$ = 4 and then blurred with the Gaussian kernel of std 2.}
	\begin{center}
		\begin{tabular}{@{\hskip 1mm}c@{\hskip 1mm}|@{\hskip 1mm}c@{\hskip 1mm}|@{\hskip 1mm}c@{\hskip 1mm}|@{\hskip 1mm}c@{\hskip 1mm}|@{\hskip 1mm}c@{\hskip 1mm}|@{\hskip 1mm}c@{\hskip 1mm}}
			\hline
			& HQ & LQ-4 & RAP-4-non-joint & RAP-4 & ARAP-4-8\\ \hline
			\hline
			Top-1  & 57.25 & 49.39 & 48.76 & 52.30 & \textbf{52.68}\\ 
			Top-5  & 76.89 & 71.29 & 70.99 & 73.80 & \textbf{74.51}\\ \hline
		\end{tabular}
	\end{center}
	
	\label{msrablurlr}
	%\vspace{-1em}
\end{table}\subsubsection{Face Identification on the MSRA-CFW Dataset}%The MSRA Dataset of Celebrity Faces on the Web (MSRA-CFW) \cite{zhang2012finding} includes $202, 792$ cropped and centered face images of size $64\times64$, with around $1600$ classes. We pick a subset including all the 123 classes with more than $300$ images. We split 90\% images of each class for training and 10\% for testing. We perform the face identification task, under highly challenging adverse conditions.

We conduct face identification on the MSRA Dataset of Celebrity Faces on the Web (MSRA-CFW) \cite{zhang2012finding}, 
which includes $202, 792$ cropped and centered face images of $64\times64$ pixels in around $1600$ classes. We select a subset including all the 123 classes of more than $300$ images, to ensure the sufficient amount of training data for our deep network model. We split 90\% images of each class for training and 10\% for testing. We perform the face identification task, under highly challenging adverse conditions, such as very low resolution, noise, blur, occlusion and mixed cases. The visual examples are displayed in Figure~\ref{fig:intro}.

For the low resolution, noise or blur case, we set $\M$ with $d = 8$,  $d_1 = 6$. The convolutional layers are configured as: $n_1 = 32$, $c_1 = 9$;  $n_2 = 16$, $c_2 = 5$;  $n_3 = 20$, $c_3 = 4$; $n_4 = 40$, $c_4 = 3$; $n_5 = 60$, $c_5 = 3$; $n_6 = 80$, $c_6 = 2$. $fc_1$ has $m_1 = 160$ and $fc_2$ has $m_2 = 123$. For occlusion, we modify $n_1 = 16$, $c_1 = 21$;  $n_2 = 8$, $c_2 = 1$, and leave other six layers unchanged. Here the low-level filters perform in-painting, and thus needs larger receptive fields to predict missing pixels from neighborhoods. 
%We find that using very large $c_1$ is critical for the occlusion case%, and the overall compact configuration ensures proper convergence. %$conv_1$, $conv_2$ and $conv_3$ are each followed by $2\times2$ max pooling, which was found to improve generalization. \paragraph{Low-Resolution, Noise, Blur}%\textbf{Low Resolution, Noise, Blur} 
The three adverse conditions follow similar settings and comparison methods to CIFAR-10. We adopt a larger downsampling factor 4 in the low-resolution case, and a larger blur std 5 for the blur case. The conclusions drawn from Tables \ref{msralr}, \ref{msrasalt} and \ref{msrablur} are also consistent to those of CIFAR-10: RAP boosts much performance in all cases compared to LQ and RAP-non-joint, and ARAP achieves considerably higher results for the two cases of low-resolution and blur.

%We adopt a larger downsampling factor $\alpha = 4$. Table \ref{msralr} compares the results by HQ, LQ-$\alpha$, RAP-$\alpha$-non-joint, RAP-$\alpha$ and ARAP-$\alpha$-$\beta$ ($\beta$ = 6). Low resolution causes the top-1 accuracy to drop nearly 7\%. RAP-$\alpha$ boosts up to 4\% and achieves the highest top-1 accuracy. ARAP-4-6 further reduces its top-5 accuracy gap with HQ to less than 2\%, while maintaining a competitive top-1 accuracy.%%\textbf{Noise} The results of HQ, LQ-$\alpha$, RAP-$\alpha$-non-joint and RAP-$\alpha$ are compared in Table \ref{msrasalt}. With $\alpha$ = 50\% salt \& pepper noise added, the top-1 accuracy witnesses a huge drop of nearly 43\%. LQ-$\alpha$-joint is able to restore the performance tremendously: its top-5 accuracy is almost as good as the HQ baseline. In contrast, RAP -$\alpha$-non-joint only slightly ameliorates the accuracy loss. %%%\textbf{Blur} We generate blurred LQ face images with a larger blur standard deviation $\alpha = 5$. HQ, LQ-$\alpha$, RAP-$\alpha$-non-joint, RAP-$\alpha$, and ARAP-$\alpha$-$\beta$ ($\beta$ = 8) are compared in Table \ref{msrablur}. The observations are consistent with the CIFAR-10 blur case: RAP-$\alpha$ gains 2.23\% over LQ-$\alpha$ in the top-1 accuracy, and ARAP-$\alpha$-$\beta$ further improves the performance.\paragraph{Occlusion}%\textbf{Occlusion} 
Prior studies in \cite{karahan2016image} discovered that the periocular occlusion degraded the face recognition performance most. We follow \cite{karahan2016image} to synthesize the occlusions for the periocular regions, in the shape of either rectangle or ellipse (chosen with equal probability). The size of either shape, as well as the pixel values within the synthetic occlusion, is drawn from uniform distributions. The center locations of synthetic occlusions are picked randomly in a bounding box, whose boundaries are determined by eye landmark points. We emphasize that the occlusion masks are unknown and changing for both training and testing, corresponding to the toughest blind inpainting problem \cite{xie2012image}. 

We evaluate HQ, LQ-$\alpha$, RAP-$\alpha$-non-joint and RAP-$\alpha$ in Table \ref{msrahole}. The parameter $\alpha$ generally denotes the controlled shape/size/location variations. We also tried large $\beta$ via enlarging the maximal size of occlusions, but observed no visible improvement from ARAP-$\alpha$-$\beta$. The occlusion causes much worse corruptions than previous adverse conditions: it completely masks a facial region that is known to be critical for recognition. The lost pixel information is harder to be restored than the salt \& pepper noise case, due to the missing neighborhood. As expected, the challenging random occlusions result in very significant drops from HQ to LQ. RAP-non-joint only marginally raises the accuracy (e.g., 2\% in top-1). RAP achieves the most encouraging improvements of 11.34\% and 10.88\%, in terms of top-1 and top-5 accuracy, respectively.
%showing remarkable robustness to occlusions.\paragraph{Mixed Adverse Conditions}%\textbf{Mixed Adverse Conditions}
In real-world applications, multiple types of degradation may appear simultaneously.
%an adverse condition hardly appears by itself. 
To this end, we examine if our algorithms remain effective under a mixture of multiple adverse conditions.  We evaluate two settings: 1) first downsampling HQ images by $\alpha$ = 2 and then adding Gaussian noise with std 25; 2) first downsampling HQ images by $\alpha$ = 4 and then blurring with the Gaussian kernel of std 2. We compare HQ, LQ-$\alpha$, RAP-$\alpha$-non-joint, RAP-$\alpha$ and ARAP-$\alpha$-$\beta$, where $\alpha$ and $\beta$ both only consider the downsampling factor for simplicity. ARAP and RAP seamlessly generalize to the mixed adverse conditions, and obtain the most promising performance in Tables \ref{msragausslr} and \ref{msrablurlr}%As expected, both Table \ref{msragausslr} and Table \ref{msrablurlr} manifest the most promising performance of ARAP and then RAP. %and ARAP-$\alpha$-$\beta$ achieves the most promising performance in both experiments.\begin{table*}[htb]
	%	\footnotesize
	\fontsize{10pt}{12pt}\selectfont
	\caption{The top-1 and top-5 digit recognition accuracy (\%) on the SVHN dataset, where LQ images are generated by blurring the original images (HQ), with the Gaussian kernel of standard deviation $\alpha$ = 2.}	
	\begin{center}
		\begin{tabular}{c|c|c|c|c|c|c}
			\hline
			& HQ & LQ-2 & RAQ-2-non-joint & RAQ-2 & ARAQ-2-5 & ARAQ-2-8 \\ 
			%			& & & non-joint & -joint & joint & joint\\ 
			\hline
			\hline
			Top-1 & 89.23 & 85.40 & 83.84 & 82.47 & \textbf{89.40} & 88.29 \\
			Top-5 & 98.57 & 97.55 & 96.92 & 96.82 &  \textbf{98.32} & 98.09 \\ \hline
		\end{tabular}
	\end{center}
	
	\label{svhnblur}
\end{table*}\begin{figure}[b]
	\centering
	%	\begin{minipage}{1\textwidth}
	%		\centering {
	\includegraphics[width=0.8\linewidth]{SVHNselect.png}
	%	}\end{minipage}
	\caption{Digit image samples from the SVHN dataset.}
	\label{fig:SVHN}
	%\vspace{-0.5em}
\end{figure}\subsubsection{Digit Recognition on the SVHN Dataset}%For digit recognition task, our classification model has a default configuration of $d = 4$, $d_1 = 2$; $n_1 = 20$, $c_1 = 5$; $n_2 = 50$, $c_2 = 5$; $m_1 = 500$; $m_2 = 10$ (class number used). $conv_1$ is followed by $2\times2$ max pooling.

The Street View House Number (SVHN) dataset \cite{netzer2011reading} contains $73, 257$ digit images of $32\times32$ pixels for training, and $26, 032$ for testing. We focus on investigating the impact of low-resolution and blur on the SVNH digit recognition. Our model has a default configuration of $d = 4$, $d_1 = 2$; $n_1 = 20$, $c_1 = 5$; $n_2 = 50$, $c_2 = 5$; $m_1 = 500$; $m_2 = 10$ (class number used). $conv_1$ is followed by $2\times2$ max pooling.

%The size of each image is $32\times32$ and some of the images have a main digit in the center and some distraction digits on the border. We have done two experiments to see the effect of low resolution and blur on digits classification.\paragraph{Low-Resolution}
Table \ref{svhnlr} compares HQ, LQ-$\alpha$, RAP-$\alpha$-non-joint, RAP-$\alpha$ and ARAP-$\alpha$-$\beta$, in the low-resolution case with $\alpha$ = 8. While the LQ-$\alpha$ accuracy drops disastrously, satisfactory top-1 and top-5 accuracy is achieved by ARAP-$\alpha$-$\beta$ ($\beta$ = 16) and RAP-$\alpha$. We observe that more than half of digit images could still be correctly predicted at the extremely low-resolution of $4\times4$ pixels by the proposed methods. 

\begin{table}
	%	\scriptsize
	\fontsize{10pt}{12pt}\selectfont
	\caption{The top-1 and top-5 digit recognition accuracy (\%) on the SVHN dataset, where LQ images are downsampling the original images (HQ) by a factor of $\alpha$ = 8.}	
	\begin{center}
		\begin{tabular}{@{\hskip 1mm}c@{\hskip 1mm}|@{\hskip 1mm}c@{\hskip 1mm}|@{\hskip 1mm}c@{\hskip 1mm}|@{\hskip 1mm}c@{\hskip 1mm}|@{\hskip 1mm}c@{\hskip 1mm}|@{\hskip 1mm}c@{\hskip 1mm}}
			\hline
			& HQ & LQ-8 & RAP-8-non-joint & RAP-8 & ARAP-8-16 \\ \hline
			\hline
			Top-1 & 89.23 & 19.60 & 45.98 & 51.00 & \textbf{51.17} \\ 
			Top-5 & 98.57 & 65.44 & 87.08 & \textbf{89.15} & 89.06 \\ \hline
		\end{tabular}
	\end{center}
	
	\label{svhnlr}
\end{table}\paragraph{Blur}
Table \ref{svhnblur} compares those methods in the Gaussian blur case with standard deviation $\alpha$ = 2. To our astonishments, ARAP-$\alpha$-$\beta$ not only improves over LQ-$\alpha$, but also surpasses the performance of HQ in terms of top-1 accuracy. That is because the original SVNH images (treated as HQ) are real-world photos that unavoidably suffer from certain blur, which can be found in Figure \ref{fig:SVHN}. Convolved with the synthetic Gaussian blur kernel ($\alpha$ = 2), the actual blur kernel's standard deviation becomes larger than 2. 
Hence ARAP-$\alpha$-$\beta$ is potentially able to remove the inherent blurs in HQ images, besides the synthetically added blurs.
%ARAP-$\alpha$-$\beta$ thus performs better than %We find low resolution by a factor of 2 or 4 has little influence on LQ images classification, thus we challenge LQ images = low resolution by a factor of 8. We first downsample the $32\stimes32$ images to size $4\stimes4$ and then upsample them back to original size using bicubic interpolation. Results are in Table \ref{svhnlr}. We find that using $4\stimes4$ information can hardly predict a digit and 19.60\% is only 9.6\% better than randomly guess (10\%). However, it is amazing that half of the digits can still be recoginized with only $4\stimes4$ information by using joint pre-training method.\begin{table*}[htb]
	%\small
	\fontsize{10pt}{12pt}\selectfont
	\caption{The top-1 and top-5 classification accuracy (\%) on the ImageNet validation set, where LQ images are downsampled by a factor of $\alpha$ = 4 or 8.}
	\begin{center}
		\begin{tabular}{c|c|c|c|c|c|c|c}
			\hline
			& HQ & LQ-4 & RAP-4-non-joint & RAP-4 & LQ-8 & RAP-8-non-joint & RAP-8  \\ 
			\hline
			\hline
			Top-1  & 71.46 & 61.92 & 61.16 & 62.03 & 46.67 & 45.37 & 47.22  \\
			%\hline
			Top-5  & 90.62 & 84.13 & 83.65 & 84.35 & 71.55 & 70.60 & 72.32  \\ 
			\hline
		\end{tabular}
	\end{center}
	
	\label{tab:imagenetlr}
	%\vspace{-1em}
\end{table*}\subsubsection{Image Classification on the ImageNet Dataset}\label{sec:imagenet}%For digit recognition task, our classification model has a default configuration of $d = 4$, $d_1 = 2$; $n_1 = 20$, $c_1 = 5$; $n_2 = 50$, $c_2 = 5$; $m_1 = 500$; $m_2 = 10$ (class number used). $conv_1$ is followed by $2\times2$ max pooling.
We validate our algorithm on a large-scale dataset, ImageNet dataset \cite{imagenet}, for image classification of 1,000 classes.
We utilize 1.2 million images of ILSVRC2012 training set for training, and 50,000 images of its validation set for testing.
We study the degradation of low-resolution on the ImageNet image classification.
In our experiment, we customize a popular classification model: \textit{VGG-16}\cite{simonyan2014very} to work on color images directly. 
Specifically, we add three convolutional layers to the beginning of VGG-16, in order to increase the model capacity for handling the low-resolution degradation.
We choose $k = 3, k_p=3$ for $\M_s$ and the configuration of the first three convolutional layers is $n_1$ = 64, $c_1$ = 9; $n_2$ = 32, $c_2$ = 1; $n_3$ = 3, $c_3$ = 5.
The rest architecture is the same as VGG-16.
We use the VGG-16 model released by its authors as the initialization of it, in order to boost the convergence rate.
We follow the conventional protocols in \cite{simonyan2014very} for data pre-processing, including image resizing, random cropping and mean removal of each color channel.



Table \ref{tab:imagenetlr} compares HQ, LQ-$\alpha$, RAP-$\alpha$-non-joint and RAP-$\alpha$, in the low-resolution case with $\alpha$ = 4 and 8.
RAP-4 outperforms LQ-4 and RAP-4-non-joint in terms of both top-1 and top-5 accuracy.
When the low-resolution degradation becomes severe, RAP-8 is superior to LQ-8 and RAP-8-non-joint by a larger margin. 
Specifically, RAP-8 beats LQ-8 by 0.55\% in top-1 accuracy and 0.77\% in top-5 accuracy, and beats RAP-8-non-joint by 1.85\% in top-1 accuracy and 1.72\% in top-5 accuracy, respectively.

\subsubsection{Face Detection on the FDDB Dataset}

We further generalize our proposed algorithm to the face detection task.
We use the training images of the WIDER Face dataset \cite{yang2016wider} as our training set, which consists of 12,880 images and the annotations of 159,424 faces.
%which consists of 32,203 images and the annotations of 393,703 faces.
and adopt the Face Detection Data Set and Benchmark (FDDB) \cite{fddbTech} as our test set, which contains the annotations for 5,171 faces in a set of 2,845 images.
We study the degradation of low-resolution for the face detection task.
In our experiment, we customize a popular detection model: \textit{Faster R-CNN}\cite{ren2015faster} to work on color images directly. 
Similar to Section \ref{sec:imagenet}, we add three convolutional layers to the beginning of Faster R-CNN, in order to increase the model capacity for handling the low-resolution degradation.
We choose $k = 3, k_p=3$ for $\M_s$ and the configuration of the first three convolutional layers is $n_1$ = 64, $c_1$ = 9; $n_2$ = 32, $c_2$ = 1; $n_3$ = 3, $c_3$ = 5.
The rest architecture is the same as Faster R-CNN.
We use the VGG-16 model in \cite{simonyan2014very} released by its authors as initialization, in order to accelerate the convergence speed.
%We follow the conventional protocols in \cite{simonyan2014very} for data pre-processing, including image resizing, random cropping and mean removal of each color channel.\begin{figure}[htbp]
	\centering
	\begin{minipage}{0.23\textwidth}
		\centering \subfigure[]{
			\includegraphics[width=\textwidth]{LQ4_Disc2.png}
	}\end{minipage}
	\begin{minipage}{0.23\textwidth}
		\centering \subfigure[]{
			\includegraphics[width=\textwidth]{LQ4_Cont2.png}
	}\end{minipage}
	\caption{(a) Discrete ROC curve and (b) Continuous ROC curve on FDDB dataset, where LQ images are downsampled by a factor of $\alpha$ = 4.}
%	\vspace{-0.5em}
	\label{fig:face_detection}
\end{figure}

Figure \ref{fig:face_detection} shows the discrete and continuous ROC curves of HQ, LQ-$\alpha$, RAP-$\alpha$-non-joint and RAP-$\alpha$, in the low-resolution case with $\alpha$ = 4.
We can observe that there is an obvious performance drop due to the low-resolution degradation.
RAP-4 outperforms LQ-4 and RAP-4-non-joint in terms of recall rate with the same number of false positives.
For example, RAP-4 recalls 50.49\% faces with 2,000 false positives, which is 0.73\% higher than RAP-4-non-joint and 2.55\% higher than LQ-4, respectively.
We obtain the same comparison result in the case of 1,500 false positives, where RAP-4 recalls 48.68\% faces, being 0.67\% higher than RAP-4-non-joint and 3.15\% higher than LQ-4, respectively.

\subsection{Analysis and Visualization}\subsubsection{Convolutional and Additive Adverse Conditions}
We have tested four adverse conditions so far. RAP and ARAP improves the recognition in all cases, which shows that the pre-training of image restoration achieves feature enhancement in the recognition model and benefits the visual recognition task.  
We note that low-resolution and blur clearly receive extra bonus from ARAP than RAP. In the other two cases, i.e., noise and occlusion, RAP and ARAP perform approximately the same. Such contrastive behaviors hint that some adverse conditions might be more suitable for ARAP to perform than the others. 

In the general image degradation model, the observed image $Y$ is usually represented as
%$Y = \mathcal{F} \ast X + e$, \begin{equation}
Y = \mathcal{F} \ast X + e,
\end{equation}
where $\mathcal{F}$ denotes the point spread function, $X$ is the clean image, and $e$ is the noise. Low-resolution and blur are usually modeled in $ \mathcal{F}$ as low-pass filters, while noise and occlusion can be incorporated in $e$ as additive perturbations. We term the former category as \textbf{convolutional adverse conditions}, and the latter as \textbf{additive adverse conditions}. 
We conjecture that the additive adverse condition causes pixel-wise corruptions but still retains some structural information, 
while the convolutional adverse condition results in global detail loss and smoothening, which may be more challenging for recognition and thus needs more robust feature extractions by purposely pre-training $\M_s$ in heavier adverse conditions.
%Visual recognition under convolutional adverse conditions significantly prefer ARAP with larger $\beta$. 
This hypothesis will be further justified experimentally when we extend our framework to video cases.

%\vspace{-0.5em}%Both RAP and ARAP have similar effects under additive ones. %visual recognition under the former is more challenging, and thus will significantly prefer ARAP with larger $\beta$, while both RAP and ARAP have similar effects under the latter. The categorization of adverse conditions are further justified experimentally when our model are extended to video cases.%low-resolution and blur: signals are largely lost %??%salt \& pepper noise: signal needs to be discriminated with additive noise %?? \subsubsection{Effects of End-to-End Tuning in RAP}\begin{figure}[htbp]
	\centering
	\begin{minipage}{0.46\textwidth}
		\centering{
			\includegraphics[width=\textwidth]{success.png}
	}\end{minipage}
	\caption{Visualized features for successful examples of joint tuning, i.e. those correctly classified by RAP but misclassified by RAP-non-joint. Column (a): original HQ images from MSRA-CFW. (b): LQ images from the first mixed adverse condition setting. (c): visualized $\mathcal{F}_k$ (intermediate features by RAP-non-joint). (d): visualized $\mathcal{F}'_k$ (intermediate features by RAP). }
	\label{success}
	%\vspace{-0.5em}
\end{figure}

To further analyze our proposed RAP, we focus on the following two questions:
How the joint tuning of $\M$ modifies the features learned in the pre-trained $\M_s$, and why it improves the recognition in almost all adverse conditions? 

%To provide an empirical answer, 
To answer these questions, we visualize and compare the features in the first $k_p$-th layers of $\M$ before and after the end-to-end tuning, denoted as $\mathcal{F}_k$ and $\mathcal{F}'_k$, respectively.
%(see supplementary for the visualization method). 
Recall that in the pre-training step of RAP, $\M_s$ reconstructs the images by feeding $\mathcal{F}_k$ to $k-k_p$ additional layers, that are removed in the joint tuning step. We pass both $\mathcal{F}_k$ and $\mathcal{F}'_k$ through the fixed mapping of these $k-k_p$ layers (obtained when training $\M_s$). The output, which is of the same dimension as HQ images, is used to visualize of $\mathcal{F}_k$ or $\mathcal{F}'_k$. Note that the visualizations of $\mathcal{F}_k$ are just the reconstruction results of $\M_s$. 
%while those of $\mathcal{F}'_k$ are expected to be enhanced versions.Figure \ref{success} presents feature visualizations for five MSRA-CFW images that are correctly classified by RAP but misclassified by RAP-non-joint. As shown in column (c), the $\mathcal{F}_k$ features from the un-tuned $\M_s$ are heavily over-smoothed, with much discriminative information lost. In contrast, the visualizations of $\mathcal{F}'_k$ yield a few impressive restoration results in column (d). The joint tuning step enables the closed-loop consideration of two information sources (HQ data and labels) for two related tasks (restoration and recognition).  It thus boosts not only the recognition accuracy, but also the restoration: column (d) results contain much richer and finer details, and are apparently more recognizable than column (c). 

%Recall that in the pre-training step of RAP, $\M_s$ reconstructs the images by feeding $\mathcal{F}_k$ to one additional layer (denoted as $\ell_{add}$), that was removed before the joint tuning step. We thus pass both $\mathcal{F}_k$ and $\mathcal{F}'_k$ through the fixed mapping of $\ell_{add}$ (obtained when training $\M_s$). The output, which will be of the same dimension as HQ images, is used to visualize of $\mathcal{F}_k$/$\mathcal{F}'_k$. The visualizations of $\mathcal{F}_k$ are just the reconstruction results of $\M_s$, while those of $\mathcal{F}'_k$ are expected to be enhanced versions.%\begin{figure}[tbp]%\centering%\begin{minipage}{0.46\textwidth}%\centering \end{minipage}%\caption{Feature visualization for failure examples of joint tuning, i.e. those that are correctly classified by LQ-non-joint but misclassified by LQ-joint. Column (a)-(d) are the same as Figure \ref{success}.}%\label{fail}%\end{figure}%\vspace{-0.5em}%We also include the visualized features of several failure cases of joint tuning in the supplementary. Despite the existence of individual failures, the superiority of LQ-joint over LQ-non-joint has been consistent and overwhelming through all the datasets that we have tested on.%Figure \ref{fail} visualizes features of several failure cases of joint tuning, i.e., those that are correctly classified using LQ-non-joint but misclassified by LQ-joint. Compared to Figure \ref{success}, we observe that column (d) results of Figure \ref{fail} fail to suppress noticeable visual artifacts. Despite the existence of such individual failures cases, the superiority of LQ-joint over LQ-non-joint has been consistent and overwhelming through all the datasets that we have tested on.%\subsubsection{Effects of Varying $\beta$ with ARAP}%%\begin{figure}[htbp]%\centering%\begin{minipage}{0.10\textwidth}%\centering \subfigure[] \end{minipage}%\begin{minipage}{0.10\textwidth}%\centering \subfigure[] \end{minipage}%\begin{minipage}{0.10\textwidth}%\centering \subfigure [] \end{minipage}%\begin{minipage}{0.10\textwidth}%\centering \subfigure [] \end{minipage}%\caption{(a) LQ image by downsampling the original image (HQ) by a factor of $\alpha$ = 2; (b) visualized $\mathcal{F}'_k$ by RAP; (c) visualized $\mathcal{F}'_k$ by ARAP with $\beta$ = 4; (d) visualized $\mathcal{F}'_k$ by ARAP with $\beta$ = 8.}%%\vspace{-0.5em}%\label{real}%\end{figure}%%%ARAP is oftenmore competitive than RAP. %To qualitatively show how the choice of $\beta$ affects ARAP, we construct a simple case where the LQ images are downsampled from the original with $\alpha = 2$ in Figure \ref{real}(a). We then train several models using ARAP-$\alpha$-$\beta$ with varied $\beta = 2, 4, 8$, and visualize $\mathcal{F}'_k$s for those models. Among Figure \ref{real} (b) - (d), although (b) might look more smoothened and visually pleasing, (c) preserves sharper details that are potentially favored by recognition, while avoiding the heavy artifacts in (d). The best recognition accuracy in this setting is achieved by $\beta = 4$.
\section{Video Recognition in Adverse Conditions}\label{sec:video}\subsection{Temporal Fusion for Video Based Models}

Temporal fusion of feature representations is usually adopted in deep learning based methods for video-related tasks.
Karpathy et al.\\cite{karpathy2014large} first provided an extensive empirical evaluation of CNNs on large-scale video classification. In addition to the single frame baseline, \cite{karpathy2014large} discussed three connectivity
patterns. The \textbf{early fusion} combines frames within a time window immediately in the
pixel level. 
%The \textbf{late fusion} places separate single frame columns that are not merged until the first fully connected layer. 
The \textbf{late fusion} separately extracts features from each frame  and does not merge them until the first fully connected layer.
The \textbf{slow fusion} is a balanced mix between the two, which slowly unifies temporal information throughout the network by progressively merging features from individual frames. 
%The authors discovered that slow fusion consistently performs better than other alternatives. \subsection{Robust Adverse Pre-training for Video Recognition}

Following \cite{karpathy2014large}, we treat each video as a number of short, fixed-sized clips. Each clip is set to contain $2T + 1$ contiguous frames in time.
% $\mathcal{F}_{t-T}$, ...,  $\mathcal{F}_{t}$, ...,  $\mathcal{F}_{t+T}$ 
The video based CNN model $\M_v$ takes a clip as its input. To extend $\M_v$ to adverse conditions, we first pre-train a single image model $\M$ using RAP or ARAP, by treating all frames as individual images and formulating an image based recognition problem. We then convert $\M$ to $\M_v$ based on different fusion strategies, and initialize the weights of $\M_v$ from $\M$ using the weight transfer proposed in \cite{kappeler2016video}. $\M_v$ is then tuned in the video setting. Since we find the late fusion results to be always inferior to the other two, we omit discussing the case of late fusion hereinafter.

For early fusion, we copy the $conv_1$ layer of $\M$ ($n_1$ filters of $c_1 \times c_1$) for $2T + 1$ times, and divide the weights of all filters by $2T+1$\footnote{
Detailed reasoning follows Section III.C of \cite{kappeler2016video}. Our early and slow fusion models resemble their architectures (a) and (b). }. 
%We then concatenate them into the new $conv_1$ layer of $\M_v$ with the size $n_1 \times c_1 \times c_1 \times (2T+1)$. 
We then use them in the new $conv_1$ layer of $\M_v$ with the size $n_1 \times c_1 \times c_1 \times (2T+1)$, to fuse information in the first layer.
All other layers of $\M_v$ are identical with $\M$ in both configuration and weight transfer. 

For slow fusion, we copy the $conv_1$ layer of $\M$ for $2T + 1$ times into the new $conv_1$ layer of $n_1 \times c_1 \times c_1 \times (2T+1)$, without changing the weights. 
%We then stack the $conv_2$ layer of $\M$  ($n_2$ filters of $c_c \times c_c$)  for $2T + 1$ times followed by dividing all filters' elements by $2T+1$, constituting the new $conv_2$ layer of $n_2 \times c_2 \times c_2 \times (2T+1)$. 
We then stack the filters of the $conv_2$ layer of $\M$  ($n_2$ filters of $c_c \times c_c$)  for $2T + 1$ times and divide all weights by $2T+1$, constituting the new $conv_2$ layer of $n_2 \times c_2 \times c_2 \times (2T+1)$ to fuse information in the second layer.
All other layers of $\M_v$ remain identical to $\M$.
% in both configuration and weight transfer. %Architecture%Early fusion: 5 x f1 x f1 x C1 -> C1 x f2 x f2 x C2;%Slow fusion: 1 x f1 x f1 x C1 (five layers like this one) -> 5C1 x f2 x f2 x C2%%First, we train the corresponding single frame VLQR model. Then we transfer weights of all layers to multi-frame model. The filters of layers 2, 3 and 4 are the same in early fusion. The first layer is initialized by w_video = 1/5*w. b_video = b. The filters of layers 1, 3 and 4 are the same in slow fusion. The second layer is initialized by w_video = 1/5*w. b_video = b. Learning rate starts from 0.0001.\subsection{Experiments on Benchmarks}

We use a video face dataset: the \textit{YouTube Face} (YTF) benchmark \cite{wolf2011face} to validate our algorithm.
We choose the 167 subject classes that contain 4 video sequences.
%We form a subset from the \textit{YouTube Face} (YTF) benchmark \cite{wolf2011face}, by choosing the 167 subject classes that contain 4 video sequences. 
For each class, we randomly pick one video for testing and the rest for training. The face regions are cropped using the given bounding boxes. As the majority of cropped faces have side lengths between 56 and 68, we slightly resize them all to $60 \times 60$ for simplicity, and refer to those as the \textit{original YTF set} hereinafter. We densely sample clips of 5 $(T = 2)$ frames  from each video with a stride of one frame, and present each clip individually to the model. The class predictions are averaged to produce an estimate of the video-level class probabilities. For the single image model,  we chose $d = 5$, $d_1 = 4$, with each layer: $n_1$ = 64, $c_1$ = 9; $n_2$ = 32, $c_2$ = 5; $n_3$ = 60, $c_3$ = 4; $n_4$ = 80, $c_4$ = 3, $m_1$ = 167. All video based models start from the same pre-trained single frame model, and then split filters differently. We enforce filter symmetry as in \cite{kappeler2016video}.  The detailed architectures are drawn in Figure~\ref{fig:video_net}.  

\begin{figure*}[htbp]
	\centering
%	\begin{minipage}{1\textwidth}
%		\centering {
			\includegraphics[width=0.9\textwidth]{Network.png}
%	}\end{minipage}
	\caption{Model architectures for YTF video recognition experiments. Top: early fusion. Bottom: slow fusion.}
	\label{fig:video_net}
	%\vspace{-0.5em}
\end{figure*}\begin{table*}[t]
	%\footnotesize
	\fontsize{10pt}{12pt}\selectfont
	\caption{The top-1 and top-5 accuracy (\%) on YTF, in the low resolution setting, with different fusion strategies.}
	\begin{center}
		\begin{tabular}{c|c|c|c|c|c|c}
			\hline
			& & HQ & LQ-2 & RAP-2 & ARAP-2-4 & ARAP-2-8 \\ \hline
			\hline
			Single  & Top-1 & 37.32 & 38.30 & 39.16 & 41.05 & 38.58 \\
			%\cline{2-7}
			Frame & Top-5 & 60.01 & 59.56 & 59.94 & 61.97 & 60.33 \\ 
			\hline
			\hline
			Early  & Top-1 & 38.11 & 37.73 & 39.83 & \textbf{41.11} & 38.05 \\
			%\cline{2-7}
			Fusion & Top-5 & 58.48 & 62.42 & 62.74 & \textbf{63.85} & 60.79 \\ 
			\hline
			\hline
			Slow  & Top-1 & 35.99 & 37.76 & 39.60 & 40.98 & 39.67\\
			%\cline{2-7}
			Fusion & Top-5 & 53.20 & 58.79 & 60.86 & 63.03 & 61.50 \\ 
			\hline
		\end{tabular}
	\end{center}
	
	\label{YTF_lr}
	%\vspace{-1em}
\end{table*}

Similarly to image based experiments, Tables \ref{YTF_lr} and \ref{YTF_noise} compare HQ, LQ-$\alpha$, RAP-$\alpha$, and ARAP-$\alpha$-$\beta$, in the settings of low resolution ($\alpha$ = 2) and salt \& pepper noise ($\alpha$ = 50\%). ARAP/RAP bring substantially improved performance within each fusion. Recall that the best fusion models in \cite{karpathy2014large} displayed only modest improvement over single frame models (from 59.3\% to 60.9\%), we consider that our 1.11\% top-5 gain by early fusion in the low resolution setting, and 13.37\% top-5 gain by slow fusion in the noise setting, are both reasonably good. 

%More interestingly, 
While \cite{karpathy2014large} advocated slow fusion for normal visual recognition problems, the situations seem more complicated when adverse conditions step in. Our results imply that additive adverse conditions favor slow fusion, while convolutional adverse conditions prefer early fusion. 
We tried experiments in the blur case, whose observations are close to the low resolution case. 
We conjecture that  early fusion becomes the preferred option when the data is already heavily ``filtered'' by degradation operators or blur kernels, such that it cannot afford extra information loss after more filtering. The diverse fusion preferences manifest the unique complication brought by adverse conditions. 
%In practice, the network fusion needs to be decided in accordance with the adverse condition type.\begin{table}[tbp]
%\footnotesize
	\fontsize{10pt}{12pt}\selectfont
	\caption{The top-1 and top-5 accuracy (\%) on YTF, in the salt \& pepper noise setting, with different fusion strategies.}
\begin{center}
\begin{tabular}{c|c|c|c|c}
\hline
 & & HQ & LQ-50\% & RAP-50\% \\ \hline
 \hline
Single  & Top-1 & 37.32 & 15.81 & 31.64 \\
%\cline{2-5}
 Frame & Top-5 & 60.01 & 30.93 & 48.48 \\ 
\hline
\hline
Early  & Top-1 & 38.11 & 18.86 & 21.20  \\
%\cline{2-5}
Fusion & Top-5 & 58.48 & 36.59 & 38.01  \\ 
\hline
\hline
Slow  & Top-1 & 35.99 & 21.97 & \textbf{34.55} \\
%\cline{2-5}
Fusion & Top-5 & 53.20 & 39.00 & \textbf{52.37}  \\ 
\hline
\end{tabular}
\end{center}

\label{YTF_noise}
%\vspace{-0.5em}
\end{table}

As the last finding, in the low resolution case, the RAP and ARAP results using LQ data can even surpass HQ results notably. We input the original YTF set to the trained ARAP-2-4 models, and also witnesses much improved accuracy in Table \ref{YTF_HQ}, than feeding the same set through the HQ models. The best top-1 and top-5 results in Table \ref{YTF_HQ} also surpass all results in Table \ref{YTF_lr}. We suspect that although the original YTF set is treated as clean and high-quality, it was actually contaminated by degradations during image collection, and is thus low-quality from that viewpoint. Applying RAP and ARAP compensates  part of the unknown information loss. From another perspective, training a model on LQ data and then applying on HQ data is related to a special  data augmentation introduced in \cite{vlrr}, that blends HQ and LQ data for training. While \cite{vlrr} confirmed its effectiveness in recognizing LQ subjects, we discover its usefulness for normal (HQ) visual recognition too. 

%$LR-2$ performs slightly better than $HQ$. Again counterintuitive as it looks, we suspect %\vspace{-0.5em}\begin{table}[h]
%\small
	\fontsize{10pt}{12pt}\selectfont
	\caption{The top-1 and top-5 accuracy (\%) by feeding the original YTF set to the trained ARAP-2-4 models.}
\begin{center}
\begin{tabular}{c|c|c|c}
\hline
 & Single Frame & Early Fusion & Slow Fusion \\ \hline
 \hline
Top-1 & 41.31 & 41.60  & \textbf{42.20} \\ 
Top-5 & 62.30 & \textbf{64.04} & 63.10\\ \hline
\end{tabular}
\end{center}

\label{YTF_HQ}
%\vspace{-0.5em}
\end{table}%high-SNR case: noise quickly suppressed and signal can be transformed multiple times for better feature representation %low SNR case: the signal and noise are now comparable, and the feature might loss faster than the noise during filtering!
\section{Coping with Unknown Adverse Conditions: A Transfer Learning Approach}\label{sec:unknown}

In all previous experiments, we train with $\{\mathbf{x}_i, \mathbf{y}_i\}_{i=1}^N$ pairs. That is equivalent to assuming a pre-known degradation process from $\{\mathbf{x}_i\}_{i=1}^N$ to $\{\mathbf{y}_i\}_{i=1}^N$. Such an assumption, as made in \cite{vlrr}, is impractical for real-world LQ data and restricts our experiments to synthesized test data so far. In this section, we develop a transfer learning approach to significantly relax this strong assumption. It ensures the wide applicability of our algorithms, even when the degradation parameters cannot be accurately inferred.

For convolutional adverse conditions, the recognition accuracy is usually peaked at some optimal $\beta^* > \alpha$.
%and gradually lowers down as $\beta$ either increases more or decreases. 
The additive adverse conditions seems insensitive to $\beta$. However, the performance ARAP-$\alpha$-$\beta$ results ($\beta > \alpha$) are observed to be always better, or at least comparable to ARAP-$\alpha$, even when $\beta$ deviates far away from $\beta^*$.
%We hardly observed any exception in all experiments. \begin{algorithm}[ht]
	\caption{Transfer ARAP Learning}
	\label{transferA}
	\begin{algorithmic}[1]
		\REQUIRE Configurations of $\M$ and $\M'$; the choice of $k$; the clean source dataset $\{\mathbf{x}_i\}$ and $\{l_i\}$; the target dataset $\{\mathbf{x}'_i\}$ and $\{l'_i\}$, with unknown $\alpha'$.
		
		\STATE Decide the major degradation type in $\{\mathbf{x}'_j\}$, and choose $\beta'$ such that it overestimates $\alpha'$.
		
		\STATE Generate $\{\mathbf{y}_i\}$ from $\{\mathbf{x}_i\}$, based on the degradation processes of the major type, parameterized by $\beta'$.
		
		\STATE Perform Steps 3 - 6 in Algorithm 1, to train $\M$ on the source dataset. 
		
		\STATE Export the first $k$ layers from $\M$ to initialize the first $k$ layers of $\M'$.
		
		\STATE Tune $\M'$ over \{$\{\mathbf{x}'_i\}$, $\{l'_i$\}\}.
		
		\ENSURE $\M'$.
	\end{algorithmic}
\end{algorithm}%\vspace{-0.5em}\begin{table}[h]
	%\small
	\fontsize{10pt}{12pt}\selectfont
	\caption{The top-1 and top-5 accuracy (\%) on the original YTF set, by transferring from the MSRA-CFW RAP-4 model.}
	\begin{center}
		\begin{tabular}{c|c|c|c|c}
			\hline
			& LQ$_d$ & LQ$_p$ & T-ARAP-non-joint & T-ARAP\\ \hline
			\hline
			Top-1 & 32.65 & 32.35 & 33.67 & \textbf{34.77} \\ 
			Top-5 & 45.37 & 47.73 & 48.11 & \textbf{53.11}\\ \hline
		\end{tabular}
	\end{center}
	
	\label{transfer}
	%\vspace{-0.5em}
\end{table}On a target dataset with real-world corruptions, it is reasonable to assume that the major type of adverse condition(s) can still be identified, but the parameter $\alpha'$ of the underlying degradation process cannot be accurately estimated. Observing the robustness of ARAP w.r.t. $\beta$, we propose the \textit{Transfer ARAP Learning} (\textbf{T-ARAP}) approach, as detailed in Algorithm \ref{transferA}. The core idea is to first choose $\beta'$ that we empirically believe $\beta' > \alpha'$, then performing RAP (with $\beta' $) to train $\M$ on a source dataset. Next, we transfer the learned sub-model of $\M$ to initialize $\M'$, which is later tuned for the target dataset. Note that $\beta'$ is not necessarily very close to $\alpha'$. In practice, one may safely start with some large $\beta'$, and scan backwards for an optimal value.  

We validate the approach via conducting the following experiment: improving face identification on (original) YTF by referring to a RAP model on MSRA-CFW. For simplicity, here we perform the task of single-image face identification, and treat the original YTF set as an image collection without utilizing temporal coherence. We visually observe that the original YTF images have inherently lower quality, which is also supported by Table \ref{YTF_HQ}. We select low resolution as our target adverse condition, and not too aggressively, choose $\beta'$ = 4. We hence take the $\M_s$ part from the RAP-4 model trained on MSRA-CFW, to initialize the first 2 layers of $\M'$. Meanwhile, we design three baselines for comparison: 
1) \textbf{LQ}$_d$ model trained directly end-to-end on YTF; 
2) \textbf{LQ}$_p$ model trained on YTF with classical unsupervised layer-wise pre-training; 
3) \textbf{T-ARAP-non-joint}, taking the untuned $\M_s$ of RAP-4 for $\M'$ initialization. In Table \ref{transfer}, T-ARAP improves the top-5 recognition accuracy by nearly 8\% over the naive \textbf{LQ}$_d$, with no strong prior knowledge about the degradation process nor its parameter, 
%--- nothing beyond some observation and reasonable guess.
which demonstrates the effectiveness of our proposed transfer learning approach.

%Experiments in Section 4 also find that feeding original images through RAP or ARAP models leads to extra improvements than HQ. That suggests that the original YTF set suffers from unknown degradations. %We also downsampled YTF by $\gamma$ times, creating three YTF$_{\gamma}$ variants: $\gamma$ = 1.5, 2, 3 ($\gamma$ is \textit{unknown} to all methods in experiments). We compare the four methods again, and confirm: (1) the consistent superiority of VLQR-$M$; (2) the margin going up when $\gamma$ increases. We will add a new section in final version to include all those results.%\section{Beyond Objects: Expression Recognition from Videos under Adverse Conditions}%%\subsection{CNN-RNN Model}%%\subsection{Experiments on the AVEC 2015 Benchmark}%%AVEC results

% used for single column version
%\input{intro_s.tex}
%\input{related_s.tex}
%\input{proposed_s.tex}
%\input{align_s.tex}
%\input{experiments_s.tex}

\section{Conclusions and Discussions}\label{sec:conc}

This paper systematically improves deep learning models via robust pre-training for image and video recognition under adverse conditions. We thoroughly evaluate our proposed algorithm on various datasets and degradation settings, and analyze our results in depth, 
which shows the effectiveness of our proposed algorithm.
A transfer learning approach is also proposed to enhance the real-world applicability.  
%Below summarize our insights gained from extensive experiments. We hope them to be of reference values for more future work.

%\section{Introduction}
% The very first letter is a 2 line initial drop letter followed
% by the rest of the first word in caps.
% 
% form to use if the first word consists of a single letter:
% \IEEEPARstart{A}{demo} file is ....
% 
% form to use if you need the single drop letter followed by
% normal text (unknown if ever used by IEEE):
% \IEEEPARstart{A}{}demo file is ....
% 
% Some journals put the first two words in caps:
% \IEEEPARstart{T}{his demo} file is ....
% 
% Here we have the typical use of a "T" for an initial drop letter
% and "HIS" in caps to complete the first word.
%\IEEEPARstart{T}{his} demo file is intended to serve as a ``starter file''
%for IEEE journal papers produced under \LaTeX\ using
%IEEEtran.cls version 1.8a and later.
% You must have at least 2 lines in the paragraph with the drop letter
% (should never be an issue)
%I wish you the best of success.

%\hfill mds
 
%\hfill September 17, 2014

%\subsection{Subsection Heading Here}
%Subsection text here.

% needed in second column of first page if using \IEEEpubid
%\IEEEpubidadjcol

%\subsubsection{Subsubsection Heading Here}
%Subsubsection text here.


% An example of a floating figure using the graphicx package.
% Note that \label must occur AFTER (or within) \caption.
% For figures, \caption should occur after the \includegraphics.
% Note that IEEEtran v1.7 and later has special internal code that
% is designed to preserve the operation of \label within \caption
% even when the captionsoff option is in effect. However, because
% of issues like this, it may be the safest practice to put all your
% \label just after \caption rather than within \caption{}.
%
% Reminder: the "draftcls" or "draftclsnofoot", not "draft", class
% option should be used if it is desired that the figures are to be
% displayed while in draft mode.
%
%\begin{figure}[!t]
%\centering
%\includegraphics[width=2.5in]{myfigure}
% where an .eps filename suffix will be assumed under latex, 
% and a .pdf suffix will be assumed for pdflatex; or what has been declared
% via \DeclareGraphicsExtensions.
%\caption{Simulation results for the network.}
%\label{fig_sim}
%\end{figure}

% Note that IEEE typically puts floats only at the top, even when this
% results in a large percentage of a column being occupied by floats.


% An example of a double column floating figure using two subfigures.
% (The subfig.sty package must be loaded for this to work.)
% The subfigure \label commands are set within each subfloat command,
% and the \label for the overall figure must come after \caption.
% \hfil is used as a separator to get equal spacing.
% Watch out that the combined width of all the subfigures on a 
% line do not exceed the text width or a line break will occur.
%
%\begin{figure*}[!t]
%\centering
%\subfloat[Case I]{\includegraphics[width=2.5in]{box}%
%\label{fig_first_case}}
%\hfil
%\subfloat[Case II]{\includegraphics[width=2.5in]{box}%
%\label{fig_second_case}}
%\caption{Simulation results for the network.}
%\label{fig_sim}
%\end{figure*}
%
% Note that often IEEE papers with subfigures do not employ subfigure
% captions (using the optional argument to \subfloat[]), but instead will
% reference/describe all of them (a), (b), etc., within the main caption.
% Be aware that for subfig.sty to generate the (a), (b), etc., subfigure
% labels, the optional argument to \subfloat must be present. If a
% subcaption is not desired, just leave its contents blank,
% e.g., \subfloat[].


% An example of a floating table. Note that, for IEEE style tables, the
% \caption command should come BEFORE the table and, given that table
% captions serve much like titles, are usually capitalized except for words
% such as a, an, and, as, at, but, by, for, in, nor, of, on, or, the, to
% and up, which are usually not capitalized unless they are the first or
% last word of the caption. Table text will default to \footnotesize as
% IEEE normally uses this smaller font for tables.
% The \label must come after \caption as always.
%
%\begin{table}[!t]
%% increase table row spacing, adjust to taste
%\renewcommand{\arraystretch}{1.3}
% if using array.sty, it might be a good idea to tweak the value of
% \extrarowheight as needed to properly center the text within the cells
%\caption{An Example of a Table}
%\label{table_example}
%\centering
%% Some packages, such as MDW tools, offer better commands for making tables
%% than the plain LaTeX2e tabular which is used here.
%\begin{tabular}{|c||c|}
%\hline
%One & Two\\
%\hline
%Three & Four\\
%\hline
%\end{tabular}
%\end{table}


% Note that the IEEE does not put floats in the very first column
% - or typically anywhere on the first page for that matter. Also,
% in-text middle ("here") positioning is typically not used, but it
% is allowed and encouraged for Computer Society conferences (but
% not Computer Society journals). Most IEEE journals/conferences use
% top floats exclusively. 
% Note that, LaTeX2e, unlike IEEE journals/conferences, places
% footnotes above bottom floats. This can be corrected via the
% \fnbelowfloat command of the stfloats package.




%\section{Conclusion}
%The conclusion goes here.





% if have a single appendix:
%\appendix[Proof of the Zonklar Equations]
% or
%\appendix  % for no appendix heading
% do not use \section anymore after \appendix, only \section*
% is possibly needed

% use appendices with more than one appendix
% then use \section to start each appendix
% you must declare a \section before using any
% \subsection or using \label (\appendices by itself
% starts a section numbered zero.)
%


%\appendices
%\section{Proof of the First Zonklar Equation}
%Appendix one text goes here.

% you can choose not to have a title for an appendix
% if you want by leaving the argument blank
%\section{}
%Appendix two text goes here.


% use section* for acknowledgment
%\section*{Acknowledgment}


%The authors would like to thank...


% Can use something like this to put references on a page
% by themselves when using endfloat and the captionsoff option.
%\ifCLASSOPTIONcaptionsoff
% \newpage
%\fi



% trigger a \newpage just before the given reference
% number - used to balance the columns on the last page
% adjust value as needed - may need to be readjusted if
% the document is modified later
%\IEEEtriggeratref{8}
% The "triggered" command can be changed if desired:
%\IEEEtriggercmd{\enlargethispage{-5in}}

% references section

% can use a bibliography generated by BibTeX as a .bbl file
% BibTeX documentation can be easily obtained at:
% http://www.ctan.org/tex-archive/biblio/bibtex/contrib/doc/
% The IEEEtran BibTeX style support page is at:
% http://www.michaelshell.org/tex/ieeetran/bibtex/
%\bibliographystyle{IEEEtran}
% argument is your BibTeX string definitions and bibliography database(s)
%\bibliography{IEEEabrv,../bib/paper}
%
% <OR> manually copy in the resultant .bbl file
% set second argument of \begin to the number of references
% (used to reserve space for the reference number labels box)
%\begin{thebibliography}{1}

%\bibitem{IEEEhowto:kopka}
%H.~Kopka and P.~W. Daly, \emph{A Guide to \LaTeX}, 3rd~ed.\hskip 1em plus
%  0.5em minus 0.4em\relax Harlow, England: Addison-Wesley, 1999.

%\end{thebibliography}

%\nocite{*} % used with package refcheck


\bibliographystyle{IEEEtran}
\bibliography{vlqr_tip}

% biography section
% 
% If you have an EPS/PDF photo (graphicx package needed) extra braces are
% needed around the contents of the optional argument to biography to prevent
% the LaTeX parser from getting confused when it sees the complicated
% \includegraphics command within an optional argument. (You could create
% your own custom macro containing the \includegraphics command to make things
% simpler here.)
%\begin{IEEEbiography}[{\includegraphics[width=1in,height=1.25in,clip,keepaspectratio]{mshell}}]{Michael Shell}
% or if you just want to reserve a space for a photo:

%\begin{IEEEbiography}{Ding Liu}
%\begin{IEEEbiography}[{\includegraphics[width=1in,height=1.25in,clip,keepaspectratio]{bio/DingLiu.eps}}]{Ding Liu}
%(S'15) received the B.S. degree from the Chinese University of Hong Kong, Hong Kong, in 2012, and the M.S. degree from the University of Illinois at Urbana-Champaign, USA, in 2014, where he is currently pursuing the Ph.D. degree under the supervision of Prof. Thomas. S. Huang.
%His research experience encompasses using deep learning to solve low-level vision problems, including image super-resolution, image restoration and image denoising. He has research interests in the broad area of computer vision, image processing and deep learning.
%\end{IEEEbiography}


%\begin{IEEEbiography}{Bowen Cheng}
%Biography text here.
%\end{IEEEbiography}
%
%
%\begin{IEEEbiography}{Zhangyang Wang}
%	Biography text here.
%\end{IEEEbiography}
%
%
%\begin{IEEEbiography}{Haichao Zhang}
%	Biography text here.
%\end{IEEEbiography}
%
%\begin{IEEEbiography}{Thomas S. Huang}
%\begin{IEEEbiography}[{\includegraphics[width=1in,height=1.25in,clip,keepaspectratio]{bio/ThomasHuang.eps}}]{Thomas S. Huang}
%(F'01) received the B.S. degree in electrical engineering from National Taiwan University, Taipei, Taiwan, R.O.C., and the M.S. and Sc.D. degrees in electrical engineering from the Massachusetts Institute of Technology (MIT), Cambridge. He was on the Faculty of the Department of Electrical Engineering at MIT from 1963 to 1973; and on the Faculty of the School of Electrical Engineering and Director of its Laboratory for Information and Signal Processing at Purdue University from 1973 to 1980. In 1980, he joined the University of Illinois at Urbana-Champaign, where he is now William L. Everitt Distinguished Professor of Electrical and Computer Engineering, and Research Professor at the Coordinated Science Laboratory, and at the Beckman Institute for Advanced Science he is Technology and Co-Chair of the Institute’s major research theme Human Computer Intelligent Interaction. His professional interests lie in the broad area of information technology, especially the transmission and processing of multidimensional signals. He has published 21 books, and over 600 papers in network theory, digital filtering, image processing, and computer vision.
%Dr. Huang is a Member of the National Academy of Engineering; a Member of the Academia Sinica, Republic of China; a Foreign Member of the Chinese Academies of Engineering and Sciences; and a Fellow of the International Association of Pattern Recognition, IEEE, and the Optical Society of America. Among his many honors and awards: Honda Lifetime Achievement Award, IEEE Jack Kilby Signal Processing Medal, and King-Sun Fu Prize of the Int. Asso. for Pattern Recognition.
%\end{IEEEbiography}

%\begin{IEEEbiography}{Ding Liu}
%Biography text here.
%\end{IEEEbiography}

% if you will not have a photo at all:
%\begin{IEEEbiographynophoto}{John Doe}
%Biography text here.
%\end{IEEEbiographynophoto}

% insert where needed to balance the two columns on the last page with
% biographies
%\newpage

%\begin{IEEEbiographynophoto}{Jane Doe}
%Biography text here.
%\end{IEEEbiographynophoto}

% You can push biographies down or up by placing
% a \vfill before or after them. The appropriate
% use of \vfill depends on what kind of text is
% on the last page and whether or not the columns
% are being equalized.

%\vfill

% Can be used to pull up biographies so that the bottom of the last one
% is flush with the other column.
%\enlargethispage{-5in}



% that's all folks
\end{document}


