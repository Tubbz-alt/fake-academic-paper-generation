
%%%%%%%%%%%%%%%%%%%%%%% file typeinst.tex %%%%%%%%%%%%%%%%%%%%%%%%%
\documentclass[runningheads,orivec,a4paper]{llncs}

\usepackage{amssymb}
%\setcounter{tocdepth}{3}
\usepackage{graphicx}
\usepackage{mathtools}
\usepackage{textcomp}
\usepackage{amsmath}
\usepackage{amsopn}
\usepackage{tabularx, booktabs}
\usepackage{array}
\newcolumntype{Y}{>{\centering\arraybackslash}X}
\newcolumntype{C}[1]{>{\centering\let\newline\\\arraybackslash\hspace{0pt}}m{#1}}
\DeclareMathOperator{\Softmax}{Softmax}

\begin{document}

\mainmatter  % start of an individual contribution

% first the title is needed
\title{Generalised Wasserstein Dice Score for Imbalanced Multi-class Segmentation\\
      using Holistic Convolutional Networks}

% a short form should be given in case it is too long for the running head
\titlerunning{Generalised Wasserstein Dice Score}

\author{Lucas Fidon\inst{1}%
	\and Wenqi Li\inst{1}\and Luis C. Garcia-Peraza-Herrera\inst{1}\and\\ 
	Jinendra Ekanayake\inst{2,3}\and Neil
        Kitchen\inst{2}\and\\S\'ebastien Ourselin\inst{1,3} \and Tom Vercauteren\inst{1,3}}

%index{Fidon, Lucas}
%index{Li, Wenqi}
%index{C. Garcia-Peraza-Herrera, Luis}
%index{Ekanayake, Jinendra}
%index{Kitchen, Neil}
%index{Ourselin, Sebastien}
%index{Vercauteren, Tom}
%
\authorrunning{Lucas Fidon et al.}

\institute{TIG, CMIC, University College London, London, UK\\
	\and
	NHNN, University College London Hospitals, London UK\\
	\and
	Wellcome / EPSRC Centre for Interventional and Surgical Sciences, \\UCL, London, UK}


\maketitle


\begin{abstract}
The Dice score is widely used for binary segmentation due to its robustness to class imbalance.
%
Soft generalisations of the Dice score allow it to be used as a loss function for training convolutional neural networks (CNN).
%
%However, the Dice score does not straightforwardly generalise to
%multi-class segmentation. This often leads to using the mean of the
%individual class Dice scores when a single criterion is required
%for  multi-class evaluation.
%
%This is the case for example when a soft generalisation of the Dice
%score is used to train a convolutional neural network (CNN).
%for multi-class brain tumour segmentation thereby optimising a criteria that is closer to the evaluation criteria.
%
Although CNNs trained using mean-class Dice score achieve state-of-the-art
results on multi-class segmentation, this loss function
does neither take advantage of inter-class relationships nor
multi-scale information.
We argue that an improved loss function should balance
misclassifications to favour predictions that are
semantically meaningful.  
%
%More recently, optimising an average soft Dice score for training convolutional neural networks (CNNs) has provided good performance for multi-class brain tumour segmentation. 
%
This paper investigates these issues in the context of multi-class brain tumour segmentation.
%
Our contribution is threefold. 1) We propose a semantically-informed
generalisation of the 
Dice score for multi-class segmentation based on the Wasserstein
distance on the probabilistic label space.
%
2) We propose a holistic CNN that embeds spatial information at multiple scales with deep supervision.
%for brain tumour segmentation with more supervision compared to classic CNNs by computing several predictions at different scales of the network thereby taking into account the class imbalance explicitly in the structure of the CNN.
%
3) We show that the joint use of holistic CNNs and generalised
Wasserstein Dice score achieves segmentations that are more semantically meaningful for brain tumour segmentation. 
\end{abstract}


\section{Introduction}

Automatic brain tumour segmentation is an active research area.
%
Learning-based methods using convolutional neural networks (CNNs) have recently emerged as the state of the art~\cite{Havaei2017,deep_medic}.
%
%The choice of the loss function has a key role for generalisation of supervised CNNs.
%%
%Classic choice for the loss function are mean cross entropy and mean square error over all voxels.
%
%The problem remains challenging largely because of severe class imbalance.
One of the challenges is the severe class imbalance.
% which makes it difficult to optimise CNNs to segment the regions of interest. 
%
Two complementary ways have traditionally been used when training CNNs to tackle
imbalance: 1) using a sampling strategy that imposes
constraints on the selection of image patches; and 2) using pixel-wise weighting
to balance the contribution of each class in the objective function. 
%
%While a sampling strategy balances the distribution over the entire
%sampled dataset, a weighted loss function balances the voxel distribution
%for any given sample. 
%
%So in general they are complementary. 
%
For CNN-based segmentation, samples should ideally be entire
subject volumes to support the use of fully convolutional network and
maximise the computational efficiency of
convolution operations within GPUs.  
%
As a result, weighted loss functions appear more promising to improve
CNN-based automatic brain tumour
segmentation. 
%
%In that case the sampling strategy cannot tackle the class imbalace.
%
Using soft generalisations of the Dice score (a popular
overlap measure for binary segmentation) directly as a loss function has
recently been proposed~\cite{v_net,Sudre2017}. By introducing global spatial
information into the loss function, the Dice loss has been shown to be
more robust to class imbalance.  
%
%Moreover, it makes sense to use a loss function during training that is close to the criteria used to evaluate the segmentation during inference.
%
However at least two sources of information are not fully utilised in this formulation: 1) the
structure of the label space; and 2) the spatial information  across scales.
%at scales in
%between the voxel and the entire volume levels. 
%
Considering the class imbalance and the hierarchical label structure illustrated in Fig.\ref{fig:tree}, both of them are likely to play an important role for multi-class brain
tumour segmentation. 

In this paper, we propose two complementary contributions that leverage
prior knowledge about brain tumour structure. 
%
First, we exploit the Wasserstein distance~\cite{wasserstein_loss,fast_emd}, which can
naturally embed semantic relationships between classes for the
comparison of label probability vectors, to generalise the Dice score
for multi-class segmentation.
%propose a generalisation of the Dice score for multi-class segmentations based on the Wasserstein distance~\cite{fast_emd}, used as a label probability vectors distance able to naturally embed semantic relationships between classes.
%
%The proposed loss function take advantage of the smoothness of the label space to balance the misclassifications and favour predictions that are semantically as close as possible to the ground truth.
%
Second, we propose a new holistic CNN architecture inspired
by~\cite{toolnet,hed} that embeds spatial information at different
scales and introduces deep supervision during the CNN training.
%thereby taking into account the
%class imbalance in the design of the CNN architecture explicitly. 
%(i) implies that current CNNs training are based on "one-versus-all" approaches to learn how to segment each class and can be interpreted as jointly train a single CNN for multiple binary segmentation tasks.
%
%Also, subdividing a label in sub-labels should ideally not degrade the quality of the label segmentation.
%
%Optimal transport-based distances like the Wasserstein distance are a natural way of introducing semantic relationship between classes.
%
%However the Wasserstein distance is based on a linear programming optimisation problem which sofar limits its application for deep learning because of the computational burden~\cite{fast_emd}.
%%
%We remark though that in the case of a crisp ground truth and for pixel-wise histogram comparison a closed-form expression exists.
%%
We show that the combination of the proposed generalised Wasserstein Dice score and our Holistic CNN achieves better generalisation
compared to both mean soft Dice score training and classic CNN architectures
for multi-class brain tumour segmentation. 
%
\begin{figure}[t!]
	\centering
	\includegraphics[width=0.78\linewidth]{class-tree_v2}
	\caption{Left: tree on BraTS label space. Edge weights have been manually selected to reflect the distance between labels. Right: illustration on a T2 scan from BraTS'15~\cite{brats}.}
	\label{fig:tree}
\end{figure}


\section{A Wasserstein approach for multi-class soft Dice score}

\subsection{Dice score for crisp binary segmentation}
The Dice score is a widely used overlap measure for pairwise comparison of binary segmentations $S$ and $G$. It can be expressed both in terms of set operations or statistical measures as:
\begin{equation}\label{dice}
D = \frac{2|S \cap G|}{|S| + |G|} = \frac{2\Theta_{TP}}{2\Theta_{TP} + \Theta_{FP} + \Theta_{FN}} = \frac{2\Theta_{TP}}{2\Theta_{TP} + \Theta_{AE}}
\end{equation}
with $\Theta_{TP}$ the number of true positives, $\Theta_{FP}$/$\Theta_{FN}$ the number of false positives/false negatives, and $\Theta_{AE}=\Theta_{FP}+\Theta_{FN}$ the number of all errors.

\subsection{Dice score for soft binary segmentation}
Extensions to soft binary
segmentations~\cite{Anbeek2005,chang2009performance} rely on the
concept of disagreement for pairs of probabilistic classifications.
%$p^i$ and $g^i$ of a voxel $i$ by: $\delta_i = |p^i - g^i|$.
The classes $S_i$ and $G_i$ of each voxel $i \in \mathbf{X}$ can be defined
as random variables on the label space $\mathbf{L}=\{0,1\}$ and the
probabilistic segmentations can be represented as label probability
maps: $p=\{p^i:=P(S_i=1)\}_{i \in \mathbf{X}}$ and
$g=\{g^i:=P(G_i=1)\}_{i \in \mathbf{X}}$. 
We denote $P({\mathbf{L}})$ the set of label
probability vectors.
%
We can now generalise $\Theta_{TP}$ and $\Theta_{AE}$ to soft
segmentations:
\begin{equation}\label{soft_dsc}
\Theta_{AE} = \sum_{i \in \mathbf{X}} |p^i - g^i|, \quad
\Theta_{TP} = \sum_{i \in \mathbf{X}}g^i(1 - |p^i - g^i|)
\end{equation}
%add in extended version
In the common case of a crisp segmentation $g$ (i.e. $\forall i \in \mathbf{X}, g^i \in \{0,1\}$), the associated soft Dice score can be expressed as:
\begin{equation}
D(p,g) = \frac{2\sum_i g^ip^i}{\sum_i (g^i + p^i)}
\end{equation}
%
A second variant has been used in~\cite{v_net}, with a quadratic term in the denominator.


\subsection{Previous work on multi-class Dice score}
The easiest way to derive a unique criterion from the soft binary Dice score for multi-class segmentation is to consider the mean Dice score:
\begin{equation}\label{mean_dice}
 D_{mean}(p,g) = \frac{1}{|\mathbf{L}|}\sum_{l \in \mathbf{L}}\frac{2\sum_i g^i_lp^i_l}{\sum_i (g^i_l + p^i_l)}
\end{equation}
where $\{g^i_l\}_{i \in \mathbf{X},\, l \in \mathbf{L}}$, $\{p^i_l\}_{i \in \mathbf{X},\, l \in \mathbf{L}}$ are the set label probability vectors for all voxels for the ground truth and the prediction.

A generalised soft multi-class Dice score has also been
proposed in~\cite{gdsc,Sudre2017} by generalising the set theory definition of
the Dice score \eqref{dice}: 
\begin{equation}\label{gdsc}
D_{FM}(p,g) = \frac{2\sum_{l}\alpha_{l}\sum_{i}\min(p^i_l,g^i_l)}{\sum_{l}\alpha_{l}\sum_{i}(p^i_l + g^i_l)}
\end{equation}
where $\{\alpha_l\}_{l \in \mathbf{L}}$ allows to weight the
contribution of each class. However, those definitions are still based
only on pairwise comparisons of probabilities associated with the same
label and don't take into account inter-class relationships.
% To
%tackle this problem it would be natural to introduce additional
%pairwise comparisons between probabilities associated with different
%labels and additional assiciated weighting coefficients
%$\{\alpha_{l,l'}\}_{l,l' \in \mathbf{L}}$. It's not clear though how
%to choose the values of those new coefficients. 

\subsection{Wasserstein distance between label probability vectors}
The Wasserstein distance (also sometimes called the \emph{Earth
	Mover's Distance}) represents the minimal cost to transform a
probability vector $p$ into another one $q$ when for all $l,l' \in
\mathbf{L}$, the cost to move a unit from $l$ to $l'$ is defined as
the distance $M_{l,l'}$ between $l$ and $l'$. 
%
This is a way to map a distance matrix $M$
(often referred to as the \emph{ground distance matrix})
on
$\mathbf{L}$, into a distance on $P({\mathbf{L}})$ that leverages
prior knowledges about $\mathbf{L}$. 
%
In the case of a finite set $\mathbf{L}$, for $p, q \in
P({\mathbf{L}})$, the Wasserstein distance between $p$ and $q$ derived
from $M$ can be defined as the solution of a linear programming
problem~\cite{fast_emd}:
\begin{equation}\label{wasserstein}
\begin{split}
&W^M(p,q) = \min_{T_{l,l'}} \sum_{l,l' \in \mathbf{L}}
T_{l,l'}M_{l,l'}, \\
\textrm{subject to }	&\forall l \in  \mathbf{L}, \, \sum_{l' \in
	\mathbf{L}}T_{l,l'} = p_l,\, \textrm{and} \quad
\forall l' \in  \mathbf{L}, \, \sum_{l \in \mathbf{L}}T_{l,l'} =
q_{l'}.
\end{split}
\end{equation}
%where $T=(T_{l,l'})_{l,l' \in \mathbf{L}} \in P({\mathbf{L}^2})$ with marginal distributions $p$ and $q$.
where $T=(T_{l,l'})_{l,l' \in \mathbf{L}}$ is a joint probability distribution for $(p,q)$ with marginal distributions $p$ and $q$.
%where $M=(d(l,l'))_{l,l' \in \mathbf{X}}$ is the distance matrix on $\mathbf{L}$ associated with $d$.
%
A value $\hat{T}$ that minimises (\ref{wasserstein}) is called an
\emph{optimal transport} between $p$ and $q$ for the distance
matrix $M$.
%, called the \emph{ground distance matrix} on
%$\mathbf{L}$.

\subsection{Soft multi-class Wasserstein Dice score}
The Wasserstein distance $W^M$ in \eqref{wasserstein} yields a natural way to 
compare two label probability vectors in a semantically meaningful
manner by supplying a distance matrix $M$ on $\mathbf{L}$.
%define a linear
%comparison between two label probability vectors that accounts for
%intra and inter-class comparison and where the coefficients of the
%linear relationship are entirely determined by a distance matrix $M$
%on $\mathbf{L}$.
% via a linear programming optimisation problem.  
%
Hence we propose using it
% derived from a
%distance matrix $M$ on the labels space $\mathbf{L}$
to generalise the
measure of disagreement between a pair of label probability vectors
%$p^i, g^i \in P({\mathbf{L}})$ of a voxel $i \in \mathbf{X}$ by
%$W^M(p^i, g^i)$
and provide the following generalisations:
\begin{align}\label{wass_AE}
\Theta_{AE} &= \sum_{i \in \mathbf{X}} W^M(p^i, g^i) \\
\Theta_{TP}^l &= \sum_{i \in \mathbf{X}}g^i_l(W^M(l, b) - W^M(p^i, g^i)),
                \quad \forall l \in \mathbf{L}\setminus\{b\}
\end{align}
%$\Theta_{AE}$: 
%\begin{equation}\label{wass_AE}
%\Theta_{AE} = \sum_{i \in \mathbf{X}} W^M(p^i, g^i)
%\end{equation}
%We generalise the concept of $\Theta_{TP}$ for any class $l \in \mathbf{L}\setminus\{b\}$:
%\begin{equation}
%\Theta_{TP}^l = \sum_{i \in \mathbf{X}}g^i(W^M(l, b) - W^M(p^i, g^i))
%\end{equation}
where $W^M(l, b)$ is shorthand for $M_{l,b}$ and $M$ is
chosen such that the background class $b$ is always the furthest away
from the other classes. 
To generalise $\Theta_{TP}$, we propose to
weight the contribution of the classes similarly to
\eqref{gdsc}: 
%Here again this is natural to use the distance matrix $M$ on $\mathbf{L}$ to weight the contribution of each class to $\Theta_{TP}$ with respect to their distance to the background class $b$:
\begin{equation}\label{wass_TP}
\Theta_{TP} = \sum_{l \in \mathbf{L}} \alpha_{l} \,\Theta_{TP}^l
\end{equation}
We chose
% $\forall l \in \mathbf{L}, \,\alpha_{l}=W^M(l, b)$,
$\alpha_{l}=W^M(l, b)$ to make sure that background voxels do not contribute to $\Theta_{TP}$.
\noindent The Wasserstein Dice score with respect to $M$ can then be defined as:
\begin{equation}\label{generalised dice}
D^M(p, g) = \frac{ 2\sum_{l} W^M(l, b)\sum_{i} g^i_l(W^M(l, b) - W^M(p^i, g^i))}{2\sum_{l}[ W^M(l, b)\sum_{i} g^i_l(W^M(l, b) - W^M(p^i, g^i)) ] + \sum_{i} W^M(p^i, g^i)}
\end{equation}
%where $M=(d(l,l'))_{l \in \mathbf{L}, l' \in \mathbf{L}}$ is the distance matrix of the labels space $\mathbf{L}$ and $b$ is the background class.
In the binary case, setting  
$M= \begin{bsmallmatrix}
0       & 1  \\
1       & 0  \\
\end{bsmallmatrix}$ leads to $W^M(p^i,g^i)=|p^i-g^i|$ and reduces the proposed
Wasserstein Dice score to the soft binary Dice score
\eqref{soft_dsc}. 

\subsection{Wasserstein Dice loss with crisp ground truth}
Previous work on Wasserstein distance-based loss functions for deep
learning have been limited because of the computational
burden~\cite{fast_emd}. However, in the case of a crisp
ground-truth $\{g^i\}_i$, and for any prediction $\{p^i\}_i$, a
closed-form solution exists for \eqref{wasserstein}.
An optimal transport is $\forall l,l' \in \mathbf{L},
\hat{T}_{l,l'}=p^i_lg^i_{l'}$ and the Wasserstein distance becomes: 
\begin{equation}\label{wass_closed-form}
W^M(p^i, g^i) = \sum_{l,l' \in \mathbf{L}} M_{l,l'}p^i_lg^i_{l'}
\end{equation}
We define the Wasserstein Dice loss derived from $M$ as $\mathcal{L}_{D^M}:=1-D^M$.
%It has the closed-form:
%\begin{equation}\label{wass_dice_loss}
%\mathcal{L}_{D^M}(p, g) = \frac{\sum_{i} \sum_{l,l'} M_{l,l'}p^i_lg^i_{l'}}{2\sum_{l}[ M_{l,b}\sum_{i} g^i_l( 1 - \sum_{l'} M_{l,l'}p^i_{l'}) ] + \sum_{i} \sum_{l,l'} M_{l,l'}p^i_lg^i_{l'}}
%\end{equation}
%For brevity we do not detail the expression of $\mathcal{L}_{D^M}$
%or its gradient.
%and its gradient with respect to $p$ which can deduced from
%\eqref{dice}, \eqref{wass_AE}, \eqref{wass_TP}, and
%\eqref{wass_closed-form}. 

%\subsection{Dice score-based loss functions}
%
%%\subsection{Motivations}
%Brain tumour segmentation is challenging because of the strong class imbalace. It exists two common ways to tackle this problem in practice: 1) using a sampling strategy that imposes constraints on the selection of samples 2) using pixel-wise weighted loss function that weights the contribution each class in the final loss.
%%
%While a sampling strategy balances only the sample distribution over all the dataset, a weighted loss function can also balance the voxel distribution over a given sample.
%%
%So in general they are complementary. However, for segmentation based on CNN one would ideally define samples as entire subject volumes to maximise the computational efficiency of convolution operations with GPUs. In that case the sampling strategy cannot tackle the class imbalace.
%
%%\subsubsection{Previous works on Dice loss.}
%\cite{v_net} first proposed to minimise a Dice loss for binary segmentation, which is derived from a soft version of the Dice coefficient and defined as:
%\[
%\mathcal{L}_{D}(p, g) = 
% 1 - \frac{\sum_{i=1}^{N}2p_ig_i}{\sum_{1}^{N} p_i^2 + \sum_{1}^{N} g_i^2}
%\]
%where $(p_i)_{i \in \mathbf{X}}$ is the predicted probability map and  $(g_i)_{i \in \mathbf{X}}$ the ground truth one-hot probability map.
%%
%The authors of~\cite{toolnet} emphasize that this loss can be reformulated as a weighted sum of pairwise MSE of segmentations of the same class defined to be more robust to class imbalace. This property accounts for the good performance of the mean Dice loss compared to classic loss functions like mean MSE or mean cross entropy.
%
%\noindent For multi-class segmentation with $C$ classes a mean Dice loss has been used successfully in~\cite{highresnet,scalenet}:
%\[
%\mathcal{L}_{D}(p, g) = 
%1 - \frac{1}{C}\sum_{c=1}^{C}\frac{\sum_{i=1}^{N}2p_{c,i}g_{c,i}}{\sum_{i=1}^{N} p_{c,i}^2 + \sum_{i=1}^{N} g_{c,i}^2}
%\]
%
%%\subsubsection{Wasserstein-Dice loss.}
%However, the mean Dice loss only evaluates pairwise comparisons of segmentation corresponding to the same class independently and doesn't take advantage of inter-class relationships.
%%
%Moreover, loss functions below are based on norm-2 quantities which doesn't fit well with the intuitive meaning of the Dice coefficient. Note that for our experiences we removed squares in the denominator to have a formula closer to the Dice score but preliminary experiments have shown similar results on the task of brain tumour segmentation with and without squares.
%
%In this paper, we propose to derive a new weighted loss function from the generalisation of Dice score defined in (\ref{generalised dice}) for imbalace multi-class segmentation which can take advantage of prior knowledge about inter-class relationship. Given a ground distance matrix $M$, we defined the multi-class Dice loss as:
%\begin{equation}
%\mathcal{L}_{D_M}(p, g) = 1 - \frac{ 2\sum_{l} d(l, b)\sum_{i|g_l^i > 0, p_l^i > 0} ( 1 - W_M(p^i, g^i))}{2\sum_{l}[ d(l, b)\sum_{i|g_l^i > 0, p_l^i > 0} ( 1 - W_M(p^i, g^i)) ] + \sum_{i} W_M(p^i, g^i)}
%\end{equation}
%
%%We propose a new definition of Dice score based loss to allow larger variety of generalisation of the binary Dice loss.
%%
%%\noindent Let's define the scarcity measure of an associated pair ($g$, $p$) of ground truth and predicted probability map as:
%%
%%\begin{alignat*}{2}
%%\mu:P^{\mathbf{X}}(\mathbf{L})\times P^{\mathbf{X}}(\mathbf{L})&\longrightarrow& \mathcal{M}_{+}(\mathbf{X}) \\
%%(\mathbf{p}, \mathbf{g})&\longmapsto& \mu_{\mathbf{p}, \mathbf{g}}:\mathbf{X} &\longrightarrow\mathbb{R}+ \\
%%&&i&\longmapsto\frac{1}{C}\sum_{c=1}^{C}\frac{2g_{c,i}}{\sum_{1}^{N} p_{c,i}^2 + \sum_{1}^{N} g_{c,i}^2}
%%\end{alignat*}
%%
%%\noindent Then the Dice loss can be reformulated as:
%%\[
%%\mathcal{L}_{DSC}(p, g) = \mathbb{E}_{\mu_{\mathbf{p}, \mathbf{g}}}[-\langle p_i, g_i \rangle_{L}]
%%\]
%%This form allows to readily disentangle two components of the Dice loss: (1)A scarcity measure (here )$\mu_{\mathbf{p}, \mathbf{g}}$) which aims at weighting the contribution of each voxel to tackle the class imbalace within a batch/patch/volume. (2) A classification loss (here $\langle \cdot, \cdot \rangle_{L}$) which measures the distance between the prediction and the ground truth at the voxel level.
%%
%%It follows that we can generalise the Dice to a familly of loss function based on the Dice coefficient by:
%%\begin{alignat*}{2}
%%\mathcal{L}:(\mu, \ell)&\longmapsto& \mathcal{L}_{\mu, \ell}:P^{\mathbf{X}}(\mathbf{L})\times P^{\mathbf{X}}(\mathbf{L}) &\longrightarrow\mathbb{R} \\
%%&&(\mathbf{p}, \mathbf{g})&\longmapsto\mathbb{E}_{\mu_{\mathbf{p}, \mathbf{g}}}[\ell(p_i, g_i)]
%%\end{alignat*}
%%
%%\noindent In this work, we focus on other choice for $\ell$ allowing to use prior knowledge about the structure of $\mathbf{L}$.
%


%%%%%%%%%%%%%%%%%%%%%%%%%%%%%%%%%%%%%%%
\section{Holistic convolutional networks for multi-scale fusion}
%\subsubsection{Motivations.}
%\begin{itemize}
%	\item Incorporate spatial information by computing loss on output with different receptive fields and using downsampling
%	\item combining "what" and "where"
%	\item include deep supervision that can improve both optimisation and generalisation
%	\item make the CNN more robust to the choice of the depth of the network
%\end{itemize}
We now describe a holistically-nested convolutional neural network
(HCNN) for imbalanced multi-class brain tumour segmentation inspired by
the holistically-nested edge detection (HED) introduced
in~\cite{hed}. HCNN has been used successfully for some imbalanced
learning tasks such as edge detection in natural images~\cite{hed} and
surgical tool segmentation~\cite{toolnet}.
The HCNN features multi-scale prediction and intermediate
supervision. It can produce a unified output using a fusion layer
while implicitly embedding spatial information in the loss. 
%
We further improve on the ability of HCNNs to deal with imbalanced
datasets by leveraging the proposed generalised Wasserstein Dice loss. 
%
To keep up with state-of-the-art CNNs, we also employ ELU as
activation function~\cite{elu} and use residual
connections~\cite{resnet}. Residual blocks include a pair of $3^3$
convolutional filters and Batch Normalisation~\cite{wideresnet}. The
proposed architecture is illustrated in
Fig.\ref{fig:hcnn_architecture}. 



\begin{figure}[b!]
	\centering
	\includegraphics[width=\linewidth]{HCNN_architecture}
	\caption{Proposed holistically-nested CNN for multi-class labelling of brain tumours.}
	\label{fig:hcnn_architecture}
\end{figure}

\subsection{Multi-scale prediction to leverage spatial consistency in the loss}
%CNNs alternate small convolutional filters and simple non-linear function over all positions of the volume to extract hierarchically intensity-based features organised in layers until the segmentation prediction at the final layer. The relevantness of those features depends on the loss function used to optimise the parameters of a CNN.
%
%In practice, it is observed that optimising CNN's parameters with a loss function derived from a soft Dice score is efficient to tackle the class imballance for 3D MRI-based multi-class segmentation tasks~\cite{highresnet}.
%
%The Wasserstein Dice loss proposed below accounts for a more semantically meaningful comparison of voxel label probability predictions compared to other multi-class soft Dice loss but doesn't penalise spatial unconsistency.
%%
%Although it would have been possible to use the Wasserstein distance to compare the label probability vectors globally instead of voxel-wise~\cite{manifold_ot} to embed spatial consistency directly in the loss function, it would make the approximation of the Wasserstein distance and its gradient intractable in practice for 3D CNN training.
%
%HCNNs implicitly embed spatial information in the loss using
%predictions at multiple layers across the network and deep
%supervision. 
%
%While classic CNNs provide only one output, Holisticaly-nested CNNs provide outputs at different layers of the network and combine them to provide a final output. Intuitively, one can interpret it as a divide and conquer strategy applied to CNNs.
%
As the receptive field 
%,which is the size of the neighbourood used to compute the features at each position for a given layer, 
increases across successive layers, predictions computed
at different layers embed spatial information at different scales. 
%As a result, the output of the HCNN
%incorporates a larger variety of spatial information compared to classic
%CNNs.
Especially for imbalanced multi-class segmentation,
different scales can contain complementary information. 
%It also makes the network more robust to the choice of the number of
%layers. 
%
In this paper, to increase the receptive field and avoid redundancy between
successive scale predictions, max pooling and
dilated convolutions (with a factor of $2$ similar to~\cite{highresnet})
have been used. 
%
As predictions are computed
regularly at intermediate scales along the network (Fig.\ref{fig:hcnn_architecture}), we chose to increase the number of features before the first prediction is made. For simplicity reasons, we then selected
the same value for all hidden layers (fixed to $70$ given memory
constraints). 

\subsection{Multi-scale fusion and deep supervision for multi-class segmentation}
While classic CNNs provide only one output, HCNNs
provide outputs $\hat{y}^{s}$ at $S$ different layers of the network,
and combine them to provide a final output $\hat{y}^{fuse}$: 
\[
(\hat{y}_{l}^{fuse})_{l \in \mathbf{L}} = \Softmax\Big(\big(\sum_{s=1}^{S} w_{l,s} \hat{y}_{l}^{s}\big)_{l \in \mathbf{L}}\Big).
\]
%Intuitively, one can interpret it as a divide and conquer strategy
%applied to CNNs. 
As different scales can be of different importance for different
classes we learn class-specific fusion weights $w_{l,s}$. 
This transformation can also be represented by a convolution layer
with kernels of size $1^3$ where the multi-scale
predictions are fused in separated branches for each class, as
illustrated in Fig.~\ref{fig:hcnn_architecture} similarly to the
scalable layers introduced in~\cite{scalenet}. 
%
%\subsubsection{Deep supervision.}
In addition to applying the loss function $\mathcal{L}$ to the fused
prediction, $\mathcal{L}$ is also applied to each scale-specific
prediction thereby providing deep supervision (coefficients $\bar{\lambda}$ and $\lambda_{s}$ are set to 1/$(S+1)$ for simplicity):
%. For S scales, the total loss is defined as: 
\[
\mathcal{L}_{Total}((\hat{y}^{s})_{s=1}^{S}, \hat{y}^{fuse}, y) = \bar{\lambda}\mathcal{L}(\hat{y}^{fuse}, y) + \sum_{s=1}^{S} \lambda_{s}\mathcal{L}(\hat{y}^{s}, y)
\]
%where the coefficients $\lambda_{s}$ are set to 1 for simplicity. 
%
%It is shown that even with the fusion layer, deep supervision can benefit to generalisation.
%Additionally, we used residual connections~\cite{resnet} to allow the gradient to flow equally in all layers.

%\subsection{Receptive field}
%To increase the difference of receptive fields and avoid redundancy between successive scale predictions down sampling using max pooling and dilated convolutions with a factor of $2$ similar to~\cite{highresnet} have been used.
%This mix of methods results of the trade off between high spatial resolution of the feature maps to provide accurate segmentations and memory consumption.

%%%%%%%%%%%%%%%%%%%%%%%%%%%%%%%%%%%%%%%%
\section{Implementation details}

\subsection{Brain tumour segmentation}
We evaluate our HCNN model and Wasserstein Dice loss functions on the
task of brain tumour segmentation using BraTS'15 training set that
provides multimodal images (T1, T1c, T2 and Flair) for 220 high-grade
gliomas subjects and 54 low-grade gliomas subjects.  We divide it
randomly into $80\%$ for training, $10\%$ for validation and $10\%$
for testing so that the proportion of high-grade and low-grade gliomas
subjects is the same in each fold. 
%
The scans are labelled with five classes (Fig.~\ref{fig:tree}): (0)
background, (1) necrotic core, (2) edema, (3) non-enhancing core and
(4) enhancing tumour. The most common evaluation criteria for BraTS is
to use the Dice scores for the whole tumour (labels 1,2,3,4), the core
tumour (labels 1,3,4) and the enhanced tumour (label 4). 
All the scans of BraTS dataset are skull stripped,
resampled to a 1mm isotropic grid and co-registered to the T1-weighted volume of each patient. 
%
Additionally, we applied histogram standardisation to each
imaging modality independently~\cite{histStd}.

\subsection{Implementation details}
We train the networks using ADAM~\cite{adam} with a
learning rate $lr=0.01$, $\beta_1 = 0.9$ and $\beta_2 = 0.999$. To
regularise the network, we use early stopping on the validation set
and dropout in all residual blocks before the last activation (as
proposed in~\cite{wideresnet}), with a probability of $0.6$. 
%
We use multi-modal volumes of size $80^3$ from one
subject concatenated as input during training and a
sampling strategy to maximise the number of
classes in each patch. 
%
Experiments have been performed using Tensorflow
1.1~\footnote{The code is publicly available as part of
  NiftyNet (http://niftynet.io)} and a Nvidia GeForce GTX Titan X GPU.



%%%%%%%%%%%%%%%%%%%%%%%%%%%%%%%%%%%%%%%%
\section{Results}

We evaluate the usefulness of the proposed soft multi-class Wasserstein Dice loss and the proposed HCNN with deep supervision. We compare the soft multi-class Wasserstein Dice loss to the state-of-the-art mean Dice score~\cite{scalenet,highresnet} for the training of our HCNN in Table \ref{tab:loss_results} and \ref{tab:confusion}. We also evaluate the segmentation at the different scales of the HCNN in Table \ref{tab:holistic_results}.

%\begin{table}[tb]
%	\centering
%	\caption{Evaluation of different multi-class Dice scores for training and testing. $\mathcal{L}_{D^{M_{tree}}-PT}$ stands for pre-training the HCNN with mean Dice score (4 epochs) and retraining it with $\mathcal{L}_{D^{M_{tree}}}$ (85 epochs).}
%	\begin{tabular}{l|c|c|c|c|c|c}
%		\hline
%		& \multicolumn{6}{c}{\bf Evaluation: Mean(std) Dice scores (\%)}\\
%		\hline
%		\bf Loss function & Whole & Core & Enh. & Mean Dice & $D^{M_{0-1}}$  & $D^{M_{tree}}$\\ 
%		\hline
%		Mean Dice                  & 83(13) & 70(21) & 68(26) & \bf60(12) & 77(11) & 80(12)\\
%		$\mathcal{L}_{D^{M_{0-1}}}$ & 86(12) & 59(29) & 69(23) & 48(5) & 82(6) & 85(5)\\
%		%	$D_{1=3}$ & 88(6) & 73(21) & 73(24) & 55(7) & 82(6) & 84(6) & 86(5)\\
%		$\mathcal{L}_{D^{M_{tree}}}$ & 88(8) & 73(23) & 70(25) & 54(7) & 84(5) & 86(5)\\
%		$\mathcal{L}_{D^{M_{tree}}-PT}$ & \bf89(6) & \bf73(22) & \bf74(23) & 59(10) & \bf84(4) & \bf87(4)\\
%		\hline
%		
%	\end{tabular}
%	\vspace{0.2cm}
%	\label{tab:loss_results}
%\end{table}

\begin{table}[tb]
	\centering
	\caption{Evaluation of different multi-class Dice scores for training and testing. $\mathcal{L}_{D^{M_{tree}}-PT}$ stands for pre-training the HCNN with mean Dice score (4 epochs) and retraining it with $\mathcal{L}_{D^{M_{tree}}}$ (85 epochs).}
	\begin{tabularx}{\textwidth}{c *{6}{Y}}
		\toprule
		\multicolumn{1}{c}{\bf Loss function}
		& \multicolumn{6}{c}{\bf Evaluation: Mean(std) Dice scores (\%)}\\
		\cmidrule(lr){2-7}
		 & Whole & Core & Enh. & Mean Dice & $D^{M_{0-1}}$  & $D^{M_{tree}}$\\ 
		\midrule
		Mean Dice                  & 83(13) & 70(21) & 68(26) & \bf60(12) & 77(11) & 80(12)\\
		$\mathcal{L}_{D^{M_{0-1}}}$ & 86(12) & 59(29) & 69(23) & 48(5) & 82(6) & 85(5)\\
		%	$D_{1=3}$ & 88(6) & 73(21) & 73(24) & 55(7) & 82(6) & 84(6) & 86(5)\\
		$\mathcal{L}_{D^{M_{tree}}}$ & 88(8) & 73(23) & 70(25) & 54(7) & 84(5) & 86(5)\\
		$\mathcal{L}_{D^{M_{tree}}-PT}$ & \bf89(6) & \bf73(22) & \bf74(23) & 59(10) & \bf84(4) & \bf87(4)\\
		\bottomrule
	\end{tabularx}
	\label{tab:loss_results}
\end{table}


\subsection{Examples of distance metrics on BraTS label space}
%
To illustrate the flexibility of the proposed generalised Wasserstein
Dice score, we evaluate two semantically driven choices for the
distance matrix $M$ on $\mathbf{L}$: 
\[
M_{0-1}= \begin{pmatrix}
0  & 1 & 1 & 1 & 1 \\
1  & 0 & 1 & 1 & 1 \\
1  & 1 & 0 & 1 & 1 \\
1  & 1 & 1 & 0 & 1 \\
1  & 1 & 1 & 1 & 0 \\
\end{pmatrix}, \quad \textrm{and} \quad
%M_{1=3}= \begin{matrix}
%0  & 1 & 1 & 1 & 1 \\
%1  & 0 & 1 & 0 & 1 \\
%1  & 1 & 0 & 1 & 1 \\
%1  & 0 & 1 & 0 & 1 \\
%1  & 1 & 1 & 1 & 0 \\
%\end{bmatrix}, \,
M_{tree}= \begin{pmatrix}
0  & 1 & 1 & 1 & 1 \\
1  & 0 & 0.6 & 0.2 & 0.5 \\
1  & 0.6 & 0 & 0.6 & 0.7 \\
1  & 0.2 & 0.6 & 0 & 0.5 \\
1  & 0.5 & 0.7 & 0.5 & 0 \\
\end{pmatrix}.
\]
$M_{0-1}$ is associated with the discrete distance on $\mathbf{L}$ with no inter-class relationship. 
%
%$M_{1=3}$ is the same as $M_{0-1}$ except that the distance between labels 1 and 3 is null, which amounts to merge label 1 and 3, this is motivated by the fact that those labels are not discriminated by the evaluation used for BraTS (see Sect.4). 
%
$M_{tree}$ is derived from the tree structure of $\mathbf{L}$
illustrated in Fig.~\ref{fig:tree}.
%
This tree is based on the tumour hierarchical structure: whole, core and enhancing tumour. We set branch weights to $0.1$ for contiguous nodes and $0.2$ otherwise.

\subsection{Evaluation and training with multi-class Dice score}

\begin{figure}[t!]
	\centering
	\includegraphics[width=\linewidth]{Loss_functions_qualitative}
	\caption{Qualitative comparison of HCNN predictions at testing after training with the proposed Generalised Wasserstein Dice loss ($\mathcal{L}_{D^{M_{tree}}-PT}$) or mean-class Dice loss. Training with $\mathcal{L}_{D^{M_{tree}}-PT}$ allows avoiding implausible misclassifications encountered in predictions after training with mean-class Dice loss (emphasized by white arrows).}
	\label{fig:loss_quali}
\end{figure}
%
The mean Dice corresponds to the mean of soft Dice scores for each class as used in~\cite{scalenet,highresnet}.
%
Results in Table~\ref{tab:loss_results} confirm that training with mean Dice score, $D^{M_{0-1}}$ or 
%tree-based Wasserstein Dice score
 $D^{M_{tree}}$ allow maximising results for the associated multi-class Dice score during inference.
%Moreover, training with $D^{M_{tree}}$ that leverages prior information about the hierarchical structure of the tumour classes yield the best results on Dice scores evaluation for whole, core and enhancing tumour.

While $D^{M_{tree}}$ takes advantage of prior information about the hierarchical structure of the tumour classes it makes the optimisation more complex by adding more constraints.
%
To relax those constraints, we propose to pretrain the network using the mean Dice score during a few epochs (4 in our experiment) and then retrain it using $D^{M_{tree}}$.
%
This approach leads to the best results for all criteria, as illustrated in the last line of Table~\ref{tab:loss_results}.
%
Moreover, it produces segmentations that are more semantically plausible compared to the HCNN trained with mean Dice only as illustrated by Fig.~\ref{fig:loss_quali}.

\subsection{Impact of the Wasserstein Dice loss on class confusion}

\begin{table}[t]
	\centering
	\caption{Dice score evaluation of the confusion after training the HCNN using different loss functions. Each line (resp. column) corresponds to the mean(standard deviation) Dice scores (\%) of a region of the ground truth (resp. prediction) with all regions of the prediction (resp. ground truth) computed on the testing set.}
	\begin{tabular}{l | C{1.9cm} | C{1.9cm} | C{1.8cm} | C{1.8cm} | C{1.8cm}}
		\hline
		Mean Dice & \multicolumn{5}{c}{\bf Prediction}\\
		\hline
		\bf Ground truth & Background & Necrotic core & Edema & Non-enh. & Enh. \\ 
		\hline
		Background     & 99.6(0) & 0(0)     &  0.8(0)   & 0.1(0)  & 0.1(0)     \\
		Necrotic core  & 0.(0)   & 36.8(30) &  1.2(3)   & 8.3(9)  & 1(1)   \\
		Edema          & 0.3(0)  & 0.9(1)   &  62.9(18) & 21.7(13)& 4.3(6)   \\
		Non-enh.       & 0.1(0)  & 8.7(9)   &  6.5(8)   & 33(15)  & 14.8(11)   \\
		Enh.           & 0(0)    & 0.9(1)   &  0.3(0)   & 6.9(7)  & 67.6(25) \\
		\hline	
	\end{tabular}
	\vspace{0.3cm}
	
	\begin{tabular}{l| C{1.9cm} | C{1.9cm} | C{1.8cm} | C{1.8cm} | C{1.8cm}}
		\hline
		$\mathcal{L}_{D^{M_{tree}}}$  & \multicolumn{5}{c}{\bf Prediction}\\
		\hline
		\bf Ground truth & Background & Necrotic core & Edema & Non-enh. & Enh. \\ 
		\hline
		Background     & 99.7(0) & 0(0) &  0.3(0)   & 0(0)     & 0(0)     \\
		Necrotic core  & 0(0)    & 0(0) &  2.4(5)   & 28.2(22) & 1.4(1)   \\
		Edema          & 0.6(0)  & 0(0) &  71.3(12) & 8.5(7)   & 3.5(5)   \\
		Non-enh.       & 0.1(0)  & 0(0) &  15.4(13) & 28.9(14) & 14.2(10) \\
		Enh.           & 0(0)    & 0(0) &  1.7(1)   & 6.9(7)   & 70.5(25) \\
		\hline
	\end{tabular}
	\vspace{0.3cm}
	
	\begin{tabular}{l| C{1.9cm} | C{1.9cm} | C{1.8cm} | C{1.8cm} | C{1.8cm}}
		\hline
		$\mathcal{L}_{D^{M_{tree}}-PT}$  & \multicolumn{5}{c}{\bf Prediction}\\
		\hline
		\bf Ground truth & Background & Necrotic core & Edema & Non-enh. & Enh. \\ 
		\hline
		Background     & 99.7(0) & 0(0)     &  0.2(0)   & 0(0)     & 0(0)     \\
		Necrotic core  & 0(0)    & 20.2(27) &  2.3(5)   & 23.2(18) & 1(1)   \\
		Edema          & 0.6(0)  & 0.3(0)   &  73.3(11) & 5.7(5)   & 3.1(4)   \\
		Non-enh.       & 0.1(0)  & 2.1(7)   &  16.1(13) & 30(17)   & 13(8) \\
		Enh.           & 0(0)    & 0(0)     &  2.4(2)   & 3.8(4)   & 73.5(22) \\
		\hline
	\end{tabular}
	\label{tab:confusion}
\end{table}

Evaluating brain tumour segmentation using Dice scores of label subsets like whole, core and enhancing tumour doesn't allow measuring the ability of a model to learn inter-class relationships and to favour voxel classifications, be it correct or not, that are semantically as close as possible to the ground truth.
%
We propose to measure class confusion using pairwise comparisons of all labels pair between the predicted segmentation and the ground truth (Table~\ref{tab:confusion}). Mathematically, for all $l,l' \in \mathbf{L}$, the quantity in row $l$ and colomn $l'$ stands for the soft binary Dice score:
\begin{equation}
	D_{l,l'} = \frac{2\sum_i g^i_lp^i_{l'}}{\sum_i (g^i_l + p^i_{l'})}
\end{equation}
%
Results in Table~\ref{tab:confusion} compare class confusion of the proposed HCNN after being trained either using mean Dice loss, tree-based Wasserstein Dice loss ($\mathcal{L}_{D^{M_{tree}}}$) or tree-based Wasserstein Dice loss pre-trained with mean Dice loss($\mathcal{L}_{D^{M_{tree}}-PT}$). The first one aims only at maximising the true positives (diagonal) while the two other additionally aim at balancing the misclassifications to produce semnatically meaningful segmentations.

The network trained with mean Dice loss segments correctly most of the voxels (diagonal in Table~\ref{tab:confusion}) but makes misclassifications that are not semantically meaningful. For example, it makes poor differentiation between the edema and the core tumour as can be seen in the line corresponding to edema in Table~\ref{tab:confusion} and in Fig.~\ref{fig:loss_quali}. 
%a large proportion of the non-enhancing tumour prediction is edema (column 4) while it would be semantically more meaningful to confuse it with the necrotic core as illustrated in Fig.~\ref{fig:tree}.

In contrast, the network trained with $\mathcal{L}_{D^{M_{tree}}}$ makes more meaningful confusion but it is not able to differentiate necrotic core and non-enhancing tumour at all (columns 2 and 4). It illustrates the difficulty to train the network with $\mathcal{L}_{D^{M_{tree}}}$ starting from a random initialisation because $\mathcal{L}_{D^{M_{tree}}}$ embeds more constraints than the mean Dice loss.

$\mathcal{L}_{D^{M_{tree}}-PT}$ allows combining advantages of both loss function:
pre-training the network using the mean Dice loss allows initialising it so that it produces quickly an approximation of the segmentation, and retraining it with $\mathcal{L}_{D^{M_{tree}}}$ allows reaching a model which provides semantically meaningful segmentations (Fig.~\ref{fig:loss_quali}) with a higher rate of true positives compared to training with $\mathcal{L}_{D^{M_{tree}}}$ or mean Dice loss alone (Table~\ref{tab:confusion}).


\subsection{Evaluation of deep supervision}

\begin{table}[tb]
	\centering
	\caption{Evaluation of scale-specific and fused predictions of the HCNN with Dice score of whole, core, enhancing tumour and $D^{M_{tree}}$ after being pre-trained with mean Dice score (4 epochs) and retrained with $\mathcal{L}_{D^{M_{tree}}}$ (85 epochs).}
	\begin{tabularx}{0.9\textwidth}{c *{5}{Y}}
		\toprule
		 \bf Prediction & \multicolumn{4}{c}{\bf Mean(Std) Dice score (\%)}\\
		\cmidrule(lr){2-5}
		 & Whole tumour & Core tumour & Enh. tumour & $D^{M_{tree}}$\\ 
		\midrule
		Scale 1 & 84(8)    & 68(23)   & 70(25)    & 84(5)\\
		Scale 2 & \bf89(5) & \bf73(22) & \bf74(23)  & \bf87(4)\\
		Scale 3 & 88(6)    & 72(23)   & 71(22)    & 86(4)\\
		Scale 4 & \bf89(5) & 72(22) & 71(21) & 86(3)\\
		%		Average & 0 & 0 & 0 & 0\\
		Fused   & 89(6) & \bf73(22) & \bf74(23) & \bf87(4)\\
		\bottomrule
	\end{tabularx}
	\label{tab:holistic_results}
\end{table}

\begin{figure}[tb]
	\centering
	\includegraphics[width=\linewidth]{Scales_qualitative}
	\caption{Qualitative comparison of fused and scales predictions at testing after training our HCNN with the proposed Generalised Wasserstein Dice loss ($\mathcal{L}_{D^{M_{tree}}-PT}$). White arrows emphasize implausible misclassifications.}
	\label{fig:scales_quali}
\end{figure}

Results in Table~\ref{tab:holistic_results} are obtained after pre-training HCNN with mean Dice score during 4 epochs and then training it with $\mathcal{L}_{D^{M_{tree}}}$ during 85 additional epochs.
%
Scales 2 to 4 and fused achieve similar Dice scores for whole, core tumour and the objective function $D^{M_{tree}}$ while scale 1 obtains lower Dice scores. Holes in tumour segmentations produced by scale 1, as illustrated in Fig.~\ref{fig:scales_quali}, suggest an unsufficient receptive field could account for those lower Dice scores.
%
The best result for the enhancing tumour is achievied by both scale 2 and fused, which was expected as this is the smallest region of interest and the full resolution is maintained until scale 2.
%
Moreover, as illustrated in Fig.~\ref{fig:scales_quali}, scales 3 and 4 fail at segmenting the thinest regions of the tumour because of their lower resolution contrary to scales 1 and 2 and fused.
%
However, scales 1 to 3 contained implausible segmentation regions contrary to scale 4 and fused.
%
This suggests trade-offs between high receptive field and high resolution that are class specific. It confirms the usefulness of the multi-scale holistic approach for the multi-class brain tumour segmentation task.


%%%%%%%%%%%

\section{Conclusion and future work}

We proposed a semantically driven generalisation of the Dice score for
soft multi-class segmentation based on the Wasserstein distance. This
embeds prior knowledge about inter-class relationships represented by
a distance  matrix on the label space. 
Additionally, we proposed a holistic convolutional network that uses
multi-scale predictions and deep supervision to make use of multi-scale
information. 
We successfully used the proposed Wasserstein Dice score as a loss
function to train our holistic networks and show the
importance of multi-scale and inter-class relationships for the
imbalanced task of multi-class brain tumour segmentation.
%
The proposed distance matrix based on the label space tree structure leads to higher Dice scores compared to the discrete distance.
Because the tree-based distance matrix used was heuristically chosen we think that better heuristics or a method to directly learn the matrix from the data could lead to further improvements.

As the memory capacity of GPUs increases, entire multi-modal volumes
could be used as input of CNN-based segmentation. However, it will also increase the class imbalance in the patches used as input.
%By moving to a
%resolution-preserving approach, 
We expect this to increase the impact of our contributions. 
%
Future work includes extending the use of Wasserstein distance by
defining a matrix distance on the entire output space
$\mathbf{X}\times\mathbf{L}$ similarly to~\cite{manifold_ot}. This would
allow embedding spatial information directly in the loss, but the
computation burden of the Wasserstein distance, in that case, remains a
challenge~\cite{fast_emd}. 
%It could also allow fusing the scale predictions with a barycenter
%with respect to the Wassertein distance. 

\subsubsection*{Acknowledgements.}\hspace*{\fill} \\
This work was supported by the Wellcome Trust (WT101957, 203145Z/16/Z, HICF-T4-275, WT 97914), EPSRC (NS/A000027/1, EP/H046410/1, EP/J020990/1, EP/K005278, NS/A000050/1), the NIHR BRC UCLH/UCL, a UCL ORS/GRS Scholarship and a hardware donation from NVidia.

%%%%%%%%%%%%%%%%%%%%%%%%%%%%%%%%%%%%%%%%%%%%%%%%%%%%%%%%%%%

%\bibliographystyle{splncs}
\bibliographystyle{splncs03}
%\bibliography{JNAbrv,library}
\bibliography{JNAbrv,bibliography}

\end{document}
