\documentclass[10pt,twocolumn,letterpaper]{article}
\usepackage{cvpr}
\usepackage{epsfig}
\usepackage{times}
\usepackage[utf8]{inputenc} % allow utf-8 input
\usepackage[T1]{fontenc}    % use 8-bit T1 fonts
\usepackage{url}            % simple URL typesetting
\usepackage{booktabs}       % professional-quality tables
\usepackage{amsfonts}       % blackboard math symbols
\usepackage{nicefrac}       % compact symbols for 1/2, etc.
\usepackage{microtype}      % microtypography
\usepackage{lipsum} 
\usepackage{color}
\usepackage{graphicx}
\usepackage{subcaption}
\usepackage{xspace}
\usepackage{booktabs} % for professional tables
\usepackage{amsmath}
\usepackage{amssymb}
\usepackage{multirow}
\usepackage{multicol}
\usepackage{algorithm}
\usepackage{algpseudocode}
\usepackage{adjustbox}
\usepackage{flushend}

% Include other packages here, before hyperref.

% If you comment hyperref and then uncomment it, you should delete
% egpaper.aux before re-running latex.  (Or just hit 'q' on the first latex
% run, let it finish, and you should be clear).
\usepackage[pagebackref=true,breaklinks=true,letterpaper=true,colorlinks,bookmarks=false]{hyperref}

%\xin{I put back to the submission format to estimate the space}
\cvprfinalcopy % *** Uncomment this line for the final submission

\def\cvprPaperID{440} % *** Enter the CVPR Paper ID here
\def\httilde{\mbox{\tt\raisebox{-.5ex}{\symbol{126}}}}

% Pages are numbered in submission mode, and unnumbered in camera-ready
\ifcvprfinal\pagestyle{empty}\fi


\newcommand{\xin}[1]{\textcolor{red}{[Xin: #1]}}
\newcommand{\ruth}[1]{\textcolor{magenta}{[Ruth: #1]}}
\newcommand{\joey}[1]{\textcolor{blue}{[Joey: #1]}}
\newcommand{\fisher}[1]{\textcolor{blue}{[Fisher: #1]}}
\newcommand{\trevor}[1]{\textcolor{red}{[Trevor: #1]}}


\newcommand{\model}{TAFE-Net\xspace}
\newcommand{\modelplural}{TAFE-Nets\xspace}

\def\vx{{\bf x}}
\def\vX{{\bf X}}
\def\vz{{\bf z}}
\def\va{{\bf a}}

\renewcommand{\vec}[1]{\mathbf{#1}}
\newcommand\minisection[1]{\vspace{2mm}\noindent \textbf{#1}}

% \title{\model: Exploring Exponential Sparsified Mixture of Experts within Single Model}
% \title{Learning Deep Dynamic Representation with Efficient Embedding}
% \title{Dynamic Representation Learning with Efficient Embedding}
% \title{Deep Dynamic Distributed Representation}
% \title{Dynamic Re-weighted Network via Joint Visual-Task Embedding}
% \title{Task Dependent Image Embedding for Low-Shot and Sparsity Learning}
% \title{Task Dependent Image Embeddings for Efficient Learning and Inference}
\title{TAFE-Net: Task-Aware Feature Embeddings for \\ Efficient Learning and Inference}

% The \author macro works with any number of authors. There are two
% commands used to separate the names and addresses of multiple
% authors: \And and \AND.
%
% Using \And between authors leaves it to LaTeX to determine where to
% break the lines. Using \AND forces a line break at that point. So,
% if LaTeX puts 3 of 4 authors names on the first line, and the last
% on the second line, try using \AND instead of \And before the third
% author name.

\author{
  Xin Wang\quad Fisher Yu\quad 
  Ruth Wang\quad
%   Azalia Mirhoseini$^2$ \\
  Trevor Darrell\quad Joseph E. Gonzalez \\
  EECS Department, UC Berkeley 
  % \quad\quad $^2$ Google Brain
}


\begin{document}
% \nipsfinalcopy is no longer used

\maketitle

\begin{abstract}
Learning good feature embeddings for images often requires substantial training data.
As a consequence, in settings where training data is limited (e.g., few-shot and zero-shot learning), we are typically forced to use a general feature embedding across prediction tasks.
Ideally, we would like to construct feature embeddings that are tuned for the given task and even input image.
In this work, we propose Task-Aware Feature Embedding Networks (\modelplural) to learn how to adapt the image representation to a new task in a meta learning fashion. 
Our network is composed of a meta learner and a prediction network, where the meta learner generates parameters for the feature layers in the prediction network based on a task input so that the feature embedding can be accurately adjusted
for that task.
We show that our \model is highly effective in generalizing to new tasks or concepts and offers efficient prediction with low computational cost. 
We demonstrate the general applicability of \model in several tasks including zero-shot/few-shot learning and dynamic efficient prediction. Our networks exceed or match the state-of-the-art on most tasks. In particular, our approach improves the prediction accuracy of unseen attribute-object pairs by 4 to 15 points on the challenging visual attributes composition task.
\end{abstract}

\section{Introduction}% \begin{enumerate}%     \item Feature embedding are central a wide range of computer vision tasks. %     \item However, learning rich embeddings requires substantial amounts of training time and data.%     \item A common practice is to reuse standard pre-trained embeddings from large supervised learning tasks and then fine tune those embeddings to new domains.%     \item However, in many few or zero shot learning settings there isn't sufficient data to fine-tune these embeddings%     \item In this work we develop a meta-learning technique to generate task specific embedding networks that can be used to adapt generic feature embeddings.%     \item We introduce a parameter generation network that combines a novel parameter factorization and secondary embedding loss to learn how to generate task specific embedding networks.% \end{enumerate}% \begin{enumerate}%     \item Fixed representations are common in many visual learning tasks, but may not have optimal efficiency when learning model parameters or performing inference on a new sample point.%     \item A wide range of problems require finding a new image embedding given a new task.% \end{enumerate}

Feature embeddings are central to 
computer vision.
By mapping images into semantically rich vector spaces, feature embeddings extract key information that can be used for a wide range of prediction tasks.  
However, learning good feature embeddings typically requires substantial amounts of training data and computation.  
As a consequence, a common practice~\cite{donahue2014decaf,girshick2014rich,zeiler2014visualizing} is to re-use existing feature embeddings from convolutional networks (e.g., ResNet50~\cite{he2016deep}) trained on large-scale labeled training datasets (e.g., ImageNet~\cite{russakovsky2015imagenet}); to achieve maximum accuracy, these general feature embedding are often fine-tuned~\cite{girshick2014rich,donahue2014decaf,zeiler2014visualizing} or transformed~\cite{Hoffman_cycada2017} using additional task specific training data. 

\begin{figure}[t]
    \centering
    \includegraphics[width=0.48\textwidth]{figs/figs_r/figure_1_v4.pdf} \caption{A cartoon illustration of Task-aware Feature Embedding. In this case there are two binary prediction task: \texttt{hasCat} and \texttt{hasDog}).
    Task-aware feature embeddings mean that the same image can have different embeddings for each task.
    As a consequence, we can adopt a single task independent classification boundary for all tasks.}
    \label{fig:tafe}
\end{figure}

In many settings, the training data are insufficient to learn or even adapt general feature embeddings to a given task.
For example, in few-shot and zero-shot prediction tasks, the scarcity of training data forces the use of generic feature embeddings.
As a consequence, in these situations much of the research instead focuses on the design of joint task and data embeddings~\cite{Changpinyo_2016_CVPR,frome2013devise,zhang2016zero} that can be generalized to unseen tasks or tasks with fewer examples. 
% We can also treat the task embedding as a linear separator and learn to generate them for new tasks~\cite{vinyals2016matching,lu2016visual}. 
Some have proposed treating the task embedding as linear separators and learning to generate them for new tasks~\cite{vinyals2016matching,lu2016visual}. 
Others have proposed hallucinating additional data points~\cite{xian2018feature,hariharan2017low, wang2018low}. 
However, in all cases, a common image embedding is shared across tasks. 
As a consequence, the common image embedding may be out of the domain or sub-optimal for any individual prediction task. 
This problem is exacerbated in settings where the number and diversity of training tasks is relatively small.

\begin{figure*}[ht]
    \centering
    \includegraphics[width=1\textwidth]{figs/arch.pdf}
    \caption{\model architecture design. \model has
    a task-aware meta learner that generates the parameters of
    the feature layers within a prediction network for classification. The generated weights are
    factorized into task-specific weights in low dimension and 
    shared weights across all tasks.}
    \label{fig:model_arch}
\end{figure*}

In this work we explore a meta-learning approach to constructing task-aware feature embeddings (TAFE).  
We introduce task-aware feature embedding networks
(\modelplural\footnote{Pronounced taffy-nets}) composed of a task-aware meta learner that generates the parameters for feature embedding layers within a standard prediction network.
% As a consequence, the latent feature embedding is task-aware. % feature embedding layers in a prediction network to produce for the given input.% Instead of generating the classifier or using nearest neighbor, 
As a consequence, we are able to learn a simple task-independent linear boundary that can separate the positive and negative examples through the use of the task aware feature embeddings. 
To address the challenge of meta-learning with limited numbers of training tasks, we couple the task embedding to the task aware feature embeddings with the addition of a novel embedding loss.
The resulting coupling improves generalization across tasks by jointly clustering both images and tasks.
% in a common embedding space.% as additional task embedding learning constraint, with the benefits that the positive features can be rallied around the task in the joint embedding domain, in tandem with the linear classifier.Directly training the meta learner to generate the weights for the feature embedding network can be challenging~\cite{bertinetto2016learning} since the prediction task requires predicting a large number of weights from a low dimensional task embedding with only a few training tasks. 
% a large number of weights must be predicted relative to the size of the task embedding.% \xin{do you mean the number of total parameters to predict or the dimension? }\joey{They are the same thing}% partly due to the limited number of task available for training. % To address this, we introduce an additional embedding loss to ensure that the latent task description embedding aligns with generated task-aware feature embeddings. % Moreover, the output space of the% generated weights is often too large~\cite{bertinetto2016learning} which makes% the weight generation infeasible, 
We therefore introduce a novel method to factorize the generated
weights into small set of predicted weights and a larger set of shared weights.

% In this work we explore a meta-learning approach to constructing task-aware feature% embeddings (TAFE). We introduce task-aware feature embeddings networks% (\modelplural\footnote{Pronounced taffy-nets}) composed of a meta learner% that takes a task specification as input and generates the parameters for feature embedding layers of a prediction network. Instead of generating the classifier or% using nearest neighbor, we learn a simple fixed linear boundary that can% separate the positive and negative feature embeddings for% different tasks. Moreover, we put the task embedding in the same% domain as feature embeddings, as additional task embedding% learning constraint, with the benefits that the positive features% can be rallied around the task in the joint embedding% domain, in tandem with the linear classifier% Directly training the meta learner to generate the parameters% for a feature embedding networks can be challenging% partly due to the limited number of task descriptions available% for training. To address this, we introduce an additional% embedding loss to ensure that the latent task description% embedding aligns with generated task-aware feature embeddings.% Moreover, the output space of the generated weights% is often too large~\cite{bertinetto2016learning} which makes the weight generation% infeasible, we therefore propose to factorize the generated% weights into task-specific weights in low dimension and a% shared weights in full size for all tasks. % Directly training the meta learner to generate the parameters for the feature embedding network can be challenging~\cite{bertinetto2016learning} since the % dimension of the parameter space is too high to be generated through task % descriptions in low dimensions. We therefore introduce a novel method to factorization scheme that decomposes the generated weights into task-specific% parameters in low dimension and shared weights across all tasks. % the generated% weights into % since a large number of network parameters must be predicted. % \xin{not the number of parameter...% it's the dimension}% partly due to the limited number of task available for training. % To address this, we introduce an additional embedding loss to ensure that the latent task description embedding aligns with generated task-aware feature embeddings. % Moreover, the output space of the% generated weights is often too large~\cite{bertinetto2016learning} which makes% the weight generation infeasible, % weights into small set of task-specific weights and shared weight.

The proposed architecture exceeds the state-of-the-art zero-shot learning on four standard benchmarks without the need for additional data generation. On the newly proposed visual-attribute composition task which is a more challenging zero-shot learning task, 
we are able to achieve an 4 to 15 point improvements over state-of-the-art descriminative models based on joint embedding. Our methods also achieve competitive results on the challenging few-shot learning benchmark based on ImageNet. Furthermore, the proposed model can be used for efficient inference
on a per-input basis with little modification 
and exceeds the performance of prior models that focus on model sparsification. 

% Furthermore, our proposed architecture can be applied in settings where feature embeddings already exist as a mechanism to construct more computationally efficient embedding functions.% By treating the input image as a form of task specification we are able to use the same meta-learning architecture to generate input specific feature embeddings.% The resulting embeddings are able to reduce computation significantly. Compared to previous techniques producing sparse channels, our method can improve the model accuracy significantly with the less computation.% more computationally efficient feature embeddings for traditional data-rich image classification tasks. % \begin{enumerate}%     \item \xin{Start with: feature embedding is important and in most of previous work the feature representation are static} Feature embedding are central a wide range of computer vision tasks. \xin{Cite a range of papers saying that it's common %     to use static/fixed feature representation in most of visual learning task}. %     \item \xin{state the problem with static representation}. However, the static feature representation may not have optimal efficiency when learning model parameters or performing inference on a new sample point. \xin{basically re-interpret%     the title that we care about both data-efficiency and runtime %     efficiency} %     \item \xin{Put ourselves in a specific problem context}. One %     important problem in computer vision and machine learning is how%     to enable model generalization with limited data and information.%     We conjecture that building a dynamic feature representation %     which could encode the "situational" information at inference time%     is key to better generalization and learning efficiency. \xin{add examples/explanation, what could be the "situational" information. For example, the unseen task descriptions could be the situational information.  We can then say that image itself could bring us %     information about the prediction task and therefore we can build %     an instance-specific representation}%     \item \xin{Talk about what people used to do in the zero-shot/few-shot literature}. One common practice to project the%     task-specific (situational) information and the static image %     features in the same embedding space and learn a metric to model%     the relationship.  The issue with this type of approaches is that%     the image features are too generic and may not better suited for%     the given situation.  Another line of research to solve the %     generalization problem is through meta learning where a meta learner is introduced to generate weights for the pupil network (learnet, hypernet, etc). These approaches are more focused on %     generating dynamic representation in a hierarchical manner. That is to say, the task embedding determines the image embedding rather%     than learning jointly. \xin{we will argue why jointly learning is good later. The main reason is there are usually limited data for %     the task embedding. we want to have additional supervision to%     address the data scarcity problem to train the controller/meta learner}.%     \item \xin{now state our approach.}  We unify both worlds by introducing \model which generate situation-aware feature embeddings with a meta-learner/controller conditioned on the %     given situation information at inference time. The situation-aware%     image features are jointly learned with the task-specific %     embeddings to address the data scarcity problem of training such %     meta learner/controller. %     \item \xin{Transition to the tasks we want to solve}. Our main goal%     here is to evaluate our model design under different situations. %     We apply our model to zero-shot learning and a more realistic %     visual concept composition task where we are given new attribute information of the category or novel visual concepts (e.g. Attribute object pair) at the test time. We also explored the cases%     where either an exampler image of the category (few shot) or the input image itself are used as situaltional information. %     We demonstrate our approach are very competitive on all tasks and %     argue that the proposed situational-aware feature embedding %     helps efficient learning and inference. % \item Static feature embedding functions are common in many visual learning tasks, but may not have optimal efficiency when learning model parameters or performing inference on a new sample point.% \item Furthermore the optimal feature embedding may depend on both th. % \item In this paper we propose Situational Aware Feature embeddings that leverage meta-learning to generate a feature embedding function conditioned on the situation of the prediction.% % and depends heavily on the prediction task. % % % \item Reusing existing embeddings + fine tuning to a specific task is a common technique but cannot be applied to zero-show or few-shot learning.% \item In this work introduce a meta-learning technique to generate a mapping from a generic embedding to a task specific embedding that can adapt to the additional information at inference time.% \item We also explore the use of task specific embeddings in tradition classification problems where we use the input image as part of the task specification.  % \end{enumerate}% Fixed representations are common in many visual learning tasks, but may not have optimal efficiency when learning model parameters or performing inference on a new sample point.% We propose to configure the image representation network and resulting image embedding based on the new tasks in a layer-wise and structured fashion. The configuration is controlled by a learned task embedding produced by a controller network whose input is the task description. The task embedding is applied to each layer through layer specific decoder. To put it in another way, the controller network generates layer parameters for the image representation network. The image embedding is passed through a single task-agnostic classier, which can tell whether the image embedding belongs to a task or not.% By definition, we have much fewer tasks than total input images, which cause difficulty in optimizing the task embedding due to limited data point (while most of the previous generation methods focus on image generation). We propose a new joint embedding for the task and image embeddings to constrain that they should be in the same space and the task embedding is prototypical of its images.% ...zero shot and few shot learning are important and require task level generalization. It is hard to achieve. Previous works focus on joint embedding (DeViSE) or embedding correlation (Learnet) of cross domains such as image and language. The image embedding is independent of the task, which can limit the generalizability of those methods.% Due to the reconfiguration nature of our network structure, we can also apply it to channel sparsification, where the controller input is the same as the image input. The controller will produce a semantic embedding or summary pertiant to the pre-defined tasks and this embedding will be decoded at each layer to select the most relevant channels.% Our results on zero-shot learning show that our task embedding is better than the existing embedding learning methods and competitive with the data generation methods. Our layer sparsification can achieve state-of-the-art results compared to the previous works in channel pruning.
\section{Related Work}\minisection{Weight generation}. Several efforts~\cite{bertinetto2016learning, ha2016hypernetworks,denil2013predicting} have studied the idea of adopting one  meta network to generate weights of another network. Our task-aware meta learner serves a similar role for the weight generation 
but in a more structured and constrained manner. We study different mechanisms to decompose the weights of the prediction network so that it can generate weights for multiple layers at once where Bertinetton~\etal~\cite{bertinetto2016learning} 
focuses on generating weights for a single layer and Denil~\etal~\cite{denil2013predicting} can generate up to 95\% parameters of a single layer due to the quadratic size of the output space. As a byproduct of our design, we also study adding sparse constraints to the generated weights or efficient inference~\cite{lin2017runtime, wang2017skipnet, wu2018blockdrop} and the resulting model outperforms the state-of-the-art dynamic channel pruning technique~\cite{lin2017runtime}.

\minisection{Joint embedding learning}. Our architecture leverages common task and 
image embeddings to improve the task efficiency when learning the task 
embedding. This approach is commonly used in work on zero- and few- shot learning~\cite{koch2015siamese, yang2018learning, vinyals2016matching, zhang2016zero, Changpinyo_2016_CVPR, frome2013devise}. 
% Several works \cite{koch2015siamese, yang2018learning, vinyals2016matching} leverage different forms of common task and image embedding. 
A metric-learning based objective \cite{koch2015siamese, yang2018learning, vinyals2016matching} is often used to jointly regularize the task embeddings and 
image embeddings for training the classification network while our embedding objective is 
used to address the data scarcity issue of the weight generation network.
% However, in contrast to this earlier work, our embedding objective is used to aid in meta-learning for the weight generation network rather than direct classification.\minisection{Meta learning}. Our work resides in the meta learning regime~\cite{finn2018learning, finn2017model,schmidhuber1987evolutionary,graves2014neural} to learn the 
structure among previously seen tasks or concepts such that this learned prior can be combined with small amounts of new data for better generalization~\cite{finn2018learning}.  
Our task-aware meta learner generalizes to new tasks with limited training data 
thansks to our novel embedding loss. 

\minisection{Feature modulation}. In the domain of visual question answering, previous works~\cite{perez2018film, de2017modulating} explore the use of question embedding network 
to modulate the features of the primary convolutional network. Our factorized weight generation
scheme for convolutional layers can also be viewed as channel-wise feature modulation. 

% to generate scaling parameters for a convolutional network applied to the input scene. % The question embedding network performs a similar role to the meta-learner in our proposed architecture.  % However, our approach generalizes this earlier work by exploring general parameter estimation and introducing an additional embedding loss to improve learning with limited numbers of tasks. 
\section{Task-Aware Feature Embedding}

The choice of the featurization of data depends on the prediction tasks~\cite{donahue2014decaf, girshick2014rich}.
% \joey{Do we have any evidence to support this claim.  It is essential to the work in this paper but it is also debatable.}\xin{maybe we can go softer and say it's our % perspecitve?}% \joey{We need to take a position here.  Is there any evidence we can use to suppor that position? Perhaps that fine-tuning of feature functions generally improves performance.  Cite someone? decaf?}
For example, identifying whether images are underwater scenes or contain a particular digit relies on different representations (e.g., color or texture). 


In many cases there is rich semantic meta-data describing the concept associated with a prediction task.
For example, we might have text describing the objects of interest or the semantic meaning of the categories.
Even the input image can be used to help characterize the prediction task. 
Encoding the prediction task as part of the learning problem can aid in generalization.
Xian~\etal~\cite{xian2018zero} and Frome~\etal~\cite{frome2013devise} demonstrate that modeling 
the semantic relationship between tasks can help the model 
generalize to new tasks. 


In this work, we explore a mechanism 
that can better incorporate
the task specifications so that the model can learn and 
predict more efficiently. 
More specifically, we are
interested in the design of models that can leverage meta-data describing the task to augment the data featurization.


To address this problem, we propose \textit{Task-aware
Feature Embedding} networks (\modelplural) that generate 
feature embeddings conditioned on the task specification.
We adopt a meta learning approach to construct \modelplural
by generating task-specific parameters for the prediction
network and introduce two key innovations in weight 
factorization and embedding loss design to train \modelplural
effectively. 

\subsection{\model Model}
We start with the \model design. There are two sub-networks
in \model as shown in Figure~\ref{fig:model_arch}: a 
task-aware meta leaner $\mathcal{G}$ and a prediction network
$\mathcal{F}$. The task-aware meta learner takes a task specification (e.g., word2vec~\cite{mikolov2013efficient} encoding or example image) $\Vec{t}$ and generates the weights of layers in the prediction network. 
The prediction network $\mathcal{F}$: 
\begin{equation}
    \Vec{y} = \mathcal{F}(\Vec{x}; \theta=\{\mathcal{G}(\Vec{t}), \theta_f\}),
\end{equation}
takes images or image features  $\Vec{x}$ as inputs and predicts the class label $\Vec{y}\in\{1...N\}$.
The prediction network $\mathcal{F}$ is parameterized by $\theta$ which is composed
of generated parameters from $\mathcal{G}$ for task-aware feature embeddings 
and shared parameters  $\theta_f$ that are shared across tasks (e.g., generic feature extraction). 
% \joey{It is super weird that the classification parameters are independent of the classification task.  We really need a better explanation of this for the next version of this paper.}
The task-aware feature embedding (TAFE) is the layer output
of $\mathcal{F}$ before the final classification layer. 

The task-aware meta learner $\mathcal{G}$ paramterized by $\eta$ is composed of an embedding network 
$\mathcal{T}(\Vec{t})$ to generate a latent task embedding $\Vec{e}_\Vec{t}$ and a set of
weight generators $\Vec{g}^i, i=\{1...K\}$ that generate parameters for K feature
layers in $\mathcal{F}$ conditioned on $\Vec{e}_\Vec{t}$. 

% The prediction network $\mathcal{F}$ is composed of several dynamic layers \joey{What is a dynamic layer?  Shouldn't we just say whose parameters are determined by the weight generator?}% either% fully connected layers or convolutional layers whose% parameters $W_\text{d}$ is generated by $\mathcal{G}$ and a share classifier% parameterized by $W_\text{s}$ that maps the generated task-aware feature embeddings% to the class label space. Optionally, the prediction network could include a% learnable (the parameters are included in $W_\text{s}$ if applicable) or a% pre-trained feature extractor to produce generic image features. % The meta weight generator $\mathcal{G}$ consists of a task transformer $\mathcal{T}$% that projects the task specification $\Vec{t}$ to a latent task embedding space % denoted by $\Vec{e_\text{t}}$ and a set of layer-specific weight generator $\Vec{g}_i(\Vec{e_\text{t}})$ where $i=\{1...N\}$ % for each dynamic layer in $\mathcal{F}$ conditioned on $\Vec{e_\text{t}}$. As % noted by prior works~\cite{learnet, hypernet}, generating full weights of the% prediction network could be challenging due to the high dimensionality. We % therefore propose a channel-wise weight factorization design which is detailed % in Section~\ref{sec:weight_factorization}. \subsection{Weight Generation via Factorization}\label{sec:weight_factorization}
We now present details of the weight generation scheme for the feature layers in 
$\mathcal{F}$. 
The feature layers that produce the task aware feature embeddings (TAFE) can either be convolutional layers or fully-connected (FC) layers. 
As noted by Bertinetto~\etal~\cite{bertinetto2016learning}, the 
number of weights that must be estimated by the meta-learner is often much larger than the task specifications and can therefore be difficult to learn from a small number of example tasks.
To ensure meta learner generalizes effectively, we propose a weight 
factorization scheme along the output dimension of each FC layer and 
the output channel dimension of a convolutional layer.
This is distinct form the low-rank decomposition used in prior meta-learning works~\cite{bertinetto2016learning}. 
The channel-wise factorization builds on the intuition 
% behind grouped convolution: 
that channels of a convolutional layer
may have different or even orthogonal functionality. 
% \joey{Does the previous statement make sense? I thought connecting to grouped convolution might make this argument stronger.}\minisection{Weight factorization for convolutions}. 
Given an input tensor $\Vec{x}^i\in \mathbb{R}^{w\times h\times c_\text{in}}$ for
the $i\text{-th}$ feature layer in $\mathcal{F}$ whose weight is $\Vec{W}^i \in\mathbb{R}^{k\times k\times c_\text{in}\times c_\text{out}}$ 
($k$ is the filter support size and $c_\text{in}$ and $c_\text{out}$ are 
the number of input and output channels) and bias is  $\Vec{b}^i\in\mathbb{R}^{c_\text{out}}$, the output $\Vec{x}^{i+1}\in \mathbb{R}^{w'\times h'\times c_\text{out}}$ of the convolutional layer is given
by 
\begin{equation}
    \Vec{x}^{i+1} = \Vec{W}^i * \Vec{x}^{i} + \Vec{b}^{i},
\end{equation}
where $*$ denotes convolution. Without loss of generality, we remove the bias term of the convolutional layer as it is often followed by the batch normalization~\cite{ioffe2015batch}. $\Vec{W}^i = \Vec{g}^i(\Vec{t})$ is
the output of the $i\text{-th}$ weight generator in $\mathcal{G}$.  
We decompose the 
weight $\Vec{W}^i$ into 
\begin{equation}
\Vec{W}^i=\Vec{W}^i_\text{sr} *_{c_{\text{out}}} \Vec{W}^i_\text{ts},
\end{equation}
where $\Vec{W}^i_\text{sr}\in\mathbb{R}^{k\times k \times c_\text{in}\times c_\text{out}}$ is a shared parameter aggregating all tasks $\{\Vec{t}_1, ...\Vec{t}_N\}$ and $\Vec{W}_\text{ts}\in\mathbb{R}^{1\times1\times c_\text{out}}$
is a task-specific parameter depending on the current task input.
$*_{c_\text{out}}$ denotes the grouped convolution along the output channel dimension,
i.e. each channel of $x*_{c_\text{out}} y$ is simply the convolution of the corresponding channels in $x$ and $y$. 

\minisection{Weight factorization for FCs}. 
Similar to the factorization of the convolution weights,
the FC layer weights $\Vec{W}^i\in\mathbb{R}^{m\times n}$ can be decomposed into 
\begin{equation}
    \Vec{W}^i = \Vec{W}^i_\text{sr} \cdot \text{diag}(\Vec{W^i_\text{ts}}),
\end{equation}
where $ \Vec{W}^i_\text{sr}\in\mathbb{R}^{m\times n}$ is the shared parameters for
all tasks and $\Vec{W}^i_\text{ts}\in\mathbb{R}^{n}$ is the task-specific parameter.

With such factorization, the weight generators only need to generate the 
task-specific parameters for each task in lower dimension and learn one set of
parameters in high dimension shared across all tasks. 

\subsection{Embedding Loss for Meta Learner}
The number of task specifications used for training the
task-aware meta learner is usually much smaller than the number of images available
for training the prediction network. The data scarcity issue may lead to a 
corrupted meta learner. We, therefore, propose to add a secondary \textit{embedding
loss}$\mathcal{L}_\text{emb}$ for the meta learner alongside the standard classification loss $\mathcal{L}_\text{cls}$ used for the 
prediction network. The overall objective is then defined as 
\begin{equation}
    \min_{\theta, \eta} \mathcal{L} = \min_{\theta, \eta} \mathcal{L}_\text{cls}
    + \beta \cdot \mathcal{L}_\text{emb},
\end{equation}
where $\beta$ is the hyper-parameter to balance the two terms. We use $\beta$
as 0.1 in our experiments if not specified. 

The idea is to project the latent task embedding $\mathcal{T}(t)$ into a
joint embedding space with the task-aware feature embedding (TAFE). We adopt a metric learning approach that for positive inputs of a
given task, 
% \joey{How can $\mathcal{F}$ predict positive or negative?  It is a multi-class classifier.  It predicts a vector.}
the corresponding TAFE is 
closer to the task embedding while for negative inputs, the corresponding TAFE is
far from the task embedding as illustrated in Figure~\ref{fig:tafe}.

We use hinged cosine similarity as the distance measurement  
(i.e. $\mathcal{D}(p, q) = \max(\texttt{cosine\_sim}(p, q), 0)$)  and a regression
loss defined as 
\begin{equation}
    \mathcal{L}_\text{emb} =\frac{1}{TS} \sum_i^T\sum_j^S{||\mathcal{D}(\text{TAFE}(\Vec{x}_j; \vec{t}_i), \mathcal{T}(\vec{t}_i)) - \Vec{q}_{i,j}||_2^2},
\end{equation}
where $\Vec{x}_j$ is the $j\text{-th}$ sample in the dataset, $\Vec{t}_i$ is 
the $i\text{-th}$ task and $\Vec{q}_{i,j}$ is 1 if $\mathcal{F}(\Vec{x}_j; \Vec{t})_i$ predicts positive and 0 otherwise.
$T$ and $S$ is the total number of task descriptions and input samples respectively.

We find in experiments this additional supervision helps training the meta learner
especially under the case where the number of training tasks is extremely limited. 
% The form of the embedding loss can be different and we will describe an alternative way of adding supervision to the meta learner in the next section.\subsection{Shallow Image Embedding as a Task Desc.}\label{sec:shallow_emb_as_task}

An interesting usage of task aware feature embeddings is to adapt the feature embedding to the input image itself.  
Moreover, the image that we are trying to classify provides some hints as to what is likely a good embedding for classifying that image.
For example, having a hint as to the likely prediction or a high-level category is a form of task specifications that can inform the choice of image embedding.

% should inform the choice of image embedding. % Even every single image of the prediction% network itself can provide task information and can be used as the input of the task-aware meta learner.  In the latter case, % the meta learner generates parameters% on a per-input basis which has the potential to configure the network more optimally% for the specific input. 

In this work we also explore the use of task aware feature embeddings as a mechanism to construct more compact feature embeddings for traditional classification tasks.
% It is well known that deep learning models tend to % over-parameterize and not efficient in terms of computational cost at runtime. 
We find that with simple sparse constraints on the weight generator, our model can be used to construct a unique thin embedding network for each input image with only minimal loss in prediction accuracy.

In this setting, the task embedding network $\mathcal{T}$ takes the image as the input and outputs a probabilistic assignment to the pre-defined tasks. 
That is to say, the embedding loss of the meta learner is defined using the cross-entropy loss with sparse constraints over the output of the weight generators ($K$ is the number of feature layers in the prediction network):
\begin{equation}
    \mathcal{L}_\text{emb} = \frac{1}{S}\sum_j^S \texttt{CE}(\mathcal{T}(\Vec{x}_j), \Vec{y}_j) + \frac{1}{K}\sum_i^K||\Vec{g}^i(\mathcal{T}(\Vec{x}_j))||_1.
    \label{eq:ef_emb}
\end{equation}

This soft assignment can be regarded as a task description for the image at hand. 
The computation for feature embedding can be reduced with the information from the meta network, since we can use the task description to decide which channels in each layer are related to the tasks of interest and which are not. To remove the irrelevant channels, we pass each of the generated parameters through a {\tt ReLU} non-linearity.
If a parameter is 0, we don't have to compute the activation map for that channel because all the parameters can be generated before any computation in feature network. 
This is similar to model cascading~\cite{wang2017idk} if the meta and feature networks are both treated as making predictions. 
But in our case, the shallow network also generates weights for the more computation intensive network.

\subsection{Model Configurations}
We now describe the network configurations. In
our work, we consider three types of task specifications: semantic representation
in a vector format (e.g. word2vec~\cite{mikolov2013efficient} of the class labels),
image features extracted from pre-trained models (the spacial dimension is collapsed) and raw images. For the first two cases detailed in Section~\ref{sec:exp},
the task embedding network $\mathcal{T}$ is a three-layer FC network with hidden unit size of 2048 except 
for the aPY dataset~\cite{farhadi2009describing} where we choose  $\mathcal{T}$ as a 2-layer FC network with the same hidden size to avoid overfitting. The weight generator $
\Vec{g}^i$ is a single FC layer with the output dimension same as the output dimension
of the corresponding feature layer in $\mathcal{F}$. To enable efficient inference,
we just add a \texttt{ReLU} to the output of $\Vec{g}^i$. In the case where raw
images are used as inputs, $\mathcal{T}$ is configured as a 5-layer convolutional
network with $3\times3$ kernels following the feature down-sampling schedule of
ResNets~\cite{he2016deep}. 

For the prediction network, the TAFE is generated through a 3-layer FC
network with the hidden size of 2048 if image features are extracted from pre-trained
models and used as the input of the prediction network (on the aPY dataset, 
we use a 1024-1024-1024-2048 FC network to avoid overfitting). If raw images 
are used as the input, we will modify the backbone prediction network (e.g. VGG-16~\cite{simonyan2014very} and ResNet-50~\cite{he2016deep}) to generate the TAFE detailed in Section~\ref{sec:efficient_inference}.

%  We used 3 fully connected layers with hidden dimension% of 2048 in the controller module and the same network for the primary prediction network for all datasets except% for the AWA1 dataset which we reduce the size of the controller network to 2 fc layers with hidden dimension% of 2048 and 4 fc layers in the prediction network with hidden dimension of 1024, 1024, 1024 and 2048 % respectively to avoid overfitting. To train our network, we set the batch size to 32 and use Adam~\cite{kingma2014adam} as% the optimizer with the initial learning rate of 1e-4 for primary prediction network and weight generators and % 10e-5 for the parameters of the task embedding module. We reduce the learning rate by 10 at epoch 30 and 45 and% train the network for 60 epochs. For AWA1 dataset, we train the network for 10 epochs and reduce the learning % rate by 10 at epoch 5. 
\newcommand{\controlmodel}{controller\xspace}\newcommand{\controlmodelplural}{controllers\xspace}\section{Experiments}\label{sec:exp}% We first showcase the performance improvement of \model on generalized zero-shot learning. 
We first conduct experiments using five zero-shot learning benchmark datasets: SUN%~\cite{xiao2010sun}
, CUB%~\cite{welinder2010caltech}
, AWA1, %~\cite{lampert2009learning}, 
AWA2 %~\cite{xian2018zero}
and aPY %~\cite{farhadi2009describing} 
following the generalized zero-shot learning setting proposed by Xian~\etal~\cite{xian2018zero}. We also provide results on the more 
challenging visual attributes composition task proposed by Misra~\etal~\cite{misra2017red} on both
MITStates~\cite{StatesAndTransformations} and StanfordVRD~\cite{lu2016visual} datasets. Our model surpasses the 
state-of-the-art methods by a significant margin. We also evaluate the model on the few-shot learning and the 
dynamic model sparsification tasks on the CIFAR and ImageNet 
benchmarks and our model matches or exceed the prior works. 

% \subsection{Implementation Details}\subsection{Generalized Zero-shot Learning}
The GZSL setting proposed by Xian~\etal ~\cite{xian2018zero} is more realistic compared to the conventional zero-shot 
learning which involves classifying test examples from both seen and unseen classes, with no
prior distinction between them. We compare our model with 
two lines of prior work in our experiments: (1) discriminative baselines~\cite{zhang2017learning, yang2018learning}% \joey{need more references}
which focus on mapping the images into a rich semantic embedding space, and (2) generative models~\cite{xian2018feature,verma2017generalized}%go ahead% \joey{need more references gans papers}
that tackle the data
scarcity problem by generating synthetic images for the unseen classes using a GAN~\cite{goodfellow2014generative, zhu2017unpaired} based
approach. 
Our work falls into the first category of research but we demonstrate in 
% Table~\ref{tab:compare_with_additional_features}
Table~\ref{tab:zs_1}
that our approach is still competitive to the generative models with additional training data.

\minisection{Datasets and evaluation metrics}. Following prior works~\cite{zhang2016zero, frome2013devise, akata2016label, akata2015evaluation}, 
we conduct our experiments on 5 benchmark datasets (Table~\ref{tab:datasets}). We follow the evaluation metrics proposed by Xian~\etal~\cite{xian2018zero} to report the per class top-1 accuracy of both unseen $acc_u$ and seen classes $acc_s$ and the harmonic mean $H = 2 \times (acc_u \times acc_s)/(acc_u+acc_s)$. 

\minisection{Training details}. We set the batch size to 32 and use Adam~\cite{kingma2014adam} as
the optimizer with the initial learning rate of $10^{-4}$ for the prediction network and weight generators, and $10^{-5}$ for the task embedding network. 
We reduce the learning rate by 10$\times$ at epoch 30 and 45, and
train the network for 60 epochs. For AWA1, we train the network for 10 epochs and reduce the learning rate by 10$\times$ at epoch 5. 

\begin{table}[h]
\small
\centering
\caption{Datasets used in GZSL}
\vspace{-1em}
\addtolength{\tabcolsep}{-1pt}
\label{tab:datasets}
\adjustbox{width=\linewidth}{
\begin{tabular}{@{}cccccc@{}}
\toprule
Dataset            & SUN   & CUB   & AWA1   & AWA2   & aPY    \\ \midrule
No. of Images      & 14,340 & 11,788 & 30,475  & 37,322  & 15,339  \\
Attributes Dim.  & 102   & 312   & 85     & 85     & 64     \\
$\mathcal{Y}$      & 717   & 200   & 50     & 50     & 32     \\
$\mathcal{Y}^{seen}$ & 645   & 150   & 40     & 40     & 20     \\
$\mathcal{Y}^{unseen}$ & 72    & 50    & 10     & 10     & 12     \\
Granularity        & fine  & fine  & coarse & coarse & coarse \\ \bottomrule
\end{tabular}}
\end{table}\minisection{Quantitative results}. We first report the performance of our \model w/o the proposed embedding loss in Table~\ref{tab:zs_1} compared to the non-generative models. Our best performing models surpass 
 the prior models by a significant margin with an improvement of roughly 16 points on AWA1 and 17 points on aPY.
 For the more challenging fine-grained SUN and CUB datasets, we are able to improve the results by 7 and 2 
 points. Compared to the more recent approaches~\cite{zhang2017learning, yang2018learning}, our models have higher accuracy in the unseen classes which leads to the boost in the harmonic mean. 

As an alternative approach to address the GZSL problem, Several~\cite{xian2018feature, verma2017generalized} propose to generate synthetic 
images of the unseen classes conditioned on the class attributes with variants of GAN models~\cite{gan_goodfellow}. We show
the comparison of our model with these generative models in Table~\ref{tab:zs_1}. Our model matches or outperforms the
baseline generative models on both CUB and AWA1 datasets without using additional training data. 
This indicates that better embedding learning may be more beneficial than the synthetic data generation. 
However, task aware embeddings and data generation are complementary and could be combined to further improve accuracy.
% which is not the main focus of this paper. \begin{figure}[t]
    \centering
    \begin{subfigure}[t]{.8\linewidth}
        \centering
        \includegraphics[width=\linewidth]{figs/figs_r/zebra.pdf}
    \end{subfigure}%
    \\
    \vspace{2em}
    \begin{subfigure}[t]{.8\linewidth}
        \centering
        \includegraphics[width=\linewidth]{figs/figs_r/donkey.pdf}
    \end{subfigure}
    \caption{Task-aware Image Feature Embedding projected into two dimensions using t-SNE~\cite{tsne} for two tasks (Zebra and Donkey). Note that changing the task produces different embeddings for the same data.}
    \label{fig:tafe-vis}
\end{figure}\begin{table*}[t]
\small
\addtolength{\tabcolsep}{-1pt}
\centering
\caption{Evaluate \model on five standard benchmarks under generalized zero-shot learning setting. Models with $^\dagger$ (f-CLSWGAN and SE) generate additional data for training while the remaining models do not.  Our model is better than all models without
additional data and also competitive compared to models with additional synthetic data.}
\label{tab:zs_1}
\adjustbox{width=\linewidth}{
\begin{tabular}{@{}lccc|ccc|ccc|ccc|ccc@{}}
\toprule
\multirow{2}{*}{Method} & \multicolumn{3}{c|}{SUN} & \multicolumn{3}{c|}{CUB} & \multicolumn{3}{c|}{AWA1} & \multicolumn{3}{c|}{AWA2} & \multicolumn{3}{c}{aPY} \\
                        & u  & s  & H     & u   & s  & H     & u   & s   & H     
                        & u   & s   & H     & u   & s     & H  \\ \midrule
LATEM~\cite{zhang2016zero}                   & 14.7    & 28.8  & 19.5  & 15.2     & 57.3  & 24.0  & 7.3      & 71.7   & 13.3  & 11.5     & 77.3   & 20.0  & 0.1 & 73.0  & 0.2   \\
ALE~\cite{akata2016label}                    & 21.8    & 33.1  & 26.3  & 23.7     & 62.8  & 34.4  & 16.8     & 76.1   & 27.5  & 14.0     & 81.8   & 23.9  & 4.6      & 73.7  & 8.7   \\
DeViSE\cite{frome2013devise}                  & 16.9    & 27.4  & 20.9  & 23.8     & 53.0  & 32.8  & 13.4     & 68.7   & 22.4  & 17.1     & 74.7   & 27.8  & 4.9      & 76.9  & 9.2   \\
SJE~\cite{akata2015evaluation}                     & 14.7    & 80.5  & 19.8  & 23.5     & 59.2  & 33.6  & 11.3     & 74.6   & 19.6  & 8.0      & 73.9   & 14.4  & 3.7      & 55.7  & 6.9   \\
% ESZSL                   & 11.0    & 27.9  & 15.8  & 12.6     & 63.8  & 21.0  & 6.6      & 75.6   & 12.1  & 5.9      & 77.8   & 11.0  & 2.4      & 70.1  & 4.6   \\
SYNC ~\cite{Changpinyo_2016_CVPR}                  & 7.9     & 43.3  & 13.4  & 11.5     & 70.9  & 19.8  & 8.9      & 87.3   & 16.2  & 10.0     & 90.5   & 18.0  & 7.4      & 66.3  & 13.3  \\
DEM~\cite{zhang2017learning} & 20.5 & 34.3 & 25.6 & 19.6 & 57.9 & 29.2 & 32.8 & 84.7 & 47.3 & 30.5 & 86.4 & 45.1 & 11.1 & 75.1 & 19.4 \\
RelationNet~\cite{yang2018learning} & - & - & - & 38.1 & 61.1 & 47.0 & 31.4 & 91.3 & 46.7 & 30.0 & 93.4 & 45.3 & - & - & - \\
\midrule
f-CLSWGAN$^\dagger$~\cite{xian2018feature} &-&-&-& 57.7 & 43.7 & \textbf{49.7}&
61.4 & 57.9 & 59.6 &-&-&-& -&-&- \\
SE$^\dagger$~\cite{verma2017generalized} &-&-&-& 53.3 & 41.5 & 46.7 
& 67.8 & 56.3 & 61.5  &-&-&-& -&-&- \\
\midrule
\model* & 27.7 & 41.1 & \textbf{33.1} & 36.5 & 60.0 & 45.4 & 46.1 & 81.4 & 58.8 & 31.9 & 91.2 & 47.2 & 19.4 & 71.3 & 30.5 \\
\model & 27.9 & 40.2 & 33.0 & 41.0 & 61.4 & 49.2 & 50.5 & 84.4 & \textbf{63.2} & 36.7 & 90.6 &\textbf{52.2} & 24.3 & 75.4 & \textbf{36.8}\\
\bottomrule
\end{tabular}}
\end{table*}% \begin{table}[ht]% \centering% \small% \caption{Compare \model with models trained with additional generated features. Our model is still competitive even without additional synthetic data.}% \begin{tabular}{@{}lccc|ccc@{}}% \toprule% \multirow{2}{*}{Method} & \multicolumn{3}{c|}{CUB} & \multicolumn{3}{c}{AWA1} \\%   & u   & s  & H     & u   & s   & H  \\ \midrule% f-CLSWGAN~\cite{xian2018feature} & 57.7 & 43.7 & \textcolor{red}{\textbf{49.7}} & 61.4 & 57.9 & 59.6 \\% SE~\cite{verma2017generalized} & 53.3 & 41.5 & 46.7 & 67.8 & 56.3 & \textcolor{blue}{61.5} \\% \midrule% \model & 41.0 & 61.4 & \textcolor{blue}{49.2} & 50.5 & 84.4 & \textcolor{red}{\textbf{63.2}}\\% \bottomrule% \end{tabular}% \label{tab:compare_with_additional_features}% \end{table}\minisection{Embedding loss ablation}. We provide numbers for our model w/o the embedding loss in
Table~\ref{tab:zs_1}. 
In general, models with the embedding loss have stronger performance than those 
without the embedding loss except for the SUN dataset whose number of
categories is about $3-22\times$ larger than the other datasets.  This observation matches our assumption
that the additional supervision on the joint embedding better addresses the data scarcity issue (i.e. fewer
class descriptions than the visual inputs) of training the controller model. 

\minisection{Embedding visualization.}
In Figure~\ref{fig:tafe-vis}, we visualize the task-aware feature embeddings 
of images from the aPY dataset. As we can see, image embeddings are projected 
into different clusters conditioned on the task specification. 

\subsection{Unseen Visual-attribute Composition}
Besides the standard zero-shot learning benchmarks, we evaluate our model in the visual-attribute composition task proposed by Misra~\etal~\cite{misra2017red}. The goal is to compose a set of visual concept primitives like attributes 
and objects (e.g. \texttt{large elephant, old building,} etc.) to obtain new visual concepts for a given image. We see this task as a more challenging ``zero-shot'' learning task which requires the model not only to predict unseen visual concept compositions but model the contextuality of the concepts. 

\minisection{Datasets and evaluation metrics}. We conduct the experiments on two
datasets: MITStates~\cite{StatesAndTransformations} and
the modified StanfordVRD~\cite{lu2016visual}. The setup is the same as Misra~\etal~\cite{misra2017red}.
Each image in the MITStates dataset is assigned a pair of (attribute, object) as
its label. The model is trained on 34K images with 1,292 label pairs and tested on
19K images with 700 unseen pairs. The second dataset is constructed based on 
the bounding boxes annotations of the StanfordVRD dataset. Each sample has an SPO (subject, predicate, object)
tuple as the ground truth label.  The dataset has 7,701 SPO tuples and 1,029 of them
are seen only in the test split. We evaluate our models only on examples with unseen
labels. We extract the image features with
pre-trained models on ImageNet. We use ResNet-101~\cite{he2016deep} as our main
feature extractor and also test features extracted 
with ResNet-18~\cite{he2016deep}, VGG-16/19~\cite{simonyan2014very} for ablation. 
For the task specifications, we concatenate the word embeddings of the attributes
and objects with word2vec~\cite{mikolov2013efficient} trained with GoogleNews. We
also consider one-hot encoding for the task id in the ablation. 

For evaluation metrics, we report the mean Average Precision (mAP) of images 
with unseen labels in the test set together with the top-$k$ accuracy where
$k=1,2,3$. We follow the same training schedule as that used in GZSL.

\minisection{Quantitative results}. We compare our model with
several baselines provided by Misra~\etal~\cite{misra2017red} and
summarize the results in
Table~\ref{tab:mit} for both MITStates and StanfordVRD
datasets. Our model surpasses the state-of-the-art models
with an improvement
of more than 6 points in mAP and 4 to 15 points in top-$k$
accuracy. Nagarajan and Grauman~\cite{nagarajan2018attrop}
recently propose an embedding learning framework for visual-attribute composition. They report the top-1 
accuracy of $12.0\%$ on the MITStates dataset with
ResNet-18 features. For fair
comparison, we use the same ResNet-18 features and obtain
the top-1 accuracy of $15.1\%$. 

% \begin{table}[ht]% \small% \centering% \caption{Evaluation on 700 unseen (attribute, object) pairs on 19K images of the MITStates Dataset and 1029 unseen SPO tuples on 1000 images of the StanfordVRD Dataset.  \model improves over the baselines by a large margin.% }% \label{tab:mit}% \adjustbox{width=\linewidth}{% \begin{tabular}{lcccc|cccc}% \toprule% &  \multicolumn{4}{c}{MITStates} & \multicolumn{4}{c}{StanfordVRD} \\% \midrule% \multirow{2}{*}{Method} & \multirow{2}{*}{AP} &  \multicolumn{3}{c}{Top-$k$ Accuracy} & \multirow{2}{*}{AP} &  \multicolumn{3}{c}{Top-$k$ Accuracy} \\% &   & 1 & 2 & 3 & & 1 & 2 & 3 \\ \midrule% Visual Product~\cite{misra2017red} & 8.8 & 9.8 & 16.1 & 20.6 & 4.9 & 3.2 & 5.6 & 7.6 \\% Label Embed (LE)~\cite{misra2017red} & 7.9 & 11.2  & 17.6 & % 22.4 &  4.3 & 4.1  & 7.2  & 10.6  \\% LEOR~\cite{misra2017red} & 4.1 & 4.5  & 6.2 & 11.8 & % 0.9 & 1.1 & 1.3 & 1.3 \\% LE + R~\cite{misra2017red} & 6.7 & 9.3 & 16.3 & 20.8 &% 3.9 & 3.9 & 7.1 & 10.4 \\% Red Wine~\cite{misra2017red} & 10.4  & 13.1 & 21.2 & 27.6 &% 5.7  & 6.3 & 9.2  & 12.7 \\ \midrule% % \model & 16.2 & 17.2 & 27.8 & 35.7 \\% \model& \textbf{16.2} & \textbf{16.9} & \textbf{27.9} & \textbf{35.1} & \textbf{12.2}  & \textbf{12.3}  &\textbf{19.7} & \textbf{27.5} \\% % \model (ResNet-18)& 14.4 & 15.1 & 24.6 & 31.4 \\% % \model* (ResNet-18) & 14.5 & 15.1 & 25.0 & 31.6 \\% \bottomrule% \end{tabular}}% \end{table}% \begin{table}[ht]% \centering% \small% \caption{Evaluation on 1029 unseen SPO tuples on 1000 images of the StanfordVRD Dataset.}% \label{tab:vrd}% \begin{tabular}{@{}lcccc@{}}% \toprule% \multirow{2}{*}{Method}   & \multirow{2}{*}{AP} & \multicolumn{3}{c}{Top-$k$ Accuracy} \\% &        & 1          & 2          & 3          \\ \midrule% % Chance   & -                   & 0.09       & 0.18       & 0.27       \\% Visual Product            & 4.9    & 3.2        & 5.6        & 7.6  \\% Label Embed (LE)          & 4.3    & 4.1        & 7.2        & 10.6 \\% LE Only Reg.\ (LEOR)      & 0.9    & 1.1        & 1.3        & 1.3  \\% LE + Reg.\ (LE + R)       & 3.9    & 3.9        & 7.1        & 10.4 \\% Red Wine~\cite{misra2017red}          & 5.7    & 6.3        & 9.2        & 12.7 \\% \midrule% % \model   & 12.0  & 12.3  & 19.4 & 26.2  \\% \model  & \textbf{12.2}  & \textbf{12.3}  &\textbf{19.7} & \textbf{27.5}  \\    % \bottomrule% \end{tabular}% \end{table}\begin{figure*}

\begin{minipage}[t]{.58\linewidth}
   \centering
   \captionof{table}{Evaluation on 700 unseen (attribute, object) pairs on 19K images of the MITStates Dataset and 1029 unseen SPO tuples on 1000 images of the StanfordVRD Dataset. \model improves over the baselines by a large margin.}
   \label{tab:mit}
   \small
   \adjustbox{width=\linewidth}{
    \begin{tabular}{lcccc|cccc}
    \toprule
    &  \multicolumn{4}{c}{MITStates} & \multicolumn{4}{c}{StanfordVRD} \\
    \midrule
    \multirow{2}{*}{Method} & \multirow{2}{*}{AP} &  \multicolumn{3}{c}{Top-$k$ Accuracy} & \multirow{2}{*}{AP} &  \multicolumn{3}{c}{Top-$k$ Accuracy} \\
    &   & 1 & 2 & 3 & & 1 & 2 & 3 \\ \midrule
    Visual Product~\cite{misra2017red} & 8.8 & 9.8 & 16.1 & 20.6 & 4.9 & 3.2 & 5.6 & 7.6 \\
    Label Embed (LE)~\cite{misra2017red} & 7.9 & 11.2  & 17.6 & 
    22.4 &  4.3 & 4.1  & 7.2  & 10.6  \\
    LEOR~\cite{misra2017red} & 4.1 & 4.5  & 6.2 & 11.8 & 
    0.9 & 1.1 & 1.3 & 1.3 \\
    LE + R~\cite{misra2017red} & 6.7 & 9.3 & 16.3 & 20.8 &
    3.9 & 3.9 & 7.1 & 10.4 \\
    Red Wine~\cite{misra2017red} & 10.4  & 13.1 & 21.2 & 27.6 &
    5.7  & 6.3 & 9.2  & 12.7 \\ \midrule
    % \model & 16.2 & 17.2 & 27.8 & 35.7 \\
    \model& \textbf{16.2} & \textbf{16.9} & \textbf{27.9} & \textbf{35.1} & \textbf{12.2}  & \textbf{12.3}  &\textbf{19.7} & \textbf{27.5} \\
    % \model (ResNet-18)& 14.4 & 15.1 & 24.6 & 31.4 \\
    % \model* (ResNet-18) & 14.5 & 15.1 & 25.0 & 31.6 \\
    \bottomrule
    \end{tabular}}
\end{minipage}\hfill
\begin{minipage}[t]{.38\linewidth}
   \centering
   \captionof{table}{Ablation study with different task encoding and base network features. The variance of performance of \model under different settings is minimal.}
   \label{tab:comp_ab}
   \adjustbox{width=\linewidth}{
   \renewcommand{\arraystretch}{1.27}% for the vertical padding
    \begin{tabular}{lccccc}
    \toprule
    \multirow{2}{*}{Task Encoding} & \multirow{2}{*}{Features}  & \multirow{2}{*}{AP} & \multicolumn{3}{c}{Top-$k$ Accuracy} \\
    &           &          & 1          & 2          & 3          \\ \midrule
    Word2vec  & ResNet-101 & 16.2 & 17.2    & 27.8       & 35.7       \\
    Onehot    & ResNet-101 & 16.1  & 16.1   & 26.8       & 33.8       \\
    \midrule
    Word2vec  & VGG16  & 16.3  & 16.4       & 26.4       & 33.0       \\
    Onehot    & VGG16  & 16.3  & 16.4       & 25.9       & 32.5       \\
    \midrule
    Word2vec  & VGG19  & 15.6  & 16.2       & 26.0       & 32.4       \\
    Onehot    & VGG19  & 16.3  & 16.4       & 26.0       & 33.1       \\
    \bottomrule
    \end{tabular}}
\end{minipage}
\end{figure*}\minisection{Ablation on the feature extractor and task specification}. We consider 
different feature extractors (ResNet-101, VGG-16 and 19) and task encodings (word2vec and one-hot encoding) for ablation and summarize the results in Table~\ref{tab:comp_ab}. The average precision difference between different feature
extractors are very minimal (within $0.1\%$) and the largest gap in Top-3 accuracy 
is within $2\%$. This indicates that \model is robust in transforming the 
generic features into task-aware feature embeddings. For the task encoding,
the one-hot encoding is comparable to the word2vec encoding and even stronger 
when using VGG-19 features. This shows that the task transformer network $\mathcal{T}$ is very expressive to extract rich semantic information simply from the
task ids. 

\begin{figure*}[t!]
    \centering
    \small
    % \includegraphics[width=\linewidth]{figs/retrievals.pdf}
    % \begin{tabular}{@{}c @{\hskip 0.05in} c @{\hskip 0.05in} c @{\hskip 0.05in} c @{\hskip 0.05in} c@{}}
    \adjustbox{width=\linewidth}{
    \begin{tabular}{c | c}
    Modern City & Ancient Town \\
    \includegraphics[width=0.49\linewidth]{figs/figs_r/retrievals/modern_city.jpg}
    &
    \includegraphics[width=0.49\linewidth]{figs/figs_r/retrievals/ancient_town.jpg}
     \\
     Modern Clock & Ancient Clock \\
    \includegraphics[width=0.49\linewidth]{figs/figs_r/retrievals/modern_clock.jpg}
    &
    \includegraphics[width=0.49\linewidth]{figs/figs_r/retrievals/ancient_clock.jpg}
    \\
    Sunny Valley & Sunny Creek \\
    \includegraphics[width=0.49\linewidth]{figs/figs_r/retrievals/sunny_valley.jpg}
    &
    \includegraphics[width=0.49\linewidth]{figs/figs_r/retrievals/sunny_creek.jpg}
    \\
    \end{tabular}}
    \caption{Top retrievals on the unseen pairs of the MITStates dataset. Our model can learn to compose new concepts from the existing attributes and objects
    while respecting their context. The second row shows some of the failure cases.}
    \label{fig:retrieval}
\end{figure*}\minisection{Visualization}. 
In Figure~\ref{fig:retrieval}, we show the top
retrievals of unseen attribute-object pairs from the MITStates dataset. 
Our model can learn to compose new concepts from the existing attributes and objects
while respecting their context.

% \begin{table}[ht]% \small% \centering% \caption{Ablation study with different task encoding and base network features. The variance of performance of \model under% different settings is minimal.}% \label{tab:comp_ab}% \adjustbox{width=\linewidth}{% \begin{tabular}{lccccc}% \toprule% \multirow{2}{*}{Task Encoding} & \multirow{2}{*}{Features}  & \multirow{2}{*}{AP} & \multicolumn{3}{c}{Top-$k$ Accuracy} \\% &           &          & 1          & 2          & 3          \\ \midrule% Word2vec  & ResNet-101 & 16.2 & 17.2    & 27.8       & 35.7       \\% Onehot    & ResNet-101 & 16.1  & 16.1   & 26.8       & 33.8       \\% \midrule% Word2vec  & VGG16  & 16.3  & 16.4       & 26.4       & 33.0       \\% Onehot    & VGG16  & 16.3  & 16.4       & 25.9       & 32.5       \\% \midrule% Word2vec  & VGG19  & 15.6  & 16.2       & 26.0       & 32.4       \\% Onehot    & VGG19  & 16.3  & 16.4       & 26.0       & 33.1       \\% \bottomrule% \end{tabular}}% \end{table}\subsection{Low-shot Image Classification}
Our model naturally fits the few-shot learning setting where one or few images
of a certain category are used as the task descriptions. Unlike prior work on meta-learning 
which experiments with few classes and low resolution images~\cite{vinyals2016matching, snell2017prototypical,finn2017model}, we evaluate
our model on the challenging benchmark proposed by Hariharan and Girshick~\cite{hariharan2017low}. The benchmark is based on the ImageNet images and
contains hundreds of classes that are divided into base classes and novel classes.
At inference time, the model is provided with one or a few examples from the novel classes
and hundreds of examples from the base classes. 
The goal is to obtain high accuracy 
on the novel classes without sacrificing the performance on the base classes. 

\minisection{Baselines}. In our experiments, the  baselines we consider are the 
state-of-the-art meta learning models Matching Network (MN)~\cite{vinyals2016matching}
and Prototypical Network (PN)~\cite{snell2017prototypical}. We also compare the logistic regression (LogReg) baseline provided by Hariharan and Girshick~\cite{hariharan2017low}. Another line of research~\cite{wang2018low, hariharan2017low} for few-shot learning is to combine
the meta-learner with a ``hallucinator'' to generate additional training data. We
regard these works as complementary approaches to our meta-learning model. 

\begin{table}[t]
\centering
\small
\caption{Few-shot ImageNet Classification on ImageNet. Our model
is competitive compared to the state-of-the-art meta learning model
without hallucinator.}
\label{tab:few-shot}
\begin{tabular}{@{}l|cc|cc@{}}
\toprule
Method & \multicolumn{2}{c}{Novel Top-5 Acc} & \multicolumn{2}{|c}{All Top-5 Acc} \\
 & n=1 & n=2  & n=1 & n=2  \\ 
 \midrule
LogReg ~\cite{hariharan2017low} & 38.4 & 51.1  & 40.8 & 49.9 \\
PN~\cite{snell2017prototypical} & 39.3 & 54.4  &  49.5 & 61.0 \\
MN~\cite{vinyals2016matching} &  43.6 & 54.0  &  54.4 & 61.0  \\
\midrule
\model & 43.0 & 53.9 & \textbf{55.7} & \textbf{61.9} \\
\midrule\midrule
LogReg w/ Analogies~\cite{hariharan2017low} & 40.7 & 50.8 & 52.2 & 59.4 \\
PN w/ G ~\cite{wang2018low} & 45.0 & 55.9 & 56.9 & 63.2 \\ 
% SGM+ MN & 41.3 & 51.3 & 62.1 & 67.8 & 71.8 & 55.0 & 61.5 & 69.3 & 73.4 & 76.2 \\ 
% SGM+\model & 41.7 & 53.5 & 63.5 & 68.9 & 71.3 & 55.4 & 62.1 & 68.2 & 71.3 & 73.2 \\
\bottomrule
\end{tabular}%
\vspace{-1em}
\end{table}\minisection{Experiment details}. We follow the prior works~\cite{hariharan2017low, wang2018low} to run five trials for each setting of n (the number of examples per
novel class, n = 1 and 2 in our experiments) on the five different data splits and report the average top-5 accuracy of both
the novel and all classes. We use the features trained with ResNet-10 using SGM
loss provided by Hariharan and Girshick~\cite{hariharan2017low} as inputs. 
For training, we sample 100
classes in each iteration and use SGD with momentum of 0.9 as the optimizer. The
initial learning rate is set to 0.1 except for the task embedding network (set to
0.01) and the learning rate is reduced by 10$\times$ every 8k iterations.
The model is trained for 
30k iterations in total. Other hyper-paramters are set to the same as Hariharan and Girshick~\cite{hariharan2017low} if not mentioned.

\minisection{Quantitative results}. As shown in Table~\ref{tab:few-shot}, our
model is on par with the state-of-the-art meta learning models in the 
novel classes while outperforming them in all categories. Attaching a ``hallucinator'' to the meta learning model improves performance in
general. Our model can be easily attached with a hallucinator and we leave
the detailed study as future work due to the time constraint. 

\subsection{Efficient Inference in Data-Rich Classification}\label{sec:efficient_inference}
As noted in Section~\ref{sec:shallow_emb_as_task}, the task embedding
network and the feature embedding network can share the same input image. 
In this case, the shallow embedding of the input image can be treated as a
``task''and conditioned on it, and the weight generator network generates weights that
are customized for the given input. Adding additional sparse constraints to 
the generated weights, our model can be used to selectively sparsify the channels
of the convolutional layers of the prediction network on a per-input basis. This
is in line with prior works~\cite{lin2017runtime, luo2017thinet, he2017channel} on channel pruning and we demonstrate that with a slight
change in the weight generator (adding sparsity constraints), our model is 
competitive with or beats existing work on model 
sparsification. 

\minisection{Baselines}. The closest baseline is the runtime neural pruning (RNP) 
proposed by Lin~\etal~\cite{lin2017runtime}, which dynamically prunes the channels
of the subsequent convolutional layers based on the previous layer outputs. We
compare with RNP on the CIFAR-100~\cite{krizhevsky2009learning} dataset using VGG-16 as the backbone following the setting as Lin~\etal~\cite{lin2017runtime}. We also
consider state-of-the-art static channel pruning works~\cite{he2017channel, luo2017thinet,huang2017data, li2016pruning}, with ResNet-50 on the 
ImageNet-2012~\cite{russakovsky2015imagenet} dataset.

\minisection{Sparse weight generation for residual blocks}. For the bottleneck residual block used in ResNet-50~\cite{he2016deep} which has 3 convolutional layers with the kernel filter
size of 1, 3 and 1, we only generate sparse
weights for the middle convolutional layer which is the computation bottleneck and share the weights of the first and last layer across inputs.

% leave the weights of the first and last layer shared across different inputs. \minisection{Training details}. For the CIFAR dataset, we start training with the learning rate of 0.1 for ResNet and 0.01 for VGG16, which is reduced by $10\times$ at the 150\textsuperscript{th} and 250\textsuperscript{th} epochs with total 350 epochs for the baselines and 270 epochs for \model joint optimization stage and another 80 epochs for fine-tuning with fixed task embedding
networks. For ImageNet, we train the network with the initial learning rate of 0.1 for 100 epochs and reduce it by 10$\times$ every 30 epochs.

\minisection{Quantitative results}. We summarize our comparison with the baselines 
in Figure~\ref{fig:rnp} and Table~\ref{tab:resnet-50-imagenet}. Our approach outperforms
the previous baselines to achieve higher accuracy with lower runtime computation measured in floating point operations per second (FLOPs). 

\begin{figure}[t]
    \centering
    \adjustbox{width=\linewidth}{
    \includegraphics[width=\linewidth]{figs/cifar_v2.pdf}}
    \vspace{-2em}
    \caption{Comparison of \model and RNP with VGG-16 on CIFAR-100. Our model outperforms RNP under different accuracy and computation trade-off.  $\beta$ controls the amount of the L1 regularization in Equation~\ref{eq:ef_emb}.}
    \label{fig:rnp}
\end{figure}\begin{table}[t]
\small
\centering
\caption{Pruned ResNet-50 on ImageNet. Top-1/5 error rate and
computation FLOPs are reported. Our model achieves lower
error rate with lower computational cost.}
\label{tab:resnet-50-imagenet}
\adjustbox{width=\linewidth}{
\begin{tabular}{l|cccc}
\toprule
Model  & Top-1 & Top-5 &  FLOPs(x$10^9$) & Reduct.(\%) \\
% \midrule
% ResNet-18 & 30.24 & 10.92 & 1.80 & - \\
% \model-DLA-34-v1 & 30.41 & 10.91 & \textbf{0.79} & 74.32\\
\midrule
SSS~\cite{huang2017data} & 26.8 & - & 3.0 & 20.3 \\
Li et al.~\cite{li2016pruning} & 27.0 & 8.9 & 3.0 & 19.0 \\
He et al.~\cite{he2017channel} & - & 9.2 & 1.9 & 50.0 \\
ThiNet~\cite{luo2017thinet} & 29.0 & 10.0 & 1.7 & 55.8 \\
% \model-DLA-34-v2 & 28.22 & 9.42 & 1.69 & 44.78 \\
\midrule
\model & \textbf{26.2} & \textbf{8.4} & \textbf{1.6} & \textbf{56.8} \\
\bottomrule
\end{tabular}}
    \vspace{-1em}
\end{table}% \paragraph{Training} % To train \model we follow common training practices~\cite{he2016deep, yu2017deep}. % For the CIFAR datasets, we start training with learning rate 0.1 for ResNet and 0.01 for VGG16, which is reduced by $10\times$ at 150 and 250 epochs with total 350 epochs for the baselines and 270 epochs for \model joint optimization stage and another 80 epochs for fine-tuning with fixed gating networks.% For ImageNet, % we train the network% with initial learning rate 0.1 for 100 epochs and reduce it by 10$\times$ every 30 epochs. % We do not further fine-tune the base convolutional network on ImageNet as we find the improvement from fine-tuning is marginal compared to that on CIFAR datasets.% We set the computational cost parameter $\lambda$ in the \model loss function (Eq.~\ref{equ:objective})% between [0.001, 8] (larger values reduce computation more).% We set $\mu = 1$ % for the CIFAR datasets to match the scale of the cross entropy loss on the base model. For ImageNet we set $\mu=0$ to improve base model feature extraction.% % Please add the following required packages to your document preamble:% % \usepackage{booktabs}% \begin{table}[]% \centering% \begin{tabular}{@{}c|cc@{}}% \toprule% Method               & Top-1 Error & FLOPs (x$10^8$) \\ \midrule% Li et al.\ ICLR 2017 & 8.50        & 0.63            \\% He et al.\ ICCV 2017 & 8.20        & 0.63            \\% \model               & 7.24        & 0.58            \\% \model               & 7.19        & 0.63            \\ \midrule% Li et al.\ ICLR 2017 & 6.94        & 0.91            \\% \model               & 7.00        & 0.70            \\% \model               & 6.80        & 0.78            \\% \model               & 6.40        & 0.89            \\ \midrule% Li et al.\ ICLR 2017 & 6.90        & 1.12            \\% \model               & 6.24        & 1.09            \\ \midrule% ResNet-56            & 6.33        & 1.25            \\% \model               & 6.16        & 1.23            \\ \bottomrule% \end{tabular}% \caption{\model \emph{vs} static pruning on ResNet-56 on CIFAR-10.}% \end{table}% % Please add the following required packages to your document preamble:% % \usepackage{booktabs}% \begin{table}[]% \centering% \begin{tabular}{@{}ccc@{}}% \toprule% Method               & Top-1 Error & FLOPs (x$10^8$) \\ \midrule% Li et al.\ ICLR 2017 & 6.70        & 1.54            \\% \model               & 6.76        & 1.13            \\% \model               & 6.20        & 1.37            \\ \midrule% Li et al.\ ICLR 2017 & 6.45        & 2.10            \\% ResNet-110           & 5.80        & 2.50            \\% \model               & 5.82        & 2.09            \\ \bottomrule% \end{tabular}% \caption{\model \emph{vs} static pruning on ResNet-110 on CIFAR-10.}% \end{table}% Please add the following required packages to your document preamble:% \usepackage{booktabs}% \begin{table}[]% \centering% \small% \caption{\model \emph{vs} RPN on VGG on CIFAR-100.\xin{convert this back to a plot?}}% \label{rnp}% \begin{tabular}{@{}ccc@{}}% \toprule% Method                  & Top-1 Error (\%) & FLOPs ($\times 10^8$) \\ \midrule% RNP                     & 40.80       & 0.38            \\% RNP                     & 39.66       & 0.87            \\% \model without finetune & 37.64       & 0.41            \\% \model                  & \textbf{32.91}       & \textbf{0.41}            \\ \midrule% RNP                     & 37.89       & 1.64            \\% \model without finetune & 32.33       & 1.45            \\% \model                  & \textbf{28.15}       & \textbf{1.45}            \\ \midrule% RNP                     & 35.98       & 3.13            \\% \model without finetune & 32.16       & 1.93            \\% \model without finetune & 31.25       & 2.28            \\% \model without finetune & 31.81       & 3.13            \\% \model      & \textbf{27.22}       & \textbf{1.93} \\% \model with finetune    & 27.17       & 2.28            \\ % \bottomrule% \end{tabular}% \end{table}
\section{Conclusion}
In this work, we explored a meta learning based approach to generate task aware feature embeddings for settings with little or no training data.
We proposed TAFE-Net, a network that generates task aware feature embeddings (TAFE) conditioned on the given task descriptions.
\model is composed of a task-aware 
meta learner that generates weights for the feature embedding layers in a standard prediction network.
To address the challenges in training the meta learner, we introduced two key
innovations: (1) adding an additional embedding loss to improve the generalization of the meta learner; (2) a novel weight factorization
scheme to generate parameters of the prediction network more effectively. 
We demonstrated the general applicability of the proposed network design on a range of applications including zero/few shot learning and dynamic
efficient prediction, and exceeded or matched the state-of-the-art on most applications. 




% \subsubsection*{Acknowledgments}

{\small
\bibliographystyle{ieee}
\bibliography{references}
}
% \clearpage
% \newpage
% \input{appendix}
% \clearpage


\end{document}
