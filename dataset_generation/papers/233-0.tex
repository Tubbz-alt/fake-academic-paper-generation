\documentclass{article}
\usepackage[utf8]{inputenc}
\usepackage[T1]{fontenc}
\begin{document}

\title{Review Feedback}
\maketitle
\begin{center}
\section*{Reviewer 1}
\end{center}

\begin{enumerate}
\item Only a total of 18 volumes (6 training, 3 validation and 9 testing) is used. This is very small.
 \\

Indeed but IBSR is golder standard dataset comprises wide range of brain volumes across age and gender. We have selected small patch of $40 \times 40 \times 3$ as an input to our U-SEGNET so we have enough data to train a deep network.  

\begin{center}
\section*{Reviewer 2}
\end{center}


\item Section 2 should be a part of Section 4, instead of being an independent section.
 \\
done

\begin{center}
\section*{Reviewer 3}
\end{center}
\item Since the number of subjects is small (only 18) this means that results are derived by training only on 9 subjects and then testing on the other 9. What if these particular test subjects were the 'easy' cases? How could the results change if we inverted the splitting? By running the experiment multiple times and averaging the results we could have a better idea about the actual performance of the proposed method. \\

Thank you so much for addressing this issue, we have not done cross validation. But to overcome the 'easy' case problem our train-test split comprises all the variation across age and gender. At training time we select $6$ volume for training and $3$ for validation and reported the dice ratio on $9$ test volumes.

\item Another point is that Fuzzy c-means performs surprizing well for WM segmentation (top performance). Thats surprising for such an 'old' method. It would be good to at least comment on this and perhaps share some insights as to why this happens. \\

Fuzzy c means use a priori spatial tissue probability maps generate from brain atlas which may have resulted the performance improvement for WM. Our method does not use a prior tissue probability map. Though we noticed dip in WM performance but the  weighted dice ratio has significantly improved. 

\end{enumerate}

\end{document}

\grid
