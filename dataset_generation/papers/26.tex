\documentclass[twocolumn,letter,10pt]{IEEEtran} %!PN
%===============================================================================================================
%===============================  Defining document properties (Author, date, ... )  ===========================
\def \deTitle{}
\def \enTitle{On the Feasibility of Generic Deep \\Disaggregation for Single-Load Extraction}
\def \student{Karim Said Barsim}   % FIXME
\def \professor{Bin Yang} % FIXME
\def \worksubject{IEEE Transactions on Smart Grid}
\def \submission{2016}
\def \doclang{english}
\def \colortype{boxed} 	% color, bw, boxed
\def \keywords{
	Energy/Load disaggregation,
	Non-Intrusive Load Monitoring (NILM),
	Convolutional Neural Networks (CNN),
	UNet, SegNet, UK-DALE
}
\def \paperAbstract{
Recently, and with the growing development of big energy datasets, data-driven learning techniques began to represent a potential solution to the energy disaggregation problem outperforming engineered and hand-crafted models.
However, most proposed deep disaggregation models are load-dependent in the sense that either expert knowledge or a hyper-parameter optimization stage is required prior to training and deployment (normally for each load category) even upon acquisition and cleansing of aggregate and sub-metered data.
In this paper, we present a feasibility study on the development of a generic disaggregation model based on data-driven learning.
Specifically, we present a generic deep disaggregation model capable of achieving state-of-art performance in load monitoring for a variety of load categories.
The developed model is evaluated on the publicly available UK-DALE dataset with a moderately low sampling frequency and various domestic loads.
}

%===============================================================================================================
%================================  Importing packages (with perioritys (1-5))  =================================
% \usepackage[export]{adjustbox}% The package provides several macros to adjust boxed content. One purpose is to supplement the standard graphics package, which defines the macros \resizebox, \scalebox and \rotatebox , with the macros\trimbox and \clipbox. The main feature is the general \adjustbox macro which extends the ?key=value? interface of \includegraphics from the graphics package and applies it to general text content. Additional provided box macros are \lapbox, \marginbox, \minsizebox, \maxsizebox and \phantombox. All macros use the collectbox package to read the content as a box and not as a macro argument. This allows for all forms of content including special material like verbatim content. A special feature of collectbox is used to provide matching environments with the identical names as the macros.\usepackage[ruled,vlined]{algorithm2e}%\usepackage[lined,plain,linesnumbered,commentsnumbered,figure,noline]{algorithm2e}% Algorithm2e is an environment for writing algorithms. An algorithm becomes a floating object (like fig­ure, table, etc.). The package provides macros that allow you to create different keywords, and a set of predefined key words is provided; you can change the typography of the keywords. The package allows vertical lines delimiting a block of instructions in an algorithm, and defines different sorts of algorithms such as Procedure or Function; the name of these functions may be reused in the text or in other algo­rithms.%\usepackage{algorithmic}% algorithmic.sty was written by Peter Williams and Rogerio Brito. This package provides an algorithmic environment fo describing algorithms. You can use the algorithmic environment in-text or within a figure environment to provide for a floating algorithm. Do NOT use the algorithm floating environment provided by algorithm.sty (by the same authors) or algorithm2e.sty (by Christophe Fiorio) as IEEE does not use dedicated algorithm float types and packages that provide these will not provide correct IEEE style captions. The latest version and documentation of algorithmic.sty can be obtained at: http://www.ctan.org/tex-archive/macros/latex/contrib/algorithms/  There is also a support site at: http://algorithms.berlios.de/index.html  Also of interest may be the (relatively newer and more customizable) algorithmicx.sty package by Szasz Janos: http://www.ctan.org/tex-archive/macros/latex/contrib/algorithmicx/%\usepackage{algorithmicx}% Algorithmicx provides a flexible, yet easy to use, way for inserting good looking pseudocode or source code in your papers. It has built in support for Pseudocode, Pascal and C, and offers powerful means to create definitions for any programming language. The user can adapt a Pseudocode style to his native language. This package provides many possibilities to customize the layout of algorithms. You can use one of the predefined layouts (pseudocode, pascal, C, and others), with or without modifications, or you can define a completely new layout for your specific needs.%\usepackage{amsfonts}% An extended set of fonts for use in mathematics, including: extra mathematical symbols; blackboard bold letters (uppercase only); fraktur letters; subscript sizes of bold math italic and bold Greek letters; subscript sizes of large symbols such as sum and product; added sizes of the Computer Modern small caps font; cyrillic fonts (from the University of Washington); Euler mathematical fonts. All fonts are proA­vided as Adobe Type 1 files, and all except the Euler fonts are provided as METAFONT source. The distribution also includes the canonical Type 1 versions of the Computer Modern family of fonts.  Plain TeX and LaTeX macros for using the fonts are provided.\usepackage[cmex10]{amsmath}%\usepackage{amsmath}% The package provides the principal packages in the AMS-LaTeX distribution. It adapts for use in LaTeX most of the mathematical features found in AMS-TeX; it is highly recommendsd as an adjunct to serious mathematical typesetting in LaTeX. When amsmath is loaded, AMS-LaTeX packages amsbsy (for bold symbols), amsopn (for operator names) and amstext (for text embdedded in mathematics) are also loaded. Amsmath is part of the LaTeX required distribution; however, several contributed packages add still furA­ther to its appeal; examples are empheq, which provides functions for decorating and highlighting mathA­ematics, and ntheorem, for specifying theorem (and similar) definitions. % A popular package from the American Mathematical Society that provides many useful and powerful commands for dealing with mathematics. If using it, be sure to load this package with the cmex10 option to ensure that only type 1 fonts will utilized at all point sizes. Without this option, it is possible that some math symbols, particularly those within footnotes, will be rendered in bitmap form which will result in a document that can not be IEEE Xplore compliant! Also, note that the amsmath package sets \interdisplaylinepenalty to 10000 thus preventing page breaks from occurring within multiline equations. Use: \interdisplaylinepenalty=2500 after loading amsmath to restore such page breaks as IEEEtran.cls normally does. amsmath.sty is already installed on most LaTeX systems. The latest version and documentation can be obtained at: http://www.ctan.org/tex-archive/macros/latex/required/amslatex/math/\usepackage{amssymb}% A big list of symbols%\usepackage{appendix}% The appendix package provides various ways of formatting the titles of appendices. Also (sub)appenA­dices environments are provided that can be used, for example, for per chapter/section appendices. The word â??Appendixâ?? or similar can be prepended to the appendix number for article class documents. The word â??Appendicesâ?? or similar can be added to the table of contents before the appendices are listed. The word â??Appendicesâ?? or similar can be typeset as a \part-like heading (page) in the body. An appendices environment is provided which can be used instead of the \appendix command.\usepackage{array}% An extended implementation of the array and tabular environments which extends the options for column formats, and provides "programmable" format specifications. This package is part of the tools bundle in the LaTeX required distribution. Frank Mittelbach's and David Carlisle's array.sty patches and improves the standard LaTeX2e array and tabular environments to provide better appearance and additional user controls. As the default LaTeX2e table generation code is lacking to the point of almost being broken with respect to the quality of the end results, all users are strongly advised to use an enhanced (at the very least that provided by array.sty) set of table tools. array.sty is already installed on most systems. The latest version and documentation can be obtained at: http://www.ctan.org/tex-archive/macros/latex/required/tools/%\ifthenelse{\equal{\doclang}{german}}{\usepackage[ngerman]{babel}}{}% The package manages culturally-determined typographical (and other) rules, and hyphenation patterns for a wide range of languages. A document may select a single language to be supported, or it may seA­lect several, in which case the document may switch from one language to another in a variety of ways. Babel uses contributed configuration files that provide the detail of what has to be done for each lanA­guage. Users of Xâ??TeX are advised to use polyglossia rather than Babel.%\usepackage{bigstrut}%\bigstrut produces a strut which is \bigstrutjot higher, lower, or both, than the standard array/table strut. Use it in table entries that are adjacent to an \hline, to leave an extra bit of space. Note that the tabls package does the job automatically, and the booktabs package has a completely dif­ferent mechanism for dealing with the problem, which obviates the need for manual adjustment.\usepackage{booktabs}% The package enhances the quality of tables in LATEX, providing extra commands as well as behind-the-scenes optimisation. Guidelines are given as to what constitutes a good table in this context. From version 1.61, the package offers longtable compatibility.\usepackage{bm}% The bm package defines a command \bm which makes its argument bold. The argument may be any maths object from a single symbol to an expression. This is closely related to the specification of the \boldsymbol command in AMS-LaTeX, but \bm is rather more careful in the way it does things. The package is part of the tools bundle in the LaTeX required distribution.%\usepackage{booktabs}% The package enhances the quality of tables in LaTeX, providing extra commands as well as behind-the-scenes optimisation. Guidelines are given as to what constitutes a good table in this context. From ver­sion 1.61, the package offers longtable compatibility.\usepackage[font={footnotesize}]{caption}%\usepackage[format=hang]{caption} 	%\usepackage[caption=false]{caption}% The caption package provides many ways to customise the captions in floating environments like figure and table, and cooperates with many other packages. Facilities include rotating captions, sideways capA­tions, continued captions (for tables or figures that come in several parts). A list of compatibility notes, for other packages, is provided in the documentation. The package also provides the â??caption outside floatâ?? facility, in the same way that simpler packages like capt-of do.  The package supersedes caption2.%\usepackage[europeanresistors,americaninductors]{circuizz}% The package provides a set of macros for naturally typesetting electrical and (somewhat less naturally, perhaps) electronic networks. It is designed as a tool that is easy to use, with a lean syntax, native to LaTeX, and directly supporting PDF output format. So is based on the very impressive pgf/TikZ package.\usepackage[final]{changes}%\usepackage[draft, markup = underlined]{changes}% The package allows the user to manually markup changes of text, such as additions, deletions, or replacements. Changed text is shown in a different colour; deleted text is crossed out. The package allows definition of additional authors and their associated colour. It also allows you to define a markup for authors or annotations. A bash script is provided for removing the changes.%\usepackage[noadjust]{cite}% cite.sty was written by Donald Arseneau V1.6 and later of IEEEtran pre-defines the format of the cite.sty package. \cite{} output to follow that of IEEE. Loading the cite package will result in citation numbers being automatically sorted and properly "compressed/ranged". e.g., [1], [9], [2], [7], [5], [6] without using cite.sty will become [1], [2], [5]--[7], [9] using cite.sty. cite.sty's \cite will automatically add leading space, if needed. Use cite.sty's noadjust option (cite.sty V3.8 and later) if you want to turn this off. cite.sty is already installed on most LaTeX systems. Be sure and use  version 4.0 (2003-05-27) and later if using hyperref.sty. cite.sty does not currently provide for hyperlinked citations. The latest version can be obtained at: http://www.ctan.org/tex-archive/macros/latex/contrib/cite/ The documentation is contained in the cite.sty file itself.%\usepackage{cleveref}% The package enhances LATEX's cross-referencing features, allowing the format of references to be determined automatically according to the type of reference. The formats used may be customised in the preamble of a document; babel support is available (though the choice of languages remains limited: currently Danish, Dutch, English, French, German, Italian, Norwegian, Russian, Spanish and Ukranian). The package also offers a means of referencing a list of references, each formatted according to its type. In such lists, it can collapse sequences of numerically-consecutive labels to a reference range.%\usepackage{color}% The color package provides both foreground (text, rules, etc.) and background colour management; it uses the device driver configuration mechanisms of the graphics package to determine how to control its ouptut. The package is part of the LaTeX graphics bundle.%\usepackage{colortbl}% This package implements a  flexibable mechanism for giving colured `panels' behind specied columns in a table. This package requires the array and color packages.\usepackage{dcolumn}%\usepackage{doublespace}% Double space environment. A LaTeX 2.09 package that defines a spacing environment which you can use anywhere in your docu­ment. This package should no longer be used: it has serious bugs, and is no longer maintained. It is super­seded by setspace.\usepackage{dsfont}% A font based on Computer Modern Roman useful for typesetting the mathematical symbols for the natural numbers (N), whole numbers (Z), rational numbers (Q), real numbers (R) and complex numbers (C); coverage includes all Roman capital letters, ‘1’, ‘h’ and ‘k’. The font is available both as METAFONT source and in Adobe Type 1 format, and LATEX macros for its use are provided. The fonts appear in the blackboard bold sampler.%	\let\labelindent\IEEElabelindent%As IEEEtran.cls provides its own means to format lists, it might be better to read its documentation and template and use these methods instead of the enumitem module. Or, if using IEEEtran is not necessary, to switch to another documentclass. As IEEEtran.cls actually defines the \labelindent only for backwards compatibility reasons:: provide for legacy IED commands/lengths when possible the macro can be re-used without side-effects.% Exporting to LaTeX, inserting the line::%	\let\labelindent\relax%before the call to enumitem (\usepackage{enumitem}) and running latex "by hand" works around the name clash.\let\labelindent\relax\usepackage{enumitem}%This pack?age pro?vides user con?trol over the lay?out of the three ba?sic list en?vi?ron?ments: enu?mer?ate, item?ize and de?scrip?tion. It su?per?sedes both enu?mer?ate and md?wlist (pro?vid?ing well-struc?tured re?place?ments for all their fun?tion?al?ity), and in ad?di?tion pro?vides func?tions to com?pute the lay?out of la?bels, and to ?clone? the stan?dard en?vi?ron?ments, to cre?ate new en?vi?ron?ments with coun?ters of their own.%\usepackage{epsfig}% This package was developed as a general solution to the problem of including graphics in LaTeX 2.09; as such there are obsolete copies to be found on the web (though no longer on the archive). These old versions should not be used with current LaTeX. The current ‘preferred’ solution is the LaTeX graphicx package, but the graphics bundle does contain a version of epsfig for use with current LaTeX.%\usepackage{eqparbox}% Also of notable interest is Scott Pakin's eqparbox package for creating (automatically sized) equal width boxes - aka "natural width parboxes". Available at: http://www.ctan.org/tex-archive/macros/latex/contrib/eqparbox/\usepackage{etoolbox}% The package is a toolbox of programming facilities geared primarily towards LATEX class and package authors. It provides LATEX frontends to some of the new primitives provided by e-TEX as well as some generic tools which are not strictly related to e-TEX but match the profile of this package. Note that the initial versions of this package were released under the name elatex. The package provides functions that seem to offer alternative ways of implementing some LATEX kernel commands; nevertheless, the package will not modify any part of the LATEX kernel.\usepackage{eucal}% A package of the AMS which rovides more scripted font. The command \mathscr{#1}, it gives access to script capitals. For example, $\mathscr{MATHEMATIC}$%. If the eucal paage is used, the option [mathscr] provides both the standard \mathcal and the \mathscr commands, producing a different script font.%\usepackage{fancyhdr}% The package provides extensive facilities, both for constructing headers and footers, and for controlling their use (for example, at times when LATEX would automatically change the heading style in use).\usepackage[bottom]{footmisc}% A collection of ways to change the typesetting of footnotes. The package provides means of changing the layout of the footnotes themselves (including setting them in ‘paragraphs’ — the para option), a way to number footnotes per page (the perpage option), to make footnotes disappear in a ‘moving’ argument (stable option) and to deal with multiple references to footnotes from the same place (multiple option). The package also has a range of techniques for labelling footnotes with symbols rather than numbers. Some of the functions of the package are overlap with the functionality of other packages. The para option is also provided by the manyfoot and bigfoot packages, though those are both also portmanteau packages. (Don't be seduced by fnpara, whose implementation is improved by the present package.) The perpage option is also offered by footnpag and by the rather more general-purpose perpage}%\usepackage{float}% Improves the interface for defining floating objects such as figures and tables. Introduces the boxed float, the ruled float and the plaintop float. You can define your own floats and improve the behaviour of the old ones. The package also provides the H float modifier option of the obsolete here package. You can select this as automatic default with \floatplacement{figure}{H}.%\usepackage[T1]{fontenc}% The package allows the user to select font encodings, and for each encoding provides an interface to â??font-encoding-specificâ?? commands for each font. Its most powerful effect is to enable hyphenation to opA­erate on texts containing any character in the font. The package supersedes t1enc; it is distributed as part of the latex distribution.%\usepackage{fixltx2e}% fixltx2e, the successor to the earlier fix2col.sty, was written by Frank Mittelbach and David Carlisle. This package corrects a few problems in the LaTeX2e kernel, the most notable of which is that in current LaTeX2e releases, the ordering of single and double column floats is not guaranteed to be preserved. Thus, an unpatched LaTeX2e can allow asingle column figure to be placed prior to an earlier double column figure. The latest version and documentation can be found at: http://www.ctan.org/tex-archive/macros/latex/base/%\usepackage{fullpage}%\usepackage[a3paper, left = 57.5mm, right = 57.5mm, top = 40mm, bottom = 135mm]{geometry}%\usepackage[left = 15mm, top = 19mm, right = 15mm, bottom = 40mm]{geometry}	%\usepackage[margin = 1.5cm, includefoot, footskip = 30pt]{geometry}	%\usepackage[right = 19.91mm, left = 19mm, top = 19mm, bottom = 31mm]{geometry}%\usepackage[right = 0.51in, left = 0.51in, top = 0.75in, bottom = 1.69in]{geometry}%\usepackage[vcentering,dvips]{geometry}% The package provides an easy and flexible user interface to customize page layout, implement­ing auto-centering and auto-balancing mechanisms so that the users have only to give the least description for the page layout. For example, if you want to set each margin 2cm without header space, what you need is just \usepackage[margin=2cm,nohead]{geometry}. The package knows about all the standard paper sizes, so that the user need not know what the nominal ‘real’ dimensions of the paper are, just its standard name (such as a4, letter, etc.). An important feature is the package’s ability to communicate the paper size it's set up to the output (whether via DVI \specials or via direct interaction with PDF(La)TeX).%\usepackage[acronym, toc, style=super4col, nonumberlist]{glossaries}%\usepackage{glossaries}%\usepackage{graphics}% The package was designed to accommodate all needs for inclusion of graphics in LaTeX docu­ments, replacing many earlier packages used in LaTeX 2.09. The package aims to give a con­sistent interface to including the file types that are understood in your output, by use of ‘printer drivers’ (now known, simply, as ‘drivers’). The distribtion of the package contains sev­eral drivers, but others (for example, pdfTeX) are distributed separately. The package also of­fers several means of manipulating graphics in the course of inserting them into a document (for example, rotation and scaling). For extended documentation see epslatex. The package is part of the graphics bundle, which is one of the collections in the LaTeX ‘re­quired’ set of packages. % graphicx was written by David Carlisle and Sebastian Rahtz. It is required if you want graphics, photos, etc. graphicx.sty is already installed on most LaTeX systems. The latest version and documentation can be obtained at: http://www.ctan.org/tex-archive/macros/latex/required/graphics/  Another good source of documentation is "Using Imported Graphics in LaTeX2e" by Keith Reckdahl which can be found as epslatex.ps or epslatex.pdf at: http://www.ctan.org/tex-archive/info/  latex, and pdflatex in dvi mode, support graphics in encapsulated postscript (.eps) format. pdflatex in pdf mode supports graphics in .pdf, .jpeg, .png and .mps (metapost) formats. Users should ensure that all non-photo figures use a vector format (.eps, .pdf, .mps) and not a bitmapped formats (.jpeg, .png). IEEE frowns on bitmapped formats which can result in "jaggedy"/blurry rendering of lines and letters as well as large increases in file sizes. You can find documentation about the pdfTeX application at: http://www.tug.org/applications/pdftex%\ifCLASSINFOpdf% \usepackage[pdftex]{graphicx}% declare the path(s) where your graphic files are% \graphicspath{{../pdf/}{../jpeg/}}% and their extensions so you won't have to specify these with% every instance of \includegraphics% \DeclareGraphicsExtensions{.pdf,.jpeg,.png}%\else% or other class option (dvipsone, dvipdf, if not using dvips). graphicx% will default to the driver specified in the system graphics.cfg if no% driver is specified.% \usepackage[dvips]{graphicx}% declare the path(s) where your graphic files are% \graphicspath{{../eps/}}% and their extensions so you won't have to specify these with% every instance of \includegraphics% \DeclareGraphicsExtensions{.eps}%\fi%\usepackage{graphicx}% The package builds upon the graphics package, providing a key-value interface for optional arguments to the \includegraphics command. This interface provides facilities that go far beyond what the graphics package offers on its own. For extended documentation, see epslatex. The package is part of the graphics bundle, which is one of the collections in the LaTeX â??requiredâ?? set of packages.%\usepackage[bookmarks=false, draft]{hyperref}\usepackage{hyperref}% The hyperref package is used to handle cross-referencing commands in LaTeX to produce hypertext links in the document. The package provides backends for the \special set defined for HyperTeX DVI processors; for embedded pdfmark commands for processing by Acrobat Distiller (dvips and Y&Yâ??s dvipsone); for Y&Yâ??s dviwindo; for PDF control within pdfTeX and dvipdfm; for TeX4ht; and for VTeXâ??s pdf and HTML backends. The package is distributed with the backref and nameref packages, which make use of the facilities of hyperref. The package depends on the authorâ??s kvoptions, ltxcmdsand refcount packages.%\usepackage{ifpdf}% Heiko Oberdiek's ifpdf.sty is very useful if you need conditional compilation based on whether the output is pdf or dvi.          Usage:	\ifpdf	% pdf code	\else	% dvi code	\fi . The latest version of ifpdf.sty can be obtained from:	http://www.ctan.org/tex-archive/macros/latex/contrib/oberdiek/ . Also, note that IEEEtran.cls V1.7 and later provides a builtin \ifCLASSINFOpdf conditional that works the same way. When switching from latex to pdflatex and vice-versa, the compiler may have to be run twice to clear warning/error messages.\usepackage{ifthen}% The package's basic command is \ifthenelse, which can use a wide array of tests. Also provided is a simple loop command \whiledo. Ifthen is a separate package within the LaTeX distribution; while it will always be present in a LaTeX distriA­bution, a \usepackage command is always needed to load it.\usepackage{indentfirst}% This minimalist package is part of the tools bundle in the LaTeX required distribution.%\usepackage[latin1]{inputenc}% The package translates various standard and other input encodings into a â??LaTeX internal languageâ??. The internal language is expressed entirely in TeX's base encoding (standard ASCII printable characters, carriage control tokens and TeX control sequences, the latter mostly defined by LaTeX). The package is part of the LaTeX distribution.%\usepackage{lastpage}% Reference the number of pages in your LATEX document through the introduction of a new label which can be referenced like \pageref{LastPage} to give a reference to the last page of a document. It is particularly useful in the page footer that says: Page N of M.%\usepackage{latexdiff}% Latexdiff is a Perl script for visual mark up and revision of significant differences between two LATEX files. Various options are available for visual markup using standard LATEX packages such as color. Changes not directly affecting visible text, for example in formatting commands, are still marked in the LATEX source. A rudimentary revision facilility is provided by another Perl script, latexrevise, which accepts or rejects all changes. Manual editing of the difference file can be used to override this default behaviour and accept or reject selected changes only.\usepackage{lipsum}%\usepackage{longtable}% Allow tables to flow over page boundaries. Longtable allows you to write tables that continue to the next page. You can write captions within the ta?ble (typically at the start of the table), and headers and trailers for pages of table. Longtable arranges that the columns on successive pages have the same widths. This last contrasts with the superficially similar supertabular package. Longtable (unlike supertabular) modifies the output routine, and consequently won't work in a multicol?umn environment (or in other circumstances where the output routine has been critically altered); it also fails in twocolumn pages. This package is part of the tools bundle in the LaTeX required distribution.%\usepackage{lscape}% Modifies the margins and rotates the page contents but not the page number. Useful, for exam­ple, with large multipage tables, as it is compatible with longtable and supertabular. Note that the package makes no special provision for PDF output, where in principle a single page can be shown at full landscape width; such an effect may be achieved using the pdflscape package instead. The package is part of the graphics bundle, which is one of the collections in the LaTeX ‘re­quired’ set of packages.\usepackage{mathtools}%Math­tools pro­vides a se­ries of pack­ages de­signed to en­hance the ap­pear­ance of doc­u­ments con­tain­ing a lot of math­e­mat­ics. The main back­bone is ams­math, so those un­fa­mil­iar with this re­quired part of the LATEX sys­tem will prob­a­bly not find the pack­ages very use­ful. Math­tools pro­vides many use­ful tools for math­e­mat­i­cal type­set­ting. It is based on ams­math and fixes var­i­ous de­fi­cien­cies of ams­math and stan­dard LATEX. It pro­vides: Ex­ten­si­ble sym­bols, such as brack­ets, ar­rows, har­poons, etc.; Var­i­ous sym­bols such as \coloneqq (:=); Easy cre­ation of new tag forms; Show­ing equa­tion num­bers only for ref­er­enced equa­tions; Ex­ten­si­ble ar­rows, har­poons and hookar­rows; Starred ver­sions of the ams­math ma­trix en­vi­ron­ments for spec­i­fy­ing the col­umn align­ment; More build­ing blocks: mult­lined, cases-like en­vi­ron­ments, new gath­ered en­vi­ron­ments; Maths ver­sions of \make­box, \llap, \rlap etc.; Cramped math styles; and more...%\usepackage{mdwmath}%\usepackage{mdwtab}% Also highly recommended is Mark Wooding's extremely powerful MDW tools, especially mdwmath.sty and mdwtab.sty which are used to format equations and tables, respectively. The MDWtools set is already installed on most LaTeX systems. The lastest version and documentation is available at: http://www.ctan.org/tex-archive/macros/latex/contrib/mdwtools/\usepackage{moresize}% A package for using font sizes up to 35.88pt, for example with the EC fonts. New commands \HUGE and \ssmall for selecting font sizes are provided together with some options working around current LATEX2ε shortcomings in using big font sizes. The package also provides options for improving the typesetting of paragraphs (or headlines) with embedded math expressions at font sizes above 17.28pt.\usepackage{multicol}% Intermix single and multiple columns. Multicol defines a multicols environment which typesets text in multiple columns (up to a maximum of 10), and (by default) balances the end of each column at the end of the environment. The package en?ables you to switch between any (permitted) number of columns at will: there is no imposed "clear page" operation, as there is in unadorned LaTeX at a switch between \onecolumn and \twocolumn sections. The multicolumn environment can also be used inside a box, thus allowing multicolumned insets in text. Multicol patches the output routine, and may clash with other packages that do the same (e.g., longtable); furthermore, there is no provision for single column floats inside a multicolumn environment, so figures and tables must be coded in-line (using, for example, the H modifier of the float package). The package is part of the tools bundle in the LaTeX required distribution.\usepackage{multirow}% The package has a lot of flexibility, including an option for specifying an entry at the “natural” width of its text. The package is distributed with the bigdelim and bigstrut packages, which can be used to ad­vantage with \multirow cells.\usepackage[square,numbers,sort&compress]{natbib}% The bundle provides a package that implements both author-year and numbered references, as well as much detailed of support for other bibliography use. Also Provided are versions of the standard BibTeX styles that are compatible with natbibâ??plainnat, unsrtnat, abbrnat. The bibliography styles produced by custom-bib are designed from the start to be compatible with natbib.%\usepackage{parskip}% Simply changing \parskip and \parindent leaves a layout that is untidy; this package (though it is no substitute for a properly-designed class) helps alleviate this untidiness%\usepackage{pdfcomment}% For a long time pdfLaTeX has offered the command \pdfannot for inserting arbitrary PDF annotations. However, the command is presented in a form where additional knowledge of the definition of the PDF format is indispensable. This package is an answer to the ? occasional ? questions in newsgroups, about how one could use the comment function of Adobe Reader. At least for the writer of LaTeX code, the package offers a convenient and user-friendly means of using \pdfannot to provide comments in PDF files. Since version v1.1, pdfcomment.sty also supports LaTeX ? dvips ? ps2pdf, LaTeX ? dvipdfmx, and XeLaTeX. Unfortunately, support of PDF annotations by PDF viewers is sparse to nonexistent. The reference viewer for the development of this package is Adobe Reader. The author is Josef Kleber. The package is Copyright ? 2008-2012 Josef Kleber.\usepackage{pdfpages}% This package simplifies the inclusion of external multi-page PDF documents in LATEX documents. Pages may be freely selected and similar to psnup it is possible to put several logical pages onto each sheet of paper. Furthermore a lot of hypertext features like hyperlinks and article threads are provided. The package supports pdfTEX (pdflatex) and VTEX. With VTEX it is even possible to use this package to insert PostScript files, in addition to PDF files.\usepackage{perpage}% The package provides a mechanism to reset counters per page and/or keep their occurences sorted in order of appearance on the page. The package also defines a command \MakePerPage that designates a counter to be reset per page. For example \MakePerPage[2]{footnote} will cause footnote numbers to start at 2 at the start of every page (the optional argument defaults to 1). The machanism may require multiple passes to work; LATEX gives the usual changed label warning if this is necessary. The package is distributed as part of the bigfoot bundle.%\usepackage{pgf-pie}% pgf-pie is a LaTeX package for drawing pie chart (and variant charts). As stated by its name, it is based on a very popular graphic package PGF/TikZ. This document presents the usage of pgf-pie and collects some pie charts as examples. pgf-pie can be downloaded from http://code.google.com/p/pgf-pie/.\usepackage{pgfplots}% PGFPlots draws high-quality function plots in normal or logarithmic scaling with a user-friendly interface directly in TeX. The user supplies axis labels, legend entries and the plot coordinates for one or more plots and PGFPlots applies axis scaling, computes any logarithms and axis ticks and draws the plots, supporting line plots, scatter plots, piecewise constant plots, bar plots, area plots, mesh-- and surface plots and some more. Pgfplots is based on PGF/TikZ (pgf); it runs equally for LaTeX/TeX/ConTeXt.%\usepackage{pgfplotstable}% Pgfplotstable displays numerical tables rounded to desired precision in various display formats (for example scientific format, fixed point format or integer), using TeX’s mathematical facili­ties for pretty printing. Furthermore, it provides methods for table postprocessing.%\usepackage{placeins}% Defines a \FloatBarrier command, beyond which floats may not pass; useful, for example, to ensure all floats for a section appear before the next \section command.%\usepackage{rotating}% Rotation tools, including rotated full-page floats. A package built on the standard LaTeX graphics package to perform all the different sorts of rotation one might like, including complete figures and tables with their captions. If you want continuous text (i.e., more than one page) set in landscape mode, use the lscape package in­stead. The rotating packages only deals in rotated boxes (or floats, which are themselves boxes), and boxes always stay on one page. If you need to use the facilities of the float in the same document, load rotating.sty via rotfloat, which smooths the path between the rotating and float packages.%\usepackage{rotfloat}% Rotate floats. The float package provides commands to define new floats of various styles (plain, boxed, ruled, and userdefined ones); the rotating package provides new environments (sidewaysfigure and side?waystable) which are rotated by 90 or 270 degrees. But what about new rotated floats, e.g. a rotated ruled one? This package makes this possible; it builds a bridge between the two packages and ex?tends the commands from the float package to define rotated versions of the new floats, too.%\usepackage[headsepline]{scrpage2}% koma-script – A bundle of versatile classes and packages. The KOMA-Script bundle provides drop-in replacements for the article/report/book classes with empha­sis on typography and versatility. There is also a letter class, different from all other letter classes. The bundle also offers: – a package for calculating type areas in the way laid down by the typographer Jan Tschichold, – a package for easily changing and defining page styles, – a package scrdate for get­ting not only the current date but also the name of the day, and – a package scrtime for getting the cur­rent time. All these packages may be used not only with KOMA-Script classes but also with the standard classes. Since every package has its own version number, the version number quoted only refers to the version of scrbook, scrreprt, scrartcl, scrlttr2 and typearea. These are the main parts of the bundle.\usepackage[nodisplayskipstretch]{setspace}% Set space between lines. Provides support for setting the spacing between lines in a document. Package options include sin­glespacing, onehalfspacing, and doublespacing. Alternatively the spacing can be changed as required with the \singlespacing, \onehalfspacing, and \doublespacing commands. Other size spacings also avail­able.%\usepackage{showframe}% The package shows a (simple, uncluttered) diagram of the page layout; similar (but more complex-looking) diagrams may be obtained from the layouts and geometry packages. The package is available as part of the eso-pic distribution.\usepackage{siunitx}\usepackage{soul}%Provides hyphenatable spacing out (letterspacing), underlining, striking out, etc., using the TEX hyphenation algorithm to find the proper hyphens automatically. The package also provides a mechanism that can be used to implement similar tasks, that have to treat text syllable by syllable. This is shown in two examples. The package itself does not support UTF-8 input in ordinary (PDF)LATEX; some UTF-8 support is offered by package soulutf8%\usepackage{spconf}% A package for formatting the ICASSP 2015 paper ...\usepackage{stackengine}% The package provides a versatile way to stack objects vertically in a variety of customizable ways. A number of useful macros are provided, all of which make use of the stackengine core.\usepackage{stfloats}% stfloats.sty was written by Sigitas Tolusis. This package gives LaTeX2e the ability to do double column floats at the bottom of the page as well as the top. (e.g., "\begin{figure*}[!b]" is not normally possible in LaTeX2e). It also provides a command:\fnbelowfloat to enable the placement of footnotes below bottom floats (the standard LaTeX2e kernel puts them above bottom floats). This is an invasive package which rewrites many portions of the LaTeX2e float routines. It may not work with other packages that modify the LaTeX2e float routines. The latest version and documentation can be obtained at: http://www.ctan.org/tex-archive/macros/latex/contrib/sttools/ Documentation is contained in the stfloats.sty comments as well as in the presfull.pdf file. Do not use the stfloats baselinefloat ability as IEEE does not allow \baselineskip to stretch. Authors submitting work to the IEEE should note that IEEE rarely uses double column equations and that authors should try to avoid such use. Do not be tempted to use the cuted.sty or midfloat.sty packages (also by Sigitas Tolusis) as IEEE does not format its papers in such ways.\usepackage{subcaption}% Support for sub-captions. The package provides a means of using facilities analagous to those of the caption package, when writ­ing captions for subfigures and the like. The package is distributed with caption.%\usepackage{subfig}	%\usepackage[font=footnotesize]{subfig}% Figures broken into subfigures. The package provides support for the manipulation and reference of small or ‘sub’ figures and tables within a single figure or table environment. It is convenient to use this package when your subfigures are to be separately captioned, referenced, or are to be included in the List-of-Figures. A new \subfigure command is introduced which can be used inside a figure environment for each subfigure. An optional first argument is used as the caption for that subfigure. This package supersedes the subfigure package (which is longer maintained). The name was changed since the package is completely backward compatible with the older package The major advantage to the new package is that the user interface is keyword/value driven and easier to use. To ease the transi­tion from the subfigure package, the distribution it includes a configuration file (subfig.cfg) which nearly emulates the subfigure package. The functionality of the package is provided by the (more recent still) subcaption package. subfig.sty, also written by Steven Douglas Cochran, is the modern replacement for subfigure.sty. However, subfig.sty requires and automatically loads Axel Sommerfeldt's caption.sty which will override IEEEtran.cls handling of captions and this will result in nonIEEE style figure/table captions. To prevent this problem, be sure and preload caption.sty with its "caption=false" package option. This is will preserve IEEEtran.cls handing of captions. Version 1.3 (2005/06/28) and later (recommended due to many improvements over 1.2) of subfig.sty supports the caption=false option directly: \usepackage[caption=false,font=footnotesize]{subfig}. The latest version and documentation can be obtained at: http://www.ctan.org/tex-archive/macros/latex/contrib/subfig/ The latest version and documentation of caption.sty can be obtained at: http://www.ctan.org/tex-archive/macros/latex/contrib/caption/%\usepackage{subfigure}	%\usepackage[tight,footnotesize]{subfigure}% Deprecated, Deprecated, Deprecated: Figures divided into subfigures. Provides support for the manipulation and reference of small or ‘sub’ figures and tables within a single figure or table environment. It is convenient to use this package when your subfigures are to be sepa­rately captioned, referenced, or are to be included in the List-of-Figures. A new \subfigure command is introduced which can be used inside a figure environment for each subfigure. An optional first argument is used as the caption for that subfigure. The package is now considered obsolete: it was superseded by subfig, but users may find the more re­cent subcaption package more satisfactory. subfigure.sty was written by Steven Douglas Cochran. This package makes it easy to put subfigures in your figures. e.g., "Figure 1a and 1b". For IEEE work, it is a good idea to load it with the tight package option to reduce the amount of white space around the subfigures. subfigure.sty is already installed on most LaTeX systems. The latest version and documentation can be obtained at: http://www.ctan.org/tex-archive/obsolete/macros/latex/contrib/subfigure/ subfigure.sty has been superceeded by subfig.sty.\usepackage{tabu}\usepackage{tabularx}\usepackage{textcomp}% LaTeX support for the Text Companion fonts. The package supports the Text Companion fonts, which provide many text symbols (such as baht, bullet, copyright, musicalnote, onequarter, section, and yen), in the TS1 encoding. Note that the package has been adopted as part of the LaTeX distribution; the reference here is to the original package, which is now little used (if at all).\usepackage{tikz}% Create PostScript and PDF graphics in TeX. PGF is a macro package for creating graphics. It is platform- and format-independent and works to­gether with the most important TeX backend drivers, including pdftex and dvips. It comes with a user-friendly syntax layer called TikZ.  Its usage is similar to pstricks and the standard picture environment. PGF works with plain (pdf-)TeX, (pdf-)LaTeX, and ConTeXt. Unlike pstricks, it can produce either PostScript or PDF output.%\usepackage{tikzscale}% Resize pictures while respecting text size. The package extends the \includegraphics command to support tikzpictures. It allows scaling of TikZ images and PGFPlots to a given width or height without changing the text size.% Handle a bug when using the "titlesec" package (add the 'subparagraph' command)\makeatletter\newcommand\subparagraph{\@startsection{subparagraph}{5}{\parindent}{3.25ex \@plus 1ex \@minus .2ex}{-1em}{\normalfont\normalsize\bfseries}}\makeatother%\usepackage{titlesec}\usepackage[compact]{titlesec}\let\subparagraph\relax% A package providing an interface to sectioning commands for selection from various title styles. E.g., marginal titles and to change the font of all headings with a single command, also providing simple one-step page styles. Also includes a package to change the page styles when there are floats in a page. You may assign headers/footers to individual floats, too.%\usepackage{tkz-kiviat}% Draw Kiviat graphs. The package allows the user to draw Kiviat Graphs directly, or with data from an external file. The drawing is done with the help of pgfplots.\usepackage[color = yellow]{todonotes}% The package lets the user mark things to do later, in a simple and and visually appealing way.  The package takes several options to enable customization / finetuning of the visual appearance. The package makes use of the tikz and xkeyval packages. %\usepackage{txfonts}% Txfonts supplies virtual text roman fonts using Adobe Times (or URW NimbusRomNo9L) with some modified and additional text symbols in the OT1, T1, and TS1 encodings; maths alphabets using Times/URW Nimbus; maths fonts providing all the symbols of the Computer Modern and AMS fonts, including all the Greek capital letters from CMR; and additional maths fonts of various other symbols. The set is complemented by a sans-serif set of text fonts, based on Helvetica/NimbusSanL, and a monospace set. All the fonts are in Type 1 format (AFM and PFB files), and are supported by TeX metrics (VF and TFM files) and macros for use with LaTeX.\usepackage{url}% Verbatim with URL-sensitive line breaks. The command \url is a form of verbatim command that allows linebreaks at certain characters or combi­nations of characters, accepts reconfiguration, and can usually be used in the argument to another com­mand. (The \urldef command provides robust commands that serve in cases when \url doesn't work in an argument.) The command is intended for email addresses, hypertext links, directories/paths, etc., which normally have no spaces, so by default the package ignores spaces in its argument. However, a package option “allows spaces”, which is useful for operating systems where spaces are a common part of file names. url.sty was written by Donald Arseneau. It provides better support for handling and breaking URLs. url.sty is already installed on most LaTeX systems. The latest version can be obtained at: http://www.ctan.org/tex-archive/macros/latex/contrib/misc/ Read the url.sty source comments for usage information. Basically, \url{my_url_here}.%\usepackage{wrapfig}% Allows figures or tables to have text wrapped around them. Does not work in combination with list enviA­ronments, but can be used in a parbox or minipage, and in twocolumn format. Supports the float packA­age. Alternatively you can use floatflt%\usepackage[table]{xcolor}%xcolor provides easy driver-independent access to several kinds of colors, tints, shades, tones, and mixes of arbitrary colors by means of color expressions like \color{red!50!green!20!blue}. It allows to select a document-wide target color model and offers tools for automatic color schemes, conversion between twelve color models, alternating table row colors, color blending and masking, color separation, and color wheel calculations.
%\usepackage[style=numeric-comp,natbib=true]{biblatex}%\AtEveryBibitem{%%	\clearfield{issn} % Remove issn%	\clearfield{doi} % Remove doi%	\ifentrytype{online}{}{\clearfield{url}} % Remove url except for @online%}\usepackage{epstopdf}%Epstopdf is a Perl script that converts an EPS file to an ‘encapsulated’ PDF file (a single page file whose media box is the same as the original EPS’s bounding box). The resulting file suitable for inclusion by PDFTEX as an image. The script is adapted to run both on Windows and on Unixalike systems. The script makes use of Ghostscript for the actual conversion to PDF. It assumes Ghostscript version 6.51 or later, and (by default) suppresses its automatic rotation of pages where most of the text is not horizontal. LATEX users may make use of the epstopdf package, which will run the epstopdf script “on the fly”, thus giving the illusion that PDFLATEX is accepting EPS graphic files.

\usepackage{nccmath}
%===============================================================================================================
%=====================================  Pre-Document declarations ==============================================
%===============================================================================================================%===========================================  Selecting the title  =============================================\ifthenelse{\equal{\doclang}{german}}{
	\def \langtitle{\deTitle}
	\def \suptitle{\enTitle}	
}{
	\def \langtitle{\enTitle}
	\def \suptitle{\deTitle}
}%===============================================================================================================%========================================  Settings for PDF Document  ==========================================\pdfstringdef\studentPDF{\student}\pdfstringdef\worktitlePDF{\langtitle}\pdfstringdef\worksubjectPDF{\worksubject}\pdfstringdef\keywordsPDF{\keywords}\hypersetup{pdfauthor=\studentPDF, 
            pdftitle=\worktitlePDF,
            pdfsubject=\worksubjectPDF,
            pdfkeywords=\keywordsPDF}%===============================================================================================================%=====================================  Adjusting TikZ and PGF Plots  ==========================================\usetikzlibrary{arrows.meta,
				automata,
				backgrounds,
                calc,
				decorations,
				decorations.pathmorphing,
				decorations.pathreplacing,
                intersections,
                ocgx,
				patterns,
				petri,
				positioning,
			   %shadows,
				shapes,
				spy,
				topaths}\pgfplotsset{compat=1.13}%\usepgfplotslibrary{groupplots}%\pgfplotsset{plot coordinates/math parser=false}%\usepgfplotslibrary{external}%\tikzexternalize% %PGF activate externalization.  The external library oers a convenient method to export every single tikzpicture into a separate .pdf (or .eps). Later runs of LATEX will simply include these graphics, thereby reducing typesetting time considerably. The library can also be used to submit documents to authors who do not even have pgfplots or TikZ installed. See pgfplots manual Section 7.1. To enable the system calls, we type pdflatex -shell-escape test%===============================================================================================================%================ Tabular booktabs rules' variables ============================================================%---------------------------------------------------------------------------------------------------------------\heavyrulewidth=0.11em
\lightrulewidth=0.05em

%===============================================================================================================%================ Headers' styles (chapter, section, subsection headers) and margins ===========================%---------------------------------------------------------------------------------------------------------------%\renewcommand{\chaptername}{}	% Supress the word "Chapter"%\renewcommand{\thechapter}{}	% Supress Chapter Numbering in margin%--------------------------------------------------------------------------------------------------------------- Requires the package "etoolbox"%\makeatletter % You can adjust the value in \vspace*{0pt}.%\patchcmd{\@makechapterhead}{50\p@}{0pt}{}{} % Supress spacing above \chappter header  %\patchcmd{\@makeschapterhead}{50\p@}{0pt}{}{} % Supress spacing above \chappter* header%\makeatother%--------------------------------------------------------------------------------------------------------------- Requires the package "titlesec"%\titleformat{\chapter}[hang]{\normalfont\huge\bfseries}{\thechapter.}{1em}{} %\titlespacing{\chapter}{0pt}{0pt}{20pt}\titlespacing{\section}{0pt}{3pt}{0pt}\titlespacing{\subsection}{0pt}{3pt}{0pt}%\titlespacing{\subsubsection}{0pt}{5pt}{0pt}\setlength{\abovedisplayskip}{2pt}\setlength{\belowdisplayskip}{3pt}\renewcommand{\baselinestretch}{0.90}%---------------------------------------------------------------------------------------------------------------%\renewcommand\thesection{\arabic{section}.}										% Supress chapter numbering%\renewcommand\thesubsection{\arabic{section}.\arabic{subsection}}					% Supress chapter numbering%\renewcommand\thesection{\arabic{chapter}.\arabic{section}}							% Numbering format of sections%\renewcommand\thesubsection{\arabic{chapter}.\arabic{section}.\arabic{subsection}}	% Numbering format of subsections%===============================================================================================================%=======================================  Defining Math Operators  =============================================\DeclareMathOperator{\sign}{sign}\newcommand{\argmin}{\operatornamewithlimits{arg\ min}}\newcommand{\argmax}{\operatornamewithlimits{arg\ max}}%\DeclareMathOperator{\argmax}{arg\!max}\newcommand{\abs}[1]{\left\vert\left\vert{#1}\right\vert\right\vert}\newcommand{\card}[1]{\left\vert{#1}\right\vert}%===============================================================================================================%=======================================  Reset footnate per page  =============================================\MakePerPage{footnote}%===============================================================================================================%==========================================  Paragraph margins  ================================================\setlength{\parskip}{0.75mm}% Space between different paragraphs%\setlength{\parsep}{0pt}%\setlength{\headsep}{0pt}%\setlength{\topskip}{0pt}%\setlength{\topmargin}{0pt}%\setlength{\topsep}{0pt}%\setlength{\partopsep}{0pt}%===============================================================================================================%==========  The name, autorefname, and title of the algorithm listings (algorithm2e package)  =================\SetAlgorithmName{Algorithm}{Algorithm}{List of algorithms}%===============================================================================================================%============================================  Autoref Names  ==================================================\renewcommand*{\sectionautorefname}{Section}%\def\chapterautorefname{Chapter}	\def\sectionautorefname{Section} \def\figureautorefname{Figure} %\def\equationautorefname{Equation} 	\def\partautorefname{Part} \def\AMSautorefname{Equation}%\def\appendixautorefname{Appendix}	\def\Itemautorefname{Item} \def\tableautorefname{Table}	%\def\subsubsectionautorefname{Subsubsection}	\def\subsectionautorefname{Subsection} %\def\paragraphautorefname{Paragraph}	\def\Hfootnoteautorefname{Footnote} %\def\subfigureautorefname{Figure} 	\def\theoremautorefname{Theorem} %\def\algorithmautorefname{Figure} 	\def\theoremautorefname{Theorem} %  \def\equationautorefname{Equation}%%  \def\footnoteautorefname{footnote}%%  \def\itemautorefname{item}%%  \def\figureautorefname{Figure}%%  \def\tableautorefname{Table}%%  \def\partautorefname{Part}%%  \def\appendixautorefname{Appendix}%%  \def\chapterautorefname{chapter}%%  \def\sectionautorefname{section}%%  \def\subsectionautorefname{subsection}%%  \def\subsubsectionautorefname{subsubsection}%%  \def\paragraphautorefname{paragraph}%%  \def\subparagraphautorefname{subparagraph}%%  \def\FancyVerbLineautorefname{line}%%  \def\theoremautorefname{Theorem}%%  \def\pageautorefname{page}%%%%\pagenumbering{gobble}%\setlength{\columnsep}{6mm}%%\setcounter{page}{100}%\def\BibTeX{{\rm B\kern-.05em{\sc i\kern-.025em b}\kern-.08em%		T\kern-.1667em\lower.7ex\hbox{E}\kern-.125emX}}%%%===============================================================================================================%=====================================  Correcting Hyphenation  ================================================%\hyphenation{op-tical net-works semi-conduc-tor}%===============================================================================================================%=============================================  ?????????  =====================================================%\pagestyle{scrheadings}%\automark{chapter}%\clearscrheadfoot%\lehead[]{\pagemark~~\headmark}%\rohead[]{\headmark~~\pagemark}%\renewcommand{\chaptermark}[1]{\markboth {\sl \hspace{8mm}#1}{}}%\renewcommand{\sectionmark}[1]{\markright{\sl \thesection~#1\hspace{8mm}}}%\addtolength{\textheight}{15mm}%\parindent0ex%\renewcommand*{\pnumfont}{\normalfont\slshape} % Seitenzahl geneigt%\renewcommand*{\sectfont}{\bfseries} % Kapitelueberschrift nicht Helvetica\hypersetup{pdfstartview=FitH, bookmarks=true, bookmarksnumbered=true, bookmarksopen=false, breaklinks=false,
		    citecolor = blue}\ifthenelse{\equal{\colortype}{color}}{
		\hypersetup{colorlinks = true, linkcolor = blue}
	}{
		\ifthenelse{\equal{\colortype}{boxed}}
		{}
		{
			\hypersetup{hidelinks=true}
		}
	}%===============================================================================================================%===========================================  Defining Colors  =================================================%\definecolor{darkblue}{rgb}{0,0,0.4}%===============================================================================================================%=====================================  User-defined declarations  =============================================%\newcommand{\matlab}{\textsc{Matlab}\raisebox{1ex}{\tiny{\textregistered}} }%\newcommand{\Z}{\mathbb{Z}}%\newcommand{\N}{\mathbb{N}}%\newcommand{\R}{\mathbb{R}}%\newcommand{\E}{\operatorname{E}}%\newcommand{\e}[1]{\operatorname{e}^{\,#1}}%\newcommand{\op}[1]{\operatorname{#1}}%\newcommand{\smtext}[1]{{\scriptscriptstyle\text{#1}}}%\newcommand{\bugfix}{\color{white}{\texttt{\symbol{'004}}}} 		% Bug-Fix Umlaute in Verbatim%\setlength\bigstrutjot{3pt}%\geometry{total={155mm,235mm}}%\newcommand{\plus}{\texttt{+}}%\newcommand{\minus}{\texttt{-}}%\pgfdeclareplotmark{o*}%{%%	\draw (0, 0) circle [radius = 1.75 pt];%	\fill (0, 0) circle [radius = 0.75 pt];%}%===============================================================================================================%======================================  Floats fractions of a page  ===========================================%\renewcommand{\topfraction}{1}%\renewcommand{\bottomfraction}{1}%\renewcommand{\textfraction}{0}%\setcounter{totalnumber}{10}%\setcounter{secnumdepth}{5}%\setcounter{tocdepth}{5}%\setlength{\parindent}{0.5cm}%===============================================================================================================%========================================  List of Symbols Glossary  ===========================================% Define a new glossary type%\newglossary[slg]{symbols}{sym}{sbl}{List of Symbols}%Den Punkt am Ende jeder Beschreibung deaktivieren%\renewcommand*{\glspostdescription}{}%Glossar-Befehle anschalten%\makeglossaries%Diese Befehle sortieren die Einträge in den%einzelnen Listen:%TeXmaker%makeindex -s %.ist -t %.alg -o %.acr %.acn%makeindex -s %.ist -t %.slg -o %.sym %.sbl%makeindex -s %.ist -t %.glg -o %.gls %.glo%TeXnicCenter%makeindex -s %tm.ist -t %tm.alg -o %tm.acr %tm.acn%makeindex -s %tm.ist -t %tm.slg -o %tm.sym %tm.sbl%makeindex -s %tm.ist -t %tm.glg -o %tm.gls %tm.glo%\input{./bib/glossary.tex}\newcommand{\on}{\emph{on}}\newcommand{\off}{\emph{off}}\newcommand{\plus}{\texttt{+}}\newcommand{\minus}{\texttt{-}}%\setlength{\marginparwidth}{4.5cm}%\presetkeys{todonotes}{fancyline, linecolor = green!30, backgroundcolor=yellow!70, bordercolor = black}{}%===============================================================================================================%============================================  Gantt diagrams  =================================================%% GanttHeader setups some parameters for the rest of the diagram%% #1 Width of the diagram%% #2 Width of the space reserved for task numbers%% #3 Width of the space reserved for task names%% #4 Number of months in the diagram%% In addition to these parameters, the layout of the diagram is influenced%% by keys defined below, such as y, which changes the vertical scale%\def\GanttHeader#1#2#3#4{%%    \pgfmathparse{(#1-#2-#3)/#4}%    \tikzset{y=7mm, task number/.style={left, font=\bfseries},%        task description/.style={text width=#3,  right, draw=none,%            font=, xshift=#2,%            minimum height=2em},%        gantt bar/.style={draw=black, fill=blue!30},%        help lines/.style={draw=black!30, dashed},%        x=\pgfmathresult pt%    }%    \def\totalmonths{#4}%    \node (Header) [task description] at (0,0) {\textbf{\large Task Description}};%    \begin{scope}[shift=($(Header.south east)$)]%        \foreach \x in {1,...,#4}%        \node[above] at (\x,0) {\footnotesize\x};%    \end{scope}%}%%% This macro adds a task to the diagram%% #1 Number of the task%% #2 Task's name%% #3 Starting date of the task (month's number, can be non-integer)%% #4 Task's duration in months (can be non-integer)%\def\Task#1#2#3#4#5{%%    \node[task number] at ($(Header.west) + (0, -#1)$) {#5};%    \node[task description] at (0,-#1) {#2};%    \begin{scope}[shift=($(Header.south east)$)]%        \draw (0,-#1) rectangle +(\totalmonths, 1);%        \foreach \x in {1,...,\totalmonths}%        \draw[help lines] (\x,-#1) -- +(0,1);%        \filldraw[gantt bar] ($(#3, -#1+0.2)$) rectangle +(#4,0.6);%    \end{scope}%}

%===============================================================================================================
%=========================================  Document glossary ==================================================
%\makeglossaries
%\input{glossary.tex}


%===============================================================================================================
%===========================================  The document =====================================================
\begin{document}

%===============================================================================================================
%=======================  Paper heading (title, authors, abstract, and keywords)  ==============================
\title{\enTitle}	% Paper title ... can use linebreaks \\ within to get better formatting as desired

%===============================================================================================================
%====================================== Author names and affiliations ==========================================
% Use a multiple column layout for up to three different affiliations
\author{
\IEEEauthorblockN{Karim Said Barsim and Bin Yang\\}
\IEEEauthorblockA{\texttt{\{karim.barsim,bin.yang\}@iss.uni-stuttgart.de}\\
                    Institute of Signal Processing and System Theory, University of Stuttgart}
}

%===============================================================================================================
%============================================ Make the title area ==============================================
\maketitle

%===============================================================================================================
%============================================= Make the abstract ===============================================
\begin{abstract}\paperAbstract\end{abstract}


%===============================================================================================================
%============================================= Make the keywords ===============================================
\begin{IEEEkeywords} \keywords \end{IEEEkeywords}


%===============================================================================================================
%========================== ****************  Main content  **************** ===================================


\section{Introduction}
\label{sec:introduction}



Energy disaggregation (or \underline{N}on-\underline{I}ntrusive \underline{L}oad \underline{M}onitoring NILM) is the process of inferring individual load profiles at the end-use level from a single or a limited number of sensing points. Promising applications of disaggregated data have motivated a growing research community to reach a widely acceptable and scalable solution. Energy disaggregation  proved to be a challenging source separation problem in which a considerably large number of parameters are to be estimated from a limited set of measurements with little constraints.

In the last decade, energy disaggregation has witnessed an unprecedented wide-spreading research which is easily observed from the wide variety of learning techniques applied to this problem alongside with the growing number of energy datasets developed specifically for this research field. More recently, and analogous to the current breakthrough in data-driven learning, deep neural networks have re-gained their interest in addressing the energy disaggregation problem, especially alongside with the recently developed large energy datasets required for training such complex models \cite{Kaman_2017,Mauch_2015,Mauch_2016,Kelly_2015,Zhang_2016_SequenceToPointLearning,He_2016_AnEmpiricalStudy, Kelly_2015_UKDALE,Parson_2015_Dataport_NILMTK}. The progress in this trend, however, is relatively slow when compared to the development in either field separately. This is sometimes attributed to the high risk of over-fitting in neural network models \cite{Makonin_2014_PhD}, insufficiency or low diversity of publicly available energy datasets \cite{Kelly_2015}, or limited insights and understanding of the learning behavior of these models \cite{Zhang_2016_SequenceToPointLearning}.

In this paper, we first present a feasibility study on the development of a generic data-driven model suitable for end-use load monitoring. The proposed disaggregation model exploits a fully convolutional neural network architecture and is generic in the sense that none of the model hyper-parameters is dependent on the load category. We assess the feasibility of such a model through empirical evaluation of the monitoring performance across various load categories in a publicly available energy dataset. 



\section{Related work}
\label{sec:related-work}

In this section, we briefly describe some of the most recent works on energy disaggregation and load monitoring that adopted data-driven learning techniques.

Mauch and Yang \cite{Mauch_2015} exploited a generic two-layer bidirectional \underline{R}ecurrent \underline{N}eural \underline{N}etwork (RNN) architecture featuring \underline{L}ong \underline{S}hort \underline{T}erm \underline{M}emory (LSTM) \cite{Hochreiter_1997_LSTM} units in extracting single load profiles. They tested their models on the \underline{R}efrence \underline{E}nergy \underline{D}isaggregation \underline{D}ataset (REDD) \cite{Kolter_2011_REDD} in a de-noised scheme \cite{Makonin_2015_NILMPerformanceEvaluation}. Additionally, they  validated the generalization of their architecture to previously unseen loads in new buildings. In a later work, Mauch and Yang \cite{Mauch_2016} used a combination of discriminative and generative models in a two-stage eventless extraction of load profiles. Kelly and Knottenbelt \cite{Kelly_2015} evaluated and compared three neural network architectures on domestic loads from the \underline{UK}-\underline{D}omestic \underline{A}ppliance \underline{L}evel \underline{E}nergy (UK-DALE) \cite{Kelly_2015_UKDALE}. The first is a bidirectional RNN architecture with LSTM units similar to the one in \cite{Mauch_2015}, the second follows the architecture of a \underline{d}e-noising \underline{A}uto-\underline{E}ncoder (dAE) \cite{Vincent_2010}, and the last is a regression-based disaggregator whose objective is to estimate the main key points of an activation cycle of the target load within a given window. 

Similarly, He and Chai \cite{He_2016_AnEmpiricalStudy} applied two architectures, namely a convolutional dAE and an RNN, to the same problem. In their architectures, they also applied parallel convolutional layers with different kernel sizes analogous to the Inception module in GoogLeNet \cite{Szegedy_2015}. Zhang et al. \cite{Zhang_2016_SequenceToPointLearning} simplified the objective of the dAE architecture in \cite{Kelly_2015} to predict a single time instance of the target load profile for a given window of the aggregate signal.
Likewise, Nascimento \cite{Nascimento_2016} applied three neural network architectures, namely basic convolutional dAE, an RNN, and a ResNet-based model \cite{He_2015_ResNet} to the same problem but on three target loads in the REDD dataset. He introduced several improvements such as redefining the loss function, exploiting batch normalization \cite{Ioffe_2015_BatchNormalization}, and applying residual connections \cite{He_2015_ResNet}.

Additionally, Lange et al. \cite{Lange_2016_BOLT} adopted a deep neural network with constrained binary and linear activation units in the last two layers. Their first objective was to retrieve subcomponents of the input signal that sum up linearly to the aggregate active and reactive powers. Finally, they estimate the \emph{on-off} activation vector of each load. Their approach, however, was applied on very high frequency current and voltage measurements (12 kHz) from the \underline{B}uilding-\underline{L}evel f\underline{U}lly labeled dataset for \underline{E}lectricity \underline{D}isaggregation (BLUED) \cite{Anderson_2012_BLUED}.

In many of these previous works \cite{Kelly_2015, He_2016_AnEmpiricalStudy, Zhang_2016_SequenceToPointLearning, Nascimento_2016}, each disaggregator is a neural network whose disaggregation window length (and consequently the width of subsequent layers) depends on the load being monitored. The disaggregation window of each load is manually adjusted in a per-load basis to fully capture a single activation cycle of the load. Moreover, the disaggregation performance widely differs amongst variant load categories and a model that achieves remarkably well on one load might drastically fail for other load types.



\section{Load Monitoring}
\label{sec:load-monitoring}

In this work, we focus essentially on \emph{single-load extraction of activation profiles}, of which we give a detailed description in the following.

\subsection{Activation profiles: definition and estimation}

In the simplest case, a load is modeled as a two-state machine which is assumed to be in the \on-state whenever the load is consuming energy from the main power source, and in the \off-state otherwise. Accordingly, load monitoring becomes a binary classification task.
Note that in contrast to previous works, the consumption profile of a load during its \on-state need not be a defined \cite{Zeifman_2011_VAST} nor a piecewise-defined function in time.

The desired signal (i.e. \emph{ground truth}) of a load $m$ in a window of $N$ time instances is the binary-valued signal $\underline{\omega}^{(m)}\in \{0, 1\}^N$ whose element $\omega^{(m)}(n)$ is set (i.e. to indicate an \on-state) whenever the load is operating in one of its activation states at time instance $n$ and unset otherwise. In this work, we refer to this signal as the \emph{activation profile}.
Applications that benefit from activation profiles include mainly activity monitoring and occupancy detection %\cite{Neighborhood NILM}
in which time-of-use information dominates the value of energy consumed.

We define the true activation profile of a load $\underline{\omega}^{(m)}$ via a threshold-based approach applied to the sub-metered real power signals and similar to the one used in \cite{Kelly_2015} as follows.
The sub-metered real-power $\underline{x}^{(m)}$ of a load $m$ is compared against predefined thresholds to detect the operation intervals of the load. In order to avoid anomalies and false activations or deactivations, the load is assumed to be in an activation state (i.e. \on) if its power draw $x^{(m)}(n)$ exceeds a given threshold $\mathcal{P}_{\text{on}}^{(m)}$ for a minimum period of time $\mathcal{N}_{\text{on}}^{(m)}$. Similarly, if the power draw drops below a predefined threshold $\mathcal{P}_{\text{off}}^{(m)}$ for a given period $\mathcal{N}_{\text{off}}^{(m)}$, the load is assumed to be disconnected. Otherwise, the load keeps its last observed state. Thus, the estimated activation profile is defined as

{\small\begin{equation*}
	\omega^{(m)}(n) = 
	\begin{cases}
	1, & \hspace{-10mm}\text{if } x^{(m)}(k) \geqslant \mathcal{P}_{\text{on}}^{(m)}, \; \text{for } n \leqslant k < n+\mathcal{N}_{\text{on}}^{(m)}\\[1mm]
	0, & \hspace{-10mm}\text{if } x^{(m)}(k) \leqslant \mathcal{P}_{\text{off}}^{(m)}, \; \text{for } n \leqslant k < n+\mathcal{N}_{\text{off}}^{(m)}\\[1mm]
	\omega^{(m)}(n-1),	& \hspace{8mm} \text{otherwise}
	\end{cases}
	\end{equation*}}%
with the initial state assumed to be \emph{off} (i.e. $\;\omega^{(m)}(0) = 0$) for all loads. Note that $\mathcal{P}^{(m)}_{\text{on}}$, $\mathcal{N}^{(m)}_{\text{on}}$, $\mathcal{P}^{(m)}_{\text{off}}$, and $\mathcal{N}^{(m)}_{\text{off}}$ are the only load-dependent parameters in this work, and they are used merely in estimating the ground truth signals. Values of these parameters are similar or close to those adopted in \cite{Kelly_2015} and are listed in Table \ref{tbl:loads} for the sake of completeness.

\subsection{Single load extraction}

In single-load extraction, each disaggregator targets exclusively a single load in the monitored circuit and normally ignores dependencies amongst loads. While exploiting loads' dependencies is expected to improve the performance of a disaggregation system in a given building \cite{Kim_2010, Kolter_2010_SparseCoding, Makonin_2014_PhD}, it is also likely to reduce the generalization capability of such a system to new, previously unseen buildings. This is because such dependencies originate not only from the physical architecture of the power line network and the assumed signal model but also from the usage behavior of end-consumers which varies widely from one building to another, especially within the residential sector \cite{Batra_2014_Comparison}.

The load monitoring problem is modeled as $K$-separate binary classification tasks. Given a window of $K$ samples of the aggregate real power signal $\underline{x}(n) = \left[x(n + k)\right]_{k=0}^{k=K-1}$ starting at the time instance $n$, the model $\underline{g}^{(m)}(\underline{x}(n), \bm{\theta})\,\in\,[0,\,1]^K$ estimates the posterior probabilities of the activation profile for the analogous $K$ time instances of $m$\textsuperscript{th} load where $\bm{\theta}$ is the model parameters (e.g. weights and biases in a neural network)

\setlength{\textfloatsep}{0.1cm}
\begin{table}[!t]
	\centering
	\footnotesize
	\caption{Load-dependent parameters for estimating the activation profiles. }
	\label{tbl:loads}
	\renewcommand{\arraystretch}{0.81}
	\footnotesize
	\begin{tabularx}{88mm}{lccc}
		\toprule
		%		\multirow{2}{*}{Load}  	 & $\mathcal{P}_{\text{on}} = \mathcal{P}_{\text{off}}$ & $\mathcal{N}_{\text{on}}$  & $\mathcal{N}_{\text{off}}$\\[1mm]
		%		                                     &             [Watt]               &            [min]         & [min]           \\[1mm]
		Load  	 & $\mathcal{P}_{\text{on}} = \mathcal{P}_{\text{off}}$ [W] & $\mathcal{N}_{\text{on}}$ [min.]  & $\mathcal{N}_{\text{off}}$ [min.]\\[1mm]
		%&             [Watt]               &            [min]         & [min]           \\[1mm]
		\midrule{Fridge \texttt{(FR)}}               & \texttt{5   }    & \texttt{1  }    & \texttt{1   }    \\[1mm]
		{Lights \texttt{(LC)}}               & \texttt{10  }    & \texttt{1  }    & \texttt{1   }    \\[1mm]
		{Dishwasher \texttt{(DW)}}           & \texttt{10  }    & \texttt{30 }    & \texttt{5   }    \\[1mm]
		{Washing machine \texttt{(WM)}}      & \texttt{20  }    & \texttt{30 }    & \texttt{5   }    \\[1mm]
		{Solar  pump \texttt{(SP)}}   		& \texttt{20  }    & \texttt{1  }    & \texttt{1   }    \\[1mm]
		{TV}                                 & \texttt{5   }    & \texttt{3  }    & \texttt{3   }    \\[1mm]
		{Boiler \texttt{(BL)}}               & \texttt{25  }    & \texttt{5  }    & \texttt{5   }    \\[1mm]
		{Kettle \texttt{(KT)}}               & \texttt{1000}    & \texttt{1/3}    & \texttt{1/6 }    \\[1mm]
		{Microwave \texttt{(MC)}}            & \texttt{50  }    & \texttt{1/6}    & \texttt{1/6 }    \\[1mm]
		{Toaster \texttt{(TS)}}              & \texttt{300 }    & \texttt{1/6}    & \texttt{1/20}    \\
		\bottomrule
	\end{tabularx}
\end{table}


{\small\begin{equation}
	p\left(\underline{\omega}^{(m)}(n)=\bm{1}\;\big|\;\underline{x}(n)\right) = \underline{g}^{(m)}\left(\underline{x}(n); \bm{\theta}\right)
	\end{equation}}%
where the disaggregator's output is bound to the valid range of a probability function via a logistic sigmoid activation in the output layer of the network

{\small\begin{equation}
	\underline{g}^{(m)}\left(\underline{x}(n); \bm{\theta}\right) = \underline{\sigma}\left(\underline{\tilde{g}}^{(m)}\left(\underline{x}(n); \bm{\theta}\right)\right)
	\end{equation}}%
where $\underline{\tilde{g}}$ represents the sub-network from the input layer to activation signals of the output layer and $\underline{\sigma}$ is the logistic sigmoid function Eq. \ref{eq:logsg-activation} applied element-wise to $\underline{\tilde{g}}$. 

In the training phase, we refer to the pair $\left(\underline{x}(n),\,\underline{\omega}(n)\right)$ as a single training segment with $K$ data samples. Training segments are extracted from the whole time series signals $\left(\underline{x},\,\underline{\omega}\right)$ using \emph{non-overlappling} windows which results in a training set whose inputs segments are

{\small\begin{equation}
	\bm{\mathcal{X}} =\left(\; \underline{x}(0),\; \underline{x}(K),\; \underline{x}(2K),\; \dots,\;\underline{x}((N_K-1)\cdot K)\right)
	\label{eq:condition-prob}
	\end{equation}}%
and the corresponding activation segments 

{\small\begin{equation}
	\bm{\Omega} = \left(\; \underline{\omega}(0),\; \underline{\omega}(K),\; \underline{\omega}(2K),\; \dots,\; \underline{\omega}((N_K-1)\cdot K)\right)
	\end{equation}}%
where $N_K = \lfloor N / K\rfloor$ is number of training segments (with the $\cdot^{(m)}$ notation omitted for brevity). Assuming all segments (and the $K$ outputs of each segment) are conditionally independent given the input vector $\underline{x}(n)$ and identically distributed (i.i.d), then the likelihood function becomes


{\small\begin{equation*}
	p\left(\bm{\Omega} |\; \bm{\mathcal{X}}, \bm{\theta} \right) = \prod_{n=0}^{N_K-1} \prod_{k=0}^{K-1} p\left(\omega(k + n\cdot K) \;|\; \underline{x}(n\cdot K)\right)
	\end{equation*}}%
with $\omega$ being a Bernoulli distributed random variable

{\small\begin{equation*}
	p\left(\bm{\Omega} |\; \bm{\mathcal{X}}, \bm{\theta} \right) = \prod_{n=0}^{N_K-1} \prod_{k=0}^{K-1} g_k(\underline{x}(n))^{\omega_k} \cdot (1 - g_k(\underline{x}(n)))^{1-\omega_k}
	\end{equation*}}%
where $g_k(\underline{x}(n))$ is the $k^\text{th}$ output of the disaggregator. The negative log-likelihood $\texttt{\textbf{-}}\mathcal{LL}$ then becomes

{\small\begin{equation*}
	\small
	\texttt{\textbf{-}}\mathcal{LL} = -\kern-0.5em\sum_{n=0}^{N_K-1} \sum_{k=0}^{K-1} \omega_k\cdot\ln g_k(\underline{x}(n)) + (1-\omega_k) \ln (1 - g_k(\underline{x}(n)))
	\end{equation*}}%
which is known as \emph{binary cross-entropy} and it is the adopted loss function in all our experiments. The choice of a logistic sigmoid activation function in the output layer together with the binary cross-entropy as the objective function is a standard combination in binary classification problems \cite{Bishop_2006}.

Finally, the following decision rule is used to estimate the final class labels

{\small\begin{equation*}
	\hat{\omega}(n)=
	\begin{cases}
	0 & \text{if } p(\omega(n) = 0\,|\,\underline{x}(n)) > p(\omega(n) = 1\,|\,\underline{x}(n)) \\[1mm]
	1 & \text{otherwise}
	\end{cases}
	\end{equation*}}%

We point out that the concept of load activation cycles, a complete cycle of operation \off $\rightarrow$ \on $\rightarrow$ \off, is not considered. In other words, an activation cycle of a load can extend over several $K$-length segments (such as lighting circuits and dishwashers) or arise more than once within the same window segment (as in fridge and kettle activations). This is an important property since a disaggregator need not wait till the deactivation of a load (i.e. switch-\off event) but rather can provide {near} real-time feedback from a partial segment of the activation, normally with some delay. 

\section{Model Architecture}
\label{sec:model-selection}

Figure \ref{fig:ae-2-architecture} shows the architecture of the proposed fully convolutional neural network model. The model consists of 46 layers in five parts (an input layer, 40 encoding and decoding layers, 4 representation layers, and an output layer) reaching 41M trainable parameters. Each layer includes a sequence of elementary operations shown in the figure and briefly introduced in the sequel. 

\begin{figure}[!htbp]
	\definecolor{bc01}{RGB}{232, 208, 169}
	\definecolor{bc02}{RGB}{183, 175, 163}
	\definecolor{bc03}{RGB}{193, 218, 214}
	\definecolor{bc04}{RGB}{172, 209, 233}
	\definecolor{bc05}{RGB}{109, 146, 155}
	\definecolor{bc06}{RGB}{245, 250, 250}
	\definecolor{bsgm}{RGB}{255, 204, 203}
	\definecolor{mainblocknormal}{RGB}{0, 153, 204}
	\definecolor{pooling}{RGB}{67,171,101}
	\definecolor{upsampling}{RGB}{81, 163, 157}
	\pgfmathsetlengthmacro\LblR{5mm}
	\pgfmathsetlengthmacro\LblL{6mm}
	\pgfmathsetlengthmacro\AllLayerWidth{1.75mm}
	\pgfmathsetlengthmacro\InnerLayerSep{3.9mm}
	\pgfmathsetlengthmacro\InnerLayerSepMerge{0.6mm}
	\pgfmathsetlengthmacro\SkpXShift{2mm}	
	
	\pgfdeclarelayer{bbackground}
	\pgfdeclarelayer{background}
	\pgfdeclarelayer{foreground}
	\pgfsetlayers{bbackground,background,main,foreground}
	
	
	\def\vsvsin{-2.75mm}
	\def\vshdin{-0.90mm}
	\def\vshdot{-1.99mm}
	\def\vsvsot{-2.75mm}
	
	\def\CONVHeight{3.25mm}
	\def\CONVWidth{33mm}
	
	\tikzstyle{CONVst}=[anchor=north, draw=black, minimum height=\CONVHeight, minimum width=\CONVWidth,
	inner xsep=0mm, inner ysep=0mm, outer xsep=0mm, outer ysep=0mm, text width=47.5mm]
	\tikzstyle{llst}=[anchor=east, align=right, inner xsep=0.3mm, inner ysep=0mm, font={\tiny}, align=right]
	\tikzstyle{rlst}=[anchor=west, align=right, inner xsep=1.0mm, inner ysep=0mm, font={\scriptsize}]
	\tikzstyle{sqar}=[line width=1.0pt, arrows={-{Latex[length=1.5mm, round]}}]
	\tikzstyle{skrs}=[]
	\tikzstyle{idad}=[]
	
	
	\centering
	\scriptsize
	\begin{tikzpicture}
	
	
	\node (inpt) [anchor=north] {Aggregate signal $\underline{x}(n)$};
	\node (x000) at ($(inpt.south) + (0, \vsvsin)$) [CONVst, fill=bsgm] {\texttt{\ \ \ \ \ \ \ \ \ CONV(7, 1)}, \texttt{BN, LogSg, GN}};↕ \node (ll00) at ($(x000.west) + (\LblL, 0)$) [llst] {\texttt{  64x}\\[-0.6mm]\texttt{10800}}; \node (rl00) at ($(x000.east) - (\LblR, 0)$) [rlst] {\texttt{\textbf{  }}};
	\node (xl10) at ($(x000.south) + (0, \vshdot)$) [CONVst, fill=bc04] {\texttt{\ \ \ \ \ \ \ \ \ CONV(7, 3)}, \texttt{LReLU, BN, GN}};↕ \node (ll10) at ($(xl10.west) + (\LblL, 0)$) [llst] {\texttt{  32x}\\[-0.6mm]\texttt{10800}}; \node (rl10) at ($(xl10.east) - (\LblR, 0)$) [rlst] {\texttt{\textbf{  }}};  
	\node (xl11) at ($(xl10.south) + (0, \vshdin)$) [CONVst, fill=bc04] {\texttt{\ \ \ \ \ \ \ \ \ CONV(7, 3)}, \texttt{LReLU, BN, GN}};↕ \node (ll11) at ($(xl11.west) + (\LblL, 0)$) [llst] {\texttt{  32x}\\[-0.6mm]\texttt{10800}}; \node (rl11) at ($(xl11.east) - (\LblR, 0)$) [rlst] {\texttt{\textbf{  }}};  
	\node (xl12) at ($(xl11.south) + (0, \vshdin)$) [CONVst, fill=bc04] {\texttt{\ \ \ \ \ \ \ \ \ CONV(7, 3)}, \texttt{LReLU, BN, GN}};↕ \node (ll12) at ($(xl12.west) + (\LblL, 0)$) [llst] {\texttt{  32x}\\[-0.6mm]\texttt{10800}}; \node (rl12) at ($(xl12.east) - (\LblR, 0)$) [rlst] {\texttt{\textbf{  }}};  
	\node (xl13) at ($(xl12.south) + (0, \vshdin)$) [CONVst, fill=bc03] {\texttt{\ \ \ \ \ \ \ \ \ CONV(7, 3)}, \texttt{LReLU, BN, GN}};↕ \node (ll13) at ($(xl13.west) + (\LblL, 0)$) [llst] {\texttt{  32x}\\[-0.6mm]\texttt{10800}}; \node (rl13) at ($(xl13.east) - (\LblR, 0)$) [rlst] {\texttt{\textbf{/3}}};  
	\node (xl20) at ($(xl13.south) + (0, \vshdot)$) [CONVst, fill=bc04] {\texttt{\ \ \ \ \ \ \ \ \ CONV(7, 3)}, \texttt{LReLU, BN, GN}};↕ \node (ll20) at ($(xl20.west) + (\LblL, 0)$) [llst] {\texttt{  64x}\\[-0.6mm]\texttt{ 3600}}; \node (rl20) at ($(xl20.east) - (\LblR, 0)$) [rlst] {\texttt{\textbf{  }}};  
	\node (xl21) at ($(xl20.south) + (0, \vshdin)$) [CONVst, fill=bc04] {\texttt{\ \ \ \ \ \ \ \ \ CONV(7, 3)}, \texttt{LReLU, BN, GN}};↕ \node (ll21) at ($(xl21.west) + (\LblL, 0)$) [llst] {\texttt{  64x}\\[-0.6mm]\texttt{ 3600}}; \node (rl21) at ($(xl21.east) - (\LblR, 0)$) [rlst] {\texttt{\textbf{  }}};  
	\node (xl22) at ($(xl21.south) + (0, \vshdin)$) [CONVst, fill=bc04] {\texttt{\ \ \ \ \ \ \ \ \ CONV(7, 3)}, \texttt{LReLU, BN, GN}};↕ \node (ll22) at ($(xl22.west) + (\LblL, 0)$) [llst] {\texttt{  64x}\\[-0.6mm]\texttt{ 3600}}; \node (rl22) at ($(xl22.east) - (\LblR, 0)$) [rlst] {\texttt{\textbf{  }}};  
	\node (xl23) at ($(xl22.south) + (0, \vshdin)$) [CONVst, fill=bc03] {\texttt{\ \ \ \ \ \ \ \ \ CONV(7, 3)}, \texttt{LReLU, BN, GN}};↕ \node (ll23) at ($(xl23.west) + (\LblL, 0)$) [llst] {\texttt{  64x}\\[-0.6mm]\texttt{ 3600}}; \node (rl23) at ($(xl23.east) - (\LblR, 0)$) [rlst] {\texttt{\textbf{/3}}};  
	\node (xl30) at ($(xl23.south) + (0, \vshdot)$) [CONVst, fill=bc04] {\texttt{\ \ \ \ \ \ \ \ \ CONV(7, 3)}, \texttt{LReLU, BN, GN}};↕ \node (ll30) at ($(xl30.west) + (\LblL, 0)$) [llst] {\texttt{ 128x}\\[-0.6mm]\texttt{ 1200}}; \node (rl30) at ($(xl30.east) - (\LblR, 0)$) [rlst] {\texttt{\textbf{  }}};  
	\node (xl31) at ($(xl30.south) + (0, \vshdin)$) [CONVst, fill=bc04] {\texttt{\ \ \ \ \ \ \ \ \ CONV(7, 3)}, \texttt{LReLU, BN, GN}};↕ \node (ll31) at ($(xl31.west) + (\LblL, 0)$) [llst] {\texttt{ 128x}\\[-0.6mm]\texttt{ 1200}}; \node (rl31) at ($(xl31.east) - (\LblR, 0)$) [rlst] {\texttt{\textbf{  }}};  
	\node (xl32) at ($(xl31.south) + (0, \vshdin)$) [CONVst, fill=bc04] {\texttt{\ \ \ \ \ \ \ \ \ CONV(7, 3)}, \texttt{LReLU, BN, GN}};↕ \node (ll32) at ($(xl32.west) + (\LblL, 0)$) [llst] {\texttt{ 128x}\\[-0.6mm]\texttt{ 1200}}; \node (rl32) at ($(xl32.east) - (\LblR, 0)$) [rlst] {\texttt{\textbf{  }}};  
	\node (xl33) at ($(xl32.south) + (0, \vshdin)$) [CONVst, fill=bc03] {\texttt{\ \ \ \ \ \ \ \ \ CONV(7, 3)}, \texttt{LReLU, BN, GN}};↕ \node (ll33) at ($(xl33.west) + (\LblL, 0)$) [llst] {\texttt{ 128x}\\[-0.6mm]\texttt{ 1200}}; \node (rl33) at ($(xl33.east) - (\LblR, 0)$) [rlst] {\texttt{\textbf{/3}}};  
	\node (xl40) at ($(xl33.south) + (0, \vshdot)$) [CONVst, fill=bc04] {\texttt{\ \ \ \ \ \ \ \ \ CONV(7, 3)}, \texttt{LReLU, BN, GN}};↕ \node (ll40) at ($(xl40.west) + (\LblL, 0)$) [llst] {\texttt{ 256x}\\[-0.6mm]\texttt{  400}}; \node (rl40) at ($(xl40.east) - (\LblR, 0)$) [rlst] {\texttt{\textbf{  }}}; 
	\node (xl41) at ($(xl40.south) + (0, \vshdin)$) [CONVst, fill=bc04] {\texttt{\ \ \ \ \ \ \ \ \ CONV(7, 3)}, \texttt{LReLU, BN, GN}};↕ \node (ll41) at ($(xl41.west) + (\LblL, 0)$) [llst] {\texttt{ 256x}\\[-0.6mm]\texttt{  400}}; \node (rl41) at ($(xl41.east) - (\LblR, 0)$) [rlst] {\texttt{\textbf{  }}};  
	\node (xl42) at ($(xl41.south) + (0, \vshdin)$) [CONVst, fill=bc04] {\texttt{\ \ \ \ \ \ \ \ \ CONV(7, 3)}, \texttt{LReLU, BN, GN}};↕ \node (ll42) at ($(xl42.west) + (\LblL, 0)$) [llst] {\texttt{ 256x}\\[-0.6mm]\texttt{  400}}; \node (rl42) at ($(xl42.east) - (\LblR, 0)$) [rlst] {\texttt{\textbf{  }}};  
	\node (xl43) at ($(xl42.south) + (0, \vshdin)$) [CONVst, fill=bc03] {\texttt{\ \ \ \ \ \ \ \ \ CONV(7, 3)}, \texttt{LReLU, BN, GN}};↕ \node (ll43) at ($(xl43.west) + (\LblL, 0)$) [llst] {\texttt{ 256x}\\[-0.6mm]\texttt{  400}}; \node (rl43) at ($(xl43.east) - (\LblR, 0)$) [rlst] {\texttt{\textbf{/5}}};  
	\node (xl50) at ($(xl43.south) + (0, \vshdot)$) [CONVst, fill=bc04] {\texttt{\ \ \ \ \ \ \ \ \ CONV(7, 3)}, \texttt{LReLU, BN, GN}};↕ \node (ll50) at ($(xl50.west) + (\LblL, 0)$) [llst] {\texttt{ 512x}\\[-0.6mm]\texttt{   80}}; \node (rl50) at ($(xl50.east) - (\LblR, 0)$) [rlst] {\texttt{\textbf{  }}};  
	\node (xl51) at ($(xl50.south) + (0, \vshdin)$) [CONVst, fill=bc04] {\texttt{\ \ \ \ \ \ \ \ \ CONV(7, 3)}, \texttt{LReLU, BN, GN}};↕ \node (ll51) at ($(xl51.west) + (\LblL, 0)$) [llst] {\texttt{ 512x}\\[-0.6mm]\texttt{   80}}; \node (rl51) at ($(xl51.east) - (\LblR, 0)$) [rlst] {\texttt{\textbf{  }}};  
	\node (xl52) at ($(xl51.south) + (0, \vshdin)$) [CONVst, fill=bc04] {\texttt{\ \ \ \ \ \ \ \ \ CONV(7, 3)}, \texttt{LReLU, BN, GN}};↕ \node (ll52) at ($(xl52.west) + (\LblL, 0)$) [llst] {\texttt{ 512x}\\[-0.6mm]\texttt{   80}}; \node (rl52) at ($(xl52.east) - (\LblR, 0)$) [rlst] {\texttt{\textbf{  }}};  
	\node (xl53) at ($(xl52.south) + (0, \vshdin)$) [CONVst, fill=bc03] {\texttt{\ \ \ \ \ \ \ \ \ CONV(7, 3)}, \texttt{LReLU, BN, GN}};↕ \node (ll53) at ($(xl53.west) + (\LblL, 0)$) [llst] {\texttt{ 512x}\\[-0.6mm]\texttt{   80}}; \node (rl53) at ($(xl53.east) - (\LblR, 0)$) [rlst] {\texttt{\textbf{/5}}};  
	\node (xl60) at ($(xl53.south) + (0, \vshdot)$) [CONVst, fill=bc04] {\texttt{\ \ \ \ \ \ \ \ \ CONV(7, 3)}, \texttt{LReLU, BN, GN}};↕ \node (ll60) at ($(xl60.west) + (\LblL, 0)$) [llst] {\texttt{1024x}\\[-0.6mm]\texttt{   16}}; \node (rl60) at ($(xl60.east) - (\LblR, 0)$) [rlst] {\texttt{\textbf{  }}};  
	\node (xl61) at ($(xl60.south) + (0, \vshdin)$) [CONVst, fill=bc04] {\texttt{\ \ \ \ \ \ \ \ \ CONV(7, 3)}, \texttt{LReLU, BN, GN}};↕ \node (ll61) at ($(xl61.west) + (\LblL, 0)$) [llst] {\texttt{1024x}\\[-0.6mm]\texttt{   16}}; \node (rl61) at ($(xl61.east) - (\LblR, 0)$) [rlst] {\texttt{\textbf{  }}};  
	\node (xl62) at ($(xl61.south) + (0, \vshdin)$) [CONVst, fill=bc04] {\texttt{\ \ \ \ \ \ \ \ \ CONV(7, 3)}, \texttt{LReLU, BN, GN}};↕ \node (ll62) at ($(xl62.west) + (\LblL, 0)$) [llst] {\texttt{1024x}\\[-0.6mm]\texttt{   16}}; \node (rl62) at ($(xl62.east) - (\LblR, 0)$) [rlst] {\texttt{\textbf{  }}};  
	\node (xl63) at ($(xl62.south) + (0, \vshdin)$) [CONVst, fill=bc04] {\texttt{\ \ \ \ \ \ \ \ \ CONV(7, 3)}, \texttt{LReLU, BN, GN}};↕ \node (ll63) at ($(xl63.west) + (\LblL, 0)$) [llst] {\texttt{1024x}\\[-0.6mm]\texttt{   16}}; \node (rl63) at ($(xl63.east) - (\LblR, 0)$) [rlst] {\texttt{\textbf{  }}};  
	\node (xl70) at ($(xl63.south) + (0, \vshdot)$) [CONVst, fill=bc01] {\texttt{\ \ \ \ \ \ \ \ \ CONV(7, 3)}, \texttt{LReLU, BN, GN}};↕ \node (ll70) at ($(xl70.west) + (\LblL, 0)$) [llst] {\texttt{ 512x}\\[-0.6mm]\texttt{   80}}; \node (rl70) at ($(xl70.east) - (\LblR, 0)$) [rlst] {\texttt{\textbf{x5}}};  
	\node (xl71) at ($(xl70.south) + (0, \vshdin)$) [CONVst, fill=bc04] {\texttt{\ \ \ \ \ \ \ \ \ CONV(7, 3)}, \texttt{LReLU, BN, GN}};↕ \node (ll71) at ($(xl71.west) + (\LblL, 0)$) [llst] {\texttt{ 512x}\\[-0.6mm]\texttt{   80}}; \node (rl71) at ($(xl71.east) - (\LblR, 0)$) [rlst] {\texttt{\textbf{  }}};  
	\node (xl72) at ($(xl71.south) + (0, \vshdin)$) [CONVst, fill=bc04] {\texttt{\ \ \ \ \ \ \ \ \ CONV(7, 3)}, \texttt{LReLU, BN, GN}};↕ \node (ll72) at ($(xl72.west) + (\LblL, 0)$) [llst] {\texttt{ 512x}\\[-0.6mm]\texttt{   80}}; \node (rl72) at ($(xl72.east) - (\LblR, 0)$) [rlst] {\texttt{\textbf{  }}};  
	\node (xl73) at ($(xl72.south) + (0, \vshdin)$) [CONVst, fill=bc04] {\texttt{\ \ \ \ \ \ \ \ \ CONV(7, 3)}, \texttt{LReLU, BN, GN}};↕ \node (ll73) at ($(xl73.west) + (\LblL, 0)$) [llst] {\texttt{ 512x}\\[-0.6mm]\texttt{   80}}; \node (rl73) at ($(xl73.east) - (\LblR, 0)$) [rlst] {\texttt{\textbf{  }}};  
	\node (xl80) at ($(xl73.south) + (0, \vshdot)$) [CONVst, fill=bc01] {\texttt{\ \ \ \ \ \ \ \ \ CONV(7, 3)}, \texttt{LReLU, BN, GN}};↕ \node (ll80) at ($(xl80.west) + (\LblL, 0)$) [llst] {\texttt{ 256x}\\[-0.6mm]\texttt{  400}}; \node (rl80) at ($(xl80.east) - (\LblR, 0)$) [rlst] {\texttt{\textbf{x5}}};  
	\node (xl81) at ($(xl80.south) + (0, \vshdin)$) [CONVst, fill=bc04] {\texttt{\ \ \ \ \ \ \ \ \ CONV(7, 3)}, \texttt{LReLU, BN, GN}};↕ \node (ll81) at ($(xl81.west) + (\LblL, 0)$) [llst] {\texttt{ 256x}\\[-0.6mm]\texttt{  400}}; \node (rl81) at ($(xl81.east) - (\LblR, 0)$) [rlst] {\texttt{\textbf{  }}};  
	\node (xl82) at ($(xl81.south) + (0, \vshdin)$) [CONVst, fill=bc04] {\texttt{\ \ \ \ \ \ \ \ \ CONV(7, 3)}, \texttt{LReLU, BN, GN}};↕ \node (ll82) at ($(xl82.west) + (\LblL, 0)$) [llst] {\texttt{ 256x}\\[-0.6mm]\texttt{  400}}; \node (rl82) at ($(xl82.east) - (\LblR, 0)$) [rlst] {\texttt{\textbf{  }}};  
	\node (xl83) at ($(xl82.south) + (0, \vshdin)$) [CONVst, fill=bc04] {\texttt{\ \ \ \ \ \ \ \ \ CONV(7, 3)}, \texttt{LReLU, BN, GN}};↕ \node (ll83) at ($(xl83.west) + (\LblL, 0)$) [llst] {\texttt{ 256x}\\[-0.6mm]\texttt{  400}}; \node (rl83) at ($(xl83.east) - (\LblR, 0)$) [rlst] {\texttt{\textbf{  }}};  
	\node (xl90) at ($(xl83.south) + (0, \vshdot)$) [CONVst, fill=bc01] {\texttt{\ \ \ \ \ \ \ \ \ CONV(7, 3)}, \texttt{LReLU, BN, GN}};↕ \node (ll90) at ($(xl90.west) + (\LblL, 0)$) [llst] {\texttt{ 128x}\\[-0.6mm]\texttt{ 1200}}; \node (rl90) at ($(xl90.east) - (\LblR, 0)$) [rlst] {\texttt{\textbf{x3}}};  
	\node (xl91) at ($(xl90.south) + (0, \vshdin)$) [CONVst, fill=bc04] {\texttt{\ \ \ \ \ \ \ \ \ CONV(7, 3)}, \texttt{LReLU, BN, GN}};↕ \node (ll91) at ($(xl91.west) + (\LblL, 0)$) [llst] {\texttt{ 128x}\\[-0.6mm]\texttt{ 1200}}; \node (rl91) at ($(xl91.east) - (\LblR, 0)$) [rlst] {\texttt{\textbf{  }}};  
	\node (xl92) at ($(xl91.south) + (0, \vshdin)$) [CONVst, fill=bc04] {\texttt{\ \ \ \ \ \ \ \ \ CONV(7, 3)}, \texttt{LReLU, BN, GN}};↕ \node (ll92) at ($(xl92.west) + (\LblL, 0)$) [llst] {\texttt{ 128x}\\[-0.6mm]\texttt{ 1200}}; \node (rl92) at ($(xl92.east) - (\LblR, 0)$) [rlst] {\texttt{\textbf{  }}};  
	\node (xl93) at ($(xl92.south) + (0, \vshdin)$) [CONVst, fill=bc04] {\texttt{\ \ \ \ \ \ \ \ \ CONV(7, 3)}, \texttt{LReLU, BN, GN}};↕ \node (ll93) at ($(xl93.west) + (\LblL, 0)$) [llst] {\texttt{ 128x}\\[-0.6mm]\texttt{ 1200}}; \node (rl93) at ($(xl93.east) - (\LblR, 0)$) [rlst] {\texttt{\textbf{  }}};  
	\node (xlA0) at ($(xl93.south) + (0, \vshdot)$) [CONVst, fill=bc01] {\texttt{\ \ \ \ \ \ \ \ \ CONV(7, 3)}, \texttt{LReLU, BN, GN}};↕ \node (llA0) at ($(xlA0.west) + (\LblL, 0)$) [llst] {\texttt{  64x}\\[-0.6mm]\texttt{ 3600}}; \node (rlA0) at ($(xlA0.east) - (\LblR, 0)$) [rlst] {\texttt{\textbf{x3}}};  
	\node (xlA1) at ($(xlA0.south) + (0, \vshdin)$) [CONVst, fill=bc04] {\texttt{\ \ \ \ \ \ \ \ \ CONV(7, 3)}, \texttt{LReLU, BN, GN}};↕ \node (llA1) at ($(xlA1.west) + (\LblL, 0)$) [llst] {\texttt{  64x}\\[-0.6mm]\texttt{ 3600}}; \node (rlA1) at ($(xlA1.east) - (\LblR, 0)$) [rlst] {\texttt{\textbf{  }}};  
	\node (xlA2) at ($(xlA1.south) + (0, \vshdin)$) [CONVst, fill=bc04] {\texttt{\ \ \ \ \ \ \ \ \ CONV(7, 3)}, \texttt{LReLU, BN, GN}};↕ \node (llA2) at ($(xlA2.west) + (\LblL, 0)$) [llst] {\texttt{  64x}\\[-0.6mm]\texttt{ 3600}}; \node (rlA2) at ($(xlA2.east) - (\LblR, 0)$) [rlst] {\texttt{\textbf{  }}};  
	\node (xlA3) at ($(xlA2.south) + (0, \vshdin)$) [CONVst, fill=bc04] {\texttt{\ \ \ \ \ \ \ \ \ CONV(7, 3)}, \texttt{LReLU, BN, GN}};↕ \node (llA3) at ($(xlA3.west) + (\LblL, 0)$) [llst] {\texttt{  64x}\\[-0.6mm]\texttt{ 3600}}; \node (rlA3) at ($(xlA3.east) - (\LblR, 0)$) [rlst] {\texttt{\textbf{  }}};  
	\node (xlB0) at ($(xlA3.south) + (0, \vshdot)$) [CONVst, fill=bc01] {\texttt{\ \ \ \ \ \ \ \ \ CONV(7, 3)}, \texttt{LReLU, BN, GN}};↕ \node (llB0) at ($(xlB0.west) + (\LblL, 0)$) [llst] {\texttt{  32x}\\[-0.6mm]\texttt{10800}}; \node (rlB0) at ($(xlB0.east) - (\LblR, 0)$) [rlst] {\texttt{\textbf{x3}}};  
	\node (xlB1) at ($(xlB0.south) + (0, \vshdin)$) [CONVst, fill=bc04] {\texttt{\ \ \ \ \ \ \ \ \ CONV(7, 3)}, \texttt{LReLU, BN, GN}};↕ \node (llB1) at ($(xlB1.west) + (\LblL, 0)$) [llst] {\texttt{  32x}\\[-0.6mm]\texttt{10800}}; \node (rlB1) at ($(xlB1.east) - (\LblR, 0)$) [rlst] {\texttt{\textbf{  }}};  
	\node (xlB2) at ($(xlB1.south) + (0, \vshdin)$) [CONVst, fill=bc04] {\texttt{\ \ \ \ \ \ \ \ \ CONV(7, 3)}, \texttt{LReLU, BN, GN}};↕ \node (llB2) at ($(xlB2.west) + (\LblL, 0)$) [llst] {\texttt{  32x}\\[-0.6mm]\texttt{10800}}; \node (rlB2) at ($(xlB2.east) - (\LblR, 0)$) [rlst] {\texttt{\textbf{  }}};  
	\node (xlB3) at ($(xlB2.south) + (0, \vshdin)$) [CONVst, fill=bc04] {\texttt{\ \ \ \ \ \ \ \ \ CONV(7, 3)}, \texttt{LReLU, BN, GN}};↕ \node (llB3) at ($(xlB3.west) + (\LblL, 0)$) [llst] {\texttt{  32x}\\[-0.6mm]\texttt{10800}}; \node (rlB3) at ($(xlB3.east) - (\LblR, 0)$) [rlst] {\texttt{\textbf{  }}};  
	\node (xlD0) at ($(xlB3.south) + (0, \vshdot)$) [CONVst, fill=bsgm] {\texttt{\ \ \ \ \ \ \ \ \ \ \ \ \ CONV(7, 3)}, \texttt{BN, LogSg    }};↕ \node (llD0) at ($(xlD0.west) + (\LblL, 0)$) [llst] {\texttt{   1x}\\[-0.6mm]\texttt{10800}}; \node (rlD0) at ($(xlD0.east) - (\LblR, 0)$) [rlst] {\texttt{\textbf{  }}};
	\node (otpt) at ($(xlD0.south) + (0, \vsvsot)$) [anchor=north]  {Activation profile $\underline{g}(\underline{x}(n))$};
	
	
	\draw (x000.north -| rl00.west) -- (x000.south -| rl00.west) (x000.north -| llD0.east) -- (x000.south -| llD0.east);
	\draw (xl10.north -| rl10.west) -- (xl10.south -| rl10.west) (xl10.north -| ll00.east) -- (xl10.south -| ll00.east);
	\draw (xl11.north -| rl11.west) -- (xl11.south -| rl11.west) (xl11.north -| ll10.east) -- (xl11.south -| ll10.east);
	\draw (xl12.north -| rl12.west) -- (xl12.south -| rl12.west) (xl12.north -| ll11.east) -- (xl12.south -| ll11.east);
	\draw (xl13.north -| rl13.west) -- (xl13.south -| rl13.west) (xl13.north -| ll12.east) -- (xl13.south -| ll12.east);
	\draw (xl20.north -| rl20.west) -- (xl20.south -| rl20.west) (xl20.north -| ll13.east) -- (xl20.south -| ll13.east);
	\draw (xl21.north -| rl21.west) -- (xl21.south -| rl21.west) (xl21.north -| ll20.east) -- (xl21.south -| ll20.east);
	\draw (xl22.north -| rl22.west) -- (xl22.south -| rl22.west) (xl22.north -| ll21.east) -- (xl22.south -| ll21.east);
	\draw (xl23.north -| rl23.west) -- (xl23.south -| rl23.west) (xl23.north -| ll22.east) -- (xl23.south -| ll22.east);
	\draw (xl30.north -| rl30.west) -- (xl30.south -| rl30.west) (xl30.north -| ll23.east) -- (xl30.south -| ll23.east);
	\draw (xl31.north -| rl31.west) -- (xl31.south -| rl31.west) (xl31.north -| ll30.east) -- (xl31.south -| ll30.east);
	\draw (xl32.north -| rl32.west) -- (xl32.south -| rl32.west) (xl32.north -| ll31.east) -- (xl32.south -| ll31.east);
	\draw (xl33.north -| rl33.west) -- (xl33.south -| rl33.west) (xl33.north -| ll32.east) -- (xl33.south -| ll32.east);
	\draw (xl40.north -| rl40.west) -- (xl40.south -| rl40.west) (xl40.north -| ll33.east) -- (xl40.south -| ll33.east);
	\draw (xl41.north -| rl41.west) -- (xl41.south -| rl41.west) (xl41.north -| ll40.east) -- (xl41.south -| ll40.east);
	\draw (xl42.north -| rl42.west) -- (xl42.south -| rl42.west) (xl42.north -| ll41.east) -- (xl42.south -| ll41.east);
	\draw (xl43.north -| rl43.west) -- (xl43.south -| rl43.west) (xl43.north -| ll42.east) -- (xl43.south -| ll42.east);
	\draw (xl50.north -| rl50.west) -- (xl50.south -| rl50.west) (xl50.north -| ll43.east) -- (xl50.south -| ll43.east);
	\draw (xl51.north -| rl51.west) -- (xl51.south -| rl51.west) (xl51.north -| ll50.east) -- (xl51.south -| ll50.east);
	\draw (xl52.north -| rl52.west) -- (xl52.south -| rl52.west) (xl52.north -| ll51.east) -- (xl52.south -| ll51.east);
	\draw (xl53.north -| rl53.west) -- (xl53.south -| rl53.west) (xl53.north -| ll52.east) -- (xl53.south -| ll52.east);
	\draw (xl60.north -| rl60.west) -- (xl60.south -| rl60.west) (xl60.north -| ll53.east) -- (xl60.south -| ll53.east);
	\draw (xl61.north -| rl61.west) -- (xl61.south -| rl61.west) (xl61.north -| ll60.east) -- (xl61.south -| ll60.east);
	\draw (xl62.north -| rl62.west) -- (xl62.south -| rl62.west) (xl62.north -| ll61.east) -- (xl62.south -| ll61.east);
	\draw (xl63.north -| rl63.west) -- (xl63.south -| rl63.west) (xl63.north -| ll62.east) -- (xl63.south -| ll62.east);
	\draw (xl70.north -| rl70.west) -- (xl70.south -| rl70.west) (xl70.north -| ll63.east) -- (xl70.south -| ll63.east);
	\draw (xl71.north -| rl71.west) -- (xl71.south -| rl71.west) (xl71.north -| ll70.east) -- (xl71.south -| ll70.east);
	\draw (xl72.north -| rl72.west) -- (xl72.south -| rl72.west) (xl72.north -| ll71.east) -- (xl72.south -| ll71.east);
	\draw (xl73.north -| rl73.west) -- (xl73.south -| rl73.west) (xl73.north -| ll72.east) -- (xl73.south -| ll72.east);
	\draw (xl80.north -| rl80.west) -- (xl80.south -| rl80.west) (xl80.north -| ll73.east) -- (xl80.south -| ll73.east);
	\draw (xl81.north -| rl81.west) -- (xl81.south -| rl81.west) (xl81.north -| ll80.east) -- (xl81.south -| ll80.east);
	\draw (xl82.north -| rl82.west) -- (xl82.south -| rl82.west) (xl82.north -| ll81.east) -- (xl82.south -| ll81.east);
	\draw (xl83.north -| rl83.west) -- (xl83.south -| rl83.west) (xl83.north -| ll82.east) -- (xl83.south -| ll82.east);
	\draw (xl90.north -| rl90.west) -- (xl90.south -| rl90.west) (xl90.north -| ll83.east) -- (xl90.south -| ll83.east);
	\draw (xl91.north -| rl91.west) -- (xl91.south -| rl91.west) (xl91.north -| ll90.east) -- (xl91.south -| ll90.east);
	\draw (xl92.north -| rl92.west) -- (xl92.south -| rl92.west) (xl92.north -| ll91.east) -- (xl92.south -| ll91.east);
	\draw (xl93.north -| rl93.west) -- (xl93.south -| rl93.west) (xl93.north -| ll92.east) -- (xl93.south -| ll92.east);
	\draw (xlA0.north -| rlA0.west) -- (xlA0.south -| rlA0.west) (xlA0.north -| ll93.east) -- (xlA0.south -| ll93.east);
	\draw (xlA1.north -| rlA1.west) -- (xlA1.south -| rlA1.west) (xlA1.north -| llA0.east) -- (xlA1.south -| llA0.east);
	\draw (xlA2.north -| rlA2.west) -- (xlA2.south -| rlA2.west) (xlA2.north -| llA1.east) -- (xlA2.south -| llA1.east);
	\draw (xlA3.north -| rlA3.west) -- (xlA3.south -| rlA3.west) (xlA3.north -| llA2.east) -- (xlA3.south -| llA2.east);
	\draw (xlB0.north -| rlB0.west) -- (xlB0.south -| rlB0.west) (xlB0.north -| llA3.east) -- (xlB0.south -| llA3.east);
	\draw (xlB1.north -| rlB1.west) -- (xlB1.south -| rlB1.west) (xlB1.north -| llB0.east) -- (xlB1.south -| llB0.east);
	\draw (xlB2.north -| rlB2.west) -- (xlB2.south -| rlB2.west) (xlB2.north -| llB1.east) -- (xlB2.south -| llB1.east);
	\draw (xlB3.north -| rlB3.west) -- (xlB3.south -| rlB3.west) (xlB3.north -| llB2.east) -- (xlB3.south -| llB2.east);
	\draw (xlD0.north -| rlD0.west) -- (xlD0.south -| rlD0.west) (xlD0.north -| llB3.east) -- (xlD0.south -| llB3.east);
	
	\foreach \x / \y in {inpt/x000, x000/xl10, xl10/xl11, xl11/xl12, xl12/xl13, xl13/xl20, xl20/xl21, xl21/xl22, xl22/xl23, xl23/xl30, xl30/xl31, xl31/xl32, xl32/xl33, xl33/xl40,
		xl40/xl41, xl41/xl42, xl42/xl43, xl43/xl50, xl50/xl51, xl51/xl52, xl52/xl53, xl53/xl60, xl60/xl61, xl61/xl62, xl62/xl63, xl63/xl70, xl70/xl71,
		xl71/xl72, xl72/xl73, xl73/xl80, xl80/xl81, xl81/xl82, xl82/xl83, xl83/xl90, xl90/xl91, xl91/xl92, xl92/xl93, xl93/xlA0, xlA0/xlA1, xlA1/xlA2,
		xlA2/xlA3, xlA3/xlB0, xlB0/xlB1, xlB1/xlB2, xlB2/xlB3, xlB3/xlD0, xlD0/otpt}
	\draw[sqar] (\x.south) -- (\y.north);
	
	\foreach \x / \y in {xl10/xl13, xl20/xl23, xl30/xl33, xl40/xl43, xl50/xl53, xl60/xl63, xl70/xl73, xl80/xl83, xl90/xl93, xlA0/xlA3, xlB0/xlB3}
	\draw[line width=1.0pt, arrows={-{Latex[length=1.5mm, round]}}, out=-180, in=180] (\x.west) to  node [midway, fill=white, circle, draw=black, line width=0.1pt, inner ysep=0.33mm, inner xsep=0.33mm] {\texttt{\textbf{+}}} (\y.west);
	
	
	
	
	\begin{pgfonlayer}{bbackground}
	\definecolor{encC}{RGB}{205, 185, 156}
	\definecolor{repC}{RGB}{204, 204, 204}
	\definecolor{decC}{RGB}{205, 185, 156}
	\node (a) at ($(xl10.north)!0.5!(xl53.south)$) [fill=encC, draw=none, minimum width =62.5mm, minimum height=88mm, opacity=0.40] {};
	\node (b) at ($(xl60.north)!0.5!(xl63.south)$) [fill=repC, draw=none, minimum width =62.5mm, minimum height=17mm, opacity=0.50] {};
	\node (c) at ($(xl70.north)!0.5!(xlB3.south)$) [fill=decC, draw=none, minimum width =62.5mm, minimum height=88mm, opacity=0.40] {};
	\node at (a.west) [anchor=south, rotate=90] {Encoder Sub-Net};
	\node at (b.west) [anchor=south, rotate=90] {Representation};
	\node at (c.west) [anchor=south, rotate=90] {Decoder Sub-Net};
	
	\tikzstyle{CONVconcat}=[CONVst, anchor=center, minimum width=43.5mm, xshift=0.7mm, yshift=0.35mm]
	\tikzstyle{skin}=[dotted, rounded corners = 3mm, arrows={-{Latex[length=1.5mm, round]}}]
	\tikzstyle{skot}=[dashed, rounded corners = 3mm, arrows={-{Latex[length=1.5mm, round]}}]
	
	\foreach \x / \y / \m in {xl13/xlB0/17, xl23/xlA0/13, xl33/xl90/9, xl43/xl80/5, xl53/xl70/1}
	{
		\node (bnode) [CONVconcat, fill=bc01] at (\y.center) {};
		\draw [skot] (\x.east) -- 
		($(\x.east) + (3mm+\m*1.3mm, -2mm)$) -- 
		($(\y.east) + (3mm+\m*1.3mm, 2mm)$) -- 
		(bnode.east);
	}
	
	\foreach \x / \y / \m in {xl10/xlB3/20, xl11/xlB2/19, xl12/xlB1/18,
		xl20/xlA3/16, xl21/xlA2/15, xl22/xlA1/14,
		xl30/xl93/12, xl31/xl92/11, xl32/xl91/10,
		xl40/xl83/8,  xl41/xl82/7,  xl42/xl81/6,
		xl50/xl73/4,  xl51/xl72/3,  xl52/xl71/2}
	{	\node (bnode) [CONVconcat, fill=bc03] at (\y.center) {};
		\draw [skin] (\x.east) -- 
		($(\x.east) + (3mm+\m*1.3mm, -2mm)$) -- 
		($(\y.east) + (3mm+\m*1.3mm, 2mm)$) -- 
		(bnode.east);
	}
	\end{pgfonlayer}
	\end{tikzpicture}
	\caption{
		Architecture of the proposed energy monitoring model. Dashed and dotted lines to the right represent \emph{outer} and \emph{inner} skip connections, respectively. Solid lines to the left represent residual connections \cite{He_2015_ResNet}. Skip connections use channel aggregation while residual connections use element-wise addition. Green-shaded layers are followed by a pooling step, while the red-yellow shaded ones are preceded by an un-pooling operation.
	}
	\label{fig:ae-2-architecture}
\end{figure}

\textbf{Dilated Convolutions} \texttt{CONV(d, k)}: The core operation of each layer is the cross-correlation defined as

{\small\begin{equation*}
	\texttt{CONV(d, k):}\; f(x) \stackrel{\text{def}}{=}\;b(n) \plus \sum_{k=\minus\lfloor\texttt{k}/2\rfloor}^{k=\lfloor\texttt{k}/2\rfloor}x(n+\texttt{d}\cdot k)\cdot \kappa(k)
	\end{equation*}}%
where \texttt{d} is the dilation rate \cite{Yu_2016}, \texttt{k} is the kernel size, $\underline{b}$ is the bias vector, and $\underline{\kappa}$ is the layer's kernel.

\textbf{Batch Normalization} \texttt{BN} \cite{Ioffe_2015_BatchNormalization}: is a composition of two  affine transformations applied to the output of each layer based on mini-batch statistics

{\small\begin{equation*}
	\texttt{BN:}\;f(x) \stackrel{\text{def}}{=}\; \gamma\, \hat{x} + \beta = \gamma\, \frac{x - \mu_\mathcal{B}}{\sigma_\mathcal{B}} + \beta
	\end{equation*}}%
where $x$ is the original output of a unit, $\mu_\mathcal{B}$ and $\sigma^2_\mathcal{B}$ are the sample mean and variance of all outputs of this neuron over the mini-batch $\mathcal{B}$, and $\gamma$ and $\beta$ are two learnable parameters.

\textbf{\underline{L}eaky \underline{Re}ctified \underline{L}inear \underline{U}nits} \texttt{LReLU} \cite{Maas_2013_RectifierNonLinearities}: is a non-linear activation function defined as

{\small\begin{equation*}
	\texttt{LReLU:}\; f(x) \mathrel{\stackon[5pt]{$=$}{$\scriptscriptstyle\alpha \,\leq\,1$}} \max(\alpha x, x)
	\end{equation*}}%


\textbf{Activation noise (noise injection)} \texttt{GN} \cite{Nair_2010_ReLU_RBM}: is a regularization technique applied during the training phase only and consists of injecting small additive Gaussian noise (with variance $\sigma^2$) to the output of the layer to avoid over-fitting
\begin{equation*}
\texttt{GN:}\; f(x) \stackrel{\text{def}}{=}\; x + z \sim \mathcal{N}(0, \sigma^2)
\end{equation*}

\textbf{Sigmoidal activations} \texttt{LogSg}: is a bounded activation function applied to the first hidden layer and the output layer of the model
\begin{equation}
\texttt{LogSg:}\; f(x) \stackrel{\text{def}}{=}\; (1+\exp(-x))^{-1}
\label{eq:logsg-activation}
\end{equation}

\textbf{Down- and up-sampling:} take place only across blocks where down-sampling is performed using MaxPooling while up-sampling is applied using forward-filling. 

\textbf{Parameter initialization and updates}: model parameters are initialized from a zero-mean uniform distribution \cite{Glorot_2010} and learned using a gradient-based stochastic  optimization \cite{Bottou_2012_StochasticGradientDescent} with an update rule based on the ADAM Algorithm \cite{Kingma_2014_ADAM} with Nesterov momentum \cite{Dozat_2015_NADAM}.

\section{Performance Measures}
\label{sec:performance-measures}

Early works on energy disaggregation tended to adopt the simple accuracy index in evaluating the performance of a disaggregation system \cite{Chang_2010, Belkin_2013, Makonin_2013_AMPds1}. Later works, however, realized the misleading interpretation of this measure (resulting from its bias towards the prevailing class) and proposed \emph{precision}, \emph{recall}, and \emph{f$_1$-score} as alternative measures for assessing the disaggregation performance \cite{Beckel_2014_ECO, Holmegaard_2016_IndustrialSettings, Kim_2010, Makonin_2015_NILMPerformanceEvaluation}. 

We, however, believe that these measures represent a one-sided rigorous solution to the biasness of the accuracy index. In fact, these metrics are fused to the assumption of scarce load usage and fail to provide valuable interpretation of performance if this assumption is violated. 

Given the raw-count contingency table

\hspace{0mm}
\begin{small}
	\setlength{\textfloatsep}{0.1cm}
	\begin{tabularx}{37.5mm}{ccccc}
		&                 &       \multicolumn{2}{l}{Predictions}        & \\[1mm]
		&                 & $\hat{\omega}^\plus$ & $\hat{\omega}^\minus$ & \\[1mm]
		\cmidrule{3-4}
		\multirow{2}{0mm}{\rotatebox[origin=c]{90}{Classes}} & $\omega^\plus$  & \texttt{TP} & \texttt{FN} & \texttt{RP} \\[2mm]
		& $\omega^\minus$ & \texttt{FP} & \texttt{TN} & \texttt{RN} \\
		\cmidrule{3-4}   
		&                 & \texttt{PP} & \texttt{PN} & \texttt{N}  \\
	\end{tabularx}
\end{small}
\begin{footnotesize}
	\begin{tabularx}{10mm}{>{\hspace{0mm}}l<{\hspace{-2mm}}l<{\hspace{0cm}}}
		\texttt{TP:} & True Positives\\
		\texttt{TN:} & True Negatives\\
		\texttt{FP:} & False Positive\\
		\texttt{FN:} & False Negatives\\
		\texttt{RP:} & Real Positives \\
		\texttt{RN:} & Real Negatives\\
		\texttt{PP:} & Positive Predictions\\
		\texttt{PN:} & Negative Predictions\\
		\texttt{N:}  & Num. of samples/events
	\end{tabularx}
\end{footnotesize}\\[1mm]
the aforementioned measure are defined as

{\small
	\begin{align}
	\text{accuracy}  &= (\texttt{TP + TN})\;/\;(\texttt{TP + TN + FP + FN}) 	\label{eq:acc-TN_TP}\\
	\text{precision} &= \texttt{TPA} = \;\texttt{TP}\,\;/\;(\texttt{TP + FP})   \nonumber \\
	\text{recall}    &= \texttt{TPR} = \;\texttt{TP}\,\;/\;(\texttt{TP + FN})   \nonumber \\
	\texttt{f$_1$-s} &= (\texttt{2 x TP})\;\;/\;(\texttt{2 x TP + FN + FP})  	\label{eq:f1s-TN_TP}
	\end{align}}%
In the case of scarce load usage, the probability of negative samples becomes relatively high and the accuracy index becomes a single-sided measure, namely the true negative rate. In this case, a trivial disaggregator (one that always predicts the prevailing class) ambiguously yields near optimal accuracy.

The information retrieval approach to alleviate this bias is to simply \emph{ignore} the prevailing term in Eq. \ref{eq:acc-TN_TP}, namely \texttt{TN}, which results in either the Jaccard index or the $f$-measure \texttt{f$_1$-s} Eq. \ref{eq:f1s-TN_TP}. We find this to be an extreme and ill-argued solution, especially in assessing energy monitoring performance. First, the scarce load usage is not always valid and is usually violated in commercial buildings or some residential loads such as refrigerators, air conditioners, space heaters, or electric vehicles. Additionally, the class of always-on loads suffers from the exact opposite situation where the class imbalance is due to the prevailing positive class and a trivial system in this case yields misleading near-optimal score for both the accuracy and the $f$-measure.

Second, when the scarce usage assumption is valid (e.g. for various miscellaneous appliances such as kettles, irons, vacuum cleaners ... etc), the extent of class imbalance varies widely amongst loads as well as users. These variations are not reflected by any means in either of the information retrieval measures. For these reasons, we claimed that precision and recall are inflexible measures since they are fused to a one-sided assumption regardless of the real distribution of classes.

Powers \cite{Powers_2011_EvaluationMetrics} introduced \emph{informedness} \texttt{B}, \emph{markedness} \texttt{M}, and their geometric mean \emph{Matthews Correlation Coefficient} \texttt{MCC} as alternative, unbiased evaluation measures
\begin{align}
\texttt{B} &= \texttt{TPR + TNR - 1} \\
\texttt{M} &= \texttt{TPA + TNA - 1} \\
\texttt{MCC} &= \sqrt{\texttt{B} \cdot \texttt{M}}
\end{align}
where \texttt{TNA} = \texttt{TN / (TN + FN)} is the inverse-precision and \texttt{TNR} = \texttt{TN / (TN + FP)} is the inverse recall. Similar to the information retrieval measures, these alternatives were proposed and adopted in similar application domains such as medical diagnostics \cite{Youden_1950,Metz_1978_ROCAnalysis} and recommender system evaluations \cite{Schroder_2011_SettingGoals}. We believe that the requirements of performance evaluation in these applications are more similar to those in energy disaggregation.


\begin{table}[!t]
	\begin{center}
		\caption{Performance comparison of 11 loads from the first building in UK-DALE \cite{Kelly_2015_UKDALE}. \texttt{rn} is the probability of the negative class in the evaluation fold and \texttt{\%-NM} is the \emph{percent-noisy measure} \cite{Makonin_2015_NILMPerformanceEvaluation}.}
		\label{tbl:model-capacity}
		\setlength{\tabcolsep}{5.65pt}
		\footnotesize
		\newcommand{\fscore}{\texttt{f}$_\texttt{1}$\texttt{-s}}
		\newcommand{\wrst}[1]{\texttt{#1}}
		\newcommand{\best}[1]{{\texttt{#1}}}
		\setlength{\tabcolsep}{4.3pt}
		%\begin{tabularx}{1.0\columnwidth}{c|cc|ccccc}
		%	\toprule                    
		%	& \texttt{rn}   & \texttt{\%-NM}  & \texttt{TPA}    & \texttt{B}    & \texttt{M}    & \fscore       & \texttt{MCC} \\
		%	\midrule 
		%	\texttt{FR}             & \wrst{0.55}   & \wrst{0.87}     & \wrst{0.916}    & \wrst{0.811}  & \wrst{0.818}  & \best{0.896}  & \best{0.815} \\
		%	\texttt{LC}             & \wrst{0.69}   & \wrst{0.92}     & \wrst{0.524}    & \wrst{0.394}  & \wrst{0.353}  & \best{0.589}  & \best{0.373} \\
		%	\texttt{DW}             & \wrst{0.98}   & \wrst{0.95}     & \wrst{0.851}    & \wrst{0.490}  & \wrst{0.838}  & \best{0.623}  & \best{0.641} \\
		%	\texttt{WM}             & \wrst{0.94}   & \wrst{0.91}     & \wrst{0.969}    & \wrst{0.994}  & \wrst{0.961}  & \best{0.979}  & \best{0.978} \\
		%	\texttt{SP}             & \wrst{0.78}   & \wrst{0.97}     & \wrst{0.460}    & \wrst{0.156}  & \wrst{0.267}  & \best{0.312}  & \best{0.204} \\
		%	\texttt{TV}             & \wrst{0.90}   & \wrst{0.97}     & \wrst{0.741}    & \wrst{0.666}  & \wrst{0.707}  & \best{0.716}  & \best{0.686} \\
		%	\texttt{BL}             & \wrst{0.91}   & \wrst{0.95}     & \wrst{0.341}    & \wrst{0.598}  & \wrst{0.311}  & \best{0.468}  & \best{0.431} \\
		%	\texttt{KT}             & \wrst{0.99}   & \wrst{0.95}     & \wrst{0.874}    & \wrst{0.866}  & \wrst{0.873}  & \best{0.870}  & \best{0.869} \\
		%	\texttt{MC}             & \wrst{0.99}   & \wrst{0.97}     & \wrst{0.624}    & \wrst{0.452}  & \wrst{0.619}  & \best{0.526}  & \best{0.529} \\
		%	\texttt{TS}             & \wrst{0.99}   & \wrst{0.98}     & \wrst{0.674}    & \wrst{0.721}  & \wrst{0.673}  & \best{0.698}  & \best{0.697} \\
		%	\texttt{KL}             & \wrst{0.87}   & \wrst{0.94}     & \wrst{0.460}    & \wrst{0.456}  & \wrst{0.390}  & \best{0.502}  & \best{0.422} \\
		%	\bottomrule
		%\end{tabularx}
		
		
		
		
		
		
		
		
		
		
		
		
		
		\begin{tabularx}{1.0\columnwidth}{c|cc|cccc|cc}
			\toprule                    
			& \texttt{rn}   & \texttt{\%-NM}  & \texttt{TPA}    & \texttt{TPR}  & \texttt{B}    & \texttt{M}    & \fscore       & \texttt{MCC} \\
			\midrule 
			\texttt{FR}             & \wrst{0.55}   & \wrst{0.87}     & \wrst{0.92}    & \wrst{0.88}  & \wrst{0.81}  & \wrst{0.82}  & \best{0.896}  & \best{0.815} \\
			\texttt{LC}             & \wrst{0.69}   & \wrst{0.92}     & \wrst{0.52}    & \wrst{0.67}  & \wrst{0.39}  & \wrst{0.35}  & \best{0.589}  & \best{0.373} \\
			\texttt{DW}             & \wrst{0.98}   & \wrst{0.95}     & \wrst{0.85}    & \wrst{0.49}  & \wrst{0.49}  & \wrst{0.84}  & \best{0.623}  & \best{0.641} \\
			\texttt{WM}             & \wrst{0.94}   & \wrst{0.91}     & \wrst{0.97}    & \wrst{0.99}  & \wrst{0.99}  & \wrst{0.96}  & \best{0.979}  & \best{0.978} \\
			\texttt{SP}             & \wrst{0.78}   & \wrst{0.97}     & \wrst{0.46}    & \wrst{0.24}  & \wrst{0.16}  & \wrst{0.27}  & \best{0.312}  & \best{0.204} \\
			\texttt{TV}             & \wrst{0.90}   & \wrst{0.97}     & \wrst{0.74}    & \wrst{0.69}  & \wrst{0.67}  & \wrst{0.71}  & \best{0.716}  & \best{0.686} \\
			\texttt{BL}             & \wrst{0.91}   & \wrst{0.95}     & \wrst{0.34}    & \wrst{0.75}  & \wrst{0.60}  & \wrst{0.31}  & \best{0.468}  & \best{0.431} \\
			\texttt{KT}             & \wrst{0.99}   & \wrst{0.95}     & \wrst{0.87}    & \wrst{0.87}  & \wrst{0.87}  & \wrst{0.87}  & \best{0.870}  & \best{0.869} \\
			\texttt{MC}             & \wrst{0.99}   & \wrst{0.97}     & \wrst{0.62}    & \wrst{0.46}  & \wrst{0.45}  & \wrst{0.62}  & \best{0.526}  & \best{0.529} \\
			\texttt{TS}             & \wrst{0.99}   & \wrst{0.98}     & \wrst{0.67}    & \wrst{0.72}  & \wrst{0.72}  & \wrst{0.67}  & \best{0.698}  & \best{0.697} \\
			\texttt{KL}             & \wrst{0.87}   & \wrst{0.94}     & \wrst{0.46}    & \wrst{0.55}  & \wrst{0.46}  & \wrst{0.39}  & \best{0.502}  & \best{0.422} \\
			\bottomrule
		\end{tabularx}
		
		
		
		%\footnotesize
		%\newcommand{\fscore}{\texttt{f}$_\texttt{1}$\texttt{-s}}
		%\newcommand{\wrst}[1]{\texttt{#1}}
		%\newcommand{\best}[1]{\textbf{\texttt{#1}}}
		%\begin{tabularx}{\textwidth}{c|cc|cccccc|cccccc}
		%   \toprule                    
		%   \multirow{2}{*}{Load}   & \multirow{2}{*}{\texttt{rn}} & \multirow{2}{*}{\texttt{\%-NM}} & \multicolumn{6}{c}{\texttt{AE.1}} & \multicolumn{6}{c}{\texttt{AE.2}} \\
		%   \cmidrule(lr){4-9}
		%   \cmidrule(lr){10-15}
		%                           &               &               & \texttt{TPA} & \texttt{TPR} & \fscore & \texttt{B} & \texttt{M} & \texttt{MCC} &
		%                                                             \texttt{TPA} & \texttt{TPR} & \fscore & \texttt{B} & \texttt{M} & \texttt{MCC}\\
		%   \midrule
		%   \texttt{FR}             & \wrst{0.55}   & \wrst{0.87}   & \wrst{0.916}  & \wrst{0.877}  & \wrst{0.896}  & \wrst{0.811}  & \wrst{0.818}  & \wrst{0.815}
		%                                                           & \best{0.953}  & \best{0.919}  & \best{0.936}  & \best{0.882}  & \best{0.888}  & \best{0.885} \\
		%                                                           
		%   \texttt{LC}             & \wrst{0.69}   & \wrst{0.92}   & \best{0.524}  & \wrst{0.672}  & \best{0.589}  & \best{0.394}  & \best{0.353}  & \best{0.373}
		%                                                           & \wrst{0.522}  & \best{0.676}  & \wrst{0.589}  & \wrst{0.394}  & \wrst{0.351}  & \wrst{0.372} \\
		%                                                           
		%   \texttt{DW}             & \wrst{0.98}   & \wrst{0.95}   & \best{0.851}  & \wrst{0.492}  & \wrst{0.623}  & \wrst{0.490}  & \best{0.838}  & \wrst{0.641}
		%                                                           & \wrst{0.804}  & \best{0.842}  & \best{0.823}  & \best{0.838}  & \wrst{0.800}  & \best{0.819} \\
		%                                                           
		%   \texttt{WM}             & \wrst{0.94}   & \wrst{0.91}   & \best{0.969}  & \best{0.997}  & \best{0.979}  & \best{0.994}  & \best{0.961}  & \best{0.978}
		%                                                           & \wrst{0.939}  & \wrst{0.984}  & \wrst{0.961}  & \wrst{0.980}  & \wrst{0.938}  & \wrst{0.959} \\
		%                                                           
		%   \texttt{SP}             & \wrst{0.78}   & \wrst{0.97}   & \best{0.460}  & \wrst{0.236}  & \wrst{0.312}  & \wrst{0.156}  & \best{0.267}  & \best{0.204}
		%                                                           & \wrst{0.346}  & \best{0.374}  & \best{0.360}  & \best{0.171}  & \wrst{0.162}  & \wrst{0.167} \\
		%                                                           
		%   \texttt{TV}             & \wrst{0.90}   & \wrst{0.97}   & \best{0.741}  & \best{0.693}  & \best{0.716}  & \best{0.666}  & \best{0.707}  & \best{0.686}
		%                                                           & \wrst{0.613}  & \wrst{0.627}  & \wrst{0.620}  & \wrst{0.582}  & \wrst{0.570}  & \wrst{0.576} \\
		%                                                           
		%   \texttt{BL}             & \wrst{0.91}   & \wrst{0.95}   & \best{0.341}  & \best{0.747}  & \best{0.468}  & \best{0.598}  & \best{0.311}  & \best{0.431}
		%                                                           & \wrst{0.204}  & \wrst{0.507}  & \wrst{0.291}  & \wrst{0.304}  & \wrst{0.144}  & \wrst{0.209} \\
		%                                                           
		%   \texttt{KT}             & \wrst{0.99}   & \wrst{0.95}   & \best{0.874}  & \best{0.866}  & \best{0.870}  & \best{0.866}  & \best{0.873}  & \best{0.869}
		%                                                           & \wrst{0.761}  & \wrst{0.841}  & \wrst{0.799}  & \wrst{0.839}  & \wrst{0.760}  & \wrst{0.799} \\
		%                                                           
		%   \texttt{MC}             & \wrst{0.99}   & \wrst{0.97}   & \wrst{0.624}  & \wrst{0.455}  & \wrst{0.526}  & \wrst{0.452}  & \wrst{0.619}  & \wrst{0.529}
		%                                                           & \best{0.855}  & \best{0.485}  & \best{0.619}  & \best{0.485}  & \best{0.851}  & \best{0.642} \\
		%                                                           
		%   \texttt{TS}             & \wrst{0.99}   & \wrst{0.98}   & \best{0.674}  & \wrst{0.723}  & \best{0.698}  & \wrst{0.721}  & \best{0.673}  & \best{0.697}
		%                                                           & \wrst{0.638}  & \best{0.731}  & \wrst{0.681}  & \best{0.729}  & \wrst{0.636}  & \wrst{0.681} \\
		%                                                           
		%   \texttt{KL}             & \wrst{0.87}   & \wrst{0.94}   & \wrst{0.460}  & \best{0.554}  & \wrst{0.502}  & \wrst{0.456}  & \wrst{0.390}  & \wrst{0.422}
		%                                                           & \best{0.484}  & \wrst{0.552}  & \best{0.516}  & \best{0.463}  & \best{0.414}  & \best{0.438} \\
		%                                                           
		%   \bottomrule
		%\end{tabularx}
	\end{center}
\end{table}


\begin{table}[!t]
	\begin{center}
		\caption{Performance comparison between the proposed model \texttt{AE} and the \emph{rectangles} architecture in \cite{Kelly_2015} \texttt{Regr} on same load instances (left) and unseen instances from new buildings (right). All values represent the $f$-measure.}
		\label{tbl:across-loads}
		\setlength\intextsep{0mm}
		\setlength{\tabcolsep}{10pt}
		\footnotesize
		\newcommand{\fscore}{\texttt{f}$_\texttt{1}$\texttt{-s}}
		\newcommand{\wrst}[1]{\texttt{#1}}
		\newcommand{\best}[1]{\textbf{\texttt{#1}}}
		\begin{tabularx}{0.94\columnwidth}{c|cc|cc}
			\toprule					
			\multirow{2}{*}{Load}	&				\multicolumn{2}{c}{Same instances} 					&		\multicolumn{2}{c}{Accross buildings} 				\\
			\cmidrule(lr){2-3}
			\cmidrule(lr){4-5}
			&	\texttt{Regr.} \cite{Kelly_2015}	&		\texttt{AE}		&		\texttt{Regr.} \cite{Kelly_2015}		& \texttt{AE}	\\
			\midrule
			\texttt{FR}				&		\wrst{0.810} 					&		\best{0.879}		&			\wrst{0.820} 					& \best{0.927}	\\
			\texttt{DW}				&		\wrst{0.720} 					&		\best{0.796}		&			\wrst{0.740} 					& \best{0.804}	\\
			\texttt{MC}				&		\wrst{0.620} 					&		\best{0.705}		&			\wrst{0.210} 					& \best{0.366}	\\
			\texttt{WM}				&		\wrst{0.490} 					&		\best{0.960}		&			\wrst{0.270} 					& \best{0.410}	\\
			\texttt{KT}				&		\wrst{0.710} 					&		\best{0.783}		&			\wrst{0.700} 					& \best{0.819}	\\
			\bottomrule
		\end{tabularx}
	\end{center}
\end{table}


\section{Experiments and Results}
\label{sec:experiments-and-results}

The developed model is evaluated on the freely available UK-DALE dataset \cite{Kelly_2015_UKDALE}, an energy dataset acquired from five residential buildings.
In this work, the 1 Hz real power measurements represent the input signals to disaggregate while the reference ones are the 1/6 Hz measurements up-sampled (using fill-forward) to 1 Hz.


Table \ref{tbl:model-capacity} shows the performance measures of the  proposed model evaluated on 11 loads from the first building in the adopted dataset with a 3-hour monitoring window for all load categories. Data folds are real power measurements from January and February of 2015 for training and validation, respectively, while the remaining 10 months of the 2015 represents the evaluation fold. While we provide these results as benchmarking ones, assessment of feasibility is observed in the following experiment.

In Table \ref{tbl:across-loads}, we compare the monitoring performance of our model \texttt{AE} with the previous work in \cite{Kelly_2015}, specifically the regression-based model \texttt{Regr.} (referred to as \emph{rectangles} architecture). We use the exact data folds adopted in \cite{Kelly_2015} for training and evaluation and define two test cases. The first trains and evaluates on the same load instances but future periods of operation while the second evaluates on new load instances (from new buildings). In both cases, the proposed model outperformed previous works in all load categories.


\section{Conclusion and future work}
\label{sec:conclusion}

In this paper, we assessed the feasibility of a generic deep disaggregation model for end-use load monitoring using a fully convolutional neural network evaluated on a variety of load categories.
The proposed model (with a fixed architecture and set of hyper-parameters) outperforms previous work and showed relatively acceptable performance across different loads.



%===============================================================================================================
%=========================================  References/Bibliography  ===========================================
{\small\vspace{-1mm}
\bibliographystyle{./IEEEtranKarim}
\bibliography{./IEEEabrv,./References}}

\end{document}



















