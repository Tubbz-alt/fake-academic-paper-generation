
%% bare_jrnl.tex
%% V1.4b
%% 2015/08/26
%% by Michael Shell
%% see http://www.michaelshell.org/
%% for current contact information.
%%
%% This is a skeleton file demonstrating the use of IEEEtran.cls
%% (requires IEEEtran.cls version 1.8b or later) with an IEEE
%% journal paper.
%%
%% Support sites:
%% http://www.michaelshell.org/tex/ieeetran/
%% http://www.ctan.org/pkg/ieeetran
%% and
%% http://www.ieee.org/

%%*************************************************************************
%% Legal Notice:
%% This code is offered as-is without any warranty either expressed or
%% implied; without even the implied warranty of MERCHANTABILITY or
%% FITNESS FOR A PARTICULAR PURPOSE!
%% User assumes all risk.
%% In no event shall the IEEE or any contributor to this code be liable for
%% any damages or losses, including, but not limited to, incidental,
%% consequential, or any other damages, resulting from the use or misuse
%% of any information contained here.
%%
%% All comments are the opinions of their respective authors and are not
%% necessarily endorsed by the IEEE.
%%
%% This work is distributed under the LaTeX Project Public License (LPPL)
%% ( http://www.latex-project.org/ ) version 1.3, and may be freely used,
%% distributed and modified. A copy of the LPPL, version 1.3, is included
%% in the base LaTeX documentation of all distributions of LaTeX released
%% 2003/12/01 or later.
%% Retain all contribution notices and credits.
%% ** Modified files should be clearly indicated as such, including  **
%% ** renaming them and changing author support contact information. **
%%*************************************************************************


% *** Authors should verify (and, if needed, correct) their LaTeX system  ***
% *** with the testflow diagnostic prior to trusting their LaTeX platform ***
% *** with production work. The IEEE's font choices and paper sizes can   ***
% *** trigger bugs that do not appear when using other class files.       ***                          ***
% The testflow support page is at:
% http://www.michaelshell.org/tex/testflow/



\documentclass[journal]{IEEEtran}
%
% If IEEEtran.cls has not been installed into the LaTeX system files,
% manually specify the path to it like:
% \documentclass[journal]{../sty/IEEEtran}





% Some very useful LaTeX packages include:
% (uncomment the ones you want to load)


% *** MISC UTILITY PACKAGES ***
%
%\usepackage{ifpdf}
% Heiko Oberdiek's ifpdf.sty is very useful if you need conditional
% compilation based on whether the output is pdf or dvi.
% usage:
% \ifpdf
%   % pdf code
% \else
%   % dvi code
% \fi
% The latest version of ifpdf.sty can be obtained from:
% http://www.ctan.org/pkg/ifpdf
% Also, note that IEEEtran.cls V1.7 and later provides a builtin
% \ifCLASSINFOpdf conditional that works the same way.
% When switching from latex to pdflatex and vice-versa, the compiler may
% have to be run twice to clear warning/error messages.






% *** CITATION PACKAGES ***
%
%\usepackage{cite}
% cite.sty was written by Donald Arseneau
% V1.6 and later of IEEEtran pre-defines the format of the cite.sty package
% \cite{} output to follow that of the IEEE. Loading the cite package will
% result in citation numbers being automatically sorted and properly
% "compressed/ranged". e.g., [1], [9], [2], [7], [5], [6] without using
% cite.sty will become [1], [2], [5]--[7], [9] using cite.sty. cite.sty's
% \cite will automatically add leading space, if needed. Use cite.sty's
% noadjust option (cite.sty V3.8 and later) if you want to turn this off
% such as if a citation ever needs to be enclosed in parenthesis.
% cite.sty is already installed on most LaTeX systems. Be sure and use
% version 5.0 (2009-03-20) and later if using hyperref.sty.
% The latest version can be obtained at:
% http://www.ctan.org/pkg/cite
% The documentation is contained in the cite.sty file itself.






% *** GRAPHICS RELATED PACKAGES ***
%
\ifCLASSINFOpdf
   \usepackage[pdftex]{graphicx}
  % declare the path(s) where your graphic files are
%   \graphicspath{{C:/Users/liuz156/Documents/My Shapes/}{network graph3/png/}}
  % and their extensions so you won't have to specify these with
  % every instance of \includegraphics
  % \DeclareGraphicsExtensions{.pdf,.jpeg,.png}
\else
  % or other class option (dvipsone, dvipdf, if not using dvips). graphicx
  % will default to the driver specified in the system graphics.cfg if no
  % driver is specified.
   \usepackage[dvips]{graphicx}
  % declare the path(s) where your graphic files are
   \graphicspath{{../eps/}}
  % and their extensions so you won't have to specify these with
  % every instance of \includegraphics
  % \DeclareGraphicsExtensions{.eps}
\fi
% graphicx was written by David Carlisle and Sebastian Rahtz. It is
% required if you want graphics, photos, etc. graphicx.sty is already
% installed on most LaTeX systems. The latest version and documentation
% can be obtained at:
% http://www.ctan.org/pkg/graphicx
% Another good source of documentation is "Using Imported Graphics in
% LaTeX2e" by Keith Reckdahl which can be found at:
% http://www.ctan.org/pkg/epslatex
%
% latex, and pdflatex in dvi mode, support graphics in encapsulated
% postscript (.eps) format. pdflatex in pdf mode supports graphics
% in .pdf, .jpeg, .png and .mps (metapost) formats. Users should ensure
% that all non-photo figures use a vector format (.eps, .pdf, .mps) and
% not a bitmapped formats (.jpeg, .png). The IEEE frowns on bitmapped formats
% which can result in "jaggedy"/blurry rendering of lines and letters as
% well as large increases in file sizes.
%
% You can find documentation about the pdfTeX application at:
% http://www.tug.org/applications/pdftex





% *** MATH PACKAGES ***
%
%\usepackage{amsmath}
% A popular package from the American Mathematical Society that provides
% many useful and powerful commands for dealing with mathematics.
%
% Note that the amsmath package sets \interdisplaylinepenalty to 10000
% thus preventing page breaks from occurring within multiline equations. Use:
%\interdisplaylinepenalty=2500
% after loading amsmath to restore such page breaks as IEEEtran.cls normally
% does. amsmath.sty is already installed on most LaTeX systems. The latest
% version and documentation can be obtained at:
% http://www.ctan.org/pkg/amsmath





% *** SPECIALIZED LIST PACKAGES ***
%
%\usepackage{algorithmic}
% algorithmic.sty was written by Peter Williams and Rogerio Brito.
% This package provides an algorithmic environment fo describing algorithms.
% You can use the algorithmic environment in-text or within a figure
% environment to provide for a floating algorithm. Do NOT use the algorithm
% floating environment provided by algorithm.sty (by the same authors) or
% algorithm2e.sty (by Christophe Fiorio) as the IEEE does not use dedicated
% algorithm float types and packages that provide these will not provide
% correct IEEE style captions. The latest version and documentation of
% algorithmic.sty can be obtained at:
% http://www.ctan.org/pkg/algorithms
% Also of interest may be the (relatively newer and more customizable)
% algorithmicx.sty package by Szasz Janos:
% http://www.ctan.org/pkg/algorithmicx




% *** ALIGNMENT PACKAGES ***
%
%\usepackage{array}
% Frank Mittelbach's and David Carlisle's array.sty patches and improves
% the standard LaTeX2e array and tabular environments to provide better
% appearance and additional user controls. As the default LaTeX2e table
% generation code is lacking to the point of almost being broken with
% respect to the quality of the end results, all users are strongly
% advised to use an enhanced (at the very least that provided by array.sty)
% set of table tools. array.sty is already installed on most systems. The
% latest version and documentation can be obtained at:
% http://www.ctan.org/pkg/array


% IEEEtran contains the IEEEeqnarray family of commands that can be used to
% generate multiline equations as well as matrices, tables, etc., of high
% quality.
\usepackage{soul}
\usepackage{threeparttable}
\usepackage{amsfonts}
\usepackage{graphicx}
\usepackage{subfigure}
\usepackage[justification=centering]{caption}
% *** SUBFIGURE PACKAGES ***
\ifCLASSOPTIONcompsoc
  \usepackage[caption=false,font=normalsize,labelfont=sf,textfont=sf]{subfig}
\else
  \usepackage[caption=false,font=footnotesize]{subfig}
\fi
% subfig.sty, written by Steven Douglas Cochran, is the modern replacement
% for subfigure.sty, the latter of which is no longer maintained and is
% incompatible with some LaTeX packages including fixltx2e. However,
% subfig.sty requires and automatically loads Axel Sommerfeldt's caption.sty
% which will override IEEEtran.cls' handling of captions and this will result
% in non-IEEE style figure/table captions. To prevent this problem, be sure
% and invoke subfig.sty's "caption=false" package option (available since
% subfig.sty version 1.3, 2005/06/28) as this is will preserve IEEEtran.cls
% handling of captions.
% Note that the Computer Society format requires a larger sans serif font
% than the serif footnote size font used in traditional IEEE formatting
% and thus the need to invoke different subfig.sty package options depending
% on whether compsoc mode has been enabled.
%
% The latest version and documentation of subfig.sty can be obtained at:
% http://www.ctan.org/pkg/subfig




% *** FLOAT PACKAGES ***
%
%\usepackage{fixltx2e}
% fixltx2e, the successor to the earlier fix2col.sty, was written by
% Frank Mittelbach and David Carlisle. This package corrects a few problems
% in the LaTeX2e kernel, the most notable of which is that in current
% LaTeX2e releases, the ordering of single and double column floats is not
% guaranteed to be preserved. Thus, an unpatched LaTeX2e can allow a
% single column figure to be placed prior to an earlier double column
% figure.
% Be aware that LaTeX2e kernels dated 2015 and later have fixltx2e.sty's
% corrections already built into the system in which case a warning will
% be issued if an attempt is made to load fixltx2e.sty as it is no longer
% needed.
% The latest version and documentation can be found at:
% http://www.ctan.org/pkg/fixltx2e


%\usepackage{stfloats}
% stfloats.sty was written by Sigitas Tolusis. This package gives LaTeX2e
% the ability to do double column floats at the bottom of the page as well
% as the top. (e.g., "\begin{figure*}[!b]" is not normally possible in
% LaTeX2e). It also provides a command:
%\fnbelowfloat
% to enable the placement of footnotes below bottom floats (the standard
% LaTeX2e kernel puts them above bottom floats). This is an invasive package
% which rewrites many portions of the LaTeX2e float routines. It may not work
% with other packages that modify the LaTeX2e float routines. The latest
% version and documentation can be obtained at:
% http://www.ctan.org/pkg/stfloats
% Do not use the stfloats baselinefloat ability as the IEEE does not allow
% \baselineskip to stretch. Authors submitting work to the IEEE should note
% that the IEEE rarely uses double column equations and that authors should try
% to avoid such use. Do not be tempted to use the cuted.sty or midfloat.sty
% packages (also by Sigitas Tolusis) as the IEEE does not format its papers in
% such ways.
% Do not attempt to use stfloats with fixltx2e as they are incompatible.
% Instead, use Morten Hogholm'a dblfloatfix which combines the features
% of both fixltx2e and stfloats:
%
% \usepackage{dblfloatfix}
% The latest version can be found at:
% http://www.ctan.org/pkg/dblfloatfix




%\ifCLASSOPTIONcaptionsoff
%  \usepackage[nomarkers]{endfloat}
% \let\MYoriglatexcaption\caption
% \renewcommand{\caption}[2][\relax]{\MYoriglatexcaption[#2]{#2}}
%\fi
% endfloat.sty was written by James Darrell McCauley, Jeff Goldberg and
% Axel Sommerfeldt. This package may be useful when used in conjunction with
% IEEEtran.cls'  captionsoff option. Some IEEE journals/societies require that
% submissions have lists of figures/tables at the end of the paper and that
% figures/tables without any captions are placed on a page by themselves at
% the end of the document. If needed, the draftcls IEEEtran class option or
% \CLASSINPUTbaselinestretch interface can be used to increase the line
% spacing as well. Be sure and use the nomarkers option of endfloat to
% prevent endfloat from "marking" where the figures would have been placed
% in the text. The two hack lines of code above are a slight modification of
% that suggested by in the endfloat docs (section 8.4.1) to ensure that
% the full captions always appear in the list of figures/tables - even if
% the user used the short optional argument of \caption[]{}.
% IEEE papers do not typically make use of \caption[]'s optional argument,
% so this should not be an issue. A similar trick can be used to disable
% captions of packages such as subfig.sty that lack options to turn off
% the subcaptions:
% For subfig.sty:
% \let\MYorigsubfloat\subfloat
% \renewcommand{\subfloat}[2][\relax]{\MYorigsubfloat[]{#2}}
% However, the above trick will not work if both optional arguments of
% the \subfloat command are used. Furthermore, there needs to be a
% description of each subfigure *somewhere* and endfloat does not add
% subfigure captions to its list of figures. Thus, the best approach is to
% avoid the use of subfigure captions (many IEEE journals avoid them anyway)
% and instead reference/explain all the subfigures within the main caption.
% The latest version of endfloat.sty and its documentation can obtained at:
% http://www.ctan.org/pkg/endfloat
%
% The IEEEtran \ifCLASSOPTIONcaptionsoff conditional can also be used
% later in the document, say, to conditionally put the References on a
% page by themselves.




% *** PDF, URL AND HYPERLINK PACKAGES ***
%
%\usepackage{url}
% url.sty was written by Donald Arseneau. It provides better support for
% handling and breaking URLs. url.sty is already installed on most LaTeX
% systems. The latest version and documentation can be obtained at:
% http://www.ctan.org/pkg/url
% Basically, \url{my_url_here}.




% *** Do not adjust lengths that control margins, column widths, etc. ***
% *** Do not use packages that alter fonts (such as pslatex).         ***
% There should be no need to do such things with IEEEtran.cls V1.6 and later.
% (Unless specifically asked to do so by the journal or conference you plan
% to submit to, of course. )


% correct bad hyphenation here
\hyphenation{op-tical net-works semi-conduc-tor}


\begin{document}
%
% paper title
% Titles are generally capitalized except for words such as a, an, and, as,
% at, but, by, for, in, nor, of, on, or, the, to and up, which are usually
% not capitalized unless they are the first or last word of the title.
% Linebreaks \\ can be used within to get better formatting as desired.
% Do not put math or special symbols in the title.
\title{Generic Model-Agnostic Convolutional Neural Network for Single Image Dehazing}
%
%
% author names and IEEE memberships
% note positions of commas and nonbreaking spaces ( ~ ) LaTeX will not break
% a structure at a ~ so this keeps an author's name from being broken across
% two lines.
% use \thanks{} to gain access to the first footnote area
% a separate \thanks must be used for each paragraph as LaTeX2e's \thanks
% was not built to handle multiple paragraphs
%

\author{Zheng Liu,
		Botao Xiao,
		Muhammad Alrabeiah,
		Keyan Wang,
        Jun Chen
        % <-this % stops a space
%\thanks{}% <-this % stops a space
%\thanks{}% <-this % stops a space
}

% note the % following the last \IEEEmembership and also \thanks -
% these prevent an unwanted space from occurring between the last author name
% and the end of the author line. i.e., if you had this:
%
\author{Zheng Liu, Botao Xiao, Muhammad Alrabeiah, Keyan Wang, Jun Chen\thanks{ Zheng Liu, Botao Xiao, Muhammad Alrabeiah, and Jun Chen are with the Department of Electrical and Computer Engineering, McMaster University, Hamilton, ON L8S 4K1, Canada (email: \{liuz156, xiaob6, alrabm, chenjun\}@mcmaster.ca).} \thanks{Keyan Wang is with the School of Tele-communication Engineering, Xidian University, Xian, 710071, China (email: kywang@mail.xidian.edu.cn).}}
% \author{....lastname \thanks{...} \thanks{...} }
%                     ^------------^------------^----Do not want these spaces!
%
% a space would be appended to the last name and could cause every name on that
% line to be shifted left slightly. This is one of those "LaTeX things". For
% instance, "\textbf{A} \textbf{B}" will typeset as "A B" not "AB". To get
% "AB" then you have to do: "\textbf{A}\textbf{B}"
% \thanks is no different in this regard, so shield the last } of each \thanks
% that ends a line with a % and do not let a space in before the next \thanks.
% Spaces after \IEEEmembership other than the last one are OK (and needed) as
% you are supposed to have spaces between the names. For what it is worth,
% this is a minor point as most people would not even notice if the said evil
% space somehow managed to creep in.



% The paper headers
\markboth{Journal of \LaTeX\ Class Files,~Vol.~14, No.~8, August~2015}%
{Shell \MakeLowercase{\textit{et al.}}: Bare Demo of IEEEtran.cls for IEEE Journals}
% The only time the second header will appear is for the odd numbered pages
% after the title page when using the twoside option.
%
% *** Note that you probably will NOT want to include the author's ***
% *** name in the headers of peer review papers.                   ***
% You can use \ifCLASSOPTIONpeerreview for conditional compilation here if
% you desire.




% If you want to put a publisher's ID mark on the page you can do it like
% this:
%\IEEEpubid{0000--0000/00\$00.00~\copyright~2015 IEEE}
% Remember, if you use this you must call \IEEEpubidadjcol in the second
% column for its text to clear the IEEEpubid mark.



% use for special paper notices
%\IEEEspecialpapernotice{(Invited Paper)}




% make the title area
\maketitle

% As a general rule, do not put math, special symbols or citations
% in the abstract or keywords.
\begin{abstract}
Haze and smog are among the most common environmental factors impacting image quality and, therefore, image analysis. This paper proposes an end-to-end generative method for image dehazing. It is based on designing a fully convolutional neural network to recognize haze structures in input images and restore clear, haze-free images. The proposed method is agnostic in the sense that it does not explore the atmosphere scattering model. Somewhat surprisingly, it achieves superior performance relative to all existing state-of-the-art methods for image dehazing even on SOTS outdoor images, which are synthesized using the atmosphere scattering model.
\end{abstract}

% Note that keywords are not normally used for peerreview papers.
\begin{IEEEkeywords}
Convolutional neural network, image dehazing, image restoration,  residual learning.
\end{IEEEkeywords}






% For peer review papers, you can put extra information on the cover
% page as needed:
% \ifCLASSOPTIONpeerreview
% \begin{center} \bfseries EDICS Category: 3-BBND \end{center}
% \fi
%
% For peerreview papers, this IEEEtran command inserts a page break and
% creates the second title. It will be ignored for other modes.
\IEEEpeerreviewmaketitle



\section{Introduction}

\IEEEPARstart{M}{any} modern applications rely on analyzing visual data to discover patterns and make decisions. Some examples could be found in intelligent surveillance, tracking, and control systems, where good quality images or frames are essential for accurate results and reliable performance. However, such systems could be significantly affected by environmentally induced distortions, the most common of which are haze and smog. Therefore, a lot of research in the computer vision community has been dedicated to addressing the problem of restoring good-quality images from their hazy counterparts, \cite{ColorAtten,DehazeNet,DarkChanPrior,berman2016non} to name a few. That problem is commonly referred to as the \textit{dehaze problem}.

The relation between the original and hazy images  \cite{narasimhan2002vision} is approximately captured by the following equation known as the atmosphere scattering model:
\begin{equation}\label{Physical}
I(x_{i})=J(x_{i})t(x)+A(1-t(x))\quad i=1,2,3 ,
\end{equation}
where for a pixel in the $i$th color channel and spatially indexed by $x$, $I(x_{i})$ is the intensity of the hazy pixel, $J(x_{i})$ is the actual intensity of that pixel, and $t(x)$ is the medium transmission function that depends on the scene depth and the atmospheric scattering coefficient $\beta$. Parameter $A$ in Equation (\ref{Physical}) is the atmosphere light intensity, which is assumed to be a global constant over the whole image. Since all variables in Equation (\ref{Physical}) are unknown except the hazy pixel intensity $I(x_{i})$, dehaze is in general an undetermined problem.

Over the past couple of decades, many methods have been proposed to solve the dehaze problem. Those methods could be loosely grouped into two categories: \textit{traditional} and \textit{Machine Learning (ML)-based} methods. The likes of \cite{DarkChanPrior}, \cite{ColorAtten}, and \cite{MarkovRandField} are some examples of the first category. They solve the underdetermined problem by exploiting some form of  prior information. 



On the other hand, works such as \cite{RandForstReg}, \cite{DehazeNet}, \cite{MultiScaleCNN}, and \cite{AllInOne} have followed a learning-based approach. They leverage the advances in classic and deep learning technologies to tackle the dehaze problem. Regardless how different those two categories may seem, they all aim to recover the original image by first estimating the unknown parameters $A$ and $t(x)$ and then inverting Equation (\ref{Physical}) to determine $J(x_{i})$:
\begin{equation}
J(x_{i})=\frac{I(x_{i})-A(1-t(x))}{t(x)}\quad i=1,2,3.
\end{equation}

\begin{figure}
	\subfigure{
		\includegraphics[width=4cm]{1}}
	\hspace{0in}
	\subfigure{
		\includegraphics[width=4cm]{1pred}}
	\captionsetup{justification=centering}
	\centering\caption{Dehazing result of synthetic image. Left: Hazy input. Right: Clear output.}
\end{figure}

From the  viewpoint of estimation theory, the methods in both categories fall under the umbrella of the plug-in principle\footnote{Consider a parametric model $\mathcal{P}=\{P_{\theta}:\theta\in\Theta\}$ and a mapping $\tau:\Theta\rightarrow\mathbb{R}$. Suppose the observation comes from $P_{\theta^*}$. The plug-in principle refers to the method of constructing an estimate of $\tau(\theta^*)$ by first deriving an estimate of $\theta^*$, denoted by $\hat{\theta}$, then plugging $\hat{\theta}$ into $\tau(\cdot)$.}, and they will all be referred to as plug-in methods. 
However, for the dehaze problem, the optimality of the plug-in principle is not completely justified. Indeed, it is unlikely that the problem of lossy reconstruction of the original image can be transformed equivalently to an estimation problem for  parameters $A$ and $t(x)$ (or their variants), at least when the two problems are subject to the same evaluation metric. Moreover, the actual relation between the original and hazy images  can be fairly complex and may not be fully captured by the atmosphere scattering model. Due to this potential mismatch, methods that rely on the atmosphere scattering model (including but not limited to plug-in methods) do not guarantee desirable generalization to natural images even if they can achieve good performance on synthetic images.

%due to model mismatch

%cannot guarantee a desirable generalization to natural even if they achieve 

%do not guarantee a desirable generalization to natural images. 

%more generic compared to GFN

%trained to recognize haze structures


%Their reliance on parameter estimation could be considered a downside, no matter how it is preformed; the ultimate goal of the dehaze problem is to restore the original image, and posing that problem as an estimation problem for parameters $A$ and $t(x)$ (or their variants) may not fully capture the ultimate goal.

Based on the aforementioned take on plug-in methods (and, more generally, model-dependent methods), this paper approaches the dehaze problem from a different, and more \textit{agnostic}, angle; it presents a dehaze neural network that solely focuses on producing a haze-free version of the input image. It utilizes the recent advances in deep learning to build an encoder-decoder network architecture that is  trained to directly restore the clear image, ignoring the parameter estimation problem altogether. The proposed method also has the potential of recognizing complex haze structures present in the training data but not captured by the atmosphere scattering model. To the best of our knowledge, such view of the dehaze problem has never been explored except in the recent work  \cite{GFN}, where a so-called Gated Fusion Network (GFN) is introduced for image dehazing. It will be seen that our proposed network has several advantages over GFN, especially in terms of architecture complexity and input-size flexibility; moreover, certain characteristics of GFN are specifically tailored to the dehaze problem whereas the architecture of our network is more generic and consequently more broadly applicable. 

The rest of the paper is organized into three sections. The following section, Section 2, presents a Generic Model-Agnostic convolutional neural Network (GMAN) for image dehazing together with a detailed explanation of  the network architecture and its building blocks. Section 3 will introduce the experimental results, showing the performance of the proposed GMAN. It also includes a description of the dataset and training procedure. Finally, Section 4 will wrap up the paper with some concluding remarks.

% The very first letter is a 2 line initial drop letter followed
% by the rest of the first word in caps.
%
% form to use if the first word consists of a single letter:
% \IEEEPARstart{A}{demo} file is ....
%
% form to use if you need the single drop letter followed by
% normal text (unknown if ever used by the IEEE):
% \IEEEPARstart{A}{}demo file is ....
%
% Some journals put the first two words in caps:
% \IEEEPARstart{T}{his demo} file is ....
%
% Here we have the typical use of a "T" for an initial drop letter
% and "HIS" in caps to complete the first word.

%\IEEEPARstart{T}{his} demo file is intended to serve as a ``starter file''
%for IEEE journal papers produced under \LaTeX\ using
%IEEEtran.cls version 1.8b and later.

% You must have at least 2 lines in the paragraph with the drop letter
% (should never be an issue)
%I wish you the best of success.

%\hfill mds
%
%\hfill August 26, 2015



\begin{figure*}
	\centering
	\includegraphics[width=0.8\linewidth]{dehaze_network}
	\captionsetup{justification=centering}
	\centering\caption{Structure and details of GMAN. The yellow blocks are convolutional layers, the green blocks are down-sampling layers and deconvolutional layers. We cascade 4 residual blocks shown as blue blocks, and the number of convolutional layers inside are 2, 2, 3, 4.}
	\label{network structure}
\end{figure*}

\section{Proposed Method}

Since the single image haze removal is an ill-posed problem, a deep neural network based on convolutional, residual, and deconvolutional blocks is devised and trained to take on a hazy image and restore its haze-free version. The network has an encoder-decoder structure as shown in Fig. \ref{network structure}. In the following subsections, the network architecture, its building blocks, and the training loss function are discussed in more detail.

\subsection{Network Architecture}

The proposed network is a fully Convolutional Neural Network (CNN). It is used to restore a clear image from a hazy input one. Functionally speaking, it is an end-to-end generative network that uses encoder-decoder structure with down- and up-sampling factor of 2. Its first two layers are constructed with 64-channel convolutional blocks. Following them are two-step down-sampling layers that encode the input image into a $56\times56\times 128$ volume. The encoded image is then fed to a residual layer built with 4 residual blocks, each containing a shortcut connection, see Fig. \ref{resblock}. This layer represents the transition from encoding to decoding, for it is followed by the deconvolutional layer that up-samples the residual layer output and reconstructs a new $224\times224\times64$ volume for another round of convolutions. The last two layers comprise convolutional blocks. They transform the up-sampled feature maps into an RGB image, which is finally added to the input image and thresholded with a ReLU to produce the haze-free version.

\subsection{Residual Learning}

%Main advantage of the middle block is that they converge faster than others (computational efficiency). Global block captures the boundary details of objects with different depths.

The network uses residual learning on two levels, local and global. In the middle layer and just right after down-sampling, the residual blocks are used to build the local residual layer. It takes advantage of the hypothesized and empirically proven \cite{kim2016accurate,zhang2017beyond,szegedy2017inception,ren2017faster} easy-to-train property of residual blocks (see \cite{ResNet}), and learns to recognize haze structures. Residual learning also appears in the overall architecture of the proposed GMAN. Specifically, the input image is fed along with the output of the final convolutional layer to a sum operator, creating one global residual block, see Fig. \ref{network structure}. The main advantage of this global residual block is that it helps the proposed network better capture the boundary details of objects with different depths in the scene.

% ResL blocks were first proposed by He \textit{et al.} \cite{ResNet} as part of their new neural network architecture. It has been hypothesized that they are relatively easier to train compared to other blocks. In the proposed network, their role is to recognize the haze structures. \ul{Although the image dehaze problem does not require extreme accuracy on pixel level, the texture detail in regions between high depth value and low depth value, like the boundary of tall buildings, are very important since the network should distinguish the difference and predict the haze structure \textbf{REPLACE WITH: the }. For this reason, the residual image between output and input is calculated} \textbf{REPLACE WITH: the features extracted by this ResL help capture those needed details}. \ul{This residual layer, in addition, has the advantage of being easier to optimize, which has been shown in}  \cite{kim2016accurate,zhang2017beyond,szegedy2017inception,ren2017faster}.
\begin{figure}
	\centering
 \includegraphics[width=0.252\textwidth,height=0.196\textheight]{residualblock}
	\captionsetup{justification=centering}
	\centering\caption{A residual block used in the middle layer of the proposed GMAN. In each block, the number of convolutional layers can be different. Relu is used as the activation function after the addition operator of every block.}
	\label{resblock}
\end{figure}

\subsection{Encoder-Decoder Architecture}

The architecture of the proposed GMAN follows the popular encoder-decoder architecture used in the deniosing problem. It is composed of three parts: encoder, hidden layers, and decoder. This architecture makes it possible to train a deep network and decrease the dimension of data. Since haze could be thought of as a form of noise, the encoder output is down-sampled and fed to the residual layer to extract important features. The network squeezes out the features of the original image and discards of the noise information. The decoder part is expected to learn and regenerate the missing data of the haze-free image, conforming the statistical distribution of the input information during the decoding period.

\begin{figure*}
	\includegraphics[width=0.13\linewidth]{1352}
	\includegraphics[width=0.13\linewidth]{DCP}
	\includegraphics[width=0.13\linewidth]{Dehazenet}
	\includegraphics[width=0.13\linewidth]{MSCNN}
	\includegraphics[width=0.13\linewidth]{AOD1}
	\includegraphics[width=0.13\linewidth]{GFN}
	\includegraphics[width=0.13\linewidth]{ours1}
	\centering
\end{figure*}
\begin{figure*}
	\subfigure[Hazy]{\includegraphics[width=0.13\linewidth]{canyon}}
	\subfigure[DCP]{\includegraphics[width=0.13\linewidth]{DCP2}}
	\subfigure[Dehazenet]{\includegraphics[width=0.13\linewidth]{Dehazenet2}}
	\subfigure[MSCNN]{\includegraphics[width=0.13\linewidth]{MSCNN2}}
	\subfigure[AOD-Net]{\includegraphics[width=0.13\linewidth]{AOD2}}
	\subfigure[GFN]{\includegraphics[width=0.13\linewidth]{canyon_GFN}}
	\subfigure[GMAN]{\includegraphics[width=0.13\linewidth]{ours2}}
	\captionsetup{justification=centering}
	\centering
	\caption{Comparison of different dehaze methods. First row has examples of synthetic hazy images. Second row has examples of natural hazy images.}
	\label{figComp}
\end{figure*}

\subsection{Loss Function: MSE and Perceptual Loss}

To train the proposed GMAN, a two-component loss function is defined. The first component measures the similarity between the output and the ground truth, and the second helps produce a visually meaningful image. The following three subsections provide more information on each component and the total loss:

\subsubsection{MSE Loss}

%\begin{equation}
%PSNR=10*\lg(\frac{MAX^2}{MSE})
%\end{equation}
Using PSNR to measure the difference between the output image and the ground truth is the most common way to show the effectiveness of an algorithm. Thus, MSE is chosen to be the first component of the loss function, namely $L_{MSE}$. The optimal value of PSNR could be reached by minimizing MSE at pixel level, which is expressed as:
\begin{equation}
L_{MSE}=\frac{1}{N}\sum_{x=1}^{N}\sum_{i=1}^{3}\parallel\hat{J}(x_{i})-J(x_{i})\parallel^2,
\end{equation}
where $\hat{J}(x_{i})$ is the output of the network, $J(x_{i})$ is the ground truth, $i$ is the channel index, and $N$ is the total number of pixels.

\subsubsection{Perceptual Loss}

In many classic image restoration problems, the quality of the output image is measured solely by the MSE loss. However, the MSE loss is not necessarily a good indicator of the visual effect. As Johnson \textit{et al.} demonstrate in \cite{johnson2016perceptual}, extracting high level features from specific layers of a pre-trained neural network can be of benefit to content reconstruction. The perceptual loss obtained from high-level features can describe the difference between two images more robustly than pixel-level losses.

Adding a perceptual loss component enables the decoder part of GMAN to acquire an improved ability to generate fine details of target images using features that have been extracted. In the present work, the network output and the ground truth are both fed to VGG16 \cite{VGG16}; following \cite{johnson2016perceptual}, we use the feature maps extracted from layers \textit{$conv1_1, conv2_2, conv3_3$} (which will be simply referred to as layers 1, 2, 3) of VGG16 to define the perceptual loss $L_{p}$ as follows:
\begin{equation}\label{perceptual loss}
L_{p}=\sum\limits_{j=1}^3\frac{1}{C_{j}H_{j}W_{j}}\parallel\phi_{j}(\hat{J})-\phi_{j}(J)\parallel^{2}_{2},
\end{equation}
where $\phi_j(\hat J)$ and $\phi_j(J)$ are the feature maps of  layer $j$ of VGG16 induced by the network output and the ground truth, respectively, and $C_j$, $H_j$, and $W_j$ are the dimensions of the feature volume of layer $j$ of VGG16.
 
% after feeding it the output hazy-free image and the ground truth image, respectively, 

%Under the belief that the perceptual similarity between the output and the ground truth could improve the performance, the haze-free output and the ground truth are both fed to a pre-trained classification CNN, e.g., VGG16 \cite{VGG16}. Using the feature maps extracted from layer $j$ for both images (layers \textit{$conv1_1, conv2_2, conv3_3$} of VGG16 are used in this work), a loss $L_{fr}$ is formulated to aid the reconstruction process. This loss is expressed as follows: :
%\begin{equation}\label{perceptual loss}
%L^{j}_{fr}(\hat{J},J)=\frac{1}{C_{j}H_{j}W_{j}}\parallel\phi_{j}(\hat{J})-\phi_{j}(J)\parallel^{2}_{2}
%\end{equation}


\subsubsection{Total Loss}
Combining both MSE and perceptual loss components results in the total loss of GMAN. In order to provide some sort of balance between the two components, the perceptual loss is pre-multiplied with $\lambda$, yielding the following expression:
\begin{equation}\label{loss}
L=L_{MSE}+\lambda_{}L_{p}.
\end{equation}

%Because of that the proposed architecture is an end-to-end single neural network, it is trained to minimize the $L$ jointly by training data. The $\lambda_{f}$ is 0.01 during the training period.

\section{Experimental Result}

This section first describes the training dataset and procedure. It then presents an evaluation of the performance of the proposed GMAN\footnote{The relevant codes can be found at https://github.com/Seanforfun/Deep-Learning/tree/master/DehazeNet.}, which is being benchmarked to some of the existing methods.

\subsection{Dataset}

According to the atmosphere scattering model, the transmission map $t(x)$ and atmosphere light intensity $A$ control the haze level of an image. Therefore, setting these two factors properly is important for building a dataset of hazy images. We use the OTS dataset from RESIDE \cite{li2017reside}, which is built using collected real-world outdoor scenes. The whole dataset contains 313,950 synthetic hazy images, generated from 8970 ground-truth images by varying the values of $A$ and $\beta$ (the depth information is estimated using \cite{liu2016learning}).
% IS IT NECESSARY?--> and for each channel of the images, $A$ is chosen uniformly from 0.6 to 1.0, $\beta$ is chosen from {0.4, 0.6, 0.8, 1.0, 1.2, 1.4, 1.6}.
  Thus, for each ground-truth image, there are 35 corresponding hazy images. We notice that the testing set of RESIDE, the SOTS, has 1000 ground-truth images, each with 35 synthetic hazy counterparts, that are all contained in the training data. This certainly can lead to some inaccuracies in testing results. Thus, the testing images were all removed from the training data (including their hazy counterparts), leading to a reduced-size training dataset of 278,950 hazy images (generated from 7970 ground-truth images).

\subsection{Training}

The proposed GMAN is trained end-to-end by minimizing the loss $L$ given by Equation (\ref{loss}). All layers in GMAN have 64 filters (kernels), except for the down-sampling ones which have 128 filters, with spatial size of $3\times3$. The network requires an input with size $224\times224$, so every image in the training dataset is randomly cropped in order to fit the input size\footnote{This restriction is only for the training phase. The trained network can be applied to images of arbitrary size}. The batch size is set to 35 to balance the training speed and the memory consumption on the GPU. For accelerated training, the Adam optimizer \cite{AdamOpt} is used with the following settings: the initial learning rate of 0.001, $\beta_1=0.9$, and $\beta_2=0.999$. The network and its training process have been implemented using TensorFlow software framework and carried out on an NVIDIA Titan Xp GPU. After 20 epochs of training, the loss function drops to a value of 0.0004, which is considered a good stopping point.

\subsection{Evaluation Results}

The proposed GMAN achieves superior performance relative to many state-of-the-art methods. According to Table \ref{t.1} \footnote{In Tables \ref{t.1} and \ref{t.2}, the performance results of other methods except GFN are quoted from \cite{li2017reside}.} below, it clearly outperforms all other competing methods under consideration on the SOTS outdoor dataset \cite{DarkChanPrior,DehazeNet,MultiScaleCNN,AllInOne}.  Moreover, as shown in Fig. \ref{figComp}, GMAN avoids darkening the image color  as well as the excessive sharpening of object edges. In contrast, it can be seen from Fig. \ref{figComp} that the DCP  method \cite{DarkChanPrior} dims the light intensity of the dehazed image, and causes color distortions in high-depth-value regions (e.g., sky); though MSCNN \cite{MultiScaleCNN} does well in these high-depth-value regions, its performance degrades in medium-depth areas of the target image. Hence, the proposed GMAN can overcome many of these issues and generate a better haze-free image.


We have also tested our network on the SOTS indoor dataset (see Table \ref{t.2}). In this case, the performance is not as impressive, and comes fourth after DehazeNet, GFN, and AOD-Net. Nevertheless, one can still see the great promise of the model-agnostic dehaze methods even on the indoor dataset. Indeed, also as a member of the family of model-agnostic networks, GFN is ranked second in terms of PSNR and ranked first (almost tied with the top-ranked DehazeNet) in terms of SSIM. Our preliminary results indicate that it is possible to design a more powerful model-agnostic network  that dominates all the existing ones (especially those based on the plug-in principle) on both SOTS outdoor and indoor datasets by integrating and generalizing the ideas underlying GMAN and GFN. This line of research will be reported in a followup work.

\begin{table}[!htbp]
	\centering
	\caption{Performance comparison on the SOTS outdoor dataset.}
	\begin{tabular}{ccccccc}
		\hline
		&DCP &DehazeNet &MSCNN &AOD-Net &GFN &GMAN\\ \hline
		PSNR &18.54 &26.84 &21.73 &24.08 &21.67 &28.19\\
		SSIM &0.7100 &0.8264 &0.8313 &0.8726 &0.8524 &0.9638\\ \hline
	\end{tabular}
	\label{t.1}
\end{table}

\begin{table}[!htbp]
	\centering
	\caption{Performance comparison on the SOTS indoor dataset.}
	\begin{tabular}{ccccccc}
%		{p{0.03\textwidth}p{0.03\textwidth}p{0.05\textwidth}p{0.05\textwidth}p{0.07\textwidth}p{0.03\textwidth}p{0.055\textwidth}}
	\hline
		&DCP &DehazeNet &MSCNN &AOD-Net &GFN &GMAN\\ \hline
		PSNR &18.87 &22.66 &20.01 &21.01 &22.44 &20.53\\
		SSIM &0.7935 &0.8325 &0.7907 &0.8372 &0.8844 &0.8081\\ \hline
	\end{tabular}
	\label{t.2}
\end{table}
%\subsubsection{Subsubsection Heading Here}
%Subsubsection text here.


% An example of a floating figure using the graphicx package.
% Note that \label must occur AFTER (or within) \caption.
% For figures, \caption should occur after the \includegraphics.
% Note that IEEEtran v1.7 and later has special internal code that
% is designed to preserve the operation of \label within \caption
% even when the captionsoff option is in effect. However, because
% of issues like this, it may be the safest practice to put all your
% \label just after \caption rather than within \caption{}.
%
% Reminder: the "draftcls" or "draftclsnofoot", not "draft", class
% option should be used if it is desired that the figures are to be
% displayed while in draft mode.
%
%\begin{figure}[!t]
%\centering
%\includegraphics[width=2.5in]{myfigure}
% where an .eps filename suffix will be assumed under latex,
% and a .pdf suffix will be assumed for pdflatex; or what has been declared
% via \DeclareGraphicsExtensions.
%\caption{Simulation results for the network.}
%\label{fig_sim}
%\end{figure}

% Note that the IEEE typically puts floats only at the top, even when this
% results in a large percentage of a column being occupied by floats.


% An example of a double column floating figure using two subfigures.
% (The subfig.sty package must be loaded for this to work.)
% The subfigure \label commands are set within each subfloat command,
% and the \label for the overall figure must come after \caption.
% \hfil is used as a separator to get equal spacing.
% Watch out that the combined width of all the subfigures on a
% line do not exceed the text width or a line break will occur.
%
%\begin{figure*}[!t]
%\centering
%\subfloat[Case I]{\includegraphics[width=2.5in]{box}%
%\label{fig_first_case}}
%\hfil
%\subfloat[Case II]{\includegraphics[width=2.5in]{box}%
%\label{fig_second_case}}
%\caption{Simulation results for the network.}
%\label{fig_sim}
%\end{figure*}
%
% Note that often IEEE papers with subfigures do not employ subfigure
% captions (using the optional argument to \subfloat[]), but instead will
% reference/describe all of them (a), (b), etc., within the main caption.
% Be aware that for subfig.sty to generate the (a), (b), etc., subfigure
% labels, the optional argument to \subfloat must be present. If a
% subcaption is not desired, just leave its contents blank,
% e.g., \subfloat[].


% An example of a floating table. Note that, for IEEE style tables, the
% \caption command should come BEFORE the table and, given that table
% captions serve much like titles, are usually capitalized except for words
% such as a, an, and, as, at, but, by, for, in, nor, of, on, or, the, to
% and up, which are usually not capitalized unless they are the first or
% last word of the caption. Table text will default to \footnotesize as
% the IEEE normally uses this smaller font for tables.
% The \label must come after \caption as always.
%
%\begin{table}[!t]
%% increase table row spacing, adjust to taste
%\renewcommand{\arraystretch}{1.3}
% if using array.sty, it might be a good idea to tweak the value of
% \extrarowheight as needed to properly center the text within the cells
%\caption{An Example of a Table}
%\label{table_example}
%\centering
%% Some packages, such as MDW tools, offer better commands for making tables
%% than the plain LaTeX2e tabular which is used here.
%\begin{tabular}{|c||c|}
%\hline
%One & Two\\
%\hline
%Three & Four\\
%\hline
%\end{tabular}
%\end{table}


% Note that the IEEE does not put floats in the very first column
% - or typically anywhere on the first page for that matter. Also,
% in-text middle ("here") positioning is typically not used, but it
% is allowed and encouraged for Computer Society conferences (but
% not Computer Society journals). Most IEEE journals/conferences use
% top floats exclusively.
% Note that, LaTeX2e, unlike IEEE journals/conferences, places
% footnotes above bottom floats. This can be corrected via the
% \fnbelowfloat command of the stfloats package.




\section{Conclusion}
The proposed GMAN in this paper explores a new direction of solving the dehaze problem. With its encoder-decoder fully convolutional architecture, GMAN learns to capture haze structures in images and restore the clear ones without referring to the atmosphere scattering model. It also avoids the deemed-unnecessary estimation of parameters $A$ and $t(x)$. Experimental results have verified the potential of GMAN in generating haze-free images and shown that it is capable of overcoming some of the common pitfalls of state-of-the-art  methods, like color darkening and excessive edge sharpening. Moreover, due to the generic architecture of GMAN, it could lay the groundwork for further research on general-purposed image restoration. Indeed, we expect that through training and some design tweaks, our network could be generalized to capture various types of image noise and distortions. In this sense, the present work not only suggests an improved solution to the dehaze problem, but also represents a progressive move towards developing a universal image restoration method.





% if have a single appendix:
%\appendix[Proof of the Zonklar Equations]
% or
%\appendix  % for no appendix heading
% do not use \section anymore after \appendix, only \section*
% is possibly needed

% use appendices with more than one appendix
% then use \section to start each appendix
% you must declare a \section before using any
% \subsection or using \label (\appendices by itself
% starts a section numbered zero.)
%


\appendices
%\section{Proof of the First Zonklar Equation}
%Appendix one text goes here.

% you can choose not to have a title for an appendix
% if you want by leaving the argument blank
%\section{}
%Appendix two text goes here.


% use section* for acknowledgment
%\section*{Acknowledgment}
%
%
%The authors would like to thank...


% Can use something like this to put references on a page
% by themselves when using endfloat and the captionsoff option.
\ifCLASSOPTIONcaptionsoff
  \newpage
\fi



% trigger a \newpage just before the given reference
% number - used to balance the columns on the last page
% adjust value as needed - may need to be readjusted if
% the document is modified later
%\IEEEtriggeratref{8}
% The "triggered" command can be changed if desired:
%\IEEEtriggercmd{\enlargethispage{-5in}}

% references section

% can use a bibliography generated by BibTeX as a .bbl file
% BibTeX documentation can be easily obtained at:
% http://mirror.ctan.org/biblio/bibtex/contrib/doc/
% The IEEEtran BibTeX style support page is at:
% http://www.michaelshell.org/tex/ieeetran/bibtex/
\bibliographystyle{IEEEtran}
% argument is your BibTeX string definitions and bibliography database(s)
%\bibliography{IEEEabrv,../bib/paper}
%
% <OR> manually copy in the resultant .bbl file
% set second argument of \begin to the number of references
% (used to reserve space for the reference number labels box)
%\begin{thebibliography}

\bibliography{IEEEabrv,citation}

%\end{thebibliography}

% biography section
%
% If you have an EPS/PDF photo (graphicx package needed) extra braces are
% needed around the contents of the optional argument to biography to prevent
% the LaTeX parser from getting confused when it sees the complicated
% \includegraphics command within an optional argument. (You could create
% your own custom macro containing the \includegraphics command to make things
% simpler here.)
%\begin{IEEEbiography}[{\includegraphics[width=1in,height=1.25in,clip,keepaspectratio]{mshell}}]{Michael Shell}
% or if you just want to reserve a space for a photo:

%\begin{IEEEbiography}{Michael Shell}
%Biography text here.
%\end{IEEEbiography}

% if you will not have a photo at all:
%\begin{IEEEbiographynophoto}{John Doe}
%Biography text here.
%\end{IEEEbiographynophoto}

% insert where needed to balance the two columns on the last page with
% biographies
%\newpage

%\begin{IEEEbiographynophoto}{Jane Doe}
%Biography text here.
%\end{IEEEbiographynophoto}

% You can push biographies down or up by placing
% a \vfill before or after them. The appropriate
% use of \vfill depends on what kind of text is
% on the last page and whether or not the columns
% are being equalized.

%\vfill

% Can be used to pull up biographies so that the bottom of the last one
% is flush with the other column.
%\enlargethispage{-5in}



% that's all folks
\end{document}


