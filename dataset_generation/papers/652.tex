% \documentclass[anonymous]{sig-alternate-10pt}
\documentclass[10pt,acmtog]{acmart}

\usepackage{multicol,graphicx,amssymb,mathrsfs,amsmath,pifont,amscd,latexsym,color,fancyhdr,CJK,url,longtable, pifont,subfig,epstopdf,setspace,multirow,colortbl,tabularx,threeparttable, booktabs, verbatim, bbm,listings, balance,color,indentfirst,xspace,hyperref,anyfontsize,mathtools}

\DeclarePairedDelimiter{\ceil}{\lceil}{\rceil}
\DeclarePairedDelimiter\floor{\lfloor}{\rfloor}

\usepackage[linesnumbered,boxed,algosection]{algorithm2e}
\SetKwComment{tcp}{$\triangleright$ }{}
\newcommand\mycommfont[1]{\footnotesize\ttfamily\textcolor{blue}{#1}}
\SetCommentSty{mycommfont}

% for force table width
\usepackage{array}
\newcolumntype{L}[1]{>{\raggedright\let\newline\\\arraybackslash\hspace{0pt}}m{#1}}
\newcolumntype{C}[1]{>{\centering\let\newline\\\arraybackslash\hspace{0pt}}m{#1}}
\newcolumntype{R}[1]{>{\raggedleft\let\newline\\\arraybackslash\hspace{0pt}}m{#1}}
\newcommand{\argmin}[1]{\underset{#1}{\operatorname{arg}\,\operatorname{min}}\;}

\newcommand{\framework}{DeepCache\xspace}
\newcommand{\sys}{\framework{}}
\newcommand{\shrink}{\emph{cache erosion}\xspace}
\newcommand{\variation}{scene variations\xspace}

\newcommand{\yunxin}[1]{{\color{blue}{#1}}}
\newcommand{\mengwei}[1]{{\color{red}{#1}}}
\newcommand{\felix}[1]{{\color{orange}{#1}}}
\newcommand{\xuanzhe}[1]{{\color{purple}{#1}}}
\newcommand{\todo}[1]{{\footnotesize \textcolor{red}{$\ll$\textsf{TODO #1}$\gg$}}}
\newcommand{\revise}[1]{{#1}}

% \copyrightyear{2018} 
% \acmYear{2018} 
\setcopyright{none}
% \acmConference[]{}{}
% \acmBooktitle{}
% \acmPrice{}
\acmDOI{10.1145/3241539.3241563}
\acmISBN{none}

\settopmatter{printacmref=false}
\fancyhead{}
\fancyfoot{}

% !TeX root = MobiCom2018-COMO.tex\usepackage{color,soul}%\usepackage{ulem}  % xzl: for strikeout, which seems more robust than \st{} provided by soul package. \newcommand\xzl[1]{\sethlcolor{yellow} \hl{xzl:#1}}% highlighted notes of other colors.\newcommand\note[1]{\sethlcolor{yellow} \hl{xzl:#1}}% highlighted notes of other colors.% \usepackage[color]{showkeys}	 % xzl: comment this out in final draft\definecolor{refkey}{rgb}{0,0,1}\definecolor{labelkey}{rgb}{0,1,0}% --- terms --- %\newcommand{\Cc}{Cascading container}\newcommand{\engine}{deep learning engine}\renewcommand{\paragraph}[1]{\vskip 3pt\noindent\textbf{#1 }}% used to be 6pt

\begin{document}
\title{\framework: Principled Cache for Mobile Deep Vision}

%\author{Mengwei Xu$^{1}$, Mengze Zhu$^{1}$, Yunxin Liu$^{2}$, Felix Xiaozhu Lin$^{3}$, Xuanzhe Liu$^{1}$\\
%$^{1}$ Key Lab of High-Confidence Software Technologies (Peking University), MoE\\
%$^{2}$ Microsoft Research  $^{3}$ Purdue University\\
%Email: \{xumengwei, zhumengze\}@pku.edu.cn, yunxin.liu@microsoft.com, xzl@purdue.edu, xzl@pku.edu.cn
%}

 \author{Mengwei Xu}
 \affiliation{%
   \institution{Peking University, MoE, Beijing, China}
   %\city{Beijing}
   }
   \email{xumengwei@pku.edu.cn}

  \author{Mengze Zhu}
 \affiliation{%
   \institution{Peking University, MoE, Beijing, China}
   %\city{Beijing}
   }
   \email{zhumz@pku.edu.cn}

% \author{Mengze Zhu}
% \affiliation{%
%   \institution{Peking University}
%   \city{Beijing}}
%\email{zhumengze@pku.edu.cn}


 \author{Yunxin Liu}
 \affiliation{%
   \institution{Microsoft Research}
   \city{Beijing, China}}
\email{yunxin.liu@microsoft.com}

 \author{Felix Xiaozhu Lin}
 \affiliation{%
%   \institution{College of Engineering, Purdue University}
   \institution{Purdue ECE}
   \city{West Lafayette, Indiana, USA}}
   \email{xzl@purdue.edu}
   
    \author{Xuanzhe Liu}\thanks{Xuanzhe Liu is the paper's corresponding author.}
 \affiliation{%
   \institution{Peking University, MoE, Beijing, China}}
\email{xzl@pku.edu.cn}

% The default list of authors is too long for headers.
% \renewcommand{\shortauthors}{Mengwei Xu et al.}

%
% The code below should be generated by the tool at
% http://dl.acm.org/ccs.cfm
% Please copy and paste the code instead of the example below. 
%
\begin{CCSXML}
<ccs2012>
<concept>
<concept_id>10003120.10003138</concept_id>
<concept_desc>Human-centered computing~Ubiquitous and mobile computing</concept_desc>
<concept_significance>500</concept_significance>
</concept>
<concept>
<concept_id>10010147.10010178.10010224.10010225</concept_id>
<concept_desc>Computing methodologies~Computer vision tasks</concept_desc>
<concept_significance>300</concept_significance>
</concept>
</ccs2012>
\end{CCSXML}

\ccsdesc[500]{Human-centered computing~Ubiquitous and mobile computing}
\ccsdesc[300]{Computing methodologies~Computer vision tasks}

\begin{abstract}
% !TeX root = ../MobiCom2018-COMO.tex\begin{comment}
Deep learning (DL) algorithm has revolutionized computer vision in recent years.
However, deploying deep learning on continuous mobile vision tasks is quite challenging due to its high computation requirement and constrained hardware resources on mobile devices, such as smartphones and head-mount devices.
In this paper, we present \framework, a transparent caching mechanism that can substantially accelerate CNN inference over continuous video streams without any efforts from app developers.
\framework is composed of two novel techniques: an image matching algorithm that judiciously and quickly identifies similar image regions between two consecutively captured images, and a cache-aware CNN inference engine that propagates the reusable regions through varied layers and reuses the computation results at layer granularity.
We implement a prototype of \framework to be transparent to application developers and able to run off-the-shelf CNN models that are trained in the traditional fashion.
Our evaluation on commodity Android devices and typical CNN models demonstrates that \framework can accelerate the execution of CNN models by \textbf{18.2\%} on average and up to \textbf{47.1\%} under various scenarios, with less than \textbf{3\%} accuracy loss.
\end{comment}%Executing deep learning models for continuous mobile vision benefits from video temporal locality by caching inference results. %Unlike traditional caches, the cache for mobile deep vision must operate under video scene variation, while trading off among cacheability, overhead, and loss in model accuracy.

We present DeepCache, a principled cache design for deep learning inference in continuous mobile vision.
%DeepCache benefits inference efficiency by exploiting temporal locality in video streams. 
DeepCache benefits model execution efficiency by exploiting temporal locality in input video streams. 
It addresses a key challenge raised by mobile vision: 
the cache must operate under video scene variation, 
while trading off among cacheability, overhead, and loss in model accuracy. 
At the input of a model, DeepCache discovers video temporal locality by exploiting the video's internal structure, for which it borrows proven heuristics from video compression; 
into the model, DeepCache propagates regions of reusable results by exploiting the model's internal structure. 
Notably, DeepCache eschews applying video heuristics to model internals which are not pixels but high-dimensional, difficult-to-interpret data. % has no good %We address these challenges in a \textit{principled} cache design called \textit{DeepCache}. %At model input, DeepCache discovers video temporal locality by exploiting the video's internal structure, for which it borrows proven heuristics from video compression; %into the model, DeepCache propagates reusable result regions by exploiting the model's internal structure. %Notably, DeepCache refrains from applying video heuristics to model internals which are not pixels but high-dimensional, difficult-to-interpret data. % has no good heuristics.%% \mengwei{Mengwei: this sentence makes sense to me but may confuse reviewers as I'm concerned.}\revise{
%We have implemented a prototype of DeepCache, which 
Our implementation of DeepCache works with unmodified deep learning models, requires zero developer's manual effort, and is therefore immediately deployable on off-the-shelf  mobile devices.
Our experiments show that DeepCache saves inference execution time by 18\% on average and up to 47\%. 
DeepCache reduces system energy consumption by 20\% on average.}%By building and evaluating DeepCache with , %from cache that exploits temporal %Due to prevalent temporal locality in mobile videos, 
\end{abstract}


\keywords{Deep Learning; Mobile Vision; Cache}

\maketitle

% !TeX root = ../MobiCom2018-COMO.tex\section{Introduction}\label{sec:intro}
With ubiquitous cameras on mobile and wearable devices, \emph{continuous mobile vision} emerges to 
%enrich human lives via ``showing computers what you see''.%Such continuous vision 
enable a variety of compelling applications, including cognitive assistance~\cite{Chen:2017:ESL:3132211.3134458}, life style monitoring~\cite{computers6010004}, and street navigation~\cite{conf/sensys/ChenRDBB15}.
%%Deep learning (DL) 
To support continuous mobile vision, Convolutional Neural Network (CNN) is recognized as the state-of-the-art algorithm: 
a software runtime, called deep learning engine, ingests a continuous stream of video images\footnote{We refer to them as a \textit{mobile video stream} in the remainder of the paper.};
for each input frame the engine executes a CNN model as a cascade of \textit{layers}, produces intermediate results called \textit{feature maps}, and outputs inference results. 
Such CNN executions are known for their high time and space complexity, stressing resource-constrained mobile devices. 
%be hungry of resources, e.g., CPU and memory. %This is exacerbated by the large amount of video frames that a mobile device keeps capturing. 
Although CNN execution can be offloaded to the cloud~\cite{conf/mobisys/HanSPAWK16,siri}, it becomes increasingly compelling to execute CNNs on device~\cite{conf/mobisys/MathurLBBFK17,conf/ipsn/LaneBGFJQK16,conf/sensys/ChenRDBB15}, which ensures fast inference, preserves user privacy, and remains unaffected by poor Internet connectivity.

% !TeX root = main.tex\begin{figure}[t!]
\centering
%\includegraphics[width=0.45\textwidth{}]{pics/overview-drawing.jpg} 
\includegraphics[width=0.45\textwidth{}]{pics/overview} 
%\missingfigure[figwidth=6cm]{Kernel support}
%\vspace{-10pt}		% use as needed
\caption{The overview of \framework{}.}
\label{fig:overview}
%\vspace{-20pt}		% use as needed
\end{figure}

To afford costly CNN on resource-constrained mobile/wearable devices, we set to exploit a mobile video stream's \textit{temporal locality}, 
i.e., rich information redundancy among consecutive video frames~\cite{conf/mobisys/MathurLBBFK17,conf/sensys/ChenRDBB15,conf/mobisys/LocLB17}.
%which are often captured when the devices are in slow motion or held still~\cite{conf/mobisys/MathurLBBFK17,conf/sensys/ChenRDBB15}. % which are often captured when the device's environments (e.g. user posture, illumination, etc.) undergo slow changes 
Accordingly, a \engine{} can \textit{cache} results when it executes CNN over a mobile video, by using input frame contents as cache keys and inference results as cache values. 
%from processing early frames for reuse in processing later frames, 
Such caching is expected to reduce the engine's resource demand significantly. 

Towards effective caching and result reusing, we face two major challenges.
\textit{1) Reusable results lookup:}%In general, a cache is a key-value; it relies on key equivalence for allows reuse of the value %In general, cache stores key-value pairs. 
Classic caches, e.g., the web browser cache, look up cached values (e.g., web pages) based on key \textit{equivalence} (e.g., identical URLs).
This does not apply to a CNN cache: 
its keys, i.e., mobile video contents, often undergo moderate scene variation over time.
The variation is caused by environmental changes such as user/camera motion, object appearance, and illumination changes~\cite{UCF101}.
%\mengwei{mengwei: \textit{\variation}}\footnote{We refer to them as the content variation of sequentially captured images for the inevitable environmental changes such as camera motion, object appearance, etc.} over time~\cite{UCF101}.
A CNN cache must systematically tolerate the variations and evaluate key \textit{similarity}. 
In doing so, the engine must trade off among cacheability, overhead, and model accuracy. 
%the cache design must identify reusable results based on the \textit{simliarity} of visual contents, %\textit{2) Result reuse at all layers:}\textit{2) Fine-grained reuse within a CNN:}
In a CNN model, expensive computations spread across multiple layers.
Besides caching the CNN's final inference outputs, 
the engine should cache the intermediate results (i.e., feature maps) produced by the internal layers. 
Furthermore, the engine should reuse the cached feature maps at fine spatial granularity. % whenever possible. %, whole or in part.
However, feature maps are high-volume, high-dimensional, barely interpretable data. 
It can be both expensive to inspect them and difficult to assess their similarity.

%This, however, may incur substantial overhead in inspecting feature maps and semantic difficulty in interpreting them. %However, features maps, unlike input images, are high dimensional data in large volume: inspecting them %discoverying reusable feature maps while tolerating input variation is difficult: as not only difficult but may also incur substantial overhead in inspecting. %1) to make the cache and its lookup algorithms not only fast but also robust against moderate variations among consecutive images in real videos; %2) to trade off between the cache's build/lookup cost and the cache's performance return. %\xzl{mention this}%\textit{cache erosion}: cacheability tends to diminish at deeper CNN layers. %Few deep-learning engines have exploited this opportunity systematically.
Few deep learning engines address the two challenges simultaneously.
Commodity engines~\cite{TensorFlow,caffe2,ncnn} process video frames in independent inference tasks with no reuse in between. 
A few recent research prototypes~\cite{conf/mobisys/LocLB17,cavigelli2017cbinfer} incorporate ad-hoc cache designs:
they either look up reusable results based on pixel-wise equivalence of image regions, or perform expensive cache lookup over feature maps at all layers inside a CNN. % at each layer of a CNN.
As a result, they often suffer from low cacheability and high lookup overhead, leaving much caching benefit untapped.

To this end, we advocate a principled cache design called \framework. 
% a pricipled cache for CNN execution over mobile video streams. 
The key ideas of \sys{}, as shown in Figure~\ref{fig:overview}, are that i) it discovers reusable image regions by exploiting \textit{the input video's internal structure}, for which it borrows the wisdom from decades of video research~\cite{tham1998novel,zhu1997new,barjatya2004block}; 
ii) it propagates the discovered reusable regions within a CNN by exploiting \textit{the CNN's internal structure}.

%Following the idea, 
As shown in Figure~\ref{fig:overview},
\sys{} stores recent input frames as cache keys and stores recent feature maps for individual CNN layers as cache values. 
To manage the cache, it provides two core mechanisms.
\begin{itemize}
\item 
%To discover reusable results in cache, 
At the engine input, \sys{} performs cache key lookup: it partitions each video frame into fine-grained regions and searches for similar regions in (cached) recent input frames. 
%(10$\times$10) regions. % (10 pixels on each dimension). 
%For each of the resultant regions, it runs a region matcher, searching for similar image regions in recent frames. 
It does so by running its region matcher. 
%For the resultant regions, it runs a region matcher to identify similar regions in recent frames.
Inspired by video compression~\cite{zhu1997new}, the matcher searches neighboring regions in specific patterns guided by video motion heuristics.
\sys{} keeps merging adjacent discovered regions in order to tackle \textit{cache erosion}, i.e., diminishing reusability at deeper layers. 
%guided by a set of video-specific heuristics. 
%To tackle cache erosion, \sys{} keeps merging adjacent discovered regions. 

In contrast to ad-hoc image comparison used by prior CNN caches~\cite{conf/mobisys/LocLB17,cavigelli2017cbinfer}, our matcher is more robust to the aforementioned \variation; 
the matcher runs fast to process more than 1,000 227$\times$227 frames per second.


%The matcher is also tailored to handle the observation of \textit{cache erosion}: cacheability tends to diminish at deeper CNN layers. 

\item 
Into the CNN execution, \sys{} maps the matched regions on input images to \textit{reusable} regions on feature maps.
It propagates the reusable regions across the feature maps of all CNN layers. 
At each layer, \sys{} transforms the reusable region boundaries based on the operators of this layer; 
it fills the reusable regions with cached feature map values in lieu of actual CNN execution. 
During the process, \sys{} weaves cache queries into CNN computations, keeping the cache queries transparent to CNN models. 

% To maximize cache benefit at low cost, \framework only executes its region matcher once per input image as the very beginning, but propagates this information among different layers during inference.
% \framework{} then \textit{propagates} the discovered reused regions (returned from cache) from the first layer towards the deeper layers in the CNN. 
% In the propagation, \framework{} tracks the changing boundaries of cached regions and transparently weaves the cache queries into the computation of each CNN layer.
\end{itemize}

With these two mechanisms, \sys{} runs its region matcher only \textit{once} per video frame at the input; it then loads cached feature maps at \textit{all} layers inside CNN. 
This contrasts to ad-hoc approaches that repeat matching processes over both images and feature maps, in and out of CNN.
%on feature maps of \textit{all layers}~\cite{conf/mobisys/LocLB17,cavigelli2017cbinfer}. 
Our rationale is that, while humans have reliable heuristics on similarity of image contents (which allows \sys{} to assess cache key similarity), they still lack knowledge on evaluating similarity of CNN's internal feature maps that are in disparate dimensions. 
By always treating feature maps as cache values not keys, \sys{} eschews high-cost, low-return searches over them, while still harvesting substantial caching benefit. 

We implement \framework in \emph{ncnn}~\cite{ncnn}, a popular deep learning engine, atop Android 6.0. 
\sys{} executes standard, unmodified CNN models such as ResNet-50~\cite{ResNet}.
%\sys{} is transparent to apps and models, requiring zero efforts from the developers.\revise{
We evaluate \framework on Nexus 6 with five popular CNN models over two large, real-world video datasets. 
Compared to a baseline engine version without enabling cache, \framework reduces the inference time by 18\% on average and up to 47\%. 
The reduction in inference time by \framework is \textbf{2$\times$} of the reduction achieved by existing CNN caches design~\cite{conf/mobisys/LocLB17}.
\sys{} reduces system energy consumption by around 20\%.
Its incurred accuracy loss is no more than 3\%.
Across all the models, \sys{} uses 2.5 MB -- 44 MB of memory, less than 2\% of the total system DRAM.}%\note{is this used by cache or whole engine?}%\mengwei{Mengwei: used by cache, so it's ``overhead''. 44MB may sound too much, maybe show an average or any other way?}

To summarize, we make the following contributions.

\noindent$\bullet$ We present \framework, a principled cache for executing CNN over mobile videos (Section~\ref{sec:overview}).
\sys{} exploits temporal locality in input mobile videos with proven video heuristics (Section~\ref{sec:matching}), propagates cacheable regions across CNN layers with the CNN knowledge (Section~\ref{sec:cache}), and eschews applying video heuristics to CNN internals.
% which are not visual contents but high-dimensional feature maps that human has no good heuristics. \noindent$\bullet$ We implement \framework in a commodity engine. 
The resultant prototype runs unmodified CNN models, requires zero effort from developers, and is immediately deployable on off-the-shelf Android devices (Section~\ref{sec:impl}).

\noindent$\bullet$ We evaluate \framework on popular CNN models with real-world datasets (Section~\ref{sec:eval}). The results show that \framework can reduce model inference time and energy consumption effectively.

\revise{
The full source code of \framework is at:
\begin{center}
\texttt{https://github.com/xumengwei/DeepCache}
\end{center}
}% The rest of paper is organized as follows.% We present the background about CNN and challenges of achieving layer-level caching in Section~\ref{sec:back}.% We describe the design overview of \framework in Section~\ref{sec:overview} and its two key components in Section~\ref{sec:matching} and Section~\ref{sec:cache}.% We explain the implementation details of \framework in Section~\ref{sec:impl}.% We evaluate \framework in Section~\ref{sec:eval}.% We discuss limitations of our work and possible future work in Section~\ref{sec:discuss}.% We survey the related work in Section~\ref{sec:related} and conclude in Section~\ref{sec:conclusion}.% Benefiting from increasingly powerful and ubiquitous cameras equipped on mobile devices, \emph{continuous mobile vision} has surprisingly enriched and facilitated human lives via ``showing computers what you see''.% The roles of such continuous vision tasks can be varied and generalized, including but not limited to% 1) \textbf{image classification}, where a 3-years old child uses a smart tablet to scan the room and learn to differentiate objects such as bookcase, air condition, etc.% 2) \textbf{face recognition}, where a user is wearing smart glasses during social activities, then a sequence of images will be captured and processed to identify the faces of the people in view. As people are identified, information about them can be displayed on glass screen.% % 3) \textbf{image-to-text translation}, where an English speaker walks around Lyon and uses her smartphone scanning the road signs written in French, then the text is translated into English and displayed on phone screen at the meantime.% To make such vision intelligence practical and efficient under mobile scenarios, both industry and academia have put tremendous efforts via developing inspiring applications~\cite{GoogleTranslate,amazon,conf/huc/LaneGQ15,conf/mobisys/KatevasLPS17,conf/www/YaoHZZA17,conf/icassp/VarianiLMMG14}, optimizing the runtime~\cite{conf/mobisys/LiKamWaZ15,conf/mobisys/LocLB17}, and building new hardware prototypes~\cite{conf/isca/LiKamWaHGPZ16,conf/mobisys/MathurLBBFK17,conf/mobisys/NaderipariziZPP17}.% % \xzl{this paragraph can be cut down to 2-3 sentences, if needed}% The state-of-the-art algorithm that drives continuous mobile vision is deep learning (DL), which feeds the captured images into a well-trained convolutional neural network (CNN) frame-by-frame and outputs task-specific results.% Though appealing, running these CNN models requires substantial hardware resources, e.g., CPU and memory, thus making it quite challenging on commodity mobile devices such as smartphones and head-mount devices.% A straightforward and traditional way to overcome this limitation cloud offloading~\cite{conf/mobisys/HanSPAWK16,siri}.% In this work, we focus on optimizing the local ``on-device'' processing for its advantages like fast response, network free, and privacy preservation.% % However, this offloading strategy requires high network bandwidth and can sometimes have substantial energy and monetary cost.% % More importantly, sending image data via network continuously introduces big privacy concerns as images captured on mobile devices often contain personal information.% % \xzl{I understand the argument against offloading. Yet, we can focus on arguing for ``on-device'' processing (fast response, privacy preserving...)}% In this paper, we explore the opportunity to exploit high temporal locality within adjacent frames in a video stream, e.g., consecutively captured images by mobile cameras might have substantial similar regions.% It is based on a key observation that mobile devices are often in slow motion or even held still when running continuous vision tasks~\cite{conf/mobisys/MathurLBBFK17,conf/sensys/ChenRDBB15}.% However, this opportunity is missed by traditional DL engines, which always run as end-to-end systems by taking the whole image as input and directly outputting the results.% Some recent work such as DeepMon~\cite{conf/mobisys/LocLB17} and CBinfer~\cite{cavigelli2017cbinfer} have implemented naive cache mechanisms.% Their approaches, however, are too coarse-grained in two aspects.% First, they only match the image blocks (or pixels) to the same position from last frame, ignoring the scenarios when cameras are not held stably or even moving.% Second, they employ \textit{each-layer matching} to identify reusable regions.% Given the large amount of layer numbers in CNN models, this approach occurs too much overhead.% % In comparison, \framework achieves a more fine-grained cache mechanism, as it reuses the intermediate computation results of identified similar regions (not the whole image) at layer granularity.% % The saved time from such caching mechanism can help deliver better user experience to end-users via a higher frame rate as well as a longer battery life.% % Designing a practical caching mechanism for CNN inference at layer granularity, however, is challenging because of the large number of layer types and complicated inter-layer processing for deep models.% % Moreover, properly identifying \emph{which part of images shall be reused} is difficult due to the presence of large environmental variations such as camera motion, object appearance, illumination conditions, etc.% % In addition, aggressive cache reuse might result in unacceptable accuracy loss as the reused image regions are unlikely to be identical.% In this paper, we present \framework (CNN Optimization on Mobile), a novel caching mechanism that accelerates CNN execution over mobile video streams.% \framework addresses the shortcomings mentioned in prior work, and maximumly reuses the computation results with minimal accuracy loss via two novel techniques.% 1) First, we design a novel image matching algorithm to identify the similar regions between two consecutive images.% The proposed algorithm is robust to deal with complicated image variation, such as camera motion, object appearance, illumination conditions, etc.% We retrofit the knowledge from video compression community~\cite{zhu1997new} for our CNN-specific caching mechanism.% The algorithm is tailored to deal with \shrink, a phenomenon that reusable regions tend to shrink during inference as defined in Section~\ref{sec:shrink}.% We also accelerate the algorithm so that it can process a 224X224 image within 10ms.% 2) Second, we dig deeply into the CNN execution procedure and build cache-awareness into the execution of a CNN inference engine. Our engine can accurately propagate reusable image regions through the internal layers and therefore reuse the computational results.% This propagation function ensures that our image matching algorithm only needs to be executed once, saving the overhead from \textit{each-layer matching}.% % develop a cache-aware CNN inference engine, ensuring that the similar regions identified in our image matching algorithm can be propagated among varied layers and the computations of these regions will be directly reused.% We implement a prototype of \framework to run standard, unmodified CNN models.% It remains transparent to the developers of apps and models, requiring zero efforts from them. % It is immediately deployable to off-the-shelf mobile devices. % We then evaluate \framework on commodity Android device with a variety of popular CNN models and real-world video benchmarks.% The results show that \framework can accelerate the inference of CNN models by 18.2\% on average and up to 47.1\% in some scenarios, but only has minimal accuracy loss.% As comparison, DeepMon has only 8.9\% acceleration.% In addition, around 19.7\% energy consumption can be saved after applying our approach.% The system overhead is acceptable given the achieved improvements.% To summarize, we make following contributions.% \noindent $\bullet$ We design \framework, a cache mechanism for CNN of mobile continuous vision (Section~\ref{sec:overview}).% The two core modules of \framework are a novel image matching algorithm for identifying similar regions among adjacent images in a video stream (Section~\ref{sec:matching}), and a cache-aware CNN inference engine that can reuse the computation results at layer granularity for each image frame (Section~\ref{sec:cache}).% \noindent $\bullet$ We implement \framework to be transparent to app developers and run unmodified CNN models on commodity devices (Section~\ref{sec:impl}).% \noindent $\bullet$ We comprehensively evaluate \framework via a variety of popular CNN models and real-world benchmarks (Section~\ref{sec:eval}). The results show that \framework can % substantially benefit end-users in consideration of end-to-end latency and energy consumption with minimal accuracy loss.% The rest of paper is organized as follows.% We survey the related work in Section~\ref{sec:related}.% We present the background about CNN and challenges of achieving layer-level caching in Section~\ref{sec:back}.% % We describe the design considerations in Section~\ref{sec:considerations}.% We discuss limitations of our work and possible future work in Section~\ref{sec:discuss} and conclude in Section~\ref{sec:conclusion}.
% !TeX root = ../MobiCom2018-COMO.tex\section{Background and Challenges}\label{sec:back}
In this section, we present CNN background and identify the major challenges to cache for continuous mobile vision.

\begin{figure}[t]
	\centering
	\includegraphics[width=0.48\textwidth]{pics/CNN_Arch}
	%\vspace*{-.6cm}
	\caption{A typical CNN model structure.}
	\label{fig:CNN_Arch}
\end{figure}\begin{table}[t]
\small
\centering
\begin{tabular}{|l|l|l|l|l|l|l|}
\hline
\textbf{Model} & \textbf{Lib} & \textbf{conv} & \textbf{fc} & \textbf{pl} & \textbf{act} & \textbf{rest} \\ \hline
\multirow{2}{*}{AlexNet~\cite{alexnet}} & TF & 79.2\% & 6.4\% & 11.1\% & 2.7\% & 0.6\% \\ \cline{2-7} 
 & ncnn & 77.9\% & 7.1\% & 12.1\% & 1.8\% & 1.1\% \\ \hline
\multirow{2}{*}{GoogLeNet~\cite{Inception}} & TF & 80.2\% & 0.1\% & 7.5\% & 8.1\% & 4.3\% \\ \cline{2-7} 
 & ncnn & 78.8\% & 0.7\% & 8.6\% & 9.3\% & 2.6\% \\ \hline
\multirow{2}{*}{ResNet-50~\cite{ResNet}} & TF & 91.8\% & 5.8\% & 0.5\% & 1.7\% & 0.2\% \\ \cline{2-7} 
 & ncnn & 93.7\% & 4.9\% & 0.8\% & 0.4\% & 0.2\% \\ \hline
\multirow{2}{*}{YOLO~\cite{yolo}} & TF & 82.4\% & 12.8\% & 2.1\% & 1.8\% & 0.9\%\\ \cline{2-7}
 & ncnn & 84.1\% & 12.2\% & 2.6\% & 0.9\% & 0.2\% \\ \hline
\multirow{2}{*}{Dave-orig~\cite{bojarski2016end}} & TF & 58.8\% & 28.6\% & 4.8\% & 2.9\% & 5.2\% \\ \cline{2-7} 
 & ncnn & 62.7\% & 25.9\% & 5.8\% & 3.7\% & 1.9\% \\ \hline
\end{tabular}
%\caption{The major latency composition of some popular CNN models tested on Nexus 6. Conv: convolutional, fc: fully-connect, pl: pooling, act: activation.}
\caption{Processing time breakdown of popular CNN models, showing that convolutional layers dominate the time. 
Layer types: convolutional (conv); fully-connected (fc); pooling (pl), activation (act). 
Hardware: Nexus 6. 
Engines: Tensorflow (TF)~\cite{TensorFlow}; ncnn~\cite{ncnn}.
}
\label{tab:latency_breakdown}
\end{table}\subsection{Convolutional Neural Network}\label{sec:cnn}
Convolutional Neural Network (CNN) is the state-of-the-art algorithm in many computer vision tasks, and is recently adopted in many mobile scenarios~\cite{conf/mobisys/ZengCZ17,TFCamera,conf/www/YaoHZZA17,conf/ica3pp/OuLSE17,TensorZoom,conf/mobisys/MathurLBBFK17}.
As shown in Figure~\ref{fig:CNN_Arch}, a typical CNN model repeatedly uses convolution and pooling layers to extract features from the whole image, and then applies fully-connected layers (fc) to finalize the 	vision tasks.
% The primary computation of CNN is in the convolutional layers (conv),
Convolutional layers (conv) apply kernels on the input data to extract embedded visual characteristics and generate output data (called \textit{feature map}).
% Max or min pooling layers (pl) cause representations to be invariant to translations, which also acts as a form of dimensionality reduction.%The difference between fully-connected layers and convolutional (pooling) layers is that the former connects each units of input and output together, while the latter only extracts simple features from input and captures local data properties.
For continuous mobile vision, CNN inference operates only on one single segment of data (i.e., an RGB image) at a time.

%\paragraph{Microbenchmark: CNN latency breakdown}\paragraph{Convolutional layers are hotspots}
Among all layer types, convolutional layers are the primary performance hotspots.
% We show this with an experiment in Table~\ref{tab:latency_breakdown}.\revise{
We summarize the latency breakdown of five popular CNN models in Table~\ref{tab:latency_breakdown}.
We use two libraries that support deep learning inference on Android to run these models on a Nexus 6 device: TensorFlow~\cite{TensorFlow} and ncnn~\cite{ncnn}.
It should be noted that each layer type (e.g., a convolutional layer) can have multiple instances in a model.}
In the breakdown, convolutional layers dominate the processing time, contributing at least 60\% and even up to 90\% (ResNet-50). %latex-cjk-common
This observation motivates us to focus on caching for convolutional layers in this work.

%\subsection{Layer-level Cache in CNN and Challenges}\subsection{Objective and Challenges}\label{sec:shrink}
Our overall approach to reduce CNN execution workloads is exploiting temporal locality on a mobile video stream.
That is, consecutive video frames often have substantial similar or overlapped regions.
In general, temporal locality in videos has been known for decades and widely exploited for video compression standards~\cite{le1991mpeg,richardson2004h}.
It is particularly pronounced in mobile videos: mobile devices (e.g., smartphones and glasses), when performing continuous vision tasks~\cite{conf/mobisys/LocLB17,conf/sensys/ChenRDBB15}, capture similar but non-identical image regions continuously. 
To this end, a deep learning engine can cache the CNN execution outcome from processing earlier frames for reuse in processing a later frame.
Of the cache, the \textit{keys} are input image contents and the \textit{values} are the corresponding inference results, i.e., feature maps. 
This objective, while simple, raises a few unique challenges.

%\noindent $\bullet$ \textbf{Tolerating Image variations.} \noindent$\bullet$\textbf{Cache lookup under \variation}~~ 
In general, cache stores key-value pairs. 
Classic caches, e.g., for web browsers or disks, look up cached values (e.g., web pages or disk blocks) by evaluating the \textit{equivalence} of keys (e.g., web URLs or block IDs). 
However, to look up reusable CNN execution results, the cache should evaluate the \textit{similarity} of keys (i.e., input image contents). 
Images consecutively captured in real world can have various aspects of differences for the presence of large variations in camera motion, object appearance, object scale, illumination conditions, etc. Those complicated conditions make it non-trivial to find out \emph{``what should be reused and what should not''}.

%\noindent $\bullet$ \textbf{Dealing with inter-layer processing of CNN.} %\noindent $\bullet$ \textbf{Fine-grained reuse at all layers}~~ \noindent$\bullet$\textbf{Fine-grained reuse of intermediate results}~~ 
The computation cost of a CNN model spreads over a cascade of internal layers, which produce feature maps as intermediate results. 
An effective CNN cache should store these feature maps and reuse them at fine spatial granularity whenever possible. 
However, deciding reusability for feature maps is challenging: 
since the data volume of feature maps is large, it incurs high overhead for the engine to inspect them; 
since feature maps consist of data points in higher dimension spaces, 
it is difficult for the engine to interpret their semantics. 

% runs as an end-to-end system by taking an image as input and producing output directly.%The intermediate processing of each layer is usually regarded as a black-box for developers.%Thus, deep insight and sophisticated design inside CNN execution engine is needed to make applications really benefit from caching. (\mengwei{Yunxin suggests we remove this challenge.})% \xzl{I am not convinced either. Agree with Yunxin - either enhance or consider removing}%\noindent $\bullet$ \textbf{Trade-off performance and accuracy.} \noindent$\bullet$\textbf{Balancing cacheability, model accuracy, and cache overhead}~~ 
In using cache, the engine will lose CNN model accuracy: it will have to reuse cached values for similar, yet nonidentical, image regions. 
This entails a complex trade-off. 
First, while relaxing the criteria for image similarity boosts cacheability, it also reduces model accuracy.
Second, while more thorough cache lookup improves cacheability, its additional overhead must be justified by sufficient performance gain. 

\begin{figure}[t]
	\centering
	\includegraphics[width=0.45\textwidth]{pics/conv_mrect}
	%\vspace*{-.6cm}
%	\caption{An example showing how the reusable (cache-able) region is shrunk by convolution operation.}
	\caption{An example showing cache erosion at a convolutional layer (kernel=3x3, stride=1, padding=1).}
	\label{fig:conv_mrect}
\end{figure}\noindent$\bullet$\textbf{Battling cache erosion}~~
%Another straightforward concern over the feasibility of \framework is that, during executing CNN models, 
Of a CNN, reusability tends to diminish at its deeper layers, a behavior we dubbed \textit{cache erosion}. 
More specifically, given an input image region deemed as similar (reusable) to an existing region of previous frame, the amount of reusable results on each layer's feature map shrinks as execution progresses into the model.
Figure~\ref{fig:conv_mrect} shows an example of a convolutional layer, for which the input is a cached region of 5x5 pixels.
However, the peripheral pixels (in gray) in the output cannot be loaded from cache as the central ones (in green), and must be exhaustively computed. 
This is because these peripheral pixels are derived from both reusable and non-reusable results in the input feature map. 
As a result, the reusable region has eroded.

Among various CNN layers, convolution, pooling, and LRN erode cache as above; 
fully-connected layer may completely destroy reusability, since each its output value depends on all its input values, which can hardly be all cached. 
\begin{figure}[t]
	\centering
	\includegraphics[width=0.48\textwidth]{pics/Alexnet_latency_breakdown}
	%\vspace*{-.6cm}
	\caption{Latency breakdown at layer granularity for AlexNet~\cite{alexnet}. Layers are presented at the order of execution: left-side layer will be executed first and the output will be fed to the right-side layer as input.}
	\label{fig:alexnet}
\end{figure}

Fortunately, in most CNN models, early layers contribute most of the computation cost and also suffer less cache erosion. 
Fully-connected layers come last in a CNN and contribute minor cost.
These are exemplified in Figure~\ref{fig:alexnet}, which breaks down the execution latency of a popular CNN model. 
Of the total latency, only 11.5\% is contributed by fully-connected layers, while the remaining 88.5\% is contributed by earlier layers that can benefit from cache. 
To further tackle cache erosion, we merge reusable regions into the largest possible ones, as will be discussed in Section~\ref{sec:matching}. 

% In the next sections, we describe how we address the above challenges through a fast image matching algorithm and a lightweight cache-aware CNN inference engine, including minimizing the \shrink effect by identifying the largest possible reusable regions in image matching. 
% \input{secs/considerations}

\section{Image Block Matching}\label{sec:matching}\begin{figure}[t]
	\centering
	\includegraphics[width=0.48\textwidth]{pics/proper_match}
	%\vspace*{-.6cm}
	\caption{Two matching examples, showing that the best matched block are not always desirable.}
	\label{fig:proper_match}
\end{figure}% \note{this is how we dealt with cache erosion}
Now we present the detailed design of our region matcher and how it deals with \shrink.
The goal of our image matching algorithm is to find ``similar'' regions (rectangles) between two images.
There are two ways to match: \textit{block-wise matching} and \textit{pixel-wise matching}.
Theoretically, identifying each pixel's matching level (pixel-wise matching) and reusing its cached results can be more fine-grained and minimize the model accuracy loss.
However, we have observed that even similar scenes in two sequential images can have relatively low matching scores of corresponding pixels (pixel mutation), due to barely unnoticeable environment variations such as light and moving objects.
Those ``unmatched'' pixels can lead to significant reduction of cache reuse due to the \shrink mentioned in Section~\ref{sec:shrink}.
Thus, we use block-wise matching rather than pixel-wise matching, taking a block (e.g., 10x10 pixels) as the basic unit to tell if it's successfully matched to a corresponding block in the previous image.
In this way, a mutated pixel will not affect the block-wise matching decision if other surrounding pixels in the block are well matched.

Two principles should be considered into the design of our block-wise matching algorithm.
First, the matching algorithm should run fast, keeping the processing overhead negligible compared to the improvement gained via cache reuse.
Second, we want the resulted blocks to be likely merged into larger blocks.
The second principle is exemplified by the case shown in Figure~\ref{fig:proper_match}: match(a) might have the highest matching scores for block B1 and B2, but it's not suitable in our cache mechanism since these small reusable blocks will quickly vanish after several layers due to \shrink (Section~\ref{sec:shrink}).
Imagine that B1 and B2 have size 5x5, and the convolutional kernel is 3x3.
After the \shrink, the reusable regions become two 3x3 rectangles, 18 pixels in total.
\revise{By contrast, match(b) finds two adjacent blocks in current frame that are similar to the blocks in previous frame, so that these two blocks can be merged into a larger one.
In this case, the reusable region becomes one 3x8 rectangle after convolution, 24 pixels in total.}

The overall flow of our matching algorithm is as follows.

\noindent$\bullet$\textbf{Step 1}. The current frame (image) is divided into an NxN grid, where each grid block contains certain number of pixels.

\noindent$\bullet$\textbf{Step 2}. For each divided grid block we find the most matched same-size block in previous frame.
Here, we denote the left-top point of \emph{i}-th block (i = 1 to $N^2$) in current frame as $(x_i,y_i)$, and the corresponding matched block position in previous frame as $(x'_i,y'_i)$.
We leverage the \emph{diamond search}~\cite{zhu1997new} algorithm which is widely used in video compression to quickly identify the most matched block.
The matching level (similarity) between two image blocks is represented by the \emph{PSNR}~\cite{zhu1997new} metric: higher \emph{PSNR} indicates that two blocks are more similar.

\noindent$\bullet$\textbf{Step 3}. We calculate the average block movement $(M_x, M_y)$ as the mean movement of the matched blocks whose \emph{PSNR} is larger than the given threshold $\mathcal{T}$.
\[(M_x,M_y)=(\dfrac{\Sigma{(x'_i-x_i)}}{K},\dfrac{\Sigma{(y'_i-y_i)}}{K}), \langle (x_i,y_i),(x'_i,y'_i) \rangle \in \mathcal{S}\]
where $\mathcal{S}$ is the collection of matched block pair whose PSNR is larger than $\mathcal{T}$, and K is the cardinality of $\mathcal{S}$.

\noindent$\bullet$\textbf{Step 4}. For each block $(x_i,y_i)$ in the current frame, we calculate its \emph{PSNR} with block $(x_i+M_x,y_i+M_y)$ in the previous frame.
If \emph{PSNR} is larger than $\mathcal{T}$, these two blocks are considered to be properly matched.

\noindent$\bullet$\textbf{Step 5}. We merge the small blocks that are properly matched in last step to larger ones.
For example, if $(x_i,y_i)$ and $(x_j,y_j)$ in current frame are adjacent, then their matched blocks in Step 4 should also be adjacent since they share the same offset $(M_x,M_y)$.
Thus, we can directly merge them into a larger rectangle as well as their matched blocks.

\begin{figure}[t]
	\centering
	\includegraphics[width=0.45\textwidth]{pics/match_elephant}
	%\vspace*{-.6cm}
	\caption{Matched rectangles in two consecutive images via our proposed algorithm.}
	\label{fig:match_elephant}
\end{figure}

Figure~\ref{fig:match_elephant} shows an output example of applying our matching algorithm on two consecutively captured images.
As observed, the second frame image is different from the first one in two aspects.
First, the camera is moving, so the overall background also moves in certain direction.
This movement is captured in Step 3 by looking into the movement of each small block and combining them together.
Second, the objects in sight are also moving.
Those moved objects (regions) should be detected and marked as non-reusable.
This detection is achieved in Step 4.

Our experiments show that most of the processing time of the above matching algorithm is spent at Step 2 and Step 4.
In Step 2, we need to explore the previous frame to identify the most matched block for every block in current image.
% The goal of such exploration is to find a proper offset $(M_x,M_y)$.
We can accelerate this step by skipping some blocks in current frame, e.g., only matching blocks at $(i*k)$-th row and $(j*k)$-th column ($i*k,j*k\leqslant N$).
Theoretically, a 2-skip (\emph{k}=2) can save 75\% of the computation time in this step, and a higher k can even achieve better improvements.
However, a higher \emph{k} might also result in inappropriately calculated $(M_x,M_y)$, resulting in fewer blocks to be properly matched at the last step.
We can further accelerate the computation of Step 4 by reusing the results in Step 2 since both of them need to calculate \emph{PSNR} between two blocks.
More specifically, if the \emph{PSNR} between $(x_i,y_i)$ (current frame) and $(x_i+M_x,y_i+M_y)$ (previous frame) is already calculated in Step 2, we simply reuse the result.
We demonstrate the efficiency of our proposed algorithm as well as these acceleration approaches in Section~\ref{sec:eval_matching}.

\section{Implementation}\label{sec:impl}

We implement our image matcher (Section~\ref{sec:matching}) in RenderScript~\cite{RenderScript}, the Android's counterpart of CUDA. 
Thanks to RenderScript, the image matcher execution can be offloaded to GPU for acceleration. 
Since RenderScript is a generic Android API, our image matcher is portable across Android devices. 

We prototype the engine feature of \sys{} atop \emph{ncnn}~\cite{ncnn}, an open-source deep learning engine optimized for mobile (Android and iOS).
%and already integrated in many popular apps such as WeChat and QQ~\footnote{WeChat and QQ are two dominant social apps in China with billions of users all over the world.}.%To apply \emph{ncnn} on mobile apps, developers train their own models via caffe~\cite{jia2014caffe} or download pre-trained caffe models from network, and convert them to \emph{ncnn} format.\emph{ncnn} works with standard CNN models.
\framework are directly compatible with those models without requiring extra model changes.

For each supported layer type, \emph{ncnn} provides a function \texttt{forward(top\_blob, bottom\_blob)}, where \texttt{top\_blob} and \texttt{bottom\_blob} encapsulate the output and input of this forward step, respectively.
We replace \texttt{forward()} with our customized \texttt{c\_forward(top\_blob, bottom\_blob, c\_blob, c\_regions)}, where \texttt{c\_blob} stores the computation results of current layer from the last frame, and \texttt{c\_regions} specifies which parts can be reused.
\texttt{c\_forward} calculates the output just as \texttt{forward} does, except that \texttt{c\_forward} skips the calculation of cached regions but copies from \texttt{c\_blob} directly.
Before \texttt{c\_forward} invoked, \texttt{c\_regions} will be propagated from last layer.
As mentioned in Section~\ref{sec:cache}, cached regions will erode (conv, pooling) or vanish  (full-connected) during the inference process, thus we use another function named \texttt{reg\_forward} which calculates how cached regions are propagated among different layers.
% Currently, we only support \texttt{c\_forward} for convolutional layer, since it dominates the processing time of overall CNN network as we have already shown.% But we can easily extend the implementation of this caching mechanism to other layer types as well.
We also implement some custom layers such as \texttt{atan} that are unsupported in the current \emph{ncnn} but necessary for our benchmark models. 
Overall, our new implementation contains 4,030 lines of code. 
%as illustrated in Table \ref{tab:LoC}.%\input{tab-loc}%\input{lst-api}\sys{} is fully compatible with \textit{ncnn} APIs. Any existing vision applications built on \textit{ncnn} will work with \sys{} out of box. 
In addition, \sys{} exposes a few cache parameters, e.g., threshold $\mathcal{T}$ (Section~\ref{sec:matching}), block size N (Section~\ref{sec:matching}), and cache expiration time (Section~\ref{sec:cache}) for developers to optionally fine control \framework behavior.
This is analogous to a browser cache exposing various policy knobs to web apps~\cite{webcache}.

% These parameters are essentially trades-off between the achieved improvements via cache reuse and the compromise of accuracy, as elaborated in Section~\ref{sec:eval_choosing}.%As observed, low-level details of rs-based image matching algorithm and cache-aware inference are completely transparent to developers.
\section{Evaluation}\label{sec:eval}
We thoroughly evaluate \framework using five typical CNN models on two real-world, large-scale datasets.
In summary, \framework saves the execution time of CNN models by 18.2\% on average and up to 47.1\%, while incurring accuracy loss as low as 3\%.
\revise{
In addition, we directly compare \sys{} with the cache mechanism presented in DeepMon~\cite{conf/mobisys/LocLB17}, a cutting-edge deep learning engine, and the results show that \sys{} outperforms DeepMon on all models and all datasets.}%saves execution time by 8.9\% at similar accuracy loss.%saves execution time by 8.9\% at similar accuracy loss.\begin{table*}[!t]
\small
\centering
\begin{tabular}{|l|l|l|c|l|l|} \hline
	\textbf{Application} & \textbf{Model Name} & \textbf{Model Architecture} & \textbf{\# of Conv} & \textbf{Model Output} & \textbf{Dataset}\\ \hline
	\multirow{3}{*}{Activity recognition} & REC\_1 & AlexNet~\cite{alexnet} & 5 & \multirow{3}{*}{human activity type} & \multirow{4}{*}{UCF101~\cite{UCF101}}\\ \cline{2-4}
	& REC\_2 & GoogLeNet~\cite{Inception} & 57 & &\\ \cline{2-4}
	& REC\_3 & ResNet-50~\cite{ResNet} & 53 & &\\ \cline{1-5}
	Object detection & DET & YOLO~\cite{yolo} & 8 & object types and positions &\\ \hline
	Self-driving & DRV & Dave-orig~\cite{bojarski2016end,Autopilot} & 5 & steering angle & Nvidia driving dataset~\cite{nvidia-driving-dataset}\\ \hline
\end{tabular}
\caption{List of CNN models used to evaluate \framework.}
\label{tab:models}
\end{table*}\begin{figure}[t]
	\centering
	\includegraphics[width=0.45\textwidth]{pics/latency_overall}
	%\vspace*{-.6cm}
	\caption{Average processing time of all five CNN models over their test scenarios.}
	\label{fig:latency_overall}
\end{figure}\begin{figure*}[t]
% \setlength{\abovecaptionskip}{2pt}
% \setlength{\belowcaptionskip}{-15pt}
\centering
\scriptsize
\subfloat[REC\_1 model]{\includegraphics[width=0.5\textwidth]{pics/eval_alexnet}}
\subfloat[REC\_2 model]{\includegraphics[width=0.5\textwidth]{pics/eval_googlenet}}\\
\subfloat[REC\_3 model]{\includegraphics[width=0.5\textwidth]{pics/eval_resnet}}
\subfloat[DET model]{\includegraphics[width=0.5\textwidth]{pics/eval_yolo}}
%\caption{Processing latency of \framework compared to \emph{no-cache} and \emph{DeepMon}.}
\caption{Per-scenario processing time of four CNN models. (For each model, the average time across all scenarios is shown in Figure~\ref{fig:latency_overall}.)}
% \vspace{-5em}
\label{fig:eval_latency}
\end{figure*}\subsection{Experimental Setup}\paragraph{Test Platform}
We use Nexus 6 (Qualcomm 2.7 GHz quad-core CPU; Adreno 420 GPU) with Android 6.0 as the test platform.

\paragraph{Benchmark Datasets}
We use two kinds of datasets to evaluate our framework.
\textbf{UCF101 dataset}~\cite{UCF101} contains 101 types of human activities and 13,421 short videos ($<$ one minute) created for activity recognition.
We randomly select 10 types from these activities and evaluate \framework across them:
\emph{Basketball} (\textbf{T1}), \emph{ApplyEyeMakeup} (\textbf{T2}), \emph{CleanAndJerk} (\textbf{T3}), \emph{Billiards} (\textbf{T4}), \emph{BandMarching} (\textbf{T5}), \emph{ApplyLipstick} (\textbf{T6}), \emph{CliffDiving} (\textbf{T7}), \emph{BrushingTeeth} (\textbf{T8}), \emph{BlowDryHair} (\textbf{T9}), and \emph{BalanceBeam} (\textbf{T10}).
In total, \textbf{55,680} images have been processed in our evaluation for each selected CNN model.
% We convert those videos into images sequence.\textbf{Nvidia driving dataset}~\cite{bojarski2016end} is collected by driving on a wide variety of roads and in a diverse set of lighting and weather conditions.
It contains 45,568 static images captured at 10 FPS and the corresponding steering angles made by the driver.
We randomly select 10 scenarios\footnote{A scenario, in video dataset lingo, refers to a video recorded at specific scene, location, and time.} (100 images for each) as the testing set.
\revise{We use ffmpeg~\cite{ffmpeg} tool to extract raw images from the above video datasets and feed the images to \framework sequentially, mimicking video ingestion in real-world continuous vision applications.}%Here we use scenario to refer to a video recorded at specific scene, location, and time.\paragraph{Workloads}
We use a variety of five CNN models to verify \framework as shown in Table~\ref{tab:models}.
For activity recognition, our models (REC\_1, REC\_2, and REC\_3) are pre-trained on ILSVRC 2012 dataset~\cite{imagenet}, and then transferred learned on UCF101.
The architectures of those models are initially used for image classification.
In our case, we use them to run each single image in the video and average the final result~\cite{conf/cvpr/KarpathyTSLSF14}.
For object detection, the model (DET) is trained via Pascal VOC 2007 dataset~\cite{voc}.
The model (DRV) used for self-driving is trained and tested on the Nvidia driving dataset mentioned above.
It is worth mentioning that these CNN models are quite generalized and can be used in many different tasks with few customization efforts.

\paragraph{Metrics}
We use accuracy, processing latency, and power consumption to evaluate the performance of our framework.
To report the \textbf{accuracy} results, we use different metrics to fit into different applications.
We report the top-k accuracy for our activity recognition models, and MSE (Mean Squared Error) as the accuracy for object detection and self-driving tasks because their outputs are continuous values.
Since the dataset used (UCF101) has no labels for object detection, we treat the output of exhaustively running complete model without cache mechanism as ground truth (observed values).
For \textbf{latency}, we log the starting time when \framework receives the image and the ending time when \framework outputs the inference result.
The subtracted duration is reported as the processing latency, including the time spent on image matching and CNN inference.
Finally, we measure the \textbf{energy consumption} via Qualcomm Snapdragon Profiler~\cite{SnapdragonProfiler}.
The baseline of phone's idle state is always subtracted.

\paragraph{\framework Configuration}
If not otherwise specified, we use a default block size of 10x10, the matching threshold $\mathcal{T}$ of 20 in our image matching algorithm (Section~\ref{sec:matching}), and the expiration time N of cache is set as 10 (Section~\ref{sec:overview}).

\paragraph{Comparison to Alternatives}
We experimentally compare the performance of \framework to two alternative approaches: \emph{no-cache}: exhaustively running the complete model without cache reuse (ground truth used in measuring accuracy); 
\emph{DeepMon}~\cite{conf/mobisys/LocLB17}:
the cache mechanism in a state-of-the-art deep learning engine.
To make the comparison fair, we have carefully ported \emph{DeepMon}'s cache to the ncnn engine executed on the CPU of our test platform, where \sys{} also runs. 
%Differently from the original prototype of \emph{DeepMon}, %we have implemented \emph{DeepMon}'s cache mechanism on CPU rather than GPU to make it comparable to \framework.
Note that we have contrasted the design of \emph{DeepMon} cache with \sys{} (Section~\ref{sec:intro}), and will present more details in related work discussion (Section~\ref{sec:related}).

\subsection{Latency Improvement}
Figure~\ref{fig:latency_overall} summarizes the achieved improvements via applying cache mechanism on average.
Our primary observation is that applying \framework can have substantial latency reduction compared to \emph{no-cache}, i.e., \textbf{18.2\%} on average, while \emph{DeepMon} has only \textbf{8.9\%}.
This improvement varies across different CNN models.
For a relatively small model REC\_1 (5 convolutions, 25 layers in total), \framework results in \textbf{28.1\%} saving up of total processing time on average, while \emph{DeepMon} only has \textbf{13.1\%} improvement.
For a deeper model REC\_2 (57 convolutions, 153 layers in total), the benefit from \framework reduces to \textbf{19.7\%}, while \emph{DeepMon} has only \textbf{10.2\%}.
For DET, \framework can have only \textbf{14.2\%} latency improvement.
The reason is that, differently from other classification models, DET is applied in object detection applications and outputs location-sensitive information.
Thus, many computation-intensive fully-connected layers reside at the end of DET, making the benefit from convolution-layer cache smaller.
The similar situation also applies for the DRV model.

We further illustrate the results under different video benchmarks (UCF101) in Figure~\ref{fig:eval_latency}.
We observe that the performance of \framework can differ a lot under different benchmarks.
Taking REC\_1 as an instance, \framework saves up \textbf{47.1\%} processing time under \emph{Billiards} (\textbf{T4}) scenarios.
We manually check the dataset videos and identify the reasons of such high speedup as following: 1) camera is in slow motion, 2) most objects are still except the player and balls, 3) indoor lighting is stable.
In comparison, \framework has only \textbf{11.0\%} latency improvement when processing \emph{BandMarching} (\textbf{T5}) videos because the camera and most objects (people) in view are moving brokenly.
Similarly, for REC\_3, \framework saves 38.7\% processing time when dealing with \textbf{T4} but only 3.8\% under \textbf{T5}.
Importantly, we observe that \framework consistently beats \emph{DeepMon} for each scenario.

\begin{figure}[t]
	\centering
	\includegraphics[width=0.48\textwidth]{pics/eval_latency_breakdown}
	%\vspace*{-.6cm}
	\caption{Processing time for individual convolutional layers in model REC\_2.}
	\label{fig:eval_latency_breakdown}
\end{figure}

We further dig into the achieved improvement at each individual convolutional layer.
As shown in Figure~\ref{fig:eval_latency_breakdown}, the latency improvement mainly comes from the first few layers due to the \shrink mentioned previously.
Fortunately, these layers often contribute to the majority of overall latency, indicating that the benefit remains meaningful when models grow deeper.
For example, the third convolutional layer takes 165ms to run, which contributes around 18.4\% to the total model.
\framework is able to save 90.2ms from this single layer since this layer resides at the beginning of the overall model.

\subsection{Accuracy Loss}\begin{figure}[t]
	\centering
	\includegraphics[width=0.44\textwidth]{pics/eval_top_accuracy}
	%\vspace*{-.6cm}
	\caption{Top-k accuracy drop of \framework.}
	\label{fig:eval_top_accuracy}
\end{figure}
We then investigate how much accuracy \framework compromises in return for the latency benefits.
The top-k accuracy drop for our activity recognition is shown in Figure~\ref{fig:eval_top_accuracy}.
In overall, \framework and \emph{DeepMon} both have very small accuracy drop ($\leqslant$\textbf{3\%} for top-1 and $\leqslant$\textbf{1.5\%} for top-3).
These loss is acceptable given the observation that our baseline (no-cache) can achieve round 62.8\% top-1 accuracy and 76.1\% top-3 accuracy.
We have even observed cases where the baseline wrongly classifies the image while our \framework does it correctly.
This is because that we have designed our image matching algorithm to carefully choose which part of computations to reuse, and this reusable information is properly propagated during inference, thus minimizing the impact on the recognition output.

\begin{figure}[t]
	\centering
	\includegraphics[width=0.44\textwidth]{pics/eval_mse}
	%\vspace*{-.6cm}
	\caption{MSE between the output of caching approaches (\framework, \emph{DeepMon}) and ground truth (\emph{no-cache}).}
	\label{fig:eval_ed_accuracy}
\end{figure}Figure~\ref{fig:eval_ed_accuracy} shows the MSE between ground truth (\emph{no-cache}) and other cache approaches (\framework and \emph{DeepMon}) when running DET and DRV models.
As observed, the median MSE of \framework is \textbf{0.00166} and \textbf{2.617} for DET and DRV respectively, quite similar to the results of \emph{DeepMon} with \textbf{0.00164} and \textbf{2.017}.
For the DRV case, the results can be interpreted that \framework leads to 2.6 degrees offset from the decision made by human driver.
Considering that \framework will periodically run the total image without cache reuse, as mentioned in Section~\ref{sec:overview}, this offset will not be accumulated.
To be compared, our above latency experiment shows that \framework can accelerate CNN models two times as \emph{DeepMon}, e.g., \textbf{18.2\%} vs. \textbf{8.9\%} on average across all models and benchmarks.

\subsection{Energy Saving}

We now investigate the energy consumption of \framework across all selected benchmarks, and illustrate results in Figure~\ref{fig:eval_energy}.
It is observed that \framework can save \textbf{19.7\%} of energy consumption on average and up to \textbf{28.6\%} (REC\_1), while \emph{DeepMon} only has \textbf{8.0\%} on average.
This saving is mostly from the reduced processing time.
Considering that vision tasks are very energy-intensive, this saving up is able to substantially lengthen battery life.
For example, applying \framework on REC\_3 to classify 10 images can help spare 66.8J energy, enough to support 40 seconds of video playing on Nexus 6 phone according to our measurement.

\begin{figure}[t]
	\centering
	\includegraphics[width=0.44\textwidth]{pics/eval_energy}
	%\vspace*{-.6cm}
	\caption{Energy consumption of \framework.}
	\label{fig:eval_energy}
\end{figure}\subsection{Choices of Parameters}\label{sec:eval_choosing}\begin{figure}[t]
\centering
\scriptsize
\includegraphics[width=0.44\textwidth]{pics/eval_parameter_T}
\caption{Effect of varied matching threshold $\mathcal{T}$ on processing latency and top-1 accuracy drop of REC\_2 model.}
\label{fig:eval_parameter_T}
\end{figure}\begin{figure}[t]
\centering
\scriptsize
\includegraphics[width=0.44\textwidth]{pics/eval_parameter_N}
\caption{Effect of varied block size on processing latency and top-1 accuracy drop of REC\_2 model.}
\label{fig:eval_parameter_N}
\end{figure}

In our matching algorithm mentioned in Section~\ref{sec:matching}, some variables can be used to make trade-off between latency improvement and accuracy drop.
Matching threshold $\mathcal{T}$ is the key to decide whether two image blocks are similar enough to be reused.
Figure~\ref{fig:eval_parameter_T} illustrates how $\mathcal{T}$ can affect the latency and accuracy (REC\_2 + \textbf{T1}).
As expected, higher $\mathcal{T}$ indicates fewer blocks can be matched, thus leading to less top-1 accuracy drop, but also higher processing latency.
In our default setting ($\mathcal{T} = 20$), \framework can achieve considerable latency improvement, e.g., \textbf{18.3\%} (from \textbf{917ms} to \textbf{748ms}), with acceptable accuracy loss (\textbf{2.1\%}).
This setting aligns with the fact that the acceptable values for wireless transmission quality loss are commonly considered to be about 20 to 25~\cite{psnr}.
However, the threshold can also be set by application developers to adapt to task-specific requirements.
For applications that are not very sensitive to the output accuracy, developers can aggressively use a smaller $\mathcal{T}$ to achieve higher latency improvement.

Another configurable parameter in our image matching algorithm is the block size.
As observed from Figure~\ref{fig:eval_parameter_N}, a larger block size results in more latency improvement but also higher accuracy loss.
This result is reasonable since splitting an image into large blocks indicates more coarse-grained matching.
As an extreme case, when block size equals to 1, the accuracy loss is very small (\textbf{0.2}) but the latency improvement is also very low (\textbf{2.19\%}).
This is actually the \emph{pixel-wise} approach discussed previously in Section~\ref{sec:overview}, and the result is consistent with our discussion.
Our empirical suggestion is setting block size around 10 for 227x227 images.

\subsection{Image Matching Performance}\label{sec:eval_matching}% !TeX root = main.tex\begin{table}[t]
\small
\centering
\begin{tabular}{L{2.5cm}ll} \hline
& \textsc{Latency (ms)} & \textsc{Match Ratio (\%)}\\\hline
DeepMon & 4.7 $\pm$ 0.7 & 46.1 \\ %\hline
ES & 33.3 $\pm$ 13.16 & 71.5\\ %\hline
TSS & 24.7 $\pm$ 9.41 & 70.8\\ %\hline
DS & 19.5 $\pm$ 6.53 & 71.2\\ %\hline
% DS + skip-block & 11.3 $\pm$ 2.68 & 69.5\\ %\hline
% DS + skip-block + reuse & 9.7 $\pm$ 2.55 & 69.5\\ \hline
DS + optimization & 9.7 $\pm$ 2.55 & 69.5\\ \hline
\end{tabular}
\caption{\revise{A comparison of image matching algorithms between DeepMon~\cite{conf/mobisys/LocLB17} (row 1), which uses histogram-based matching, and \sys{} (the remaining rows) that uses different block matching algorithms in combination with optimization techniques mentioned in Section~\ref{sec:matching}.
\sys{} achieves much higher match ratios with minor increase in latency.}
}
% \xzl{finish the above. 2-skip sounds a bit counter-intuitive. how about skip-block?}
% \vspace{-1em}
\label{tab:eval_matching}
\end{table}\revise{
Finally, we report the performance of our renderscript-based implementation of image matching algorithm individually.
Our current matching algorithm mentioned in Section~\ref{sec:matching} is based on the diamond search (DS), i.e., DS as an ``algorithm unit'' (used in Step 2). In addition to the DS, there are several other block matching algorithms that can be plugged into our image matching algorithm to replace DS, such as the Exhaustive Search (ES) and the Three Step Search (TSS). The details and differences of these algorithms are summarized in the survey effort~\cite{barjatya2004block}. In this part of evaluation, we also implement the ES-based and the TSS-based image matching to compare.
We run preceding algorithms on 10,000 images that are randomly picked from UCF101 and resized to 227x227, and log the processing time (\textit{latency}) and the proportion of matched regions (\textit{match ratio}).


As shown in Table~\ref{tab:eval_matching}, our image matching algorithm can achieve around 70\% match ratio.
The use of different block matching algorithms has minor impacts on the match ratio, but the DS-based implementation is much faster than the ES-based and TSS-based implementation, i.e., 19.5ms vs. 33.3ms \& 24.7ms.
Another important observation is that the acceleration techniques mentioned in Section~\ref{sec:matching}, i.e., k-skip and reusing, can significantly improve the processing latency from \textbf{19.5ms} to \textbf{9.7ms} on average, with only \textbf{2.4\%} loss in the match ratio.
These results indicate that our image matching algorithm works well for our CNN cache mechanism, as it occurs quite negligible overhead ($\leqslant$ \textbf{10ms}) compared to the benefit gained from cache reusing.
To be compared, the histogram-based matching algorithm used in \emph{DeepMon} matches only 46.1\% of image areas, while only runs 5ms faster.
}

In our above experiments, we treat the image matching and CNN inference as two sequential stages so that the time consumed on the image matching diminishes the benefits gained from cache-reuse.
Though the matching algorithm is accelerated, it still has non-trivial impacts on the performance of \framework especially when the model is relatively small such as DRV.
But in practice, these two stages can often be carried out asynchronously when the images can be captured at a higher rate than our CNN inference.
More specifically, \framework can run the image matching algorithm on $i$-th image and CNN inference on $(i+1)$-th image at the same time.
In our case, since we implement these two stages on different mobile processors (GPU and CPU), their processing should not interfere each other, therefore \framework can further improve the overall performance.

\subsection{Memory Overhead}\begin{figure}[t]
	\centering
	\includegraphics[width=0.45\textwidth]{pics/memory_overhead}
	%\vspace*{-.6cm}
	\caption{Memory overhead of \framework.}
	\label{fig:memory_overhead}
\end{figure}

Figure~\ref{fig:memory_overhead} shows the memory overhead of \framework.
Besides the 5 models used above, we also test on other three popular CNN models: MobileNet~\cite{MobileNet}, SqueezeNet~\cite{squeezenet}, and DeepFace~\cite{DeepFace}.
Here we assume that all model parameters are read into memory once initialized without I/O transmission during the inference.
We report the memory peak usage during the inference here.
As observed, the memory overhead occurred by \framework ranges from \textbf{2.5MB} to \textbf{43.8MB} depending on the internal structure of models.
This overhead is quite trivial since nowadays mobile devices are usually equipped with large size of memory, e.g., 3GB in Nexus 6.
Note that we only cache and reuse the computation results of convolutional layers so that no extra memory usage will be wasted on other computation-light layers.
\section{Discussion}\label{sec:discuss}\revise{
% Currently we have only tested \framework on five typical CNN models.
% We believe the results can be generalized to other novel model architectures, such as SqueezeNet~\cite{squeezenet}, MobileNet~\cite{MobileNet}, DenseNet~\cite{huang2017densely}, etc.
% This is because the novelty of these models primarily reside in two aspects: inter-layer organization, such as dense connection~\cite{huang2017densely}, and intra-layer design, such as depth-wise convolution~\cite{MobileNet,chollet2017xception}, dilated convolution~\cite{yu2015multi} that are used to replace traditional convolution.
% These new fashions don't hurt the temporal locality that \framework can exploit in video stream. 
% Besides, the observation made in Section~\ref{sec:back}, i.e., convolutional layers, especially the first ones, contribute most to the overall computations of CNN models, can be universally found on these new models as well.
% Furthermore, the implementation of \framework can be generalized to other hardwares as well.
% Taking GPU as an example, the redundant processing can be also saved by reducing the GPU kernels that are required to compute the output feature map. For other hardware-specialized accelerators like FPGA, it may be easier as one could revise the hardware logic to support caching.
% We plan to extend \framework in the above aspects as future work.
% Being confident in the generality of \sys{}, we plan to test it on devices in different form factors (e.g., head-mount glasses, collar-worn camera, etc) and more CNN models (e.g., VGG, RCNN, etc).
% For now we focus on optimizing vision tasks, however, our caching principle may benefit other deep learning applications where inputs show temporal locality.
% We plan to explore the possibility of applying our technique to natural language and audio processing.
% Last, we plan to accelerate the block matching algorithm (Section~\ref{sec:matching})
% with video encoding hardware widely available on mobile SoCs~\cite{MediaCodec}.

% The idea and high-level design of \sys{} can be applied on other non-mobile videos, such as on-cloud video analysis, as long as there's redundancy between sequential frames. Currently \sys{} is designed to optimize mobile vision since: (1) real-time mobile vision usually has image redundancy, but it does not hold for many videos such as movies that have been edited; (2) mobile devices suffer from long user-perceived latency and huge energy consumption for running CNN-based vision tasks, while such problems are not so serious on PC desktops or servers.
% }

% \xzl{proof read the following carefully}

\paragraph{Applicability to other CNN models}
This paper reports only the results of \framework on five typical CNN models.
Yet, we expect that \framework{} applies to emerging CNN models, such as the SqueezeNet~\cite{squeezenet}, the MobileNet~\cite{MobileNet}, and the DenseNet~\cite{huang2017densely}. 
Intuitively, the new models, with their innovated inter-layer organizations and intra-layer optimizations, still preserve the temporal locality that \framework hinges on.
Furthermore, our observation on the domaninating cost of early convolutional layers (Section~\ref{sec:back}) is true for these new models.

\paragraph{Implementation on accelerators}
While we prototype the inference stage of \framework{} on CPU, we expect that it can be ported to and benefit from hardware accelerators. 
Taking GPU as an example, \framework{} is capable of reducing the redundant processing by avoiding GPU kernels for computing output feature maps. 
For FPGA, we expect that our caching mechanism can be implemented as the hardware logic for further speedup. 

\paragraph{Applicability to other video types}
The idea and high-level design of \sys{} can be applied on other non-mobile videos as long as there's redundancy between adjacent frames. Currently \sys{} is optimized for the mobile vision, because (1) mobile videos contain much richer temporal locality than other video types such as edited movies, and (2) mobile devices are much more sensitive to the latency and the energy consumption in deep vision as compared to other platforms such as desktops or servers.

}%-- a notable advantage of leveraging conventional wisdom.% \note{this paragraph may hurt us. suggest removal}% Our image matching algorithm is designed to handle scenarios when the device is % sliding without other kinds of geometric transformation.% For example, we observe that (even small degree) rotation of device can lead to lots of reduction of the matched areas.% We plan to further make our algorithm more efficient and robust under those complicated scenarios.
\section{Related Work}\label{sec:related}% \note{1. This can be moved to the end of paper since we've already differentiate ourselves in Intro. 2. Check recent Mobisys/Mobicom. 3. I moved ``caching'' to the start as it is most related}% In this section, we discuss existing literature studies that relate to our work in this paper.\paragraph{Convolutional Layer Caching}
As most related efforts, DeepMon~\cite{conf/mobisys/LocLB17} and CBinfer~\cite{cavigelli2017cbinfer}  incorporate CNN caches that we deem ad-hoc.
%However, they are both short in several aspects compared to \framework.
First, they match the image blocks (or pixels) in only the same positions, therefore are unable to tolerate the scene variation as we highlighted in Section~\ref{sec:intro}.
By contrast, \sys{} retrofits proven video techniques to systematically search for nearby similar image blocks.
%making them not feasible when the mobile camera is not held stably or even moving.%As a comparison, \framework adopts a novel image matching algorithm that is more robust to environmental conditions.
Second, 
%they leverage \textit{each-layer matching}, requiring the matching algorithm to be called in each convolutional layer.
they execute cache lookup over feature maps at all layers.
\revise{Such each-layer matching strategy not only incurs too much runtime overhead, but also requires extra efforts from application/model developers to manually set a ``proper'' matching threshold for each layer.
By contrast, \sys{} runs lookup only once at the input raw images, and propagates the reusable region boundaries across all the layers.
In a concurrent project, $EVA^2$~\cite{buckler2018eva} proposes hardware optimization for exploiting temporal redundancy in live computer vision. By contrast, \sys{} is designed and implemented to run on general-purpose processors that are widely available on commodity mobile devices.
Besides, $EVA^2$ requires a model to be manually separated into two parts, and the output of the prefix part will be saved and reused while the suffix part will be fully executed. In \sys{}, such manual efforts are naturally avoided by our propagation mechanism mentioned in Section~\ref{sec:cache}.
Potluck~\cite{guo2018potluck} enables the cross-application cache reuse of computations on a similar video input.
However, unlike \sys{} that identifies which parts of image regions shall be reused, the cache mechanism of Potluck is rather coarse-grained since it can reuse \textbf{only} the whole output.
}% As a result, \sys{} incurs much lower overhead.%In \framework, the matching algorithm only runs once at the beginning, and the information of reusable regions will be properly propagated inside our inference engine.% Third, DeepMon simply matches image blocks based on the histogram distribution and CBinfer uses pixel-level matching, which is either not accurate enough or causes too much overhead compared to our image matching algorithm presented in Section~\ref{sec:matching}.% Third, neither DeepMon nor CBinfer makes a detailed design on how cached regions shall be propagated and altered during DL inference as discussed in Section~\ref{sec:cache}.%Overall, our work presents a deeper and more thorough study on this specific caching mechanism in CNN.\paragraph{Continuous Mobile Vision}
Emerging mobile vision systems span from commercial products~\cite{GoogleTranslate,amazon} to research prototypes~\cite{conf/ica3pp/OuLSE17,conf/mobisys/ZhuYZZ17,conf/cvpr/DasDWWSS17,conf/huc/HodgesWBISBSKW06,conf/mobisys/ZengCZ17,conf/mobisys/JainMC15,hwang2017raven}.
To optimize mobile vision tasks,
\cite{conf/mobisys/LiKamWaPPZB13,likamwa2013energy} made the early energy characterization and optimization towards continuous mobile vision.
Starfish~\cite{conf/mobisys/LiKamWaZ15} allows concurrent vision applications to share computation and memory objects.
RedEye~\cite{conf/isca/LiKamWaHGPZ16} reduces image sensor energy by offloading CNN layers to analog domain. % before quantization.% DeepMon~\cite{conf/mobisys/LocLB17} designs a suite of optimization techniques to efficiently offload CNN to mobile GPUs.
DeepEye~\cite{conf/mobisys/MathurLBBFK17} enables rich analysis of images in near real-time via novel, small form factor wearable camera. 
Such high interest in mobile vision motivates \sys{}. 
% match-box sized collar or lapel-worn camera wearable.%All these existing work are motivational to us.% \textbf{DL Cloud Offloading.}% Running deep neural networks (DNN), especially deep CNN models for vision tasks, often requires large amount of computation resource.% A common wisdom has been that commercial smartphones cannot support such high computational workloads with reasonable end-to-end latency and energy consumption.% Thus, a traditional way of utilizing DL algorithms is offloading the execution procedure from mobile devices to high-end clouds.% Many companies such as Google and Apple have already built powerful web services to support such offloading tasks from intelligent applications~\cite{TPU,siri}.% In addition to the industrial efforts, some academic projects go further in some certain aspects.% DjiNN~\cite{conf/isca/HauswaldKLCLMDM15} is a novel service infrastructure in warehouse scale computers (WSCs) specialized to handle large-scale DNN tasks offloaded from remote applications.% Zhang et al.~\cite{journals/tc/ZhangYC16} propose a privacy-preserving deep model using the BGV encryption scheme to encrypt the private data and employing cloud servers to perform the high-order back-propagation algorithm on the encrypted data.% Neurosurgeon~\cite{conf/asplos/KangHGRMMT17} can automatically partition DNN computation between mobile devices and data-centers at the granularity of neural network layers.% This partition strategy is based on the consideration of latency and energy consumption of mobile devices, and can adapt to different hardware specifications and other environment elements such as network connectivity.\paragraph{Optimizing Deep Learning Execution for Mobile}% In recent years, executing DL tasks on local smartphones have become a realistic idea due to the hardware evolution on mobile platforms.% Compared to cloud offloading, local execution have two key advantages: 1) it requires no network bandwidth, 2) it keeps privacy data on local device.% However, local executing can still suffer from resource constrain.
Extensive work is done on making deep learning affordable on mobile devices.
The approaches include making models much smaller to fit mobile devices~\cite{conf/huc/LaneGQ15,conf/icassp/ChenPH14,conf/icassp/VarianiLMMG14,MobileNet},
specializing hardware to deep learning algorithms~\cite{conf/asplos/ChenDSWWCT14,conf/fpga/ZhangLSGXC15,conf/isca/ChenES16,conf/isca/HanLMPPHD16},
compressing existing models~\cite{conf/mobisys/KatevasLPS17,conf/ipsn/LaneBGFJQK16,conf/cvpr/WuLWHC16,conf/nips/DentonZBLF14,conf/mobisys/HanSPAWK16,yao2017deepiot}, etc.
%For example, DeepX~\cite{conf/ipsn/LaneBGFJQK16} dramatically reduces the resource overhead by leveraging Runtime Layer Compression (RLC) and Deep Architecture Decomposition (DAD).% Those approaches essentially made trade-off between the accuracy and resource consumption. % Among various compression methodologies, weight pruning~\cite{journals/corr/YangCS16a}~\cite{molchanov2016pruning}~\cite{journals/corr/LiKDSG16}~\cite{luo2017thinet} is a widely explored approach to optimizing executing CNN models.% The key of this approach is selecting the ``proper'' weights to prune or compress.% A recent effort~\cite{journals/corr/YangCS16a} proposed to preferentially prune the weights of nodes that are predicted to be energy-hungry. 
Complementary to these techniques, \framework speeds up mobile deep vision through systematically exploiting temporal locality in input data, across multiple inference tasks.
\sys{} can coexist with these techniques in one engine.
%All the above techniques can be leveraged atop \framework~to optimize the DL processing collaboratively.

\section*{Acknowledgment}
This work was supported by the National Key R\&D Program under the grant number 2018YFB1004801, the National Natural Science Foundation of China under grant numbers 61725201, 61528201, 61529201, and a Google Faculty Award. 
% --- already noted in the front page --- %Xuanzhe Liu is the corresponding author of this work. 

\bibliographystyle{ACM-Reference-Format}
\balance
\bibliography{secs/ref}

\end{document}
