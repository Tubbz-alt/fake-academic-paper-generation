
% *** Authors should verify (and, if needed, correct) their LaTeX system  ***
% *** with the testflow diagnostic prior to trusting their LaTeX platform ***
% *** with production work. IEEE's font choices and paper sizes can       ***
% *** trigger bugs that do not appear when using other class files.       ***                          ***
% The testflow support page is at:
% http://www.michaelshell.org/tex/testflow/



\documentclass[journal]{IEEEtran}
%
% If IEEEtran.cls has not been installed into the LaTeX system files,
% manually specify the path to it like:
% \documentclass[journal]{../sty/IEEEtran}





% Some very useful LaTeX packages include:
% (uncomment the ones you want to load)


% *** MISC UTILITY PACKAGES ***
%
\usepackage{ifpdf}
% Heiko Oberdiek's ifpdf.sty is very useful if you need conditional
% compilation based on whether the output is pdf or dvi.
% usage:
% \ifpdf
%   % pdf code
% \else
%   % dvi code
% \fi
% The latest version of ifpdf.sty can be obtained from:
% http://www.ctan.org/tex-archive/macros/latex/contrib/oberdiek/
% Also, note that IEEEtran.cls V1.7 and later provides a builtin
% \ifCLASSINFOpdf conditional that works the same way.
% When switching from latex to pdflatex and vice-versa, the compiler may
% have to be run twice to clear warning/error messages.






% *** CITATION PACKAGES ***
%
\usepackage[noadjust]{cite}
% cite.sty was written by Donald Arseneau
% V1.6 and later of IEEEtran pre-defines the format of the cite.sty package
% \cite{} output to follow that of IEEE. Loading the cite package will
% result in citation numbers being automatically sorted and properly
% "compressed/ranged". e.g., [1], [9], [2], [7], [5], [6] without using
% cite.sty will become [1], [2], [5]--[7], [9] using cite.sty. cite.sty's
% \cite will automatically add leading space, if needed. Use cite.sty's
% noadjust option (cite.sty V3.8 and later) if you want to turn this off
% such as if a citation ever needs to be enclosed in parenthesis.
% cite.sty is already installed on most LaTeX systems. Be sure and use
% version 5.0 (2009-03-20) and later if using hyperref.sty.
% The latest version can be obtained at:
% http://www.ctan.org/tex-archive/macros/latex/contrib/cite/
% The documentation is contained in the cite.sty file itself.






% *** GRAPHICS RELATED PACKAGES ***
%
\ifCLASSINFOpdf
   \usepackage[pdftex]{graphicx}
  % declare the path(s) where your graphic files are
  % \graphicspath{{../pdf/}{../jpeg/}}
  % and their extensions so you won't have to specify these with
  % every instance of \includegraphics
  % \DeclareGraphicsExtensions{.pdf,.jpeg,.png}
\else
  % or other class option (dvipsone, dvipdf, if not using dvips). graphicx
  % will default to the driver specified in the system graphics.cfg if no
  % driver is specified.
  % \usepackage[dvips]{graphicx}
  % declare the path(s) where your graphic files are
  % \graphicspath{{../eps/}}
  % and their extensions so you won't have to specify these with
  % every instance of \includegraphics
  % \DeclareGraphicsExtensions{.eps}
\fi
% graphicx was written by David Carlisle and Sebastian Rahtz. It is
% required if you want graphics, photos, etc. graphicx.sty is already
% installed on most LaTeX systems. The latest version and documentation
% can be obtained at: 
% http://www.ctan.org/tex-archive/macros/latex/required/graphics/
% Another good source of documentation is "Using Imported Graphics in
% LaTeX2e" by Keith Reckdahl which can be found at:
% http://www.ctan.org/tex-archive/info/epslatex/
%
% latex, and pdflatex in dvi mode, support graphics in encapsulated
% postscript (.eps) format. pdflatex in pdf mode supports graphics
% in .pdf, .jpeg, .png and .mps (metapost) formats. Users should ensure
% that all non-photo figures use a vector format (.eps, .pdf, .mps) and
% not a bitmapped formats (.jpeg, .png). IEEE frowns on bitmapped formats
% which can result in "jaggedy"/blurry rendering of lines and letters as
% well as large increases in file sizes.
%
% You can find documentation about the pdfTeX application at:
% http://www.tug.org/applications/pdftex


\usepackage[table,xcdraw]{xcolor}
\usepackage{multirow}

% *** MATH PACKAGES ***
%
\usepackage[cmex10]{amsmath}
% A popular package from the American Mathematical Society that provides
% many useful and powerful commands for dealing with mathematics. If using
% it, be sure to load this package with the cmex10 option to ensure that
% only type 1 fonts will utilized at all point sizes. Without this option,
% it is possible that some math symbols, particularly those within
% footnotes, will be rendered in bitmap form which will result in a
% document that can not be IEEE Xplore compliant!
%
% Also, note that the amsmath package sets \interdisplaylinepenalty to 10000
% thus preventing page breaks from occurring within multiline equations. Use:
%\interdisplaylinepenalty=2500
% after loading amsmath to restore such page breaks as IEEEtran.cls normally
% does. amsmath.sty is already installed on most LaTeX systems. The latest
% version and documentation can be obtained at:
% http://www.ctan.org/tex-archive/macros/latex/required/amslatex/math/





% *** SPECIALIZED LIST PACKAGES ***
%
\usepackage{algorithmic}
% algorithmic.sty was written by Peter Williams and Rogerio Brito.
% This package provides an algorithmic environment fo describing algorithms.
% You can use the algorithmic environment in-text or within a figure
% environment to provide for a floating algorithm. Do NOT use the algorithm
% floating environment provided by algorithm.sty (by the same authors) or
% algorithm2e.sty (by Christophe Fiorio) as IEEE does not use dedicated
% algorithm float types and packages that provide these will not provide
% correct IEEE style captions. The latest version and documentation of
% algorithmic.sty can be obtained at:
% http://www.ctan.org/tex-archive/macros/latex/contrib/algorithms/
% There is also a support site at:
% http://algorithms.berlios.de/index.html
% Also of interest may be the (relatively newer and more customizable)
% algorithmicx.sty package by Szasz Janos:
% http://www.ctan.org/tex-archive/macros/latex/contrib/algorithmicx/




% *** ALIGNMENT PACKAGES ***
%
\usepackage{array}
% Frank Mittelbach's and David Carlisle's array.sty patches and improves
% the standard LaTeX2e array and tabular environments to provide better
% appearance and additional user controls. As the default LaTeX2e table
% generation code is lacking to the point of almost being broken with
% respect to the quality of the end results, all users are strongly
% advised to use an enhanced (at the very least that provided by array.sty)
% set of table tools. array.sty is already installed on most systems. The
% latest version and documentation can be obtained at:
% http://www.ctan.org/tex-archive/macros/latex/required/tools/


% IEEEtran contains the IEEEeqnarray family of commands that can be used to
% generate multiline equations as well as matrices, tables, etc., of high
% quality.




% *** SUBFIGURE PACKAGES ***
%\ifCLASSOPTIONcompsoc
%  \usepackage[caption=false,font=normalsize,labelfont=sf,textfont=sf]{subfig}
%\else
%  \usepackage[caption=false,font=footnotesize]{subfig}
%\fi
% subfig.sty, written by Steven Douglas Cochran, is the modern replacement
% for subfigure.sty, the latter of which is no longer maintained and is
% incompatible with some LaTeX packages including fixltx2e. However,
% subfig.sty requires and automatically loads Axel Sommerfeldt's caption.sty
% which will override IEEEtran.cls' handling of captions and this will result
% in non-IEEE style figure/table captions. To prevent this problem, be sure
% and invoke subfig.sty's "caption=false" package option (available since
% subfig.sty version 1.3, 2005/06/28) as this is will preserve IEEEtran.cls
% handling of captions.
% Note that the Computer Society format requires a larger sans serif font
% than the serif footnote size font used in traditional IEEE formatting
% and thus the need to invoke different subfig.sty package options depending
% on whether compsoc mode has been enabled.
%
% The latest version and documentation of subfig.sty can be obtained at:
% http://www.ctan.org/tex-archive/macros/latex/contrib/subfig/

\usepackage{caption}
\usepackage{subcaption}



% *** FLOAT PACKAGES ***
%
\usepackage{fixltx2e}
% fixltx2e, the successor to the earlier fix2col.sty, was written by
% Frank Mittelbach and David Carlisle. This package corrects a few problems
% in the LaTeX2e kernel, the most notable of which is that in current
% LaTeX2e releases, the ordering of single and double column floats is not
% guaranteed to be preserved. Thus, an unpatched LaTeX2e can allow a
% single column figure to be placed prior to an earlier double column
% figure. The latest version and documentation can be found at:
% http://www.ctan.org/tex-archive/macros/latex/base/


\usepackage{stfloats}
% stfloats.sty was written by Sigitas Tolusis. This package gives LaTeX2e
% the ability to do double column floats at the bottom of the page as well
% as the top. (e.g., "\begin{figure*}[!b]" is not normally possible in
% LaTeX2e). It also provides a command:
%\fnbelowfloat
% to enable the placement of footnotes below bottom floats (the standard
% LaTeX2e kernel puts them above bottom floats). This is an invasive package
% which rewrites many portions of the LaTeX2e float routines. It may not work
% with other packages that modify the LaTeX2e float routines. The latest
% version and documentation can be obtained at:
% http://www.ctan.org/tex-archive/macros/latex/contrib/sttools/
% Do not use the stfloats baselinefloat ability as IEEE does not allow
% \baselineskip to stretch. Authors submitting work to the IEEE should note
% that IEEE rarely uses double column equations and that authors should try
% to avoid such use. Do not be tempted to use the cuted.sty or midfloat.sty
% packages (also by Sigitas Tolusis) as IEEE does not format its papers in
% such ways.
% Do not attempt to use stfloats with fixltx2e as they are incompatible.
% Instead, use Morten Hogholm'a dblfloatfix which combines the features
% of both fixltx2e and stfloats:
%
% \usepackage{dblfloatfix}
% The latest version can be found at:
% http://www.ctan.org/tex-archive/macros/latex/contrib/dblfloatfix/




%\ifCLASSOPTIONcaptionsoff
%  \usepackage[nomarkers]{endfloat}
% \let\MYoriglatexcaption\caption
% \renewcommand{\caption}[2][\relax]{\MYoriglatexcaption[#2]{#2}}
%\fi
% endfloat.sty was written by James Darrell McCauley, Jeff Goldberg and 
% Axel Sommerfeldt. This package may be useful when used in conjunction with 
% IEEEtran.cls'  captionsoff option. Some IEEE journals/societies require that
% submissions have lists of figures/tables at the end of the paper and that
% figures/tables without any captions are placed on a page by themselves at
% the end of the document. If needed, the draftcls IEEEtran class option or
% \CLASSINPUTbaselinestretch interface can be used to increase the line
% spacing as well. Be sure and use the nomarkers option of endfloat to
% prevent endfloat from "marking" where the figures would have been placed
% in the text. The two hack lines of code above are a slight modification of
% that suggested by in the endfloat docs (section 8.4.1) to ensure that
% the full captions always appear in the list of figures/tables - even if
% the user used the short optional argument of \caption[]{}.
% IEEE papers do not typically make use of \caption[]'s optional argument,
% so this should not be an issue. A similar trick can be used to disable
% captions of packages such as subfig.sty that lack options to turn off
% the subcaptions:
% For subfig.sty:
% \let\MYorigsubfloat\subfloat
% \renewcommand{\subfloat}[2][\relax]{\MYorigsubfloat[]{#2}}
% However, the above trick will not work if both optional arguments of
% the \subfloat command are used. Furthermore, there needs to be a
% description of each subfigure *somewhere* and endfloat does not add
% subfigure captions to its list of figures. Thus, the best approach is to
% avoid the use of subfigure captions (many IEEE journals avoid them anyway)
% and instead reference/explain all the subfigures within the main caption.
% The latest version of endfloat.sty and its documentation can obtained at:
% http://www.ctan.org/tex-archive/macros/latex/contrib/endfloat/
%
% The IEEEtran \ifCLASSOPTIONcaptionsoff conditional can also be used
% later in the document, say, to conditionally put the References on a 
% page by themselves.




% *** PDF, URL AND HYPERLINK PACKAGES ***
%
\usepackage{url}
% url.sty was written by Donald Arseneau. It provides better support for
% handling and breaking URLs. url.sty is already installed on most LaTeX
% systems. The latest version and documentation can be obtained at:
% http://www.ctan.org/tex-archive/macros/latex/contrib/url/
% Basically, \url{my_url_here}.




% *** Do not adjust lengths that control margins, column widths, etc. ***
% *** Do not use packages that alter fonts (such as pslatex).         ***
% There should be no need to do such things with IEEEtran.cls V1.6 and later.
% (Unless specifically asked to do so by the journal or conference you plan
% to submit to, of course. )


% correct bad hyphenation here
\hyphenation{op-tical net-works semi-conduc-tor}


\begin{document}
%
% paper title
% Titles are generally capitalized except for words such as a, an, and, as,
% at, but, by, for, in, nor, of, on, or, the, to and up, which are usually
% not capitalized unless they are the first or last word of the title.
% Linebreaks \\ can be used within to get better formatting as desired.
% Do not put math or special symbols in the title.
\title{EddyNet: A Deep Neural Network For Pixel-Wise Classification of Oceanic Eddies}
%
%
% author names and IEEE memberships
% note positions of commas and nonbreaking spaces ( ~ ) LaTeX will not break
% a structure at a ~ so this keeps an author's name from being broken across
% two lines.
% use \thanks{} to gain access to the first footnote area
% a separate \thanks must be used for each paragraph as LaTeX2e's \thanks
% was not built to handle multiple paragraphs
%

\author{Redouane~Lguensat,~\IEEEmembership{Member,~IEEE,}
		Miao~Sun, %~\IEEEmembership{Member,~IEEE,}       
        Ronan~Fablet,~\IEEEmembership{Senior~Member,~IEEE,}  
        Evan~Mason, %~\IEEEmembership{Member,~IEEE,}
        Pierre~Tandeo, %~\IEEEmembership{Member,~IEEE,}
        and~Ge~Chen% <-this % stops a space
\thanks{R. Lguensat and R. Fablet and P. Tandeo are with IMT Atlantique; UBL; Lab-STICC; 29200 Brest, France. E-mail: redouane.lguensat@imt-atlantique.fr.}% <-this % stops a space
\thanks{M. Sun is with the National Marine Data and Information Service; Key Laboratory of Digital Ocean; 300171 Tianjing, China.}% <-this % stops a space
\thanks{G. Chen is with Department of Marine Information Technology; College of Information Science and Engineering; Ocean University of China; 266000 Qingdao, China.}% <-this % stops a space
\thanks{E. Mason is with the Mediterranean Institute for Advanced Studies (IMEDEA/CSIC-UIB), 07190 Esporles - Balearic Islands, Spain.}% <-this % stops a space

%\thanks{Manuscript received xxxx; revised xxxx.}
}

% note the % following the last \IEEEmembership and also \thanks - 
% these prevent an unwanted space from occurring between the last author name
% and the end of the author line. i.e., if you had this:
% 
% \author{....lastname \thanks{...} \thanks{...} }
%                     ^------------^------------^----Do not want these spaces!
%
% a space would be appended to the last name and could cause every name on that
% line to be shifted left slightly. This is one of those "LaTeX things". For
% instance, "\textbf{A} \textbf{B}" will typeset as "A B" not "AB". To get
% "AB" then you have to do: "\textbf{A}\textbf{B}"
% \thanks is no different in this regard, so shield the last } of each \thanks
% that ends a line with a % and do not let a space in before the next \thanks.
% Spaces after \IEEEmembership other than the last one are OK (and needed) as
% you are supposed to have spaces between the names. For what it is worth,
% this is a minor point as most people would not even notice if the said evil
% space somehow managed to creep in.



% The paper headers
\markboth{Submitted}%
{Lguensat \MakeLowercase{\textit{et al.}}: Bare Demo of IEEEtran.cls for Journals}
% The only time the second header will appear is for the odd numbered pages
% after the title page when using the twoside option.
% 
% *** Note that you probably will NOT want to include the author's ***
% *** name in the headers of peer review papers.                   ***
% You can use \ifCLASSOPTIONpeerreview for conditional compilation here if
% you desire.




% If you want to put a publisher's ID mark on the page you can do it like
% this:
%\IEEEpubid{0000--0000/00\$00.00~\copyright~2014 IEEE}
% Remember, if you use this you must call \IEEEpubidadjcol in the second
% column for its text to clear the IEEEpubid mark.



% use for special paper notices
%\IEEEspecialpapernotice{(Invited Paper)}




% make the title area
\maketitle

% As a general rule, do not put math, special symbols or citations
% in the abstract or keywords.
\begin{abstract}
This work presents EddyNet, a deep learning based architecture for automated eddy detection and classification from Sea Surface Height (SSH) maps provided by the Copernicus Marine and Environment Monitoring Service (CMEMS). EddyNet consists of a convolutional encoder-decoder followed by a pixel-wise classification layer. The output is a map with the same size of the input where pixels have the following labels \{'0': Non eddy, '1': anticyclonic eddy, '2': cyclonic eddy\}. Keras Python code, the training datasets and EddyNet weights files are open-source and freely available on \url{https://github.com/redouanelg/EddyNet}.

\end{abstract}

% Note that keywords are not normally used for peerreview papers.
\begin{IEEEkeywords}
Mesoscale eddy, Segmentation, Classification, Deep learning, Convolutional Neural Networks.
\end{IEEEkeywords}






% For peer review papers, you can put extra information on the cover
% page as needed:
% \ifCLASSOPTIONpeerreview
% \begin{center} \bfseries EDICS Category: 3-BBND \end{center}
% \fi
%
% For peerreview papers, this IEEEtran command inserts a page break and
% creates the second title. It will be ignored for other modes.
\IEEEpeerreviewmaketitle



\section{Introduction}
% The very first letter is a 2 line initial drop letter followed
% by the rest of the first word in caps.
% 
% form to use if the first word consists of a single letter:
% \IEEEPARstart{A}{demo} file is ....
% 
% form to use if you need the single drop letter followed by
% normal text (unknown if ever used by IEEE):
% \IEEEPARstart{A}{}demo file is ....
% 
% Some journals put the first two words in caps:
% \IEEEPARstart{T}{his demo} file is ....
% 
% Here we have the typical use of a "T" for an initial drop letter
% and "HIS" in caps to complete the first word.
\IEEEPARstart{G}{oing} "deeper" with artificial neural networks (ANNs) by using more than the original three layers (input, hidden, output) started the so-called deep learning era. The developments and discoveries which are still ongoing are producing impressive results and reaching state-of-the-art performances in various fields. The reader is invited to read \cite{goodfellow2016deep} for a general introduction to deep learning. In particular, Convolutional Neural Networks (CNN) sparked-off the deep learning revolution in the image processing community and are now ubiquitous in computer vision applications. This has led numerous researchers from the remote sensing community to investigate the use of this powerful tool for tasks like object recognition, scene classification, etc... More applications of deep learning for remote sensing data can be found in \cite{zhang2016deep,DLforRS} and references therein.

By standing on the shoulders of recent achievements in deep learning for image segmentation we present "EddyNet", a deep neural network for automated eddy detection and classification from Sea Surface Height (SSH) maps provided by the Copernicus Marine and Environment Monitoring Service (hereinafter denoted by AVISO-SSH). EddyNet is inspired by ideas from widely used image segmentation architectures, in particular U-shaped architectures such as U-Net \cite{ronneberger2015u}. We investigate the use of Scaled Exponential Linear Units (SELU) \cite{klambauer2017self,clevert2015fast} instead of the classical ReLU + Batch Normalization (R+BN) and show that we greatly speed up the training process while reaching comparable results. We adopt a loss function based on the Dice coefficient (also known as the F1 measure) and illustrate that we reach better scores for the two most relevant classes (cyclonic and anticyclonic) than with using the categorical cross-entropy loss. We also supplement dropout layers to our architecture that prevents EddyNet from overfitting.

Our work joins the emerging cross-fertilization between the remote sensing and machine learning communities that is leading to significant contributions in addressing the segmentation of remote sensing images \cite{maggiori,audebert2016semantic,volpi2017dense}. To the best of our knowledge, the present work is the first to propose a deep learning based architecture for pixel-wise classification of eddies, dealing with the challenges of this particular type of data.

This letter is organized as follows: Section \uppercase\expandafter{\romannumeral2} presents the eddy detection and classification problem and related work. Section \uppercase\expandafter{\romannumeral3} describes the data preparation process. Section \uppercase\expandafter{\romannumeral4} presents the architecture of EddyNet and details the training process. Section \uppercase\expandafter{\romannumeral5} reports the different experiments considered in this work and discusses the results. Our conclusion and future work directions are finally stated in Section \uppercase\expandafter{\romannumeral6}. 

\section{Problem statement and related work}

Ocean mesoscale eddies can be defined as rotating water masses, they are omnipresent in the ocean and carry critical information about large-scale ocean circulation \cite{holland1978role,chelton2011global}. Eddies transport different relevant physical quantities such as carbon, heat, phytoplankton, salt, etc. This movement helps in regulating the weather and mixing the ocean \cite{mcwilliams2008nature}. Detecting and studying eddies helps also considering their effects in ocean climate models \cite{le2011parameterization}. With the development of altimeter missions and since the availability of two or more altimeters at the same time, merged products of Sea Surface Height (SSH) reached a sufficient resolution to allow the detection of mesoscale eddies \cite{faghmous2015daily,pascual2006improved}. SSH maps allow us distinguish two classes of eddies: i) anticyclonic eddies that are recognized by their positive SLA (Sea Level Anomaly which is SSH anomaly with regard to a given mean) and ii) cyclonic eddies that are characterized by their negative SLA.
% * <evanmason@gmail.com> 2017-11-10T14:05:26.481Z:
% 
% > cite
% Cite here Pascual et al 2006
% Pascual, A., Y. Faugère, G. Larnicol, and P.-Y. Le Traon (2006), Improved description of the ocean mesoscale variability by combining four satellite altimeters, Geophys. Res. Lett., 33, L02611, doi:10.1029/2005GL024633.
% 
% ^.
 
In recent years, several studies were conducted with the aim of detecting and classifying eddies in an automated fashion \cite{faghmous2012eddyscan}. Two major families of methods prevail in the literature, namely, physical parameter-based methods and geometrical contour-based methods. The most popular representative of physical parameter-based methods is the Okubo-Weiss parameter method \cite{okubo1970horizontal,weiss1991dynamics}. The Okubo-Weiss parameter method is however criticized for its expert-based and region-specific parameters and also for its sensitivity to noisy SSH maps \cite{chelton2007global}. Other methods were since then developed using other techniques such as wavelet decomposition \cite{turiel2007wavelet}, winding angle \cite{sadarjoen1999geometric}, etc. Geometric-based methods rely on considering the eddies as elliptic shapes and use closed contour techniques, the most popular method remains Chelton et al. method \cite{chelton2011global} (hereinafter called CSS11). Methods that combines ideas from both worlds are called hybrid methods (e.g. \cite{yi2014enhancing,isern2003identification}). Machine learning methods were also used in the past to propose a solution to the problem \cite{castellani2006identification,hai2008automatic}, recently they are again getting an increasing attention \cite{ashkezari2016oceanic,deepeddy}.  
% * <evanmason@gmail.com> 2017-11-10T14:10:13.190Z:
% 
% Include here:
% Isern-Fontanet, J., E. García-Ladona, and J. Font, 2003: Identification of Marine Eddies from Altimetric Maps. J. Atmos. Oceanic Technol., 20, 772–778, https://doi.org/10.1175/1520-0426(2003)20<772:IOMEFA>2.0.CO;2 
% 
% Chelton etal 2011 https://doi.org/10.1016/j.pocean.2011.01.002
% 
% Chelton etal 2007:
%  Chelton, D. B., M. G. Schlax, R. M. Samelson, and R. A. de Szoeke (2007), Global observations of large oceanic eddies, Geophys. Res. Lett., 34, L15606, doi:10.1029/2007GL030812.
% 
% ^.

We propose in this work to benefit from the advances in deep learning to address ocean eddy detection and classification. Our proposed deep learning based method requires a training database consisting of SSH maps and their corresponding eddy detection and classification results. In this work, we train our deep learning methods from the results of the \textit{py-eddy-tracker} SSH-based approach (hereinafter PET14) \cite{mason2014new}, the algorithm developed by Mason et al. is closely related to CSS11 but has some significant differences such as not allowing multiple local extremum in an eddy. An example of a PET14 result is given in Figure \ref{fig:eddies} which shows eddies identified in the southwest Atlantic (see \cite{mason17}). 
% * <evanmason@gmail.com> 2017-11-10T14:16:13.565Z:
% 
% > Mason etal 2017
% Mason, E., A. Pascual, P. Gaube, S. Ruiz, J. L. Pelegrí, and A. Delepoulle (2017), Subregional characterization of mesoscale eddies across the Brazil-Malvinas Confluence, J. Geophys. Res. Oceans, 122, 3329–3357, doi:10.1002/2016JC012611.
% 
% ^.
The outputs of the eddy tracker algorithm provide the center coordinates of each classified eddy along with its speed and effective contours. Since we aim for a pixelwise classification, i.e., each pixel is classified, we transform the outputs into segmentation maps such as the example shown in Figure \ref{fig:exampletraining}. We consider here the speed contour which corresponds to the closed contour that has the highest mean geostrophic rotational current. The speed contour can be seen as the most energetic part of the eddy and is usually smaller than the effective radius. The next section describes further the data preparation process that yields the training database of pixelwise classification maps.

\begin{figure}[t]
\centering
\includegraphics[width=5cm]{eddies.png}
\caption{A snapshot of a SSH map from the Southern Atlantic Ocean with the detected eddies by PET14 algorithm, red shapes represent anticyclonic eddies while green shapes are cyclonic eddies}
\label{fig:eddies}
\end{figure}

\section{Data preparation}

As stated in the previous section, we consider PET14 outputs as a training database for our deep-neural-network based algorithms. We use 15 years (1998-2012) of daily detected and classified eddies. The corresponding SSH maps (AVISO-SSH) are provided by the Copernicus Marine Environment Monitoring Service (CMEMS). The resolution of the SSH maps is 0.25$^\circ$.

Due to memory constraints, the input image of our architectures is $128 \times 128$ pixels. The first 14 years are used as a training dataset and the last year (2012) is left aside for testing our architecture. We consider the Southern Atlantic Ocean region depicted in Figure \ref{fig:eddies} and cut the top region where no eddies were detected. Then we randomly sample  one $128 \times 128$ patch from each SSH map, which leaves us with 5100 training samples. A significant property of this type of data is that its dynamics are slow, a single eddy can live for several days or even more than a year. In addition to the fact that a $128 \times 128$ patch can comprise several examples of cyclonic and anticyclonic eddies, we believe that data augmentation (adding rotated versions of the patches to the training database for example) is not needed; we observed experiments (not shown here) that even resulted in performance degradation. The next step consists of extracting the SSH $128 \times 128$ patches from AVISO-SSH. For land pixels or regions with no data we replaced the standard fill value by a zero; this helps to  avoid outliers and does not affect detection since eddies are located in regions with non zero SSH. The final and essential step is the creation of the segmentation masks of the training patches. This is done by creating polygon shapes using the speed contour coordinates mapped onto the nearest lattices in the AVISO-SSH 0.25$^\circ$ grid. Pixels inside each polygon are then labeled with the class of the polygon representing the eddy \{'0': Non eddy/land/no data, '1': anticyclonic eddy, '2': cyclonic eddy\}. An example of the coupled \{SSH map, segmentation map\} from the training dataset is given in Figure \ref{fig:exampletraining}.

\begin{figure}[t]
\centering
\includegraphics[scale=0.3]{exampletraining}
\caption{Example of a SSH-Segmentation training couple, anticyclonic (green), cyclonic (brown), non eddy (blue)}
\label{fig:exampletraining}
\end{figure}

\section{Our proposed method}
\subsection{EddyNet architecture}
The EddyNet architecture is based on the U-net architecture \cite{ronneberger2015u}. It starts with an encoding (downsampling) path with 3 stages, where each stage consists of two $3\times 3$ convolutional layers followed by either a Scaled Exponential Linear Unit (SELU) activation function \cite{klambauer2017self} (referred to as EddyNet\_S) or by the classical ReLU activation + Batch Normalization (referred to as EddyNet), then a $2\times 2$ max pooling layer that halves the resolution of the input. The decoding (upsampling) path uses transposed convolutions (also called deconvolutions) \cite{zeiler2010deconvolutional} to return to the original resolution. Like U-net, Eddynet benefits from skip connections from the contracting path to the expanding path to account for information originating from early stages. Preliminary experiments with the original architecture of U-Net showed a severe overfitting given the low number of training samples compared to the capacity of the architecture. Numerous attempts and hyperparameter tuning led us to finally settle on a 3-stage all-32-filter architecture as shown in Figure \ref{fig:eddynet}. EddyNet has the benefit of having a small number of parameters compared to widely used architecture, thus resulting in low memory consumption. Our neural network can still overfit the data which shows that it can capture the nonlinear inverse problem of eddy detection and classification. Hence, we add dropout layers before each max pooling layer and before each transposed convolutional layer; we chose these positions since they are the ones involved in the concatenations where the highest number of filters (64) is present. Dropout layers helped to regularize the network and boosted the validation loss performance. 
% * <evanmason@gmail.com> 2017-11-10T14:26:50.878Z:
% 
% > max
% maximum?
% 
% ^.
Regarding EddyNet\_S, we mention three essential considerations: i) The weight initialization is different than with EddyNet, we detail this aspect in the experiment section. ii) The theory behind the SELU activation function stands on the self-normalizing property which aims to keep the inputs close to a zero mean and unit variance through the network layers. Classical dropout that randomly sets units to zero could harm this property; \cite{klambauer2017self} propose therefore a new dropout technique called AlphaDropout that addresses this problem by randomly setting activations on the negative saturation value. iii) SELU theory is originally derived for Feed Forward Networks, applying them to CNNs needs careful setting. In preliminary experiments, using our U-net like architecture with SELU activations resulted in a very noisy loss that even explodes sometimes. We think this could be caused by the skip connections that can violate the self-normalizing property desired by the SELU, and hence decided to keep Batch Normalization in EddyNet\_S after each of the maxpooling, transposed convolution and concatenation layers. 

\begin{figure}[t]
\centering
\includegraphics[scale=0.5]{unetschema}
\caption{EddyNet architecture}
\label{fig:eddynet}
\end{figure}

\subsection{Loss metric}
While multiclass classification problems in deep learning are generally trained using the categorical cross-entropy cost function, segmentation problems favor the use of overlap based metrics. The dice coefficient is a popular and largely used cost function in segmentation problems. Considering the predicted region $P$ and the groundtruth region $G$, and by denoting $|P|$ and $|G|$ the sum of elements in each area, the dice coefficient is twice the ratio of the intersection over the sum of areas:
\begin{equation}
\text{DiceCoef}(P,G)=\frac{2 |P \cap G|}{|P|+ |G|}.
\end{equation}
 A perfect segmentation result is given by a dice coefficient of 1, while a dice coefficient of 0 refers to a completely mistaken segmentation. Seeing it from a F1-measure perspective, the dice coefficient is the harmonic mean of the precision and recall metrics.

The implementation uses one-hot encoding vectors, an essential detail is that the loss function of EddyNet uses a soft and differentiable version of the dice coefficient which considers the output of the softmax layer as it is without binarization:
\begin{equation}
\text{softDiceCoef}(P,G)=\frac{2 \sum_i p_i * g_i}{\sum_i p_i+ \sum_i g_i},
\end{equation}
where the $p_i$ are the probabilities given by the softmax layer $0\leq p_i \leq 1$, and the $g_i$ are either 1 for the correct class and 0 either. We found later that a recent study used another version of a soft dice loss \cite{milletari2016v}; a comparison of both versions is out of the scope of this work.

Since we are in the context of a multiclass classification problem, we try to maximize the performance of our network using the mean of three one-vs-all soft dice coefficients of each class. The loss function that our neural network aims to minimize is then simply:
\begin{equation}\label{eq:diceloss}
\text{Dice Loss}= 1 - \text{softMeanDiceCoef}
\end{equation}
%%%%%%%%%%%%%%%%%%%%%%%%%%%%%%%%%%%%%%%%%%%%%%%%%%%%%%%%%%%%%%%%%%%%%%%%%%%%%%%%%%%%%%%%%%%%%%%%%%%%%%%%%%%%%%%%%%%%%%%%%%%%%%%%%%%%%%%%%%%%%%%%%%%%%%%%%%%%%%%%%%%%%%%%%%%%%%%%%%%%%%%%%%%%%%%%%%%%%%%%%%%%
\section{Experiments}
\subsection{Assessment of the performance}
Keras framework \cite{chollet2015keras} with a Tensorflow backend is considered in this work. EddyNet is trained on a Nvidia K80 GPU card using ADAM optimizer \cite{kingma2014adam} and mini-batches of 16 maps. The weights were initialized using truncated Gaussian distributed weights of zero mean and \{2/number of input units\} variance \cite{he2015delving} for EddyNet, while we use weights drawn from a truncated Gaussian distribution of zero mean and \{1/number of input units\} variance for EddyNet\_S. The training dataset is split into 4080 images for training and 1020 for validation. We also use an early-stopping strategy to stop the learning process when the validation dataset loss stops improving in five consecutive epochs. EddyNet weights are then the ones resulting in the lowest validation loss value.

EddyNet and EddyNet\_S are then compared regarding the use of the classical ReLU+BN and the use of SELU. We also compare the use of overlap based metric represented by the Dice Loss (Equation \ref{eq:diceloss}), with the classical Categorical Cross-Entropy (CCE). Table \ref{tab:eddynet_results} compares the four combination in terms of global accuracy and mean dice coefficient (original not soft) averaged on 50 random sets of 360 SSH $120\times 120$ maps from 2012. Training EddyNet\_S takes nearly half the time needed for training EddyNet. Comparison regarding the training loss function shows that training with the dice loss results in a higher dice coefficient for our two classes of interest (cyclonic and anticyclonic) in both EddyNet and EddyNet\_S; dice loss yields a better overall mean dice coefficient than training with CCE loss. Regarding the effect of the activation function, we obtained better metrics with EddyNet at the cost of a longer training procedure. Visually Eddynet and EddyNet\_S give close outputs as can be seen in Figure \ref{fig:exampleeddynet}.

\begin{figure}
    \centering
    \begin{subfigure}[b]{0.3\textwidth}
       \centering \includegraphics[width=\textwidth]{example_eddynet}
       \caption{}\label{fig:example1}
    \end{subfigure}
%\unskip\ \vrule\
\begin{subfigure}[b]{0.3\textwidth}
       \centering 
       \includegraphics[width=\textwidth]{example_eddynet3}
       \caption{}\label{fig:example2}
    \end{subfigure}
\caption{Examples of the eddy segmentation results using Eddynet and EddyNet\_S: anticyclonic eddies (green), cyclonic (brown), non eddy (blue)}
\label{fig:exampleeddynet}
\end{figure}

\begin{table*}[t]
\centering
\caption{Metrics calculated from the results of 50 random sets of 360 SSH patches from the test dataset, we report the mean value and put the standard variation between parenthesis.}
\label{tab:eddynet_results}
\begin{tabular}{llll|l|l|l|ll}
\cline{5-7}
                                                        &                                         &                                         &                                        & Anticyclonic             & Cyclonic                 & Non Eddy                 &                                               &                                               \\ \cline{2-9} 
\multicolumn{1}{l|}{}                                   & \multicolumn{1}{l|}{\#Param}       & \multicolumn{1}{l|}{Epoch time}     & Train loss                          & \multicolumn{3}{c|}{Dice Coef}                                                & \multicolumn{1}{l|}{Mean Dice Coef}          & \multicolumn{1}{l|}{Global Accuracy}          \\ \hline
\multicolumn{1}{|l|}{}                                  & \multicolumn{1}{l|}{177,571}                   & \multicolumn{1}{l|}{}                   & \cellcolor[HTML]{FFCE93}Dice Loss      & \cellcolor[HTML]{FFCE93}\textbf{0.708} (0.002) & \cellcolor[HTML]{FFCE93}\textbf{0.677} (0.001) & \cellcolor[HTML]{FFCE93}0.929 (0.001) & \multicolumn{1}{l|}{\cellcolor[HTML]{FFCE93}\textbf{0.772 }(0.001)} & \multicolumn{1}{l|}{\cellcolor[HTML]{FFCE93}88.60\% (0.10\%)} \\ \cline{4-9} 
\multicolumn{1}{|l|}{\multirow{-2}{*}{EddyNet}} & \multicolumn{1}{l|}{}                   & \multicolumn{1}{l|}{\multirow{-2}{*}{$\sim$12 min}} & \cellcolor[HTML]{FFFFC7}CCE & \cellcolor[HTML]{FFFFC7}0.695 (0.003) & \cellcolor[HTML]{FFFFC7}0.651 (0.001) & \cellcolor[HTML]{FFFFC7}0.940 (0.001) & \multicolumn{1}{l|}{\cellcolor[HTML]{FFFFC7}0.762 (0.001)} & \multicolumn{1}{l|}{\cellcolor[HTML]{FFFFC7}89.92\% (0.07\%)} \\ \cline{1-1} \cline{3-9} 
\multicolumn{1}{|l|}{}                                  & \multicolumn{1}{l|}{}                   & \multicolumn{1}{l|}{$\sim$7 min}                   & \cellcolor[HTML]{FFCE93}Dice Loss      & \cellcolor[HTML]{FFCE93}0.694 (0.003) & \cellcolor[HTML]{FFCE93}0.665 (0.001) & \cellcolor[HTML]{FFCE93}0.933 (0.001) & \multicolumn{1}{l|}{\cellcolor[HTML]{FFCE93}0.764 (0.001)} & \multicolumn{1}{l|}{\cellcolor[HTML]{FFCE93}88.98\% (0.09\%)} \\ \cline{4-9} 
\multicolumn{1}{|l|}{\multirow{-2}{*}{EddyNet\_S}}    & \multicolumn{1}{l|}{\multirow{-4}{*}{}} & \multicolumn{1}{l|}{\multirow{-2}{*}{}} & \cellcolor[HTML]{FFFFC7}CCE & \cellcolor[HTML]{FFFFC7}0.682 (0.002) & \cellcolor[HTML]{FFFFC7}0.653 (0.002) & \cellcolor[HTML]{FFFFC7}0.939 (0.001) & \multicolumn{1}{l|}{\cellcolor[HTML]{FFFFC7}0.758 (0.001)} & \multicolumn{1}{l|}{\cellcolor[HTML]{FFFFC7}89.83\% (0.08\%)} \\ \hline
\end{tabular}
\end{table*}

\subsection{Ghost eddies}
The presence of ghost eddies is a frequent problem encountered in eddy detection and tracking algorithms \cite{faghmous2015daily}. Ghost eddies are eddies that are found by the detection algorithm then disappear between consecutive maps before reappearing again. To point out the position of the missed ghost eddies, PET14 uses linear temporal interpolation between centers of detected eddies and stores the positions of the centers of ghost eddies. Using EddyNet we check if the pixels of ghost eddy centers correspond to actual eddy detections. We found that EddyNet assigns the centers of ghost eddies to the correct eddy classes 55\% of the time for anticyclonic eddies, and 45\% for cyclonic eddies. EddyNet could be a relevant method to detect ghost eddies that are missed out by conventional methods. Figure \ref{fig:ghost} illustrates two examples of ghost eddy detection. 

\section{Conclusion}
This work investigates the use of recent developments in deep learning based image segmentation for an ocean remote sensing problem, namely, eddy detection and classification from Sea Surface Height (SSH) maps. We propose EddyNet, a deep neural network architecture inspired from architectures and ideas widely adopted in the computer vision community. We transfer successfully the knowledge gained to the problem of eddy classification by dealing with various challenges. Future work involves investigating the use of temporal volumes of SSH and deriving a 3D version inspired by the works of \cite{milletari2016v}. Adding other surface information such as Sea Surface Temperature might also help improving the detection. Another extension would be the application of EddyNet over the globe, and assessing its general capacity over other regions. Post-processing by constraining the eddies to verify additional criteria and tracking the eddies was omitted in this work and could also be developed in future work.  

Beyond the illustrative aspect of this contribution, we offer to the oceanic remote sensing community an easy and powerful tool that can save handcrafting model efforts. Any user can employ his own eddy segmentation "ground truth" and train the model from scratch if he/she has the necessary memory and computing resources, or simply use EddyNet provided weights as an initialization then perform fine-tuning using his/her dataset. One can also think of averaging results from classical contour-based methods and EddyNet. In the spirit of reproducibility, Python code is available at \url{https://github.com/redouanelg/eddynet}, and we also share the training and testing data used for this work to encourage competing methods and, especially, other deep learning architectures.


\begin{figure}
    \centering
    \begin{subfigure}[b]{0.5\textwidth}
       %\centering 
       \includegraphics[width=\textwidth]{figure_ghost}
       \caption{}\label{fig:ghost1}
    \end{subfigure}
    ~ 
    \begin{subfigure}[b]{0.5\textwidth}
       %\centering 
       \includegraphics[width=\textwidth]{figure_ghost2}
       \caption{}\label{fig:ghost2}
    \end{subfigure}
\caption{Detection of ghost eddies: [left] SSH map where ghost eddies centers are marked: anticyclonic (red dots), cyclonic (blue dots). [center] PET14 segmentation. [right] EddyNet segmentation: anticyclonic (green), cyclonic (brown), non eddy (blue)}
\label{fig:ghost}
\end{figure}


% use section* for acknowledgment
\section*{Acknowledgment}


The authors would like to thank Antoine Delepoulle, Bertrand Chapron and Julien Le Sommer for their constructive comments. This  work  was  supported  by  ANR (Agence Nationale de la Recherche, grant ANR-13-MONU-0014) and Labex Cominlabs (grant  SEACS). Evan Mason is supported by the Copernicus Marine Environment Monitoring Service (CMEMS) MedSUB project.


% Can use something like this to put references on a page
% by themselves when using endfloat and the captionsoff option.
\ifCLASSOPTIONcaptionsoff
  \newpage
\fi



% trigger a \newpage just before the given reference
% number - used to balance the columns on the last page
% adjust value as needed - may need to be readjusted if
% the document is modified later
%\IEEEtriggeratref{8}
% The "triggered" command can be changed if desired:
%\IEEEtriggercmd{\enlargethispage{-5in}}

% references section

% can use a bibliography generated by BibTeX as a .bbl file
% BibTeX documentation can be easily obtained at:
% http://www.ctan.org/tex-archive/biblio/bibtex/contrib/doc/
% The IEEEtran BibTeX style support page is at:
% http://www.michaelshell.org/tex/ieeetran/bibtex/
\bibliographystyle{IEEEtran}
% argument is your BibTeX string definitions and bibliography database(s)
\bibliography{AnDA_biblio,IEEEabrv,bibtex/bib/IEEEexample}
%
% <OR> manually copy in the resultant .bbl file
% set second argument of \begin to the number of references
% (used to reserve space for the reference number labels box)
%\begin{thebibliography}{1}

%\bibitem{IEEEhowto:kopka}
%H.~Kopka and P.~W. Daly, \emph{A Guide to \LaTeX}, 3rd~ed.\hskip 1em plus
%  0.5em minus 0.4em\relax Harlow, England: Addison-Wesley, 1999.

%\end{thebibliography}

% biography section
% 
% If you have an EPS/PDF photo (graphicx package needed) extra braces are
% needed around the contents of the optional argument to biography to prevent
% the LaTeX parser from getting confused when it sees the complicated
% \includegraphics command within an optional argument. (You could create
% your own custom macro containing the \includegraphics command to make things
% simpler here.)
%\begin{IEEEbiography}[{\includegraphics[width=1in,height=1.25in,clip,keepaspectratio]{mshell}}]{Michael Shell}
% or if you just want to reserve a space for a photo:

% \begin{IEEEbiography}{Michael Shell}
% Biography text here.
% \end{IEEEbiography}

% % if you will not have a photo at all:
% \begin{IEEEbiographynophoto}{John Doe}
% Biography text here.
% \end{IEEEbiographynophoto}

% insert where needed to balance the two columns on the last page with
% biographies
%\newpage

% \begin{IEEEbiographynophoto}{Jane Doe}
% Biography text here.
% \end{IEEEbiographynophoto}

% You can push biographies down or up by placing
% a \vfill before or after them. The appropriate
% use of \vfill depends on what kind of text is
% on the last page and whether or not the columns
% are being equalized.

%\vfill

% Can be used to pull up biographies so that the bottom of the last one
% is flush with the other column.
%\enlargethispage{-5in}



% that's all folks
\end{document}