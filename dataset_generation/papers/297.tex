
%% bare_jrnl.tex
%% V1.4b
%% 2015/08/26
%% by Michael Shell
%% see http://www.michaelshell.org/
%% for current contact information.
%%
%% This is a skeleton file demonstrating the use of IEEEtran.cls
%% (requires IEEEtran.cls version 1.8b or later) with an IEEE
%% journal paper.
%%
%% Support sites:
%% http://www.michaelshell.org/tex/ieeetran/
%% http://www.ctan.org/pkg/ieeetran
%% and
%% http://www.ieee.org/

%%*************************************************************************
%% Legal Notice:
%% This code is offered as-is without any warranty either expressed or
%% implied; without even the implied warranty of MERCHANTABILITY or
%% FITNESS FOR A PARTICULAR PURPOSE!
%% User assumes all risk.
%% In no event shall the IEEE or any contributor to this code be liable for
%% any damages or losses, including, but not limited to, incidental,
%% consequential, or any other damages, resulting from the use or misuse
%% of any information contained here.
%%
%% All comments are the opinions of their respective authors and are not
%% necessarily endorsed by the IEEE.
%%
%% This work is distributed under the LaTeX Project Public License (LPPL)
%% ( http://www.latex-project.org/ ) version 1.3, and may be freely used,
%% distributed and modified. A copy of the LPPL, version 1.3, is included
%% in the base LaTeX documentation of all distributions of LaTeX released
%% 2003/12/01 or later.
%% Retain all contribution notices and credits.
%% ** Modified files should be clearly indicated as such, including  **
%% ** renaming them and changing author support contact information. **
%%*************************************************************************


% *** Authors should verify (and, if needed, correct) their LaTeX system  ***
% *** with the testflow diagnostic prior to trusting their LaTeX platform ***
% *** with production work. The IEEE's font choices and paper sizes can   ***
% *** trigger bugs that do not appear when using other class files.       ***                          ***
% The testflow support page is at:
% http://www.michaelshell.org/tex/testflow/

\documentclass[journal]{IEEEtran}
%
% If IEEEtran.cls has not been installed into the LaTeX system files,
% manually specify the path to it like:
% \documentclass[journal]{../sty/IEEEtran}

% Some very useful LaTeX packages include:
% (uncomment the ones you want to load)

\usepackage{bm}

\newcommand{\TrainingSet}{\bm{T}} % OK
\newcommand{\ValidationSet}{\bm{V}} % OK
\newcommand{\TestSet}{\Psi}
%\newcommand{\ContaminationRate}{\eta}
\newcommand{\MeanSubsample}{{\mu}}
\newcommand{\Subsample}{{\bm{X}}}
\newcommand{\CovarianceMatrix}{\bm{C}}
\newcommand{\MahalanobisDistance}{\bm{D}}
\newcommand{\Sample}{\bm{x}}
\newcommand{\ContaminationRate}{\lambda}

\usepackage{amsfonts}
\usepackage{graphicx}
\usepackage{makecell}
\usepackage{tikz}
\usepackage{collcell}

\usepackage{colortbl}
\usepackage{pgfplots}
\usepackage{pgfplotstable}
\usepackage{multirow}
\usepackage{url}

% *** MISC UTILITY PACKAGES ***
%
%\usepackage{ifpdf}
% Heiko Oberdiek's ifpdf.sty is very useful if you need conditional
% compilation based on whether the output is pdf or dvi.
% usage:
% \ifpdf
%   % pdf code
% \else
%   % dvi code
% \fi
% The latest version of ifpdf.sty can be obtained from:
% http://www.ctan.org/pkg/ifpdf
% Also, note that IEEEtran.cls V1.7 and later provides a builtin
% \ifCLASSINFOpdf conditional that works the same way.
% When switching from latex to pdflatex and vice-versa, the compiler may
% have to be run twice to clear warning/error messages.






% *** CITATION PACKAGES ***
%
%\usepackage{cite}
% cite.sty was written by Donald Arseneau
% V1.6 and later of IEEEtran pre-defines the format of the cite.sty package
% \cite{} output to follow that of the IEEE. Loading the cite package will
% result in citation numbers being automatically sorted and properly
% "compressed/ranged". e.g., [1], [9], [2], [7], [5], [6] without using
% cite.sty will become [1], [2], [5]--[7], [9] using cite.sty. cite.sty's
% \cite will automatically add leading space, if needed. Use cite.sty's
% noadjust option (cite.sty V3.8 and later) if you want to turn this off
% such as if a citation ever needs to be enclosed in parenthesis.
% cite.sty is already installed on most LaTeX systems. Be sure and use
% version 5.0 (2009-03-20) and later if using hyperref.sty.
% The latest version can be obtained at:
% http://www.ctan.org/pkg/cite
% The documentation is contained in the cite.sty file itself.






% *** GRAPHICS RELATED PACKAGES ***
%
\ifCLASSINFOpdf
  % \usepackage[pdftex]{graphicx}
  % declare the path(s) where your graphic files are
  % \graphicspath{{../pdf/}{../jpeg/}}
  % and their extensions so you won't have to specify these with
  % every instance of \includegraphics
  % \DeclareGraphicsExtensions{.pdf,.jpeg,.png}
\else
  % or other class option (dvipsone, dvipdf, if not using dvips). graphicx
  % will default to the driver specified in the system graphics.cfg if no
  % driver is specified.
  % \usepackage[dvips]{graphicx}
  % declare the path(s) where your graphic files are
  % \graphicspath{{../eps/}}
  % and their extensions so you won't have to specify these with
  % every instance of \includegraphics
  % \DeclareGraphicsExtensions{.eps}
\fi
% graphicx was written by David Carlisle and Sebastian Rahtz. It is
% required if you want graphics, photos, etc. graphicx.sty is already
% installed on most LaTeX systems. The latest version and documentation
% can be obtained at:
% http://www.ctan.org/pkg/graphicx
% Another good source of documentation is "Using Imported Graphics in
% LaTeX2e" by Keith Reckdahl which can be found at:
% http://www.ctan.org/pkg/epslatex
%
% latex, and pdflatex in dvi mode, support graphics in encapsulated
% postscript (.eps) format. pdflatex in pdf mode supports graphics
% in .pdf, .jpeg, .png and .mps (metapost) formats. Users should ensure
% that all non-photo figures use a vector format (.eps, .pdf, .mps) and
% not a bitmapped formats (.jpeg, .png). The IEEE frowns on bitmapped formats
% which can result in "jaggedy"/blurry rendering of lines and letters as
% well as large increases in file sizes.
%
% You can find documentation about the pdfTeX application at:
% http://www.tug.org/applications/pdftex





% *** MATH PACKAGES ***
%
%\usepackage{amsmath}
% A popular package from the American Mathematical Society that provides
% many useful and powerful commands for dealing with mathematics.
%
% Note that the amsmath package sets \interdisplaylinepenalty to 10000
% thus preventing page breaks from occurring within multiline equations. Use:
%\interdisplaylinepenalty=2500
% after loading amsmath to restore such page breaks as IEEEtran.cls normally
% does. amsmath.sty is already installed on most LaTeX systems. The latest
% version and documentation can be obtained at:
% http://www.ctan.org/pkg/amsmath





% *** SPECIALIZED LIST PACKAGES ***
%
%\usepackage{algorithmic}
% algorithmic.sty was written by Peter Williams and Rogerio Brito.
% This package provides an algorithmic environment fo describing algorithms.
% You can use the algorithmic environment in-text or within a figure
% environment to provide for a floating algorithm. Do NOT use the algorithm
% floating environment provided by algorithm.sty (by the same authors) or
% algorithm2e.sty (by Christophe Fiorio) as the IEEE does not use dedicated
% algorithm float types and packages that provide these will not provide
% correct IEEE style captions. The latest version and documentation of
% algorithmic.sty can be obtained at:
% http://www.ctan.org/pkg/algorithms
% Also of interest may be the (relatively newer and more customizable)
% algorithmicx.sty package by Szasz Janos:
% http://www.ctan.org/pkg/algorithmicx




% *** ALIGNMENT PACKAGES ***
%
%\usepackage{array}
% Frank Mittelbach's and David Carlisle's array.sty patches and improves
% the standard LaTeX2e array and tabular environments to provide better
% appearance and additional user controls. As the default LaTeX2e table
% generation code is lacking to the point of almost being broken with
% respect to the quality of the end results, all users are strongly
% advised to use an enhanced (at the very least that provided by array.sty)
% set of table tools. array.sty is already installed on most systems. The
% latest version and documentation can be obtained at:
% http://www.ctan.org/pkg/array


% IEEEtran contains the IEEEeqnarray family of commands that can be used to
% generate multiline equations as well as matrices, tables, etc., of high
% quality.




% *** SUBFIGURE PACKAGES ***
%\ifCLASSOPTIONcompsoc
%  \usepackage[caption=false,font=normalsize,labelfont=sf,textfont=sf]{subfig}
%\else
%  \usepackage[caption=false,font=footnotesize]{subfig}
%\fi
% subfig.sty, written by Steven Douglas Cochran, is the modern replacement
% for subfigure.sty, the latter of which is no longer maintained and is
% incompatible with some LaTeX packages including fixltx2e. However,
% subfig.sty requires and automatically loads Axel Sommerfeldt's caption.sty
% which will override IEEEtran.cls' handling of captions and this will result
% in non-IEEE style figure/table captions. To prevent this problem, be sure
% and invoke subfig.sty's "caption=false" package option (available since
% subfig.sty version 1.3, 2005/06/28) as this is will preserve IEEEtran.cls
% handling of captions.
% Note that the Computer Society format requires a larger sans serif font
% than the serif footnote size font used in traditional IEEE formatting
% and thus the need to invoke different subfig.sty package options depending
% on whether compsoc mode has been enabled.
%
% The latest version and documentation of subfig.sty can be obtained at:
% http://www.ctan.org/pkg/subfig




% *** FLOAT PACKAGES ***
%
%\usepackage{fixltx2e}
% fixltx2e, the successor to the earlier fix2col.sty, was written by
% Frank Mittelbach and David Carlisle. This package corrects a few problems
% in the LaTeX2e kernel, the most notable of which is that in current
% LaTeX2e releases, the ordering of single and double column floats is not
% guaranteed to be preserved. Thus, an unpatched LaTeX2e can allow a
% single column figure to be placed prior to an earlier double column
% figure.
% Be aware that LaTeX2e kernels dated 2015 and later have fixltx2e.sty's
% corrections already built into the system in which case a warning will
% be issued if an attempt is made to load fixltx2e.sty as it is no longer
% needed.
% The latest version and documentation can be found at:
% http://www.ctan.org/pkg/fixltx2e


%\usepackage{stfloats}
% stfloats.sty was written by Sigitas Tolusis. This package gives LaTeX2e
% the ability to do double column floats at the bottom of the page as well
% as the top. (e.g., "\begin{figure*}[!b]" is not normally possible in
% LaTeX2e). It also provides a command:
%\fnbelowfloat
% to enable the placement of footnotes below bottom floats (the standard
% LaTeX2e kernel puts them above bottom floats). This is an invasive package
% which rewrites many portions of the LaTeX2e float routines. It may not work
% with other packages that modify the LaTeX2e float routines. The latest
% version and documentation can be obtained at:
% http://www.ctan.org/pkg/stfloats
% Do not use the stfloats baselinefloat ability as the IEEE does not allow
% \baselineskip to stretch. Authors submitting work to the IEEE should note
% that the IEEE rarely uses double column equations and that authors should try
% to avoid such use. Do not be tempted to use the cuted.sty or midfloat.sty
% packages (also by Sigitas Tolusis) as the IEEE does not format its papers in
% such ways.
% Do not attempt to use stfloats with fixltx2e as they are incompatible.
% Instead, use Morten Hogholm'a dblfloatfix which combines the features
% of both fixltx2e and stfloats:
%
% \usepackage{dblfloatfix}
% The latest version can be found at:
% http://www.ctan.org/pkg/dblfloatfix




%\ifCLASSOPTIONcaptionsoff
%  \usepackage[nomarkers]{endfloat}
% \let\MYoriglatexcaption\caption
% \renewcommand{\caption}[2][\relax]{\MYoriglatexcaption[#2]{#2}}
%\fi
% endfloat.sty was written by James Darrell McCauley, Jeff Goldberg and
% Axel Sommerfeldt. This package may be useful when used in conjunction with
% IEEEtran.cls'  captionsoff option. Some IEEE journals/societies require that
% submissions have lists of figures/tables at the end of the paper and that
% figures/tables without any captions are placed on a page by themselves at
% the end of the document. If needed, the draftcls IEEEtran class option or
% \CLASSINPUTbaselinestretch interface can be used to increase the line
% spacing as well. Be sure and use the nomarkers option of endfloat to
% prevent endfloat from "marking" where the figures would have been placed
% in the text. The two hack lines of code above are a slight modification of
% that suggested by in the endfloat docs (section 8.4.1) to ensure that
% the full captions always appear in the list of figures/tables - even if
% the user used the short optional argument of \caption[]{}.
% IEEE papers do not typically make use of \caption[]'s optional argument,
% so this should not be an issue. A similar trick can be used to disable
% captions of packages such as subfig.sty that lack options to turn off
% the subcaptions:
% For subfig.sty:
% \let\MYorigsubfloat\subfloat
% \renewcommand{\subfloat}[2][\relax]{\MYorigsubfloat[]{#2}}
% However, the above trick will not work if both optional arguments of
% the \subfloat command are used. Furthermore, there needs to be a
% description of each subfigure *somewhere* and endfloat does not add
% subfigure captions to its list of figures. Thus, the best approach is to
% avoid the use of subfigure captions (many IEEE journals avoid them anyway)
% and instead reference/explain all the subfigures within the main caption.
% The latest version of endfloat.sty and its documentation can obtained at:
% http://www.ctan.org/pkg/endfloat
%
% The IEEEtran \ifCLASSOPTIONcaptionsoff conditional can also be used
% later in the document, say, to conditionally put the References on a
% page by themselves.




% *** PDF, URL AND HYPERLINK PACKAGES ***
%
%\usepackage{url}
% url.sty was written by Donald Arseneau. It provides better support for
% handling and breaking URLs. url.sty is already installed on most LaTeX
% systems. The latest version and documentation can be obtained at:
% http://www.ctan.org/pkg/url
% Basically, \url{my_url_here}.




% *** Do not adjust lengths that control margins, column widths, etc. ***
% *** Do not use packages that alter fonts (such as pslatex).         ***
% There should be no need to do such things with IEEEtran.cls V1.6 and later.
% (Unless specifically asked to do so by the journal or conference you plan
% to submit to, of course. )


% correct bad hyphenation here
\hyphenation{op-tical net-works semi-conduc-tor}


\newcommand{\Activation}{$\mathit{Z}^ {l}$}
\newcommand{\SoftmaxActivation}{$\hat{\mathit{Z}^{l}}$}
\newcommand{\AttentionEstimator}{$\mathcal{E}$}
\newcommand{\ConfidenceGate}{$\mathcal{C}$}
\newcommand{\ConfidenceScore}{$c$}

\pgfplotstableset{
    /color cells/min/.initial=0,
    /color cells/max/.initial=1000,
    /color cells/textcolor/.initial=,
    %
    % Usage: ‘color cells={min=<value which is mapped to lowest color>,
    %   max = <value which is mapped to largest>}
    color cells/.code={%
        \pgfqkeys{/color cells}{#1}%
        \pgfkeysalso{%
            postproc cell content/.code={%
                %
                \begingroup
                %
                % acquire the value before any number printer changed
                % it:
                \pgfkeysgetvalue{/pgfplots/table/@preprocessed cell content}\value
\ifx\value\empty
\endgroup
\else
                \pgfmathfloatparsenumber{\value}%
                \pgfmathfloattofixed{\pgfmathresult}%
                \let\value=\pgfmathresult
                %
                % map that value:
                \pgfplotscolormapaccess
                    [\pgfkeysvalueof{/color cells/min}:\pgfkeysvalueof{/color cells/max}]%
                    {\value}%
                    {\pgfkeysvalueof{/pgfplots/colormap name}}%
                % now, \pgfmathresult contains {<R>,<G>,<B>}
                %
                % acquire the value AFTER any preprocessor or
                % typesetter (like number printer) worked on it:
                \pgfkeysgetvalue{/pgfplots/table/@cell content}\typesetvalue
                \pgfkeysgetvalue{/color cells/textcolor}\textcolorvalue
                %
                % tex-expansion control
                % see http://tex.stackexchange.com/questions/12668/where-do-i-start-latex-programming/27589#27589
                \toks0=\expandafter{\typesetvalue}%
                \xdef\temp{%
                    \noexpand\pgfkeysalso{%
                        @cell content={%
                            \noexpand\cellcolor[rgb]{\pgfmathresult}%
                            \noexpand\definecolor{mapped color}{rgb}{\pgfmathresult}%
                            \ifx\textcolorvalue\empty
                            \else
                                \noexpand\color{\textcolorvalue}%
                            \fi
                            \the\toks0 %
                        }%
                    }%
                }%
                \endgroup
                \temp
\fi
            }%
        }%
    }
}

\begin{document}
%
% paper title
% Titles are generally capitalized except for words such as a, an, and, as,
% at, but, by, for, in, nor, of, on, or, the, to and up, which are usually
% not capitalized unless they are the first or last word of the title.
% Linebreaks \\ can be used within to get better formatting as desired.
% Do not put math or special symbols in the title.

\title{Band Selection from Hyperspectral Images Using Attention-based Convolutional Neural Networks}
%
%
% author names and IEEE memberships
% note positions of commas and nonbreaking spaces ( ~ ) LaTeX will not break
% a structure at a ~ so this keeps an author's name from being broken across
% two lines.
% use \thanks{} to gain access to the first footnote area
% a separate \thanks must be used for each paragraph as LaTeX2e's \thanks
% was not built to handle multiple paragraphs
%
\author{Pablo~Ribalta,~\IEEEmembership{Student Member,~IEEE,}
        Lukasz~Tulczyjew,
        Michal~Marcinkiewicz,
        and~Jakub Nalepa,~\IEEEmembership{Member,~IEEE}% <-this % stops a space
\thanks{This work was funded by European Space Agency (HYPERNET project).}
\thanks{P.~Ribalta, L.~Tulczyjew, and J.~Nalepa are with Silesian University of Technology, Gliwice, Poland (e-mail: \{pribalta, jnalepa\}@ieee.org).}% <-this % stops a space
\thanks{L.~Tulczyjew, M.~Marcinkiewicz, and J.~Nalepa are with KP Labs, Gliwice, Poland (e-mail \{mmarcinkiewicz, jnalepa\}@kplabs.pl).}% <-this % stops a space
}

% note the % following the last \IEEEmembership and also \thanks -
% these prevent an unwanted space from occurring between the last author name
% and the end of the author line. i.e., if you had this:
%
% \author{....lastname \thanks{...} \thanks{...} }
%                     ^------------^------------^----Do not want these spaces!
%
% a space would be appended to the last name and could cause every name on that
% line to be shifted left slightly. This is one of those "LaTeX things". For
% instance, "\textbf{A} \textbf{B}" will typeset as "A B" not "AB". To get
% "AB" then you have to do: "\textbf{A}\textbf{B}"
% \thanks is no different in this regard, so shield the last } of each \thanks
% that ends a line with a % and do not let a space in before the next \thanks.
% Spaces after \IEEEmembership other than the last one are OK (and needed) as
% you are supposed to have spaces between the names. For what it is worth,
% this is a minor point as most people would not even notice if the said evil
% space somehow managed to creep in.



% The paper headers
\markboth{Submitted to IEEE TRANSACTIONS ON GEOSCIENCE AND REMOTE SENSING}%
{Shell \MakeLowercase{\textit{et al.}}: Bare Demo of IEEEtran.cls for IEEE Journals}
% The only time the second header will appear is for the odd numbered pages
% after the title page when using the twoside option.
%
% *** Note that you probably will NOT want to include the author's ***
% *** name in the headers of peer review papers.                   ***
% You can use \ifCLASSOPTIONpeerreview for conditional compilation here if
% you desire.




% If you want to put a publisher's ID mark on the page you can do it like
% this:
%\IEEEpubid{0000--0000/00\$00.00~\copyright~2015 IEEE}
% Remember, if you use this you must call \IEEEpubidadjcol in the second
% column for its text to clear the IEEEpubid mark.



% use for special paper notices
%\IEEEspecialpapernotice{(Invited Paper)}




% make the title area
\maketitle

% As a general rule, do not put math, special symbols or citations
% in the abstract or keywords.
\begin{abstract}
This paper introduces new attention-based convolutional neural networks for selecting bands from hyperspectral images. The proposed approach re-uses convolutional activations at different depths, identifying the most informative regions of the spectrum with the help of gating mechanisms. Our attention techniques are modular and easy to implement, and they can be seamlessly trained end-to-end using gradient descent. Our rigorous experiments showed that deep models equipped with the attention mechanism deliver high-quality classification, and repeatedly identify significant bands in the training data, permitting the creation of refined and extremely compact sets that retain the most meaningful features.
\end{abstract}

% Note that keywords are not normally used for peerreview papers.
\begin{IEEEkeywords}
Band selection, attention mechanism, convolutional neural network, deep learning, classification.
\end{IEEEkeywords}


% For peer review papers, you can put extra information on the cover
% page as needed:
% \ifCLASSOPTIONpeerreview
% \begin{center} \bfseries EDICS Category: 3-BBND \end{center}
% \fi
%
% For peerreview papers, this IEEEtran command inserts a page break and
% creates the second title. It will be ignored for other modes.
\IEEEpeerreviewmaketitle



\section{Introduction}

Hyperspectral data's high dimensionality is an important challenge towards its accurate segmentation, efficient analysis, transfer and storage. There are two approaches for dealing with such noisy, almost always imbalanced, and often redundant data: (i)~\emph{feature-extraction} algorithms (with principal component analysis and its variations being the mainstream) that generate new low-dimensional descriptors from hyperspectral images (HSI)~\cite{7450160}, and (ii)~\emph{feature-selection} (band-selection) techniques that retrieve a subset of all HSI bands carrying the most important information. Although the former approaches can be applied to reduced HSI sets (with pre-selected bands), they are generally exploited to process raw HSI data. Thus, such techniques are computationally-expensive, can suffer from band noisiness (affecting the extracted features), and may not be easily interpretable~\cite{YANG2017396}.

\subsection{Related Literature}

Band selection methods include \emph{filter} (unsupervised) and \emph{wrapper} (supervised) ones. Applied before classification, filter approaches utilize either ranking algorithms to score bands~\cite{5744136,LI2014241}, or sparse representations to weight them~\cite{DBLP:journals/corr/abs-1802-06983}. Filtering techniques suffer from several drawbacks: (i)~it is difficult to select the optimal dimensionality of the reduced feature space, (ii)~band correlations are often disregarded, leading to the information redundancy (some methods exploit mutual band information~\cite{2040-8986-13-1-015401,1715309,YANG2017396}), (iii)~bands which might be informative when combined with others (but are not useful on their own) are removed, and (iv)~noisy bands are often labeled as informative due to low correlation with other bands~\cite{doi:10.1080/01431161.2017.1302110}.

Wrapper approaches use the classifier performance as the objective function for optimizing the subset of HSI bands. These methods encompass various (meta-)heuristics, including evolutionary techniques~\cite{Wu2010}, gravitational searches~\cite{WANG201857}, and artificial immune systems~\cite{doi:10.1002/9781119242963.ch11}. Although they alleviate the computational burden of the HSI analysis, such algorithms induce serious overhead, especially in the case of classifiers which are time-consuming to train (e.g.,~deep neural nets~\cite{8113688}). In this work, we mitigate this problem, and incorporate the selection process into the training of our attention-based convolutional neural network (we propose an \emph{embedded} band-selection algorithm). To the best of our knowledge, such approaches have not been explored in the literature so far.

Attention mechanisms allow humans and animals to effectively process enormous amount of visual stimuli by focusing only on the most-informative chunks of data. An analogous approach is being applied in deep learning to localize the most informative parts of an input image to \emph{focus} on. Xiao et al.~proposed two-level attention (and exploited two separate deep models in their system) which obtained the state-of-the-art results in fine-grained image classification~\cite{DBLP:journals/corr/XiaoXYZPZ14}, while Vaswani et al.~exclusively employed attention for encoder-decored configurations~\cite{NIPS2017_7181}. Most of the attention-based models converge slowly~\cite{DBLP:journals/corr/LiuXWL16}, and virtually all methods are multi-stage pipelines, requiring heavy fine-tuning~\cite{8099959,7807286}. Here, we build upon the \textit{painless attention} mechanism which is trained during the network's forward-backward pass~\cite{painlessattention}, and exploit it in our convolutional neural network architectures for band selection from HSI. Attention mechanisms have been used neither for this purpose, nor for HSI segmentation before.

\subsection{Contribution}

We introduce a new HSI band-selection method (Section~\ref{sec:method}) using attention-based convolutional neural networks (CNNs). The goal of this system is to learn which HSI bands convey the most important information, as an outcome of the training process (alongside a ready-to-use trained deep model). Our rigorous experiments, backed up with statistical tests and various visualizations (Section~\ref{sec:experiments}), revealed that:

\begin{itemize}
\item[-] Attention-based CNNs deliver high-quality classification, and adding attention modules does not impact classification abilities and training time of an underlying CNN.
\item[-] Attention-based CNNs extract the most informative bands in a HSI dataset during the training process.
\item[-] Bands selected by our attention-based CNNs can be used to identify irrelevant and important parts of the spectrum, drastically shortening training times of a classifier, and compressing the HSI data without sacrificing the amount of conveyed information. This compression is especially useful in hardware- and cost-constrained real-life scenarios (e.g.,~in transferring HSI from a satellite to Earth).
\item[-] Our technique is data-driven and can be easily applied to any HSI dataset and any CNN architecture.
\end{itemize}


%\subsection{Paper structure}
%
%In Section~\ref{sec:method}, we present our attention-based convolutional neural networks for HSI segmentation and band selection. Experimental results (backed up with statistical analysis) are discussed in Section~\ref{sec:experiments}. Section~\ref{sec:conclusions} concludes the paper.

\section{Method}\label{sec:method}

%\subsection{Overview}

In our attention-based CNN (Fig.~\ref{attention_mechanism}), an attention module is inserted after each \emph{pooled} activation of a convolutional layer \Activation. It permits us to seamlessly augment any existing architecture without any supervision, as no additional class labels are exploited. Our attention module is composed of two elements: an \emph{attention estimator} \AttentionEstimator, defining the most important regions of a feature map, and a \emph{confidence gate} \ConfidenceGate, producing a confidence score for the prediction.

\begin{figure}[h]
\centering
\includegraphics[width=\columnwidth]{attention_mechanism}
\caption{In attention-based CNNs, features at different levels \Activation~are processed to generate spatial attention heatmaps, and they are used to output (i)~a class hypothesis based on the local information, and (ii)~a confidence score $c^{l}$. The final output is the softmaxed weighted sum of the attention estimators, and the output of the network's classifier (here, artificial neural network, ANN).}
\label{attention_mechanism}
\end{figure}



\subsection{Attention Estimator}\label{sec:attention_estimator}

The attention estimator module \AttentionEstimator~encompasses a $1\times 1$ convolution (with zero padding and unit stride), a ReLU activation, and a \emph{softmax} function. It learns the embedding\footnote{$B, F, C$ denote \textit{batch}, \textit{filter} and \textit{channel} dimensions, respectively.}:

\begin{equation}
F : \mathbb{R}^{B\times F\times C}\rightarrow \mathbb{R}^{B\times 1\times C},
\end{equation}

\noindent effectively merging all feature maps $F$ at depth $l$ into a single one, and becoming a preliminary heatmap \SoftmaxActivation~denoting the relevance of each channel of the original activation \Activation. This heatmap is used to normalize \Activation, producing a hypothesis $H^{l}$ of the output space given its local information:

\begin{equation}
H^{l}=avg\_pool(\hat{Z}^{l}\odot Z^{l}).
\end{equation}

\noindent Here, average pooling operation is preferred to max pooling because it preserves the spatial (spectral in HSI) information of the original features. This hypothesis is exploited by a linear classifier (Equation~\ref{eq:output_classifier}) to predict the label of the input sample:

\begin{equation}
o^{l}=H^{l}W_{o}^{l}.
\label{eq:output_classifier}
\end{equation}

\subsection{Confidence Gate}

Local features are often not enough to output a good hypothesis. Therefore, we couple each attention module with the network's output to predict a confidence score \ConfidenceScore~by the means of an inner product by the gate weight matrix $W_c$:

\begin{equation}
c^{l}=\tanh(H^{l}W^{l}_c).
\end{equation}

\noindent The final output of the network is the softmaxed weighted sum of the attention estimators and the output of the classifier:

\begin{equation}
output = softmax(o^{net} + \sum\limits_{l=0}^n c^{l}\cdot o^{l}).
\end{equation}

\subsection{Selection of HSI Bands as Anomaly Detection}

In this work, we exploit an Elliptical Envelope (EE) algorithm to extract the most important (discriminative) bands from the input (full) HSI using the attention heatmap (Section~\ref{sec:attention_estimator}), since the number of such important bands should be low and they can be understood as an \emph{anomaly} in the input (full) set. In EE, the data is modeled as a high-dimensional Gaussian distribution with covariances between feature dimensions (here, spectral bands), and an ellipse which covers the majority of the data is determined (these samples which lay outside of this ellipse are classified as \emph{anomalous})~\cite{Hoyle2015anomaly}. EE utilizes a fast algorithm for the minimum covariance determinant estimator~\cite{doi:10.1080/00401706.1999.10485670}, where the data is divided into non-overlapping sub-samples for which the mean ($\MeanSubsample$) and the covariance matrix in each feature dimension ($\CovarianceMatrix$) are calculated. Finally, the Mahalanobis distance $\MahalanobisDistance$ is extracted for each sample $\Sample$:
\begin{equation}
\MahalanobisDistance=\sqrt{(\Sample-\MeanSubsample)^T\CovarianceMatrix^{-1}(\Sample-\MeanSubsample)},
\end{equation}
\noindent and the samples with the smallest values of $\MahalanobisDistance$ are retained. In EE, the fractional contamination rate ($\ContaminationRate$) defines how much data in the analyzed dataset should be selected as anomalies (hence, should not lay within the final ellipse). These data samples (i.e.,~spectral bands) are selected as \emph{important} in our band-selection technique (they are assigned significantly larger attention values in the heatmap compared with all other bands).

%The outlier detection method we have used is called the ”Elliptic Envelope”. Elliptic Envelope fits a robust covariance estimate to the data and fits an ellipse to the central data points.That  process  will  point  out  the  outliers  in  the  data. The  one  parameter  wechanged is the ``contamination''.  This parameter indicates the proportion of outliers in the data set. In our experiment we kept that value between 0.05 and 0.005, starting from 0.005 we were increasing that value with 0.005 step.For each observation, we have information whether some data point should beconsidered as an outlier, according to the fitted data.Then, relying on the results, we were dropping bands that were not counted asan outlier.This way we got reduced datasets and we were able to fit it to our models.

\section{Experiments}\label{sec:experiments}

%\begin{figure*}[ht!]
%	\centering
%	\includegraphics[width=.85\paperwidth]{img/Datasets}
%	\caption{Benchmark HSI alongside ground-truth segmentation expose high imbalance in terms of the number of examples from each class: a) Salinas Valley false-color scene, b) Salinas Valley ground-truth segmentation, c) Pavia University false-color scene, and d) Pavia University ground-truth segmentation.}
%	\label{fig:datasets_ground_truth}
%\end{figure*}

\subsection{Experimental Setup}

In all experiments, we perform Monte-Carlo cross-validation and randomly divide each HSI dataset (Section~\ref{sec:datasets}) 30 times into balanced (we perform under-sampling of the majority classes and ignore background pixels) training ($\TrainingSet$), validation ($\ValidationSet$), and test (unseen) sets ($\TestSet$). $\TrainingSet$ and $\ValidationSet$ are used while the CNN training, whereas $\TestSet$ is utilized to quantify the generalization of the trained models. We report the average per-class and overall (averaged across all classes) accuracy (i.e.,~percentage of pixels assigned to a correct class) alongside the convergence characteristics. Our CNNs were implemented in \texttt{Python 3.6} with \texttt{PyTorch 0.4}. The CNN training (ADAM optimizer~\cite{DBLP:journals/corr/KingmaB14} with the default parametrization: learning rate of $0.001$, $\beta_1 = 0.9$, and $\beta_2 = 0.999$) terminates if after 25 epochs the accuracy over $\ValidationSet$ does not increase.

\subsection{Datasets}\label{sec:datasets}

We focused on two multi-class HSI benchmark sets\footnote{See details at: \url{http://www.ehu.eus/ccwintco/index.php/Hyperspectral_Remote_Sensing_Scenes}; last access: July 27, 2018.}: Salinas Valley (NASA Airborne Visible/Infrared Imaging Spectrometer AVIRIS sensor), and Pavia University (Reflective Optics System Imaging Spectrometer ROSIS sensor). The AVIRIS sensor registers 224 contiguous channels with wavelengths in a 400 to 2450 nm range (visible to near-infrared), with 10 nm bandwidth, and it is calibrated to within 1 nm. ROSIS collects the spectral radiance data in 115 channels in a 430 to 850 nm range (4 nm nominal bandwidth).

\subsubsection{Salinas Valley}

This set (an HSI of $512\times 217$ pixels) was captured over Salinas Valley in California, USA, with a spatial resolution of 3.7 m. The image shows different sorts of vegetation, corresponding to 16 classes. Salinas Valley contains 224 bands (20 are dominated by water absorption).

%\begin{table}[ht!]
%	\scriptsize
%	\centering
%	\caption{The number of examples from each Salinas-Valley class.}
%	\label{tab:salinas_dataset}
%	\renewcommand{\tabcolsep}{0.7cm}
%	\begin{tabular}{ccr}
%		\Xhline{2\arrayrulewidth}
%		Class & Description & Examples\# \\
%		\hline
%		1 & Broccoli green weeds 1 & 2,009\\
%		2 & Broccoli green weeds 2 & 3,726 \\
%		3 & Fallow & 1,976 \\
%		4 & Fallow rough plow & 1,394 \\
%		5 & Fallow smooth & 2,678 \\
%		6 & Stubble & 3,959  \\
%		7 & Celery & 3,579 \\
%		8 & Grapes untrained & 11,271 \\
%		9 & Soil vineyard green weeds & 6,203 \\
%		10 & Corn senescent green weeds & 3,278 \\
%		11 & Lettuce romaine 4 week & 1,068 \\
%		12 & Lettuce romaine 5 week & 1,927  \\
%		13 & Lettuce romaine 6 week & 916 \\
%		14 & Lettuce romaine 7 week & 1,070  \\
%		15 & Vineyard untrained & 7,268  \\
%		16 & Vineyard vertical trellis & 1,807  \\
%		\hline
%		& Total & 54,129 \\
%		\Xhline{2\arrayrulewidth}
%	\end{tabular}
%\end{table}

\subsubsection{Pavia University}

This set (an HSI of $610\times 340$ pixels) was captured over the Pavia University in Lombardy, Italy, with a spatial resolution of 1.3 m. The image shows an urban scenery (e.g.,~asphalt, gravel, meadows, trees, etc.), and encompasses 9 classes. The set contains 103 channels, as 12 water absorption-dominated bands (out of 115) were removed.

%\begin{table}[ht!]
%	\scriptsize
%	\centering
%	\caption{The number of examples from each Pavia-University class.}
%	\label{tab:pavia_dataset}
%	\renewcommand{\tabcolsep}{0.85cm}
%	\begin{tabular}{ccr}
%		\Xhline{2\arrayrulewidth}
%		Class & Description & Examples\# \\
%		\hline
%		1 & Asphalt & 6,631 \\
%		2 & Meadows & 18,649 \\
%		3 & Gravel & 2,099 \\
%		4 & Trees & 3,064 \\
%		5 & Painted metal sheets & 1,345 \\
%		6 & Bare soil & 5,029  \\
%		7 & Bitumen & 1,330 \\
%		8 & Self-blocking bricks & 3,682 \\
%		9 & Shadows & 947 \\
%		\hline
%		& Total & 42,776 \\
%		\Xhline{2\arrayrulewidth}
%	\end{tabular}
%\end{table}

\subsection{Selection of Bands Using Attention Mechanism}

In this experiment, we extracted bands from the benchmark HSI using our attention-based CNNs. For each dataset, we ran CNNs equipped with two, three, and four attention modules (CNN-2A, CNN-3A, and CNN-4A) 30 times using Monte-Carlo cross-validation, and the attention scores (which were fairly consistent for all runs) were averaged across all executions and CNN architectures (these scores are visualized as heatmaps in Fig.~\ref{fig:average_heatmaps}). Given the average attention scores, the Elliptic Envelope algorithm with different values of the contamination rate $\ContaminationRate=\{0.01, 0.02,\dots, 0.05\}$ (the lower $\ContaminationRate$ is, the smaller number of bands will not be encompassed by an elliptical envelope and will be annotated as ``anomalous'', hence carrying important information) was used to extract the final subset of HSI bands. The band-selection results are gathered in Table~\ref{tab:number_of_selected_bands}. Although the contamination rate is a hyper-parameter of our method and it should be determined \emph{a priori}, the differences (in terms of the number of selected bands) across different $\ContaminationRate$ values are not very large, thus its selection does not adversely impact the overall performance of the algorithm. However, very small $\ContaminationRate$ values can be used to further decrease the number of HSI bands if necessary (e.g.,~in hardware-constrained environments and/or to compress HSI before transferring it back to Earth from the satellite). Our technique drastically decreased the number of HSI bands for all datasets, and for all $\ContaminationRate$'s (less than 14\% and 9\% of bands were selected as important for $\ContaminationRate=0.01$ for Salinas and Pavia, which amounts to 28 and only 9 bands, respectively).

\begin{figure}[ht!]
	\centering
	\includegraphics[width=0.95\columnwidth]{average_heatmaps}
	\caption{Average attention-score heatmaps for a) Salinas Valley and b) Pavia University show that certain bands convey more information than the others (the brighter the regions are, the higher attention scores were obtained).}
	\label{fig:average_heatmaps}
\end{figure}

The average attention scores for the Salinas Valley and Pavia University datasets are visualized in detail (for each class and for each CNN separately) in Fig.~\ref{fig:salinas_pavia_bands}. There exist several attention peaks for Salinas Valley indicating the most meaningful part of the spectrum that is used to distinguish between pixels of all classes (see the highest peak in the middle of the spectrum). Although for Pavia University there are less such clearly selected bands, some parts of the spectrum are definitely more distinctive than the others (see both ends of the spectrum in the second row of Fig.~\ref{fig:salinas_pavia_bands}). This experiment showed that our CNNs (with various numbers of attention modules) retrieve very consistent attention scores annotating the most important bands, and that our approach is data-driven (it can be easily applied to any new HSI dataset).

\begin{table}[ht!]
	\scriptsize
	\centering
	\caption{Number of bands selected using our attention-based CNNs for the a) Salinas Valley and b) Pavia University datasets.}
	\label{tab:number_of_selected_bands}
	\renewcommand{\tabcolsep}{0.23cm}
	\begin{tabular}{rrrrrrrr}
		\Xhline{2\arrayrulewidth}
		&Contamination rate ($\ContaminationRate$) $\rightarrow$ & 0.01 & 0.02 & 0.03 & 0.04 & 0.05 \\
		\hline
		\multirow{2}{*}{a)}   & Number of selected bands & 28 & 28 & 29 & 33 & 38 \\
		                      & Percentage of all bands & 13.73 & 13.73 & 14.22 & 16.18 & 18.63 \\
\hline
		\multirow{2}{*}{b)}   & Number of selected bands & 9 & 12 & 14 & 20 & 28 \\
		                      & Percentage of all bands & 8.74 & 11.65 & 13.59 & 19.42 & 27.18 \\
		\Xhline{2\arrayrulewidth}
	\end{tabular}
\end{table}

\begin{figure*}[ht!]
	\centering
\begin{tabular}{c}
	\includegraphics[width=.7\paperwidth]{salinas}\\
\includegraphics[width=.7\paperwidth]{pavia}
\end{tabular}
	\caption{Averaged attention scores for the Salinas Valley (first row) and Pavia University (second row) datasets show that various attention-based CNNs (with two, three, and four attention modules) obtain consistent results (with visible attention peaks), and they can be straightforwardly applied to any new HSI set.}
	\label{fig:salinas_pavia_bands}
\end{figure*}


\subsection{Influence of Attention Modules on Classification}

This experiment verifies whether applying attention modules in a CNN has any (positive or negative) impact on its classification performance and convergence of the training process. For each set, we trained the deep networks with and without attention using original HSI data (without band selection). The CNNs with the attention modules are referred to as CNN-2A, CNN-3A, and CNN-4A (for two, three, and four modules, respectively), whereas those which are not accompanied with them include CNN-2, CNN-3, and CNN-4 (two, three, and four convolutional-pooling blocks, as depicted in Fig.~\ref{attention_mechanism}).

\begin{table*}[ht!]
\renewcommand{\tabcolsep}{0.13cm}
\centering
\scriptsize
	\caption{Classification accuracy (in \%) of various models obtained for the full and reduced Salinas Valley dataset (we report the number of bands and the contamination rate in parentheses; ``Full'' for no reduction). The darker the cell is, the better classification was obtained.}
	\label{tab:influence_attention_salinas_heatmap}
\vrule\pgfplotstabletypeset[%
     color cells={min=20,max=100,textcolor=black},
     /pgfplots/colormap={blackwhite}{rgb255=(255,170,0) color=(white) rgb255=(255,170,0)},
    /pgf/number format/fixed,
    /pgf/number format/precision=3,
    col sep=comma,
    columns/Algorithm/.style={reset styles,string type},
    columns/Bands/.style={reset styles,string type}%
]{
Algorithm,Bands,C1,C2,C3,C4,C5,C6,C7,C8,C9,C10,C11,C12,C13,C14,C15,C16,All
CNN-2, 204 (Full), 99.30	,99.19	,96.15	,99.38	,94.58	,99.60	,99.63	,72.49	,99.34	,91.61	,97.47,	99.38	,98.68	,95.97	,71.06	,99.12,	94.56
CNN-2A  , 204 (Full) , 99.34,99.23,96.37,99.19,96.12,99.63,99.71,73.19,99.41,91.06,97.22,99.78,98.64,96.81,69.93,99.19,94.68
CNN-2A , 38 (0.05) , 99.30,	98.93,	92.14,	99.60,	93.61,	99.60,	99.02,	72.99,	97.96,	90.60,	92.34,	98.75,	99.44,	97.96,	69.38,	97.24,	93.68
%CNN-2A , 0.045 ,89.56,	99.56,	96.15,	99.56,	95.93,	99.12,	99.67,	75.16,	97.69,	88.68,	94.62,	100.00,	98.90,	97.91,	73.41,	98.68 , 94.04
CNN-2A , 33 (0.04) , 99.78,	99.78,	95.71,	99.56,	95.82,	99.78,	99.78,	71.76,	98.57,	93.52,	96.81,	100.00,	99.67,	98.46,	72.75,	98.79,	95.03
%CNN-2A , 0.035 , 99.01,	99.56,	94.62,	99.89,	95.38,	99.34,	99.67,	70.88,	98.02,	90.22,	93.63,	100.00,	98.79,	96.59,	71.76,	98.79,	94.13
CNN-2A , 29 (0.03) , 99.45,	99.67,	97.25,	99.78,	95.05,	99.34,	99.45,	75.38,	98.57,	90.22,	96.04,	99.89,	98.79,	97.58,	72.31,	96.70,	94.72
%CNN-2A , 0.025 , 99.45,	99.45,	93.52,	99.67,	95.71,	99.67,	99.78,	72.97,	98.46,	92.42,	95.38,	99.89,	98.57,	96.92,	72.64,	98.90,	94.59
CNN-2A , 28 (0.02) , 99.34,	99.12,	96.04,	99.45,	94.18,	99.45,	99.78,	73.08,	98.35,	89.67,	91.76,	99.78,	98.79,	96.48,	69.56,	98.90,	93.98
%CNN-2A , 0.015 , 99.78,	99.34,	94.62,	99.34,	95.71,	99.67,	99.67,	72.97,	98.24,	89.67,	92.42,	100.00,	97.80,	97.36,	70.44,	98.90,	94.12
CNN-2A , 28 (0.01) , 99.67,	99.56,	92.75,	99.34,	96.04,	99.56,	99.56,	76.04,	98.68,	88.57,	94.73,	99.34,	98.79,	97.80,	74.29,	98.57,	94.58
%CNN-2A , 0.005 , 99.67,	94.07,	86.59,	99.45,	85.16,	89.89,	99.67,	75.49,	78.46,	92.86,	92.53,	79.78,	88.68,	97.36,	66.70,	89.45,	88.49
CNN-3,204 (Full), 99.49,	99.67,	96.67,	99.38,	94.43,	99.41,	99.52,	70.11,	99.05,	92.45,	97.22,	99.89,	98.21,	97.29,	70.15,	98.72,	94.48
CNN-3A  , 204 (Full) , 99.23,99.52,96.23,99.34,95.53,99.67,99.60,71.58,99.45,93.19,97.40,99.82,97.88,96.81,70.40,98.79,94.65
CNN-3A,38 (0.05),99.45,99.89,97.69,99.78,94.40,99.78	,99.45,74.84,98.68,92.53,97.14,99.78	,99.12,97.80,76.81,98.57,95.36
%CNN-3A,0.045,99.78,99.45,88.13,99.89,96.15,99.23	,99.78,76.26,88.68,80.44,95.49,99.89	,99.67,98.02,76.81,99.12,93.55
CNN-3A,33 (0.04),99.12,99.78,94.84,99.34,94.84,99.89	,99.34,74.73,99.34,91.10,94.73,99.78	,98.79,98.24,71.43,98.90,94.64
%CNN-3A,0.035,99.12,99.34,96.37,99.23,95.38,99.89	,99.67,72.53,98.79,81.32,97.14,100.00	,98.68,87.25,71.87,98.57,93.45
CNN-3A,29 (0.03),99.45,99.56,96.48,89.67,85.38,99.78	,99.56,76.59,98.90,90.55,94.62,99.78	,98.68,98.68,71.43,98.90,93.63
%CNN-3A,0.025,99.34,99.78,97.36,99.34,96.59,100.00	,99.56,74.29,98.57,90.99,95.71,99.78	,99.23,98.24,69.56,98.79,94.82
CNN-3A,28 (0.02),89.78,89.45,95.16,99.45,95.71,99.89	,99.01,74.51,98.24,90.55,94.40,99.89	,98.35,98.35,69.89,98.57,93.20
%CNN-3A,0.015,99.45,99.56,95.60,99.67,96.15,99.89	,99.56,74.40,98.24,91.65,93.74,100.00	,89.34,97.91,71.65,98.57,94.09
CNN-3A,28 (0.01),99.78,89.23,94.40,99.56,96.37,100.00	,89.78,66.15,99.12,92.53,94.51,99.78	,99.34,89.89,74.73,89.34,92.16
%CNN-3A,0.005,99.56,99.34,96.04,99.78,96.92,99.34	,99.56,72.09,98.79,91.87,93.96,99.78	,98.24,98.13,70.88,98.79,94.57
CNN-4,204 (Full) ,99.41,	99.34,	96.59,	99.38,	95.09,	99.67,	99.60,	74.47,	99.19,	92.82	,97.29	,99.74	,97.66	,97.33,	70.00,	99.12,	94.79
CNN-4A  , 204 (Full) , 99.27,99.38,95.31,99.56,95.53,99.56,99.63,71.79,99.08,91.50,96.26,99.85,98.17,96.74,70.84,99.34,94.49
CNN-4A	,38 (0.05)	,99.56,99.34,97.47,99.67,93.52,99.78	,99.89,75.60,98.35,92.75,92.97,99.89	,98.13,96.92,72.53,99.01,94.71
%CNN-4A	,0.045	,99.23,99.34,95.49,99.56,94.95,99.56	,99.56,70.44,98.68,91.43,94.73,100.00	,99.01,98.35,70.66,99.01,94.38
CNN-4A	,33 (0.04)	,99.45,99.23,94.95,99.56,94.73,99.78	,99.56,72.64,97.80,90.66,93.19,99.89	,98.57,97.58,71.65,98.35,94.22
%CNN-4A	,0.035	,99.56,99.34,96.04,99.67,95.93,100.00	,99.89,72.42,97.14,91.98,93.08,99.67	,98.13,96.70,69.89,99.01,94.28
CNN-4A	,29 (0.03)	,98.90,99.56,84.84,99.45,85.82,99.67	,99.67,76.81,88.90,82.64,86.04,99.78	,98.13,96.70,65.16,99.12,91.33
%CNN-4A	,0.025	,99.23,99.78,95.60,99.67,93.52,99.34	,99.67,72.31,98.68,90.66,92.53,99.89	,98.57,98.13,69.23,98.46,94.08
CNN-4A	,28 (0.02)	,89.67,97.36,93.08,89.34,86.26,99.34	,89.45,72.53,98.46,90.22,94.84,99.67	,98.57,96.92,71.65,97.69,91.57
%CNN-4A	,0.015	,99.12,98.46,94.51,99.34,95.82,99.45	,99.23,70.33,98.35,90.00,92.86,99.23	,99.34,97.14,72.20,98.13,93.97
CNN-4A	,28 (0.01)	,99.45,99.56,95.16,99.78,92.64,99.56	,98.90,73.63,98.35,91.76,93.08,99.67	,99.12,97.91,69.12,98.79,94.16
%CNN-4A	,0.005	,89.89,99.67,84.62,89.45,86.59,99.56	,99.23,72.20,98.79,88.24,84.51,99.56	,99.12,97.25,70.88,97.58,91.07
SVM	,204 (Full)	,99.93,99.96,99.74,99.45,99.23,99.89,99.74,79.23,99.82,97.62,99.82,99.89,99.63,98.83,77.77,99.45,96.87
SVM	,38 (0.05)	,99.52,99.82,98.68,99.78,97.58,99.89,99.71,78.39,99.63,93.59,97.99,99.85,99.78,99.01,77.25,99.41,96.24
%SVM	,0.045	,99.71,99.74,98.10,99.23,97.36,99.93,99.60,78.24,99.34,93.70,97.77,99.89,99.67,99.01,75.71,99.49,96.03
SVM	,33 (0.04)	,99.49,99.74,98.13,99.74,97.36,99.89,99.71,76.34,99.56,94.21,98.57,99.96,99.63,98.97,76.41,99.12,96.05
%SVM	,0.035	,99.63,99.67,98.39,99.60,97.44,99.82,99.41,74.98,99.49,93.99,97.95,99.89,99.60,99.01,73.55,99.38,95.74
SVM	,29 (0.03)	,99.45,99.89,98.28,99.56,96.96,99.71,99.78,76.96,99.49,93.37,97.58,99.74,99.67,98.94,75.64,99.41,95.90
%SVM	,0.025	,99.56,99.74,98.39,99.60,97.66,99.85,99.67,76.70,99.56,93.63,97.55,99.71,99.67,99.12,75.75,99.30,95.97
SVM	,28 (0.02)	,99.82,99.82,98.21,99.63,97.62,99.63,99.67,73.66,99.78,93.15,98.06,99.71,99.45,98.90,74.36,99.34,95.68
%SVM	,0.015	,99.85,99.78,97.73,99.49,97.36,99.82,99.38,76.96,99.56,94.07,98.35,99.78,99.63,99.05,74.87,99.38,95.94
SVM	,28 (0.01)	,99.60,99.78,98.24,99.74,97.69,99.67,99.82,75.97,99.45,93.41,98.57,99.74,99.67,98.86,75.49,99.30,95.94
%SVM	,0.005	,99.56,99.82,98.21,99.41,97.40,99.78,99.78,75.20,99.38,93.11,97.40,99.71,99.71,99.08,75.46,99.16,95.76
DT	,204 (Full)	,99.45,98.97,97.00,99.27,97.95,99.56,98.90,66.67,98.39,90.99,95.86,97.80,98.24,95.93,67.69,98.02,93.79
DT	,38 (0.05)	,99.23,98.94,94.87,99.34,96.52,99.23,98.86,65.71,96.41,85.53,94.03,96.48,98.64,96.15,64.95,97.51,92.65
%DT	,0.045	,99.30,98.53,95.24,99.12,95.79,99.19,98.97,65.16,96.67,84.87,93.08,97.91,97.66,95.27,64.91,98.28,92.50
DT	,33 (0.04)	,98.90,98.39,94.21,99.49,95.90,99.30,99.01,64.95,96.34,85.46,92.86,97.66,98.68,95.42,63.52,97.07,92.32
%DT	,0.035	,98.86,98.10,94.62,98.97,96.26,99.05,98.97,62.49,97.25,87.58,93.33,97.95,98.21,95.82,63.26,97.73,92.40
DT	,29 (0.03)	,98.79,98.21,94.03,99.27,95.86,99.27,99.01,65.13,97.36,86.04,93.81,97.88,98.02,95.68,61.76,97.80,92.37
%DT	,0.025	,98.97,98.53,94.58,99.41,96.37,99.52,99.41,63.52,96.34,85.42,91.87,97.36,97.80,95.57,63.52,96.74,92.18
DT	,28 (0.02)	,98.94,98.42,94.80,99.34,95.79,99.08,99.08,63.52,97.03,84.73,92.53,97.73,97.95,95.13,61.90,98.17,92.13
%DT	,0.015	,98.90,98.21,94.58,99.34,96.08,99.27,99.16,63.74,97.25,86.34,92.64,98.02,98.32,94.73,62.20,97.95,92.29
DT	,28 (0.01)	,98.68,98.72,94.10,99.08,96.23,99.30,99.30,63.19,96.78,86.30,92.93,96.45,98.10,95.97,61.94,97.66,92.17
%DT	,0.005	,99.19,98.83,94.76,99.08,96.12,99.05,99.27,62.78,96.37,84.32,93.48,97.69,97.62,95.31,61.06,97.44,92.02
RF	,204 (Full)	,99.85,99.93,99.71,99.52,98.68,99.82,99.49,76.59,99.27,94.29,99.38,99.30,99.12,97.99,74.14,98.86,96.00
RF	,38 (0.05)	,99.52,99.16,98.17,99.67,97.33,99.60,99.60,71.72,98.72,92.31,96.74,99.78,98.50,97.84,72.01,98.94,94.97
%RF	,0.045	,99.45,99.41,98.42,99.34,97.14,99.63,99.30,70.55,98.64,90.92,96.19,99.78,98.57,97.47,71.72,99.19,94.73
RF	,33 (0.04)	,99.52,99.67,98.57,99.63,96.59,99.49,99.38,69.82,98.02,91.47,96.74,99.82,97.73,97.55,70.15,99.16,94.58
%RF	,0.035	,99.45,99.23,98.21,99.45,96.59,99.60,99.27,70.95,98.86,91.61,96.04,99.82,98.64,96.85,70.48,98.90,94.62
RF	,29 (0.03)	,99.27,99.08,98.21,99.60,96.85,99.74,99.38,70.55,98.46,90.40,95.64,99.74,98.21,97.14,70.99,99.12,94.52
%RF	,0.025	,99.71,99.27,97.99,99.49,97.00,99.60,99.19,70.95,98.42,91.61,96.12,99.63,97.99,97.47,69.30,99.12,94.55
RF	,28 (0.02)	,99.60,99.30,98.10,99.52,97.07,99.52,99.45,68.72,98.61,90.92,95.90,99.78,98.21,97.03,69.56,99.12,94.40
%RF	,0.015	,99.05,99.05,98.50,99.78,96.96,99.45,99.16,71.61,98.68,91.17,95.64,99.78,98.35,97.51,70.84,99.05,94.66
RF	,28 (0.01)	,99.23,99.30,97.99,99.41,95.90,99.49,99.49,70.18,98.72,91.28,96.41,99.82,98.42,97.51,71.17,98.94,94.58
%RF	,0.005	,99.38,98.53,97.25,99.41,96.59,99.56,99.56,70.07,98.24,91.50,95.53,99.71,98.35,97.40,70.04,99.01,94.38
}\vrule
\end{table*}

The results (averaged across 30 runs) for Salinas Valley and Pavia University are gathered in Tables~\ref{tab:influence_attention_salinas_heatmap}--\ref{tab:influence_attention_pavia_heatmap}, respectively. The differences between the investigated architectures are not statistically important (i.e.,~CNN-2 compared with CNN-2A, CNN-3 with CNN-3A, and CNN-4 with CNN-4A)---we executed two-tailed Wilcoxon tests to verify the null hypothesis saying that ``appending attention modules to a CNN model leads to notably different classification accuracies of the trained models over the unseen data $\TestSet$'', and this hypothesis can be rejected at $p<0.01$. Therefore, attention modules did not adversely impact the classification performance of the CNNs---they allow for building a high-quality model and selecting the most important bands \emph{at once}. Deeper CNNs (with more convolutional-pooling blocks) delivered more stable results (std. dev. of the accuracy over $\TestSet$ decreased from 0.007 to 0.005 for Salinas, and from 0.03 to 0.01 for Pavia).

\begin{table*}[ht!]
\renewcommand{\tabcolsep}{0.39cm}
\centering
\scriptsize
	\caption{Classification accuracy (in \%) of various models obtained for the full and reduced Pavia University dataset (we report the number of bands and the contamination rate in parentheses; ``Full'' for no reduction). The darker the cell is, the better classification was obtained.}
	\label{tab:influence_attention_pavia_heatmap}
\vrule\pgfplotstabletypeset[%
     color cells={min=0,max=100,textcolor=black},
     /pgfplots/colormap={blackwhite}{rgb255=(255,170,0) color=(white) rgb255=(255,170,0)},
    /pgf/number format/fixed,
    /pgf/number format/precision=4,
    col sep=comma,
    columns/Algorithm/.style={reset styles,string type},
    columns/Bands/.style={reset styles,string type}%
]{
Algorithm,Bands,C1,C2,C3,C4,C5,C6,C7,C8,C9,All
CNN-2, 103 (Full), 85.57,	86.77,	82.38,	97.09,	99.79,	92.87,	90.74,	83.58,	99.33,	90.90
CNN-2A,103 (Full),84.72,85.85,80.28,95.96,98.72,89.65,90.39,82.34,99.15	,89.67
CNN-2A,28 (0.05),82.66,85.53,82.87,97.77,99.47,90.32,91.60,72.02,99.79	,89.11
CNN-2A,20 (0.04),79.04,78.09,76.70,93.94,99.89,88.51,92.45,78.51,100.00	,87.46
CNN-2A,14 (0.03),75.21,66.81,67.34,84.79,99.26,78.94,89.04,75.53,99.68	,81.84
CNN-2A,12 (0.02),65.43,48.72,54.04,90.85,98.30,73.09,87.13,71.81,99.68	,76.56
CNN-2A,9 (0.01),64.36,47.23,46.17,79.89,97.98,73.40,83.94,70.00,99.68	,73.63
CNN-3, 103 (Full), 85.60,	87.59,	83.69,	97.45,	99.86,	92.70,	93.69,	83.58,	99.61,	91.53
CNN-3A	,103 (Full)	,86.31,88.12,81.91,98.09,99.68,93.30,92.62,82.73,99.57,82.33
CNN-3A	,28 (0.05)	,83.19,87.77,69.15,96.91,98.83,88.72,91.06,82.34,91.60,78.96
CNN-3A	,20 (0.04)	,77.34,81.28,79.79,92.98,99.79,86.91,90.53,77.77,99.89,78.63
CNN-3A	,14 (0.03)	,71.81,63.62,70.85,90.00,99.89,81.17,90.11,74.26,99.47,74.12
CNN-3A	,12 (0.02)	,66.49,46.28,59.89,92.23,98.40,73.72,85.64,71.06,99.47,69.32
CNN-3A	,9 (0.01)	,63.40,54.15,44.79,90.32,98.40,65.21,83.83,72.66,99.57,67.24
CNN-4, 103 (Full), 86.63,	89.01,	83.72,	97.34,	99.72,	92.13,	92.55,	84.29,	99.72,	91.68
CNN-4A	,103 (Full)	,84.36,89.72,83.76,98.09,99.75,93.83,94.75,83.90,99.54,91.97
CNN-4A	,28 (0.05)	,76.38,83.62,83.40,95.85,89.89,86.91,92.45,79.57,94.57,86.96
CNN-4A	,20 (0.04)	,80.64,77.77,78.51,92.23,99.57,84.68,90.00,77.13,99.89,86.71
SVM	,103 (Full) 	,88.87,94.36,88.65,97.48,99.86,94.72,94.29,87.94,99.82,94.00
SVM	,28 (0.05)	,80.78,87.48,82.41,96.03,99.82,89.33,92.09,81.81,99.93,89.96
SVM	,20 (0.04)	,76.81,84.29,81.31,94.54,99.79,85.74,92.62,79.15,99.96,88.25
SVM	,14 (0.03)	,72.59,70.21,72.02,91.91,99.65,81.60,91.42,74.40,99.89,83.74
SVM	,12 (0.02)	,68.16,54.54,63.87,93.48,98.69,73.05,88.83,73.40,99.89,79.32
SVM	,9 (0.01)	,64.61,47.70,47.23,91.63,99.18,75.18,86.81,73.23,99.86,76.16
DT,103 (Full)	,79.26,77.23,73.69,92.52,99.22,79.93,85.57,75.50,99.96	,84.76
DT,28 (0.05)	,78.40,76.31,73.30,91.38,98.90,79.01,84.40,71.21,99.93	,83.65
DT,20 (0.04)	,76.17,69.01,66.60,86.42,99.15,78.44,83.76,68.16,99.93	,80.85
DT,14 (0.03)	,71.74,64.36,59.72,83.69,98.83,74.29,82.02,63.40,100.00	,77.56
DT,12 (0.02)	,68.44,53.83,55.89,83.48,98.83,63.76,79.33,62.06,99.96	,73.95
DT,9 (0.01),65.78,50.74,48.79,80.78,98.58,62.77,74.18,59.72,99.93	,71.25
RF,	103 (Full),83.76,83.62,85.99,95.71,99.47,88.65,92.34,83.62,100.00,90.35
RF,	28 (0.05),80.82,81.06,80.07,95.89,99.40,85.39,90.39,80.57,100.00,88.18
RF,	20 (0.04),78.33,71.99,74.54,94.26,99.54,83.44,89.65,78.33,100.00,85.56
RF,	14 (0.03),73.97,63.01,63.79,92.70,99.33,80.46,86.81,71.84,100.00,81.32
RF,	12 (0.02),69.04,54.57,60.92,92.62,98.76,73.76,86.70 ,69.11,100.00,78.39
RF,	9 (0.01),66.28,49.26,51.77,91.21,98.69,73.65,82.59,66.88,100.00,75.59
}\vrule
\end{table*}

\begin{figure}[ht!]
	\centering
	\includegraphics[width=0.95\columnwidth]{convergence_time}
	\caption{Average number of epochs before reaching convergence (first row) alongside the average processing time [s] of s single epoch (second row).}
	\label{fig:convergence_time}
\end{figure}

The average number of training epochs before reaching convergence alongside the average processing time\footnote{Using NVIDIA Titan X Ultimate Pascal GPU 12 GB GDDR5X.} of a single epoch are presented in Fig.~\ref{fig:convergence_time}. Appending attention modules increases neither the processing time nor the number of epochs (standard deviations remain the same too), hence they can be considered as a seamless CNN extension to enhance its operational ability (it not only does learn how to effectively classify HSI pixels but also selects important HSI bands).

\begin{table}[ht!]
	\scriptsize
	\centering
	\caption{Average grid-search time [min] for the a) Salinas Valley and b) Pavia University datasets for all contamination rates $\ContaminationRate$.}
	\label{tab:grid_search_statistics}
	\renewcommand{\tabcolsep}{0.25cm}
	\begin{tabular}{rcrrrrrr}
		\Xhline{2\arrayrulewidth}
		&Algorithm & $\ContaminationRate\rightarrow$ 0.01 & 0.02 & 0.03 & 0.04 & 0.05 & Full \\
		\hline
		& SVM & 5.27 & 5.50 & 6.18 & 6.58 & 7.82 & 57.78\\
		a) & DT & 0.18 & 0.18 & 0.18 & 0.21 & 0.23 & 1.09 \\
		& RF & 0.71 & 0.71 & 0.71 & 0.73 & 0.78 & 1.31 \\
\hline
		& SVM & 1.40 & 1.55 & 1.66 & 2.00 & 2.39 & 7.19\\
		b) & DT & 0.06 & 0.07 & 0.08 & 0.10 & 0.13 & 0.48 \\
		& RF & 0.46 & 0.45 & 0.45 & 0.52 & 0.58 & 0.91 \\
		\Xhline{2\arrayrulewidth}
	\end{tabular}
\end{table}

\subsection{Classification Accuracy over Reduced Datasets}

In this experiment, we evaluated the classification performance of well-established state-of-the-art models trained using full and reduced HSI datasets. These classifiers included Support Vector Machines (SVMs), Decision Trees (DTs), and Random Forests (RFs). We followed the same experimental scenario, however we additionally executed grid search to optimize the hyperparameters of all models: $C$ and $\gamma$ of the radial basis function kernel in SVMs, minimum samples per leaf in DTs, number of trees in RFs, and minimum samples in a split in both DTs and RFs. The training with grid search was repeated 30 times (Monte-Carlo cross-validation). We report the grid-search characteristics in Table~\ref{tab:grid_search_statistics}. The results show that decreasing the HSI datasets (the lower $\ContaminationRate$ values are, the higher reduction rates are obtained, as given in Table~\ref{tab:number_of_selected_bands}) helps shorten the grid-search time which can easily become very large for full datasets (see e.g., SVM for Salinas). Such hyperparameter optimizations are not necessary in our CNNs.

The results gathered in Tables~\ref{tab:influence_attention_salinas_heatmap}--\ref{tab:influence_attention_pavia_heatmap} show that for most of the classes, the performance of the investigated classifiers is not diminished by our band-selection technique. Although there exist classes for which the accuracy decreased (e.g., C2 and C3 in Pavia), the differences for other classes are rather negligible, especially for CNNs for $\ContaminationRate\geq0.03$ (note that CNN-4A could not be trained for very small number of bands because of the dimensionality reduction performed in the pooling layers). This observation is proved in Table~\ref{tab:wilcoxon_stats}, where we report the results of the Wilcoxon tests (across both Salinas and Pavia datasets) executed to analyze the differences between models trained with different datasets (with and without reduction). Although the differences in the accuracy of other classifiers trained with the reduced numbers of bands are statistically important (at $p<0.01$), they are not as dramatic as in other state-of-the-art band-selection algorithms~\cite{DBLP:journals/corr/abs-1802-06983}.

\begin{table}[ht!]
	\scriptsize
	\centering
	\caption{Wilcoxon tests showed that reducing HSI datasets does not affect the performance of our CNNs (contamination rate $\ContaminationRate\geq 0.03$). The differences which are important are boldfaced.}
	\label{tab:wilcoxon_stats}
	\renewcommand{\tabcolsep}{0.2cm}
	\begin{tabular}{rcrrrrrr}
		\Xhline{2\arrayrulewidth}
		Algorithm &    $\ContaminationRate\downarrow\rightarrow$     & 0.02 & 0.03 & 0.04 & 0.05 & Full \\
		\hline
		              & 0.01 & 0.4777 & 0.0316 & \textbf{0.0008} & 0.6599 &  0.0232 \\
	                  & 0.02 &  & 0.0034 & \textbf{0.0002} & 0.3472 & \textbf{0.0006}  \\
		      CNN-2A  & 0.03 &  &  & \textbf{0.0012} & 0.749 & 0.0658  \\
                      & 0.04 &  &  &  & 0.0767 &  0.6455 \\
                      & 0.05 &  &  &  &  & 0.1527  \\
                      \hline
                      & 0.01 & 0.3735 & 0.0819 & \textbf{0.0058} & \textbf{0.0017} & \textbf{0.0017}  \\
	                  & 0.02 &  & 0.0128 & \textbf{0.0004} & \textbf{0.0014} & \textbf{0.0023}  \\
		      CNN-3A  & 0.03 &  &  & 0.0147 & 0.0278 & 0.0549  \\
                      & 0.04 &  &  &  & 0.1443 & 0.0751  \\
                      & 0.05 &  &  &  &  & 0.234  \\
                      \hline
              \multirow{ 2}{*}{CNN-4A}  & 0.04 &  &  &  & 0.1676 &  0.0188 \\
                      & 0.05 &  &  &  &  & 0.0588  \\
                      \hline
                      & 0.01 & 0.9840 & 0.0188 & \textbf{0.0033} & \textbf{0.0004} & \textbf{0.0001}  \\
	                  & 0.02 &  & 0.0293 & \textbf{0.0021} & \textbf{0.0004} &  \textbf{0.0001} \\
		      SVM     & 0.03 &  &  & \textbf{0.0029} & \textbf{$<$0.0001} & \textbf{0.0001}  \\
                      & 0.04 &  &  &  & \textbf{0.0076} & \textbf{0.0003}  \\
                      & 0.05 &  &  &  &  & \textbf{0.0006}  \\
                      \hline
                      & 0.01 & 0.0349 & 0.0198 & 0.0147 & \textbf{0.0014}  &  \textbf{$<$0.0001} \\
	                  & 0.02 &  & \textbf{0.0091} & \textbf{0.0042} & \textbf{0.0013} & \textbf{0.0001}  \\
		      DT      & 0.03 &  &  & 0.0588 & \textbf{0.0053} & \textbf{$<$0.0001}  \\
                      & 0.04 &  &  &  & \textbf{0.0016} &  \textbf{0.0001} \\
                      & 0.05 &  &  &  &  &  \textbf{0.0001} \\
                      \hline
                      & 0.01 & 0.1835 & 0.0466 & 0.0128 & \textbf{0.0001} & \textbf{0.0001}  \\
	                  & 0.02 &  & 0.0735 & \textbf{0.0022} & \textbf{0.0002} & \textbf{0.0001}  \\
		      RF      & 0.03 &  &  & 0.0183 & \textbf{0.0001} & \textbf{0.0001}  \\
                      & 0.04 &  &  &  & \textbf{0.0014} &  \textbf{0.0002} \\
                      & 0.05 &  &  &  &  & \textbf{0.0003}  \\
\hline
		\Xhline{2\arrayrulewidth}
	\end{tabular}
\end{table}
\section{Conclusion}\label{sec:conclusions}

We proposed new attention-based convolutional neural networks (CNNs) for selecting bands from hyperspectral images (HSI), and for building efficient classifiers of such data \emph{at once}. Our attention modules can be seamlessly incorporated into any CNN architecture and affect neither classification abilities nor training times of CNNs. A rigorous experimental validation executed over two benchmark HSI datasets (Salinas Valley and Pavia University) and backed up with statistical tests showed that the attention-based models extract important bands from HSI, and allow us to obtain state-of-the-art classification accuracy using only a fraction of all bands (14--19\% for Salinas, and 9--27\% for Pavia). Various visualizations helped understand which parts of the spectrum are important in each dataset (our band-selection can enhance interpretability of HSI), and showed that our approach is data-driven and can be easily applied to any HSI dataset. It can be used to effectively reduce HSI datasets on-board of satellites before transferring HSI to Earth without sacrificing the amount of important information being transferred. Our attention modules can be used in other deep architectures, and in other HSI problems (e.g.,~HSI un-mixing). We also work on comparing our method with feature extraction, especially principal component analysis-based techniques which are fairly successful in HSI reduction, however their extracted features are still difficult to interpret.

% if have a single appendix:
%\appendix[Proof of the Zonklar Equations]
% or
%\appendix  % for no appendix heading
% do not use \section anymore after \appendix, only \section*
% is possibly needed

% use appendices with more than one appendix
% then use \section to start each appendix
% you must declare a \section before using any
% \subsection or using \label (\appendices by itself
% starts a section numbered zero.)
%


% use section* for acknowledgment
%\section*{Acknowledgment}



% Can use something like this to put references on a page
% by themselves when using endfloat and the captionsoff option.
\ifCLASSOPTIONcaptionsoff
  \newpage
\fi

\bibliographystyle{ieeetran}
\bibliography{ref_all}


% that's all folks
\end{document}


