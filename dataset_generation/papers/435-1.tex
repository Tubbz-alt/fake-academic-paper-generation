\documentclass[10pt,twocolumn,letterpaper]{article}

\usepackage{cvpr}
\usepackage{times}
\usepackage{epsfig}
\usepackage{graphicx}
\usepackage{amsmath}
\usepackage{amssymb}
\usepackage{xspace}
\usepackage{booktabs}
\usepackage{array}
\usepackage{multirow}
\usepackage{subcaption}
\usepackage{enumitem}
\newcolumntype{P}[1]{>{\centering\arraybackslash}p{#1}}
\newcommand{\Model}{\textsc{TeaStud}\xspace}
\newcommand{\tea}{\textsc{Teacher}\xspace}
\newcommand{\stud}{\textsc{Student}\xspace}
\newcommand{\full}{\textsc{Teacher-Full}}
\newcommand{\uniform}[1]{\textsc{Teacher-Uniform-}$#1$}
\newcommand{\random}[1]{\textsc{Teacher-Random-}$#1$}
\newcommand{\final}[1]{\textsc{Student-}$#1$\textsc{-Final}}
\newcommand{\inter}[1]{\textsc{Student-}$#1$\textsc{-Intermediate}}
% Include other packages here, before hyperref.

% If you comment hyperref and then uncomment it, you should delete
% egpaper.aux before re-running latex.  (Or just hit 'q' on the first latex
% run, let it finish, and you should be clear).
\usepackage[pagebackref=true,breaklinks=true,letterpaper=true,colorlinks,bookmarks=false]{hyperref}

\cvprfinalcopy % *** Uncomment this line for the final submission

\def\cvprPaperID{0011} % *** Enter the CVPR Paper ID here
\def\httilde{\mbox{\tt\raisebox{-.5ex}{\symbol{126}}}}

% Pages are numbered in submission mode, and unnumbered in camera-ready
\ifcvprfinal\pagestyle{empty}\fi
\begin{document}

%%%%%%%%% TITLE
\title{\textit{I Have Seen Enough:} A Teacher Student Network for Video Classification Using Fewer Frames}

\author{\\
Shweta Bhardwaj\\
Indian Institute of Technology Madras\\
{\tt\small cs16s003@cse.iitm.ac.in}
% For a paper whose authors are all at the same institution,
% omit the following lines up until the closing ``}''.
% Additional authors and addresses can be added with ``\and'',
% just like the second author.
% To save space, use either the email address or home page, not both
\and
\\
Mitesh M. Khapra\\
Indian Institute of Technology Madras\\
{\tt\small miteshk@cse.iitm.ac.in}
}

\maketitle
%\thispagestyle{empty}

%%%%%%%%% ABSTRACT
\begin{abstract}
Over the past few years, various tasks involving videos such as classification, description, summarization and question answering have received a lot of attention. Current models for these tasks compute an encoding of the video by treating it as a sequence of images and going over every image in the sequence. However, for longer videos this is very time consuming. In this paper, we focus on the task of video classification and aim to reduce the computational time by using the idea of distillation. Specifically, we first train a \textit{teacher} network which looks at all the frames in a video and computes a representation for the video. We then train a \textit{student} network whose objective is to process only a small fraction of the frames in the video and still produce a representation which is very close to the representation computed by the \textit{teacher} network. This smaller student network involving fewer computations can then be employed at inference time for video classification. We experiment with the YouTube-8M dataset and show that the proposed student network can reduce the inference time by upto $30\%$ with a very small drop in the performance. 

%observing that humans typically fast forward a   defatco paradigm for video processing is deep learning wherein the video is treated as a sequence of images and a recurrent neural network is used to encode the video based on the encoding of each obtained from a convolutional neural network.  models developed for these tasks use a combination of convolutional neural networks and recurrent neural networks. The video processing has received a lot of attention with emphasis on tasks such as video classification , 
%The popularity of social media platforms explosion of video content on the web 

%Today video content has become extremely prevalent on the internet touching all aspects of our life including education, entertainment, sports, \textit{etc.} Today video content has become extremely prevalent on the internet touching all aspects of our life including education, entertainment, sports, \textit{etc.} Over the past few years, video processing has received a lot of attention with emphasis on tasks such as 
%The popularity of social media platforms explosion of video content on the web 
\end{abstract}

%%%%%%%%% BODY TEXT
\section{Introduction}
% With abundant flooding of video content on internet, there is a need for large scale datasets in understanding the visual information for various applications
Today video content has become extremely prevalent on the internet influencing all aspects of our life such as education, entertainment, sports \textit{etc.} This has led to an increasing interest in automatic video processing with the aim of identifying activities \cite{action-recog, video_beyond_short_snippet_classify}, generating textual descriptions \cite{lstm-description}, generating summaries \cite{video_summ}, answering questions \cite{tgif-qa} and so on. Current state of the art models for these tasks are based on the neural encode-attend-decode paradigm \cite{Bahdanau2,Bahdanau}. Specifically, these approaches treat the video as a sequence of images (or frames) and compute a representation of the video by using a Recurrent Neural Network (RNN). The input to the RNN at every time step is an encoding of the corresponding image (frame) at that time step as obtained from a Convolutional Neural Network. Computing such a representation for longer videos can be computationally very expensive as it requires running the RNN for many time steps. Further, for every time step the corresponding frame from the video needs to pass through a convolutional neural network to get its representation. Such computations are still feasible on a GPU but become infeasible on low end devices which have power, memory and computational constraints. 
%For example, one of the state of the art models for video classification (in the context of the YouTube8M dataset) takes 0.034 seconds to process a single video at inference time on a Tesla K80 GPU.

In this work, we focus on the task of video classification \cite{Youtube8M} and aim to reduce the computational time. We take motivation from the observation that when humans are asked to classify a video or recognize an activity in a video they do not typically need to watch every frame or every second of the video. A human would typically fast forward through the video essentially seeing only a few frames and would still be able to recognize the activity (in most cases). Taking motivation from this we propose a model which can compute a representation of the video by looking at only a few frames of the video. Specifically, we use the idea of distillation wherein we first train a computationally expensive \textit{teacher} network which computes a representation for the video by processing all frames in the video. We then train a relatively inexpensive \textit{student} network whose objective is to process only a few frames of the video and produce a representation which is very similar to the representation computed by the teacher. This is achieved by minimizing the squared error loss between the representations of the student network and the teacher network. At inference time, we then use the student network for classification thereby reducing  the time required for processing the video. We experiment with the YouTube-8M dataset and show that the proposed student network can reduce the inference time by upto $30\%$ and still give a classification performance which is very close to that of the expensive teacher network. 
%Large scale video datasets play major role in understanding the visual information inundating the internet for various applications like Action recognition in videos, Summarization of movie clips, Real time object detection on traffic data etc. But the analysis of such data comes with major challenges. It is evident that analyzing the video needs more complex and sophisticated approaches as compared to the image related tasks (e.g. image classification, image question answering) as images are static in nature. On the other hand, videos involve the extra dimension of time.
% the good performance comes at the cost of large computations in these models.
%In all the video related tasks, temporal information that is spanned across the different frames of video is significant in understanding the semantic meaning of a video in its entirety. In most of the current applications for video analysis like video classification, summarization, video question answering e.t.c, the respective state-of-art model use a recurrent neural network to capture the sequential (in other words temporal) information and convolutional neural networks to get the spatial features of videos or aggregate both these video features at the same time. In these networks, the generic setting of the model processes all the frames of video for the given task. For processing the videos which are very long, it can be computationally expensive because we first need to pass each frame through CNN and then the sequence of frames through a recurrent neural network. For example in YouTube8-M dataset \cite{Youtube8M}, on average each video is of 200 seconds. However, this is not the way human beings process the videos to determine the activity performed in the video. We typically fast forward the video and only look at same sampled frames and that should be enough to decide the class of video.

%Motivated by this insight, we would like to implement the similar idea in a neural network and hence, have proposed a \textit{teacher-student} network, where the teacher actually processes the entire video and learns an encoding of video. The student on the other hand looks at only few frames from the video and we ensure that the encoding learnt by the student is very similar to the encoding learnt by the teacher. 

 %Another motivation behind this paper is that we can see that good performance of complex neural networks comes at the cost of extensive computations performed in the processing of all the frames of video. This makes it difficult for the deep learning models to run on some target machine with low computation capacity and energy requirements. In this paper, we have shown that by using our proposed method, we can achieve the same accuracy as achieved by using the entire video, by processing only one-tenth or even one-twentieth of the frames of the video. This makes it possible to deploy these models on low power devices. 

% In this paper, we have memorizing the generic embedded representation of video learnt by bigger network, by training a smaller network side-by-side. This method ensures that the final representation learnt by the smaller network captures the significant information needed to perform the task at hand (for e.g. classification in this case). The major contribution of this method is the significant reduction in number of computations performed given the x$\%$ capacity of target machine

% \subsection{Mitesh sir}
% For processing videos, if the video is very long, it can be computationally very expensive because for every frame in the video we first need to pass it through a convolutional neural network, and then the sequence of frames needs to pass through a recurrent neural network.
% For example, in the YouTube dataset, on an average each video is of length 200 seconds.
% However, this is not how humans process videos; we don't look at every frame.
% If I want to determine what is the activity being performed in the video, I would typically fast forward the video and only look at some sampled frames and that should be enough for me to decide the class of the video.
% We would like to implement the same idea in a neural network, and the way we propose to do this is to use a teacher-student network, where the teacher actually processes the entire video and learns an encoding of the video.
% The student on the other hand looks at only few frames from the video and we ensure that the encoding learnt by the student is very similar to the encoding learnt by the teacher using an appropriate loss function to minimize the difference between the student encoding and the teacher encoding.
% We show that by using this method at inference time we can achieve the same accuracy as achieved by using the entire video by processing only one-tenth, or even one-twentieth of the frames of the video.
% We perform experiments with the YouTube dataset.

%-------------------------------------------------------------------------

\begin{figure*}[t]
\centering
%\includegraphics[scale=0.6]{diagrams/figure-teacher-student-image.pdf}
\includegraphics[width=12cm, height=5cm]{diagrams/figure-teacher-student-image.pdf}
\caption{\label{diagram} Architecture of \textsc{Teacher-Student} network for video classification}
\end{figure*}

\section{Related Work}
We focus on video classification in the context of the YouTube-8M dataset \cite{Youtube8M}. On average the videos in this dataset have a length of $200$ seconds. Each video is represented using a sequence of frames where every frame corresponds to one second of the video. These one-second frame representations are pre-computed and provided by the authors. The authors also proposed a simple baseline model which treats the entire video as a sequence of these one-second frames and uses an Long short-term memory networks (LSTM) to encode this sequence. Apart from this, they also propose some simple baseline models like Deep Bag of Frames (DBoF) and Logistic Regression \cite{Youtube8M}.
Various other classification models \cite{willow, monkey-typing, temporal-models-yt8m,aggregate-frame-features, deep-models-videos} have been proposed and evaluated on this dataset which explore different methods of: 1) feature aggregation in videos
(temporal as well as spatial) \cite{aggregate-frame-features,willow}, 2) capturing the interactions between labels \cite{monkey-typing} and 3) learning new non-linear units to model the interdependencies among the activations of the network \cite{willow}. 
%and have done extensive evaluation of different ensemble methods \cite{deep-models-videos} for video classification on YouTube-8M dataset.
We focus on one such state of the art model, \textit{viz.}, a hierarchical model whose performance is close to that of %best performing 
the best model on this dataset. We take this model as the teacher network and train a comparable student network as explained in the next section.\\
%which is a hierarchical model containing two layers to encode the video. The idea is to divided the video into k sequences where each sequence consists $n$ one second frames. first layer encodes $k$ one-second frames as 3
%\cite{cnn_for_video_classify, video_beyond_short_snippet_classify}, we would also like to mention about other video processing tasks such as summarization \cite{video_summ, seq-to-seq}, video question answering \cite{tgif-qa, movie-qa} and so on. Most state of the art methods extract features from individual frames or blocks of frames using spatial features from convolutional neural networks \cite{resnet} and temporal features from recurrent neural networks \cite{video_beyond_short_snippet_classify} or spatio-temporal from 3D convolutional networks\cite{c3d-videos}. All these models are computationally expensive as they process every frame of the video. In this work we focus on one of the state of the art models \cite{hierarchiccal model} proposed for video classification on the YouTube8M dataset. This model uses a hierarchical encoder decoder model wherein the lower encoder
\hspace*{3mm}Our work is inspired by the work on model compression in the context of image classification. For example, \cite{do-deep-really-deep,know-distill,fitnets} use \textit{Knowledge Distillation} to learn a more compact \textit{student} network from a computationally expensive \textit{teacher} network. The key idea is to train a shallow student network to mimic the deeper teacher network, by ensuring that the final output representation and the intermediate hidden representations produced by the student network are very close to those produced by the teacher network. While in their case the teacher and student differ in the number of layers, here, the teacher and student network differ in the number of time steps of frames processed by two networks. %which can be achieved by training the student on soft targets from the teacher rather than hard targets. This enables the student network to learn the generic representations of teacher which are critical for task at hand, while maintaining the same performance of the student network. %There are several other variants of this technique for e.g. \cite{}, which extend this idea to allow the training of student model which not only learn the outputs but also the intermediate representations learned by the teacher as \textit{hints} to improve the general representations and final performance of the student. 
%Our proposed idea is based on the \textit{Knowledge Distillation} technique, that we have extended to video representation learning in a recurrent neural network on classification task \cite{Youtube8M}. \textit{To the best of our knowledge, this is the first work which attempts to implement the video representation learning in the teacher-student setting on video classification task.}
\section{Proposed Approach}\label{section3}
Our model contains a teacher network and a student network. The teacher network can be any state of the art video classification model but in this work we consider the hierarchical RNN based model. This model assumes that each video contains a sequence of $b$ equal sized blocks. Each of these blocks in turn is a sequence of $m$ frames thereby making the entire video a sequence of sequences. In the case of the YouTube-8M dataset, these frames are one-second shots of the video and each block $b$ is a collection of $m$ such one-second frames. The model contains a lower level RNN to encode each block (sequence of frames) and higher level RNN to encode the video (sequence of blocks). As is the case with all state of the art models for video classification, this teacher network looks at all the $N$ frames of video ($F_{0}, F_{1},\dots,F_{N-1}$) and computes an encoding $\mathcal{E}_{T}$ of the video, which is then fed to a simple feedforward neural network with a multi-class output layer containing a sigmoid neuron for each of the $C$ classes (a video can have multiple labels). The parameters of the teacher network as well as the output layer are learnt using  a standard multi-label classification loss $\mathcal{L}_{model}$, which is a sum of the cross-entropy losses between the true labels $y$ and predictions $\hat{y}$ for each of the $C$ classes, given by:\[ \mathcal{L}_{model} = (- 1) \sum_{i=1}^{C} y_{i} \log(\hat{y}_{i}) + (1- y_{i}) \log(1- \hat{y}_{i}) \]
%igure \ref{diagram}).  
%Generally in any video related task, we learn the representation of video by using all the frames of video and then pass it to the end task model, which is video classifier in our case. As we have discussed that processing the whole video (all the frames) is very computationally expensive task so, we have proposed a new idea of learning a student model which works with fewer frames and hence, is less computationally expensive. 
%The teacher network looks at all the frames of video ($F_{0}, F_{1},\dots,F_{N}$) and learns the encoding of video as $\mathcal{E}_{T}$ which is then fed into the video classifier (see figure \ref{diagram}). The parameters of teacher network and video classifier are trained using standard multi-label classification loss $\mathcal{L}_{model}$ between predictions of model $n^{'}$ and the corresponding true class labels $n$ for all the $C$ classes.
%\begin{equation*}
%\mathcal{L}_{model} = (- 1) \sum_{i=1}^{C} n_{i} \log(n_{i}^{'}) + (1- n_{i}) \log(1- n_{i}^{'})
%\end{equation*}
In addition to this teacher network, we introduce a student network which only processes every $j^{th}$  frame ($F_0, F_j, F_{2j}, \dots, F_{\frac{N}{j}-1} $) of the video and computes a representation $\mathcal{E}_S$ of the video from these $\frac{N}{j}$ frames (which constitutes $\frac{100}{j}$ = $k$ $\%$ of $N$ frames). At the time of evaluation, this representation is fed to the feedforward network with a multi-class output layer. We introduce an additional loss function as shown below which ensures that the representation computed by the student network is very similar to the representation computed by the teacher network.  
%that is very similar to the encoding learnt by the teacher ($\mathcal{E}_T$) using an appropriate loss function to minimize the difference between the student encoding and the teacher encoding (see figure [\ref{diagram}]). To further explore, we have experimented with second variant of our approach, in which the student network learns the intermediate states along with the final encoding of video from the teacher network.
%We have used square error loss between video encodings from the teacher network and the student network to train the parameters of student. The student network looks at every $k^{th}$ frame and tries to mimic the teacher by minimizing the $\mathcal{L}_{student}$. The video encoding from the teacher is passed to the video classifier and the whole network is trained in parallel with the student network. 
\begin{equation*}
 \mathcal{L}_{student} = || \mathcal{E}_{T} - \mathcal{E}_{S} ||^{2}
\end{equation*}
We also try a simple variant of the model, where in addition to ensuring that the final representations $\mathcal{E}_S$ and $\mathcal{E}_T$  are similar, we also ensure that the intermediate representations of the models are similar. In particular, we ensure that the representation of the frames $j$, $2j$ and so on computed by the teacher and  student network are very similar by minimizing the squared error distance between the corresponding intermediate representations. The parameters of the teacher network, student network and output layer are trained jointly as shown in the Figure \ref{diagram}. Note that for ease of illustration, in the figure, we show a simple RNN model as opposed to a hierarchical RNN model. 

\begin{table*}
\centering
 \resizebox{0.58\linewidth}{!}{
\centering
% \begin{subtable}{0.6\linewidth}
\begin{tabular}{l|c@{\quad}c@{\quad}c@{\quad}c}
\toprule
\textsc{Model}           & \textsc{AVG-Hit}@1 & \textsc{PERR}  & m\textsc{AP}   & \textsc{GAP} \\
\toprule
\full       & 0.862    & 0.736 & 0.402 & 0.809  \\
\midrule
%\uniform{50}       & 0.859    & 0.731 & 0.390  & 0.804  \\
\uniform{50} & 0.859 &	0.731	& 0.390	& 0.804 \\
\uniform{25}      & 0.855	& 0.725	& 0.371	& 0.798 \\
\uniform{10}         & 0.848    & 0.716 & 0.362 & 0.788 \\
\uniform{5}         & 0.834    & 0.698 & 0.333 & 0.770 \\
\midrule
\random{50}      & 0.841    & 0.702 & 0.305 & 0.775 \\
\random{25}	& 0.832	   & 0.693 & 0.297 & 0.765  \\
\random{10}        & 0.829    & 0.693 & 0.320  & 0.765 \\
\random{5} & 0.804	& 0.665	& 0.288	& 0.731 \\
\midrule
\final{50} & 0.860 &	0.733	& 0.360	& 0.803 \\
\final{25} & 0.857	& 0.727	& 0.385	& 0.802 \\
\final{10} & 0.852    & 0.721 & 0.375 & 0.795 \\
\final{5} & 0.842  & 0.710  & 0.359 & 0.783 \\
\midrule
\inter{50} & 0.862	& 0.739	& 0.385	& 0.805 \\
\inter{25} & 0.851 & 0.718& 0.346&	0.792 \\
\inter{10} & 0.854  & 0.725  & 0.382	& \textbf{0.799}\\
\inter{5} & 0.845 &	0.720	& 0.356	& 0.787 \\
\bottomrule
\end{tabular}
}
\newline
\caption{Performance comparison of proposed \final{k} and \inter{k} models with different baselines on YouTube-8M dataset. Here, \stud-$k$ refers to $k\%$ of frames used by student network. \textsc{Final} encoding and \textsc{Intermediate} encoding refer to two simple variants of the proposed framework.}
\label{table1}
% \end{subtable}\qquad\qquad\begin{subtable}{0.3\linewidth}
% \centering
% \begin{tabular}{@{}l@{}c@{}}
% \toprule
% \multirow{2}{*}{\textsc{Model}}    & \textsc{Evaluation }\\
%  & \textsc{Time (min)} \\
% \toprule
% Original       & \\
% \midrule
% Every-2        & \\
% Every-10         & 360.0 \\
% Every-20         &  350.0  \\
% \midrule
% \Model-every-10 Final &  370.0 \\
% \Model-every-20 Final &  356.7\\
% \midrule
% \Model-every-10 Intermediate &  371.2\\
% \Model-every-20 Intermediate & 354\\
% \midrule\\
% \end{tabular}
% \label{table2}
% \newline
% \caption{Comparison of Evaluation time of different models on validation} 

% \end{subtable}
\end{table*}

\section{Experimental Setup}
In this section, we describe the dataset used for our experiments, the hyperparameters that we considered, the baseline models that we compare with and the performance of the two variants of our model.\\

%We have empirically evaluated the proposed approach on video classification task using the YouTube8-Million dataset \cite{Youtube8M}. 
\noindent \textbf{1. Dataset:}
The YouTube-8M dataset \cite{Youtube8M} contains 8
million videos with multiple classes associated with each video. The average length of a video is $200s$ and the maximum length of a video is $300s$. The authors of the dataset have provided pre-extracted audio and visual features for every video such that every second of the video is encoded as a single frame feature. The original dataset consists of 5,786,881 training ($70\%$), 1,652,167 validation ($20\%$) and 825,602 test examples ($10\%$). Since \cite{Youtube8M} does not provide access to the test set, we have reported results on the validation dataset. In this work, we do not use any validation set as we experiment with a fixed set of hyperparameters as explained below.\\

\noindent \textbf{2. Hyperparameters:}
For all our experiments, we used Adam Optimizer with the initial learning rate set to $0.001$  and then decrease it exponentially with $0.95$ decay rate. We used a batch size of $256$. For both the student and teacher networks we used a $2$-layered MultiRNN Cell with cell size of $1024$ for both the layers of the hierarchical model. The size of the hidden representation of the LSTM was 2048. %and evaluation is done with batch size of $512$ on NVIDIA-DGX gpus.
For regularization, we used dropout ($0.5$) and $\mathbf{L}_{2}$ regularization penalty of $2$ for all the parameters. We trained all the models for 5 epochs. For the teacher network we chose the value of $m$ (number of frames per block) to be 20 and for the student network we set the value of $m$ to 5. We first train the teacher, student and output layer jointly using the two loss functions described in Section \ref{section3}. After that, we remove the teacher network and finetune the student network and the output layer. \\

%As reported by \cite{willow}, the performance on validation and test datasets are comparable. We have reported the results on validation data using the following evaluation metrics : 
\noindent \textbf{3. Evaluation Metrics:} We used the following metrics for evaluating the performance of different models \cite{Youtube8M}:
\begin{itemize}[leftmargin=*,noitemsep]
\item GAP (Global Average Precision): is defined as
\[GAP = \sum_{i=1}^{P} p(i) \nabla r(i)\]
where $p(i)$ is the precision of prediction $i$, $r(i)$ is the
recall of prediction $i$ and $P$ is the number of predictions
(label/confidence pairs). We limit our evaluation to only top-$20$ predictions for each video as mentioned in the YouTube-8M
Kaggle competition.
\item AVG-Hit@$t$ : Fraction of test samples for which the model predicts at least
one of the ground truth labels in the top $t$ predictions. 
\item PERR (Precision at Equal Recall Rate) : For each sample (video), we compute the precision of the top $L$ scoring labels, where $L$ is the number of labels in the ground truth for that sample. The PERR metric is the average of these precision values across all the samples.  
\item mAP (Mean Average Precision) : The mean average precision is computed
as the unweighted mean of all the per-class average precisions.
\end{itemize}
%\section{Implementation Details :}
%For all the experiments, we have used Adam Optimizer with learning rate of $0.01$ and then decrease it exponentially with $0.95$ decay rate. All the models have been trained with batch size of $256$ and evaluation is done with batch size of $512$ on NVIDIA-DGX gpus. The teacher model, along with the classifier is trained with cross entropy loss $\mathcal{L}_{Model}$ whereas the student model is trained with squared error loss between the student network output and teacher network output. For regularisation of models, we have used dropout ($0.5$) and $l_{2}$ regularization penalty of $2$ on all the parameters. 
\noindent \textbf{4. Baseline Models:} As mentioned earlier the student network only processes $k\%$ of the frames in the video. We report results with different values of $k$ : $5$, $10$, $25$ or $50$ and compare the performance of the student network with the following versions of the teacher network:
\begin{enumerate}[leftmargin=*,label=\alph*)]
\item \full: The original hierarchical model which processes all the frames of the video.
\item \uniform{k} : A hierarchical model trained from scratch which only processes $k\%$ of the frames of the video. These frames are separated by a constant interval and are thus equally spaced. However, unlike the student model this model does not try to match the representations produced by the full teacher network. 
\item \random{k}:  A hierarchical model trained from scratch which only processes $k\%$ of the frames of the video. These frames are sampled randomly from the video and may not be equally spaced.
\end{enumerate}

We refer to our proposed student network which processes $k\%$ of the frames and matches its final representation to that of the teacher as \final{k}. We refer to the student network which matches all the intermediate representations of the teacher network in addition to the final representation as  \inter{k}.


%The student network in proposed model `\Model' has a similar architecture as the default one, with number of inputs to the first layer as $5$ . These simple baselines are used to evaluate the working of proposed model (\Model).

\section{Results} 
The results of our experiments are summarized in Tables \ref{table1} (performance) and \ref{table2} (computation time). We can show that the observed results are enough to convey the main findings of our work as discussed below. \\
%Some of our experiments are still running and Table \ref{table1} is incomplete. In particular, the experiments for \random{5}, \final{25} and \inter{25}, \inter{5} are still running. However, the above results are enough to convey the main findings of our work as discussed below. 
\noindent\textbf{1. Performance comparison against baselines:} As the percentage of frames processed decreases, there is a gap in the performance of \tea and \uniform{50}. However, this gap is not very large. In particular, even when we process only $10\%$ of the frames (\uniform{10}) the drop in AVG-Hit@1, PERR, mAP and GAP is only $2$-$4\%$. As expected, sampling equally spaced frames from the video (\textsc{Uniform}) gives better performance than randomly sampling frames from the video (\textsc{Random}). Further, the gap between the performance of the student network and teacher network is even smaller. In particular, \random{k}$<$ \uniform{k} $<$ \stud-$k$ $<$ \full. This suggests that the student network indeed learns better representations which are comparable to the representations learned by the \tea network. In fact, when we train the student network to match all the intermediate representations produced by the teacher network then we get the best performance. 

\noindent\textbf{2. Computation time of different models:} 
As expected, the computation time of all the models that process only $k \%$ of the frames ($<N$) is much less than the computation time of the teacher network which processes all ($N$) frames of the video (see Table \ref{table2}). We would like to highlight that the \inter{10} gives a drop of $1.2 \%, 0.8 \%, 1.1\%$ and $2 \%$ in GAP, AVG-Hit@1, PERR and mAP scores respectively while the inference time drops by $30\%$.

%\subsection{Variant 1: Performance of \Model using Final Encoding of Teacher}
%We have experimented with two variants of proposed model : 1) Learn the final encoding of teacher 2) Learn the intermediate states of teacher along with the final encoding. 
%\Model model is trained by learning student along with the teacher. After training the model for two epochs, we can finetune the whole network by replacing the teacher with student. By comparing the performance of default, random and every $k^{th}$ baselines in table \ref{table1}, we can say that the average performance of the model improves by increasing the number of frames sampled by it. It is clearly evident from the results that the random baseline models are much worse than every $k^{th}$. Also, we can see that sampling every $2^{nd}$ frame is very close to baseline but it comes with very less improvement in computation time. So, we have mainly focused on sampling $10\%$ frames which has more drop in computation time ($x\%$). On comparing performance of our model \Model with other baselines, we can see the significant reduction in \textit{validation} time with approximately $1\%$ drop in performance. Even in case of \Model model with selection of every $20^{th}$ frame, the proposed method has better performance than simple every $20^{th}$ baseline. 
% \begin{table}
% \centering
% \begin{tabular}{lc}
% \toprule
% \multirow{2}{*}{\textsc{Model}}    & \textsc{Evaluation }\\
%  & \textsc{Time (hrs.)} \\
% \toprule
% TEACHER-FULL       &  13.00 \\
% \midrule
% $10\%$        &  9.11\\
% $25\% $        &  11.00\\
% $50\% $        &   12.50 \\
% \bottomrule
% \end{tabular}
% \caption{Comparison of evaluation time of models using $k\%$ of frames and the TEACHER-FULL (original) model  on validation set using Tesla k80 machines}
% \label{table2}
% \end{table}
\vspace*{-4mm}
\begin{table}
\centering
\begin{tabular}{@{}lcccc@{}}
\toprule
\textsc{Model}    & \full & $10\%$ &  $25\% $  & $50\% $     \\
\midrule
\textsc{Time (hrs.)}&  13.00  &  9.11 &  11.00 &   12.50 \\
\bottomrule
\end{tabular}
\vspace*{-3mm}
\caption{Comparison of evaluation time of models using $k\%$ of frames and the \full (original) model  on validation set using Tesla k80 machines}
\label{table2}
\vspace*{-4mm}
\end{table}
%\subsection{Variant 2 : Performance of \Model using Intermediate Encoding of Teacher}
%In the second variant of \Model Model, we have computed squared error loss between intermediate states of student and every $k^{th}$ step of teacher along with the final encoding loss, to learn the parameters of student network. This variant shows more promising results with $x\%$ reduction in computation time (validation) with just $<=1\%$ performance drop. This infers that the proposed model surely fits our hypothesis of human analogy of processing the video frames, which is the main motivation behind this work.

%  \begin{table*}[t]
% \centering
% \label{table1}
% \begin{tabular}{llllll}
% \textsc{Model}           & \textsc{AvgHit}@1 & \textsc{PERR}  & \textsc{MAP}   & \textsc{GAP} & Validation Time \\
% \hline
% Baseline       & 0.862    & 0.736 & 0.402 & \textbf{0.809} & \\
% \hline
% Every10         & 0.848    & 0.716 & 0.362 & 0.788 & \\
% TeaStud-every10-i & 0.854  & 0.725  & 0.382	& \textbf{0.799} & \\
% \hline
% Every20         & 0.834    & 0.698 & 0.333 & 0.770  & \\
% TeaStud-every20-i & running  &  &  & & \\
% \hline
% \\
% \end{tabular}

% \caption{Performance comparison of different baseline models  with \Model Intermediate-encoding (proposed model) on Youtube8Million Dataset}
% \end{table*}
\section{Conclusion and Future Work}
We proposed a method to reduce the computation time for video classification using the idea of distillation. Specifically, we first train a teacher network which computes a representation of the video using all the frames in the video. We then train a student network which only processes $k$ \% of the frames of the video. We add a loss function which ensures that the final representation produced by the student is the same as that produced by the teacher. We also propose a simple variant of this idea where the student is trained to also match the intermediate representations produced by the teacher for every $j^{th}$ frame. We evaluate our model on the YouTube-8M dataset and show that the computationally less expensive student network can reduce the computation time by upto $30\%$ while giving similar performance as the teacher network. 
%From the results, we can conclude that our proposed model \Model which uses less number of frames, performs at par with the original model, the latter consuming all the frames of video. We observe that it clearly has slight edge in terms of overall performance as compared to the simple baselines for sampling frames from the video. Thus, the model can easily cater to the needs of low power and energy requirement target devices which can afford to work with only small fraction of frames. 

As future work, we would like to evaluate our model on other video processing tasks such as summarization, question answering and captioning. We would also like to experiment with different teacher networks other than the hierarchical RNN considered in this work.
%One possible future direction for this work would be to augment the attention based models with the proposed \Model approach. Also, it is interesting the analyze the behaviour of proposed approach in case of different configurations of the student and teacher network such as different \textit{cell size} of RNN, \textit{loss function} and separate network for matching of videos features/encodings. This approach can be applied to other video related tasks as well for e.g, Video Question Answering and Video summarization, in which the selection of frames is really critical for the model performance. 



%\subsection{References} \bibliography will add References. You don't need a separate subsection


{\small
\bibliographystyle{ieee}
\bibliography{egbib}
}

\end{document}
