%%%%%%%%%%%%%%%%%%%%%%%%%%%%%%%%%%%%%%%%%%%%%%%%%%%%%%%%%%%%%%%%%%%%%%%%%%%%%%%%
%2345678901234567890123456789012345678901234567890123456789012345678901234567890
%        1         2         3         4         5         6         7         8

\documentclass[letterpaper, 10 pt, conference]{ieeeconf}  % Comment this line out if you need a4paper
%\documentclass[a4paper, 10pt, conference]{ieeeconf}      % Use this line for a4 paper

\IEEEoverridecommandlockouts                              % This command is only needed if
                                                          % you want to use the \thanks command

\overrideIEEEmargins                                      % Needed to meet printer requirements.

%In case you encounter the following error:
%Error 1010 The PDF file may be corrupt (unable to open PDF file) OR
%Error 1000 An error occurred while parsing a contents stream. Unable to analyze the PDF file.
%This is a known problem with pdfLaTeX conversion filter. The file cannot be opened with acrobat reader
%Please use one of the alternatives below to circumvent this error by uncommenting one or the other
%\pdfobjcompresslevel=0
%\pdfminorversion=4

% See the \addtolength command later in the file to balance the column lengths
% on the last page of the document

% The following packages can be found on http:\\www.ctan.org
%\usepackage{graphics} % for pdf, bitmapped graphics files
%\usepackage{epsfig} % for postscript graphics files
%\usepackage{mathptmx} % assumes new font selection scheme installed
%\usepackage{times} % assumes new font selection scheme installed
%\usepackage{amsmath} % assumes amsmath package installed
%\usepackage{amssymb}  % assumes amsmath package installed
\usepackage{subcaption}
\usepackage{graphicx}
\usepackage{comment}

\usepackage[normalem]{ulem} % for sout
\usepackage{amsmath}

\usepackage{enumerate}
\usepackage{xcolor}
\usepackage{graphicx}
\usepackage{hyperref}
\usepackage{cite}
\usepackage{adjustbox}
\usepackage{soul}


% \hypersetup{
%     colorlinks=true,
%     linkcolor=blue,
%     filecolor=magenta,
%     urlcolor=cyan,
%     citecolor=blue
% }

% Colored annotation
\newcommand{\phil}[1]{\textcolor{blue}{#1}}
\newcommand{\jp}[1]{\textcolor{orange}{#1}}
\newcommand{\abdeslam}[1]{\textcolor{magenta}{#1}}
\newcommand{\chaitanya}[1]{\textcolor{red}{#1}}

\title{\LARGE \bf
Learning Object Localization and 6D Pose Estimation from Simulation and Weakly Labeled Real Images
}


 \author{Jean-Philippe Mercier$^{1}$, Chaitanya Mitash$^{2}$, Philippe Gigu\`ere$^{1}$ and Abdeslam Boularias$^{2}$%
 \thanks{$^{1}$ Laval University, Quebec, Canada.}
 \thanks{$\enspace $ jean-philippe.mercier.2@ulaval.ca, philippe.giguere@ift.ulaval.ca}
 \thanks{$^{2}$ Rutgers University, NJ, USA}%
 \thanks{$\enspace$ \{cm1074,ab1544\}@rutgers.edu.}
 }

\begin{document}

%\IEEEpeerreviewmaketitle
\maketitle
\thispagestyle{empty}
\pagestyle{empty}




%%%%%%%%%%%%%%%%%%%%%%%%%%%%%%%%%%%%%%%%%%%%%%%%%%%%%%%%%%%%%%%%%%%%%%%%%%%%%%%%
\begin{abstract}
Accurate pose estimation is often a requirement for robust robotic grasping and manipulation of objects placed in cluttered, tight environments, such as a shelf with multiple objects. When deep learning approaches are employed to perform this task, they typically require a large amount of training data. However, obtaining precise 6 degrees of freedom for ground-truth can be prohibitively expensive. This work therefore proposes an architecture and a training process to solve this issue. More precisely, we present a weak object detector that enables localizing objects and estimating their 6D poses in cluttered and occluded scenes. To minimize the human labor required for annotations, the proposed detector is trained with a combination of synthetic and a few weakly annotated real images (as little as 10 images per object), for which a human provides only a list of objects present in each image (no time-consuming annotations, such as bounding boxes, segmentation masks and object poses). To close the gap between real and synthetic images, we use multiple domain classifiers trained adversarially. During the inference phase, the resulting class-specific heatmaps of the weak detector are used to guide the search of 6D poses of objects. Our proposed approach is evaluated on several publicly available datasets for pose estimation. We also evaluated our model on classification and localization in unsupervised and semi-supervised settings. The results clearly indicate that this approach could provide an efficient way toward fully automating the training process of computer vision models used in robotics.



% Accurate pose estimation is often a requirement for robust robotic grasping and manipulation of objects placed in cluttered, tight environments, such as a shelf with multiple objects.
% \phil{When deep learning approaches are employed to perform this task, they typically require a large amounts of training data. However, obtaining precise 6 degrees of freedom for ground-truth can be prohibitively expensive.} This work therefore proposes \phil{an architecture and a training process} to solve this issue. More precisely, we present a point-wise object detector that enables localizing objects and estimates their 6D poses, tested in cluttered and occluded scenes. To minimize the human labor required for annotation, the proposed object detector is first trained in simulation by using automatically annotated synthetic images. We then show that the performance of the detector can be substantially improved by using a small set of {\it weakly annotated} real images, where a human provides only a list of objects present in each image without indicating the location of the objects \phil{if they don't need to provide the pose, we should emphasis it here}. To close the gap between real and synthetic images, we \phil{include in our architecture} a adopt a domain adaptation mechanism through {\it adversarial training}.
% \phil{I think this is too detailed for the abstract: The detector resulting from this training process can be used to localize objects by using its per-object activation maps. In this work, we use the activation maps to guide the search of 6D poses of objects.} Our proposed approach is evaluated on several publicly available datasets for pose estimation. We also evaluated our model on classification and localization in unsupervised and semi-supervised settings. The results clearly indicate that this approach could provide an efficient way toward fully automating the training process of computer vision models used in robotics. \phil{indicate how little data we need: 10? 100?}

\end{abstract}


%%%%%%%%%%%%%%%%%%%%%%%%%%%%%%%%%%%%%%%%%%%%%%%%%%%%%%%%%%%%%%%%%%%%%%%%%%%%%%%%

\section{Introduction}\label{sec:intro}

Robotic manipulators are increasingly deployed in challenging  situations that include significant occlusion and clutter. Prime examples are warehouse automation and logistics, where such manipulators are tasked with picking up specific items from dense piles of a large variety of objects, as illustrated in Fig.~\ref{fig_pose_estimation}. The difficult nature of this task was highlighted during the recent Amazon Robotics Challenges~\cite{Correll:2016aa}. These robotic manipulation systems are generally endowed with a perception pipeline that starts with object recognition, followed by the object's six degrees-of-freedom (6D) pose estimation. It is known to to be a computationally challenging problem, largely due to the combinatorial nature of the corresponding global search problem. A typical strategy for pose estimation methods~\cite{hinterstoisser2016going,krull2015learning,brachmann2014learning,michel2017global} consists in generating a large number of candidate 6D poses for each object in the scene and refining hypotheses with the Iterative Closest Point (ICP)~\cite{besl1992method} method or its variants. The computational efficiency of this search problem is directly affected by the number of pose hypotheses. Reducing the number of candidate poses is thus an essential step towards real-time grasping of objects.

%An accurate point-wise object detector is necessary for reducing the number of candidate poses, by focusing the search only on relevant parts of the image. It thus constitute an essential step towards real-time grasping of objects. %\jp{I'm really not sure about the rest of this paragraph. Do we have examples of papers that do this approach?? I think the message we want to pass is that feature-based methods requires ICP (long) and that end-to-end deep learning requires time-consuming annotations.}%(~\cite{xiang2017posecnn,hinterstoisser2016going,li2018deepim,oberweger2018making,krull2015learning,brachmann2014learning})\begin{figure}[t!]
      \begin{center}
%      \includegraphics[width=0.95\linewidth]{images/pose_estimation_pipeline_cropped}
      \includegraphics[width=1\linewidth]{images/pipeline}
      \end{center}
      \caption{Overview of our approach for 6D pose estimation at inference time. This figure shows the pipeline for the drill object of the YCB-video dataset~\cite{xiang2017posecnn}. A deep learning model is trained with {\it weakly annotated} images. Extracted class-specific heatmaps, along with 3D models and the depth image, guide the Stochastic Congruent Sets (StoCS) method~\cite{mitash2018robust} to estimate 6D object poses. Further details of the network are available in Section \ref{sec:approach}.}
      \label{fig_pose_estimation}
      \vspace{-0.45cm}
\end{figure}%As robot are increasingly deployed in uncontrolled environments such as large warehouses, the need for robustness to changes in illumination, occlusion%This component's robustness is becoming increasingly important, as robots are being deployed in environments less structured than traditional manufacturing setups. Warehouse automation and logistics are prime examples of application domains where a robotic manipulator is tasked with picking up specific items from a dense pile of a large variety of objects, placed in complex arrangements where only a small part of each object is potentially visible to the robot's camera, as illustrated in Figure~\ref{}. The challenging nature of this task was highlighted during the recent Amazon Robotics Challenges~\cite{Correll:2016aa}.

Training Convolutional Neural Networks (CNN) for tasks such as object detection and segmentation~\cite{Princeton,hernandez2016team,shelhamer2016fully} makes it possible to narrow down the regions that are used for searching for object poses in RGB-D images. However, CNNs typically require large amounts of annotated images to achieve a good performance. While such large datasets are publicly available for general-purpose computer vision, specialized datasets in certain areas such as robotics and medical image analysis tend to be significantly scarcer and time-consuming to obtain. In a warehouse context (our target context), new items are routinely added to inventories. It is thus impractical to collect and manually annotate a new dataset every time an inventory gets updated, particularly if it must cover all possible lighting and arrangement conditions that a robot may encounter during deployment. This is even more challenging if one wants this dataset to be collected by non-expert workers. The main goal of our approach is thus to reduce such a need for manual labeling, including completely eliminating bounding boxes, segmentation masks and 6D ground truth manual annotations.

Our first solution to reduce manual annotations is to leverage synthetic images generated with a CAD model rendered on diverse backgrounds. However, the visual features difference between real and synthetic images can be large to the point of leading to poor performance on real objects. The problem of learning from data sampled from non-identical distributions is known as {\it domain adaptation}. Domain adaptation has been increasingly seen as a solution to bridge the gap between domains~\cite{wang2018deep,csurka2017domain}. Roughly speaking, domain adaptation tries to generalize the learning from a \textit{source domain} to a \textit{target domain}, or in our case, from synthetic to real images. Since labeled data in the target domain is unavailable or limited, the standard way %of doing domain adaptation
is to train on labeled source data, while trying to minimize the distribution discrepancy between source and target domains. %%%%%%%%%%%%%%% I REMOVED THIS FROM THE ICRA SUBMISSION%%%%%%%%%%%%%%%%%%%%%%%%%% \jp{ \textbf{TO REMOVE} One such successful approach is called Domain-Adversarial training of Neural Networks (DANN)~\cite{ganin2016domain}. %However, it has a detrimental tendency to align the whole feature distribution together, instead of in a class-specific way. %This in turns decreases the discriminative power of the approach. To reduce feature misalignments between classes, %The Multi-Adversarial Domain Adaptation (MADA)~\cite{pei2018multi} alleviates this issue by using one domain discriminator for each class, and weighting their input features by their associated class probability. % outputted by a classification module.%Another approach, called \textit{domain randomization}, circumvents the whole issue by generating so many variant of domains that the true one can be perceived as yet another instance of domain. It has been used in the context of object detection by \cite{tremblay2018training}, in which they show that deep networks can perform as well or even better on unrealistic synthetic data than on realistic. They also show the power of fine-tuning with labeled real images, even on a small set, which we also explore in our work.}%%%%%%%%%%%%%%% I REMOVED THIS FROM THE ICRA SUBMISSION%%%%%%%%%%%%%%%%%%%%%%%%%

While having a small labeled dataset on a target domain allows to boost performances, it may still require significant human effort for the annotations. Our second solution is to use {\it weakly supervised learning}, which significantly decreases annotation efforts, albeit with a reduced performance compared to fully-annotated images. Some methods~\cite{oquab2015object,durand2017wildcat} have been shown to be able to retrieve a high level representation of the input data (such as object localization) while only being trained for object classification. To the best of our knowledge, this promising kind of approach has not yet been applied within a robotic manipulation context.

% We do so by training a weakly supervised network (\cite{oquab2015object,zhou2017brief}) to localize objects simply from class labels and by leveraging %  This is particularly true for tasks such as object detection, segmentation and pose estimation for which bounding boxes, pixelwise annotations and 6D poses are required. Weakly supervised learning methods significantly decrease this annotation effort, albeit with reduced performance compared to fully-annotated. Some~\cite{oquab2015object,durand2017wildcat} have been shown to be able to retrieve a high level representation of the input data (such as object localization) while only being trained for object classification, without any localization information. To the best of our knowledge, this promising kind of approach has not yet been applied within a robotic manipulation context.%to reduce the search by focusing relevant parts of the image is a plausible solution to reduce%\phil{I feel that these applies to pretty much any vision problem: Applications such as warehouse automation and logistics are affected by changes in lighting and shadowing conditions and involve varying object arrangements. It is difficult to provide an annotated dataset that covers all possible conditions that a robot may encounter during the testing phase.} Moreover, new items are routinely added to  inventories and others are removed. It is thus impractical to collect and manually annotate a new dataset every time an inventory gets updated. \phil{Discussion of trying to reduce the need of experts to label such datasets?}%o generate a large number of synthetic images in simulation, as they can be. %However, the visual features of real and synthetic images are significantly different and training only on synthetic data without compensating in any way for the gap between real and synthetic data results in poor performances~\cite{}. The problem of learning from data sampled from non-identical distributions is known as {\it domain adaptation}. %be sufficiently large that are significantly different and training only on synthetic data without compensating in any way for the gap between real and synthetic data results in poor performances~\cite{}. %A simple solution that is becoming increasingly popular in robot learning to alleviate this annotated dataset problem is to %\phil{Should we also discuss other methods that exist \emph{to reduce the need for training data}, our main point. How about data augmentation? semi-supervised learning? autoencoders?}%For most domain adaptation methods, image classification is the application of choice~\cite{}. There has been rather little focus on certain tasks, such as object detection and localization, that are crucial for robotic grasping and manipulation. One of the few work that does trained deep networks for  % \phil{JP: there was a grasping paper that was using DANN. Can you cite it?} \jp{\cite{bousmalis2017using}, not sure if we want to plug it directly or need to add a few sentences describing different papers doing different tasks with domain adaptation}% These techniques frequently consist in training with labeled data from the source domain and unlabeled or partially labeled data from the target domain. % These methods can also be used with partially labeled target data, but might not work as well as using a transfer technique such as fine-tuning.%Indeed, it has been shown in~\cite{oliver2018realistic} that simple fine-tuning performs better than any state-of-the-art semi-supervised learning method on several datasets. This conclusion may also generalize to semi-supervised domain adaptation when only a few labels are available (\phil{on the target domain?}). %and we believe it is a promising approach to increase the annotation process speed while also reducing the human effort.%\sout{Performance of weakly supervised approaches on tasks such as object detection and segmentation can however be weaker than networks specifically trained for these tasks with full annotations~\cite{}. They are not optimized to perform well for specific metrics, but they still have a good capacity to perform these tasks.} \phil{I'd be tempted to push this in Section 3: Pose estimation methods such as Stochastic Congruent Sets (StoCS)~\cite{mitash2018robust} can perform well with noisy segmentation maps and can therefore be a good fit with weakly supervised methods to perform pose estimation.}%the data coming from the source domain is annotated and partially or not annotated in the target domain.  %Transfer learning techniques have helped to alleviate this problem to some extent by using the knowledge gained from training a network on a different task or dataset where annotated data is abundant and transferring that knowledge to the task of interest. For instance, many computer vision methods use a network pre-trained on the {\it ImageNet} dataset [refs here?] and fine-tune the network on the desired dataset. This approach works well when the training dataset is of sufficient size and variety [refs here?]. However, it is still likely that the network ends up overfitting the training data. %A reason for that could be that it is easier to get annotated data for classification and therefore easier to evaluate. \\%\phil{Ici peut-tre faire des liens avec un des effets secondaires de global average pooling. voir mes notes de cours glo4030, actate 4 http://www2.ift.ulaval.ca/~pgiguere/cours/DeepLearning/09-Attention.pdf .}%While training networks for classification, the last layers can be interpreted as a high level representations of the input data. Methods such as \cite{oquab2015object} use that information to retrieve the position of objects without requiring bounding box annotations. This kind of approach is called "Weakly Supervised". \phil{faudrait une dfinition un peu plus formelle de weakly supervised} This is an interesting way of training an object detector, as it does not need the time-consuming effort of annotating images. \\ In this paper, we propose a two-step approach for 6D pose estimation, as shown in Fig.~\ref{fig_pose_estimation}. First, we train a network for classification through domain adaptation, by using a combination of weakly labeled synthetic and real color images. During the inference phase, the weakly supervised network generates class-specific heatmaps that are subsequently refined with an independent 6D pose estimation method called Stochastic Congruent Sets (StoCS)~\cite{mitash2018robust}. Our complete method achieves competitive results on the YCB-video object dataset~\cite{xiang2017posecnn} and Occluded Linemod~\cite{krull2015learning} while using only synthetic images and few weakly labeled real images (as little as 10) per object in training. %\phil{I feel that something like Contrary to other approaches, we need significantly less data or something like that.}.
We also empirically demonstrate that for our test case, using domain adaptation in semi-supervised settings is preferable than training in unsupervised settings and fine-tuning on available weakly labeled real images, a commonly-accepted strategy when only a few images from the target domain are available.

% So, in this paper, we propose to leverage weakly-annotated images % and \phil{Do we describe semi-supervised in the intro?: semi-supervised learning}% in the context of 6D object pose estimation. More precisely, our approach consists first in training a classifier through domain adaptation, using weakly labeled synthetic and real color images. This allows to retrieve class-specific heatmaps that are used to localize objects. These localization heatmaps are then subsequently refined with an independent 6D pose estimation method called StoCS. Our complete method achieves competitive results on the YCB-video object dataset~\cite{xiang2017posecnn} and Occluded Linemod~\cite{krull2015learning} while using only synthetic images and a \phil{singificantly fewer? few} weakly labeled real images \phil{(as little as x)} per object in training. %\phil{I feel that something like Contrary to other approaches, we need significantly less data or something like that.}.% We also empirically demonstrate that for our test case, using domain adaptation in semi-supervised settings is preferable than training in unsupervised settings and fine-tuning on available weakly labeled real images\phil{, a commonly-accepted strategy}. %first in supervised training of a network with fully-annotated synthetic images, and then fine-tuning it with a few weakly annotated real images \phil{through domain adaptation and a specialized cosine training regimen}. %The pose estimation %We also show that training the weakly supervised network in multiple steps is advantageous. First, we train the network combined with a domain adaptation method on synthetic images and a bigger unlabeled real dataset. Then, we fine-tune the network with the smaller and weakly labeled real dataset. The normalized output of the network is then used as probability maps in StoCS for pose estimation. Our method achieve near state-of-the-art results on YCB-video object dataset \jp{add more datasets and review results} while using only synthetic images and a few weakly labeled real images per object.%We use this knowledge to propose a way to train a weakly supervised network with only a few weakly annotated real images. This is based on the fact that getting a lot of images for all classes with sufficient variations to prevent overfitting may still be time-consuming. Therefore, we limit the acquisition and annotation efforts. To circumvent the limited data, we generate synthetic data and use a domain adaptation method to improve the performance in the .
\section{Related Works}\label{sec:rel_work}

In this paper, we aim at performing object localization and 6D pose estimation with a deep network, with minimal human labeling efforts. Our approach is based on training from synthetic and weakly labeled real images, via domain adaptation. These various concepts are discussed below.

%jp There is no pre-training in the final solution (training all at once)%TODO% Add normalization in figure\begin{figure*}[thpb]
      \begin{center}
      %\includegraphics[width=0.85\linewidth]{images/architecture_corl_cropped}
      \includegraphics[width=0.75\linewidth]{images/architecture_icra}
      \end{center}
      \caption{Overview of the proposed approach for object localization and 6D pose estimation with domain adaptation, using a mix of synthetic images and weakly labeled real images.}
 \label{fig_wildcat_mada}
 \vspace{-0.55cm}
\end{figure*}% \phil{Il manque de colle logique entre les diffrents lments. Il faudrait expliquer le lien entre les deux approches mentionnes ici-bas et le domain adaptation ou transfert learning.}%\phil{I changed the order of the paragraph from the original CoRL submission.}\textbf{6D Pose Estimation} Recent literature in pose estimation focuses on learning to predict 6D poses using deep learning techniques. For example, \cite{xiang2017posecnn} predicts separately the object center in images for translation and regresses over the quaternion representation for predicting the rotation. Another approach is to first predict 3D object coordinates, followed by a RANSAC-based scheme to predict the object's pose\cite{brachmann2014learning,michel2017global}. Similarly, \cite{michel2017global} uses geometric consistency to refine the predictions from the learned model. These methods, however, need access to several images that are manually labeled with the full object poses, which is time-consuming to acquire. Some other approaches make use of the object segmentation output to guide a global search process for estimating object poses in the scene \cite{mitash2018robust, narayanan2016discriminatively, mitash2017improving}. Although the search process could compensate for errors in prediction when the segmentation module is trained with synthetic data, the domain gap could be large, and a computationally expensive search process may be needed to bridge this gap.

\textbf{Learning with Synthetic Data}
Training with synthetic data has recently gained significant traction, as shown by the multiple synthetic datasets recently available~\cite{gaidon2016virtual,mayer2016large,qiu2016unrealcv,ros2016synthia,johnson2017driving, richter2017playing}, with some focusing on optimizing the realism of the generated images. While the latter can decrease to a certain degree the gap between real and synthetic images, it somehow defeats the purpose of using simulation as a cost-effective way to create training data. To circumvent this issue, \cite{dwibedi2017cut,georgakis2017synthesizing} proposed instead to create images using segmented object instances copied on real images. This type of approach, akin to data augmentation, is however limited to the number of object views and illuminations that are available in the original dataset.
Recently,~\cite{hinterstoisser2017pre,tremblay2018training} showed promising results by training object detectors with 3D models rendered in simulation with randomized parameters, such as lighting, number of objects, object poses, and backgrounds. While in~\cite{hinterstoisser2017pre} they only uses synthetic images in training, \cite{tremblay2018training} demonstrated the benefits of fine-tuning on a limited set of real labeled images. The last one also showed that using photorealistic synthetic images does not necessarily improve object detection, compared to training on a less realistic synthetic dataset generated with randomized parameters.

%While some datasets try to mimic a corresponding real dataset, an approach called domain randomization \cite{sadeghi2016rl, tobin2017domain} has shown that image realism is not always necessary to generalize well to real images. \phil{I am not sure you should say the following, unless you make explicit link to your work:}In fact, generating images under varying conditions might prove to be better than realistic images.%A recent approach is called domain randomization \cite{sadeghi2016rl, tobin2017domain}. For this method, many simulation parameters are randomized to generate synthetic data. The core of the idea is to generate large variations so that real data is seen as a variation of the simulation world. In \cite{tremblay2018training}, they used that idea to train an object detector on their randomized synthetic data. In their experiments, models train on their data performs better on the real dataset than when trained on \phil{check spelling:} Virual Kitty, which tries to mimic the real dataset. It therefore shows that realism is not necessarily important to always bridge the reality gap. They also investigated the idea of \cite{hinterstoisser2017pre}, in which they claim that freezing the weights of the early layers of a pre-trained network is helpful. However, in their experiments, they found it to be more harmful than helpful. They also reported that a huge improvement by using pretrained networks instead of training from scratch, even with huge datasets of 1 million images. Their performance saturates with synthetic dataset size of 10k images when pretrained. \textbf{Domain Adaptation}
Domain adaptation techniques~\cite{wang2018deep,csurka2017domain} can serve to decrease the distribution discrepancy between different domains, such as real vs. synthetic. The popular DANN~\cite{ganin2016domain} approach relies on two classifiers: one for the desired task, trained on labeled data from a source domain, and another one (called \emph{domain classifier}) that classifies whether the input data is from the source or target domain. Both classifiers share the first part of the network, which acts as a feature extractor. The network is trained in an adversarial manner: domain classifier parameters are optimized to minimize the domain classification loss, and shared parameters are optimized to maximize the domain classification loss. It is possible to achieve this minimax optimization in a single step by using a gradient reversal layer that reverses the sign of the gradient between shared and non-shared parameters of the domain classifier. To the best of our knowledge, the present work is the first use a DANN-like approach for point-wise object localization, a fundamental problem in robotic manipulation.

% DANNs \cite{ganin2016domain} are one of the methods commonly used to make features independent between an input domain and a target domain. The network can be divided into 3 parts: feature extractor, task network and domain classifier. The goal of the architecture is to be able to have good classification on the source domain, while being unable to differentiate between domains. The latter is achieved by using a gradient reversal layer between the feature extractor and the domain classifier. In semi-supervised scenarios, the task network is trained on the annotated source data (annotated target data not available).% \jp{Domain Randomization: C'est pratiquement ce qu'on fait. On randomise les backgrounds, les objets, leur position/rotation. Durant le training, on fait de l'augmentation qui randomise le scale, couleur/contraste/eclairage. Je pense que c'est pour cette raison que la detection fonctionne aussi bien sur les donnes synthetiques seulement}% In \cite{rad2017feature}, they observed that for paired images, only a few feature coefficients were \phil{je comprends pas: creating the domain gap}. To solve the problem, they trained a network to minimize the $L_2$ loss between synthetic image features and transformed real ones. In their case, it reduced the large differences that caused the domain gap, thus allowing them to train their network for pose estimation. They have also shown that their method outperforms DANN for pose estimation on LineMod dataset.\textbf{Weakly Supervised Learning}
We are interested in weakly supervised learning with inexact supervision, for which only coarse-grained labels are available~\cite{zhou2017brief}. In~\cite{oquab2015object}, a network was trained only with weak image-level labels (classes that are present in images, but not their position) and max-pooling was used to retrieve approximate location of objects. The proposed {\it WILDCAT} model~\cite{durand2017wildcat} performs classification and weakly supervised point-wise detection and segmentation. This architecture learns multiple localized features for each class, and uses a spatial pooling strategy that generalizes to many ones (max pooling, global average pooling and negative evidence). In the present work, we push the paradigm of minimum human supervision \emph{even further}. To this effect, we propose to train WILDCAT with synthetic images, in addition to weakly supervised real ones, and use MADA (a variant of DANN) for domain adaptation.

% \paragraph{Weakly Supervised Networks}% First, a \textbf{weakly supervised network} is trained with only the class of the objects that are in an image. In our case, we train a weakly supervised network for classification and detection. By doing that, we are able to train a network for object detection without needing bounding box annotations. Here's a recap of the inputs and outputs for training. \\ \\% Input: Image + class \\% Output: Class + Maximum Activation Per Class (see Fig \ref{fig_wildcat}, there is 1 "heatmap" per class") \\% \paragraph{Semi-Supervised Networks}% \par For semi-supervised networks, it means that for training, there are no or only a few annotations available for the target dataset. %-------------------------------------------------------------------------%In the case of the detection experiments, the target dataset is real images and as a source dataset, we've generated 4 different synthetic datasets at the moment:% \begin{itemize}% \item 3D models rendered on basic black backgrounds (bottom row in Fig. \ref{fig_training_examples})% 	\begin{enumerate}% 	\item Models centered and same distance from the camera%     \item Models not centered and at different distance from the camera (different scale)% 	\end{enumerate}% \item 3D models rendered on backgrounds from LSUN dataset (top row in Fig. \ref{fig_training_examples})% 	\begin{enumerate}% 	\item Models centered and same distance from the camera%     \item Models not centered and at different distance from the camera (different scale)% 	\end{enumerate}% \end{itemize}%We can see examples of instances of these datasets in Fig. \ref{fig_training_examples}. \\% \begin{figure}[thpb]%       \centering%       \includegraphics[width=0.7\linewidth]{images/training_examples}%       \caption{Examples of different source datasets used in experiments for semi-supervised detection}%       \label{fig_training_examples}%    \end{figure}
\section{Proposed Approach}\label{sec:approach}
We present here our approach to object localization and 6D pose estimation. It is trained using a mix of synthetic and real images and only requires weak annotations (only class-presence) in both domains.

\vspace{-0.55cm}\subsection{Overview}% \begin{figure}[thpb]%       \begin{center}%       \fbox{\includegraphics[width=0.98\linewidth]{images/pose_estimation_pipeline_cropped}}%       \end{center}%       \caption{Pose estimation pipeline}}%  \label{fig_pose_estimation}% \end{figure}

Figure~\ref{fig_wildcat_mada} depicts an overview of our proposed system. It comprises {\it i}) a {\it ResNet-50} model pre-trained on {\it ImageNet} as a feature extractor (green), {\it ii}) a weak classifier inspired from the WILDCAT model~\cite{durand2017wildcat} (blue), {\it iii}) the Stochastic Congruent Sets ({\it StoCS}) for 6D pose estimation (red)~\cite{mitash2018robust}, and {\it iv}) the MADA domain adaptation network to bridge the gap between synthetic and real data. %\sout{derived from Multi-Adversarial Domain Adaptation network ({\it MADA})~\cite{pei2018multi} with a gradient reversal layer to learn domain-independent features. } \sout{The different parts of the system and the training process are explained in the following sections.}%The ResNet extracts features from the images that are useful for object classification (\phil{, localization and pose estimation}. \sout{which could also be used for point-wise localization by WILDCAT.}
During the inference phase, the domain adaptation part of the network is discarded. Given a test image, class-specific heatmaps are generated by the network. These heatmaps indicate the most probable locations of each object in the image. This probability distribution is then fed to StoCS, a robust pose estimation algorithm that is specifically designed to deal with noisy localization.
%\phil{Here i feel you re-explain what DANN is, which you have done before. Focus should be on the difference with MADA.}
To force the feature extractor to extract similar features for both synthetic and real images, a MADA module (described below) is employed.
%is trained to classify any input image according to its domain $\{\mathrm{real, synthetic} \}$ and the weights of ResNet are updated such that this classification's error increases.
MADA's purpose is to generate gradients during training (via a reversal layer) in order to improve the generalization capabilities of the feature extractor.

%\phil{CONFIRM THE FOLLOWING!!! Once training is achieved, the MADA module is discarded.} ---> I've moved it higher in the paragraph (JP)% Thus, ResNet is trained to extract features that make it hard for MADA to distinguish between synthetic and real images. \phil{This is not part of architecture, but training regimen: MADA's role ends once the training of ResNet and WILDCAT is over.} %\phil{Are all parts active at all times? Are some only used during training? during testing?}% TODO% Add architecture and training regimen in another paragraph\subsection{Synthetic Data Generation}
For synthetic data generation, we used a modified version of the SIXD toolkit\footnote{\url{https://github.com/thodan/sixd_toolkit}}. This toolkit generates color and depth images of 3D object models rendered on black backgrounds. Virtual camera viewpoints are sampled on spheres of different radii, following the approach described in~\cite{hinterstoisser2008simultaneous}. We extended the toolkit with the functionality of rendering more than one object per image, and also used random backgrounds taken from the LSUN dataset~\cite{YuZSSX15}. % instead of generic black backgrounds.
Similarly to recent {\it domain randomization} techniques~\cite{DBLP:conf/iros/TobinFRSZA17}, we observed from our experiments that these simple modifications help transferring from simulation to real environments where there are multiple objects of interest, occlusions and diverse backgrounds. Figure~\ref{fig_wildcat_mada} displays some examples of the generated synthetic images that we used to train our network.

% \begin{figure}%  \begin{center}%   \begin{subfigure}[b]{0.25\textwidth}%     \includegraphics[width=\textwidth]{images/syn_ycb_1}%   \end{subfigure}%   % %   \begin{subfigure}[b]{0.25\textwidth}%     \includegraphics[width=\textwidth]{images/syn_ycb_2}%   \end{subfigure}%   %%   \begin{subfigure}[b]{0.25\textwidth}%     \includegraphics[width=\textwidth]{images/syn_ycb_3}%   \end{subfigure}% \end{center}%   \caption{Examples of synthetic images used to train the proposed system for the {\it YCB} objects~\cite{calli2017yale}.}%   \label{fig_synthetic_images}% \end{figure}\subsection{Weakly Supervised Learning with WILDCAT}
The images used for training our system are weakly labeled: only a list of object classes present in the image is provided. In order to recover localization from such weak labels, we leverage the WILDCAT architecture~\cite{durand2017wildcat}. Indeed, WILDCAT is able to recover localization information through its high-level feature map, even though it is only trained with a classification loss. %\sout{Despite being only trained for classification, WILDCAT implicitly recovers localization information for each object through  responses in specific feature maps. }%, which is illustrated in blue in Figure \ref{fig_wildcat_mada}.
As a feature extractor, we employ a ResNet-50 (pretrained on ImageNet) for which the last layers (global average pooling and fully connected layers) are removed, as depicted in Figure~\ref{fig_wildcat_mada}. The WILDCAT architecture added on top of this ResNet-50 comprises three main modules: a \textit{multimap transfer layer}, a \textit{class pooling layer} and a \textit{spatial pooling layer}. The \textit{multimap transfer layer} consists of $1 \times 1$ convolutions that extracts $M$ class-specific modalities per class $C$, with $M=8$ as per the original paper~\cite{durand2017wildcat}. The \textit{class pooling} module is an average pooling layer that reduces the number of feature maps from $MC$ to $C$. Then, the \textit{spatial pooling} module selects $k$ regions with maximum/minimum activations to calculate scores for each class. The classification loss for this module is a multi-label one-versus-all loss based on max-entropy (\textit{MultiLabelSoftMarginLoss} in PyTorch). The classification scores are then rescaled between 0 and 1 to cooperate with MADA.

%\phil{These scores will be used for domain adaptation, but with a normalization procedure that we have developed, described in the next section. } % be normalized in a specific way before being passed to a domain adaptation module based on MADA, explained in the next section. %\phil{CONFIRM! I see below that the normalization is for StoCS, but the text makes it sound it is for MADA. I do not know which one is correct.}% \paragraph{Multimap Transfer Layer}% This layer consists of $1 \times 1$ convolutions that takes features from the feature extractor as input and outputs $M$ feature maps per class. For instance, for an input image of dimensions $W \times H \times 3$, the output dimensions of this layer is $W/32 \times H/32 \times MC $, where $M$ is the number of modalities per class and $C$ the number of classes. The objective of having multiple modalities $M$ per class is to learn class-specific features. For our approach, we used $M=8$, as it is value that reports the best performance in~\cite{durand2017wildcat}. % \paragraph{Class Pooling}% This layer is an average pooling layer that reduces the number of feature maps from $MC$ to $C$. % \paragraph{Spatial Pooling}% This layer use the maximum and minimum activations for each feature maps corresponding to a class to output classification scores. The score for each class $s^c$ is defined as: % \begin{equation}% s^c = \max_{\mathbf{h} \in \mathcal{H}_{k^+}}  \frac{1}{k^+} \sum_{i,j} h_{i,j} \bar{z}_{i,j}^c + \alpha \bigg( \min_{\mathbf{h} \in \mathcal{H}_{k^-}} \sum_{i,j} h_{i,j} \bar{z}_{i,j}^c \bigg)% \label{eq_wildcat_classification}% \end{equation}% where $\mathcal{H}_{k}$ correspond to the $k$ selected regions with maximum/minimum activations, $h_{i,j} \in \{0,1\}$ and $\bar{z}_{i,j}^c$ is the input coming from the class pooling layer. Indices $(i,j) \in (1..w, 1..h)$ where $w$ and $h$ are the width and height of the feature maps. We normalize the scores by standard deviation when the output is passed to the domain adaptation module.\subsection{Multi-Adversarial Domain Adaptation with MADA}
We used the {\it Multi-Adversarial Domain Adaptation} (MADA) approach~\cite{pei2018multi} to bridge the ``reality gap''. MADA extends the {\it Domain Adversarial Networks} (DANN) approach~\cite{ganin2016domain} by using one domain discriminator per class, instead of a single global discriminator as in the original version of DANN~\cite{ganin2016domain}. Having one discriminator per class has been found to help aligning class-specific features between domains. In MADA, the loss $L_d$ for the $K$ domain discriminators and input $\mathbf{x_i}$ is defined as:
\begin{equation}
L_d = \frac{1}{n} \sum_{k=1}^{K} \sum_{\mathbf{x_i} \in D_s \cup D_t} L_d^k\bigg(G_d^k \Big(\hat{y}_i^k G_f(\mathbf{x_i})\Big), d_i\bigg),
\label{eq_mada_loss}
\end{equation}
wherein $i \in \{1, \dots, n\}$, and $n = n_s + n_t$ is the total number of training images in source domain $D_s$ (synthetic images) and the target domain $D_t$ (real images). $G_f$ is the feature extractor (the same for both domains), $\hat{y}_i^k$ is the probability of label $k$ for image $\mathbf{x_i}$. This probability $\hat{y}_i^k$ is the output of the weak classifier WILDCAT. $G_d^k$ is the $k$-th domain discriminator and $L_d^k$ is its cross-entropy loss, given the ground truth domain $d_i\in \{\mathrm{synthetic, real}\}$ of image $\mathbf{x_i}$. Our global objective function is:
% \begin{equation}% C = \frac{1}{n_s} \sum_{\mathbf{x_i} \in D_s } L_y \bigg( G_y \Big( G_f(\mathbf{x_i}) \Big), y_i \bigg) - \frac{\lambda}{n} \sum_{k=1}^{K} \sum_{\mathbf{x_i} \in D} L_d^k\bigg(G_d^k \Big(\hat{y}_i^k G_f(\mathbf{x_i})\Big), d_i\bigg).% \label{eq_mada_loss_global}% \end{equation}%\begin{multline}\begin{equation}
C = \frac{1}{n} \sum_{\mathbf{x_i} \in D } L_y \bigg( G_y \Big( G_f(\mathbf{x_i}) \Big), y_i \bigg) - \lambda L_d  \quad ,
%  - \frac{\lambda}{n} \sum_{k=1}^{K} \sum_{\mathbf{x_i} \in D} L_d^k\bigg(G_d^k \Big(\hat{y}_i^k G_f(\mathbf{x_i})\Big), d_i\bigg)
\label{eq_mada_loss_global}
\end{equation}%\end{multline}
where $L_y$ is the classification loss, $L_d$ the domain loss and $\lambda$ has been found to work well with a value of 0.5. The heat-map probability distribution extracted from WILDCAT is used to guide the StoCS algorithm in its search for 6D poses, as explained in the next section.  %\phil{I have simplified above equation. Please confirm! How do you find the value of $\lambda$?} (\jp{pretty much trial-and-error, so many hyperparameters...})\subsection{Pose Estimation with Stochastic Congruent Sets (StoCS)}
The StoCS method~\cite{mitash2018robust} is a robust pose estimator that predicts the 6D pose of an object in a depth image from its 3D model and a probability heatmap.
%\phil{In order to employ the score heatmap generated from WILDCAT, we developed a specific normalization approach (as earlier attempts at using unnormalized heatmaps were unsuccessful.)}
We employ a min-max normalization on the class-specific heatmaps of the Wildcat network, transforming them into a probability heatmaps $w_{p_i}$, using the per-class minimum ($w_{min}$) and maximum ($w_{max}$) values:
%it is generated by normalizing over an intermediate output of the WILDCAT network with activation $w_{p_i}$, \begin{equation}
\label{eq_normalization}
\pi_{p_i \rightarrow O_k} = \frac{w_{p_i} - w_{min}}{w_{max} - w_{min}}.
\end{equation}
This generates a heatmap providing the probability $\pi$ of an object $O_k$ being located at a given pixel $p_i$. The StoCS algorithm then follows the paradigm of a randomized alignment technique. It does so by iteratively sampling a set of four points, called a base $B$, on the point cloud $S$ and finds corresponding set of points on the object model $M$. Each corresponding set of four points defines a rigid transformation $T$, for which an alignment score is computed between the transformed model cloud and the heatmap for that object. The optimization criteria is defined as\\
\begin{eqnarray}
T_{opt} = arg\,max_{T}\sum_{m_i \in M_k}f( m_i, T,
S_k),\\
f(m_i, T, S_k ) =
\pi_k(s*), if \mid T(m_i) - s* \mid < \delta_s.
\end{eqnarray}
The base sampling process in this algorithm considers the joint probability of all four points belonging to the object in question, given as\\
\begin{equation}
Pr(B \rightarrow O_k) = \frac{1}{Z}
\prod_{i=1}^{4} \{ \phi_{node}(b_i) \prod_{j=1}^{j<i} \phi_{edge}(b_i,
b_j) \}.
\end{equation}
where $\phi_{node}$ is obtained from the probability heatmap and $\phi_{edge}$ is computed based on the point-pair features of the pre-processed object model. Thus, the method combines the normalized output of the Wildcat network with the geometric model of objects to obtain base samples which belong to the object with high probability.

In the next two Sections, we demonstrate the usefulness of our approach. First in Section~\ref{sec:w_exp}, we quantify the importance of each component (Wildcat, MADA) in order to train a network that generates \emph{relevant} feature maps from weakly labeled images. In Section~\ref{sec:6d_exp}, we then evaluate the performance of using these heatmaps with StoCS for rapid 6D pose estimation, which is the final goal of our paper.

% \begin{figure}[thpb]%       \begin{center}%       \fbox{\includegraphics[width=0.4\linewidth]{images/Stocs_Overview}}%       \end{center}%       \caption{Input (left) and output (right) of Stochastic Congruent Sets (StoCS) for 6D pose estimation}%       \label{fig_stocs_overview}% \end{figure}%WSL transfer layer (see Fig. \ref{fig_wildcat}). % \phil{Justifier pourquoi avoir pris un ResNet-50 au lieu d'autres features extractors (ResNet-xxx, DenseNet, etc).}
\section{Weakly Supervised Learning Experiments for object detection and classification}\label{sec:w_exp}% \begin{figure}%  \begin{center}%   \begin{subfigure}[b]{0.2\textwidth}%     \includegraphics[width=\textwidth]{images/img01_rgb}%     \caption{Weak/Class labels}%     \label{fig_supervision_rgb}%   \end{subfigure}%   % %   \begin{subfigure}[b]{0.2\textwidth}%     \includegraphics[width=\textwidth]{images/img01_bbox}%     \caption{Bounding boxes}%     \label{fig_supervision_bbox}%   \end{subfigure}%   %%   \begin{subfigure}[b]{0.2\textwidth}%     \includegraphics[width=\textwidth]{images/img01_seg}%     \caption{Pixelwise}%     \label{fig_supervision_seg}%   \end{subfigure}%   %%   \begin{subfigure}[b]{0.2\textwidth}%     \includegraphics[width=\textwidth]{images/img01_pose}%     \caption{6D poses}%     \label{fig_supervision_pose}%   \end{subfigure}%  \end{center}% \caption{Different supervision levels for training pose estimation methods. In (a), class of objects in the image are known. }% \label{fig_supervision_levels}% \end{figure}In this first experimental section, we perform an ablation study to evaluate the impact of various components for classification and point-wise localization. We first tested our approach %with unsupervised learning (no human annotations at all),
without any human labeling, as a baseline. We then evaluated the gain obtained by employing various numbers of weakly labeled images for four semi-supervised strategies. % \sout{when an increasing number of weakly labeled real images are available. These experiments serve to demonstrate that minimalistic human effort can results in significant gains, as well as highlight the winning strategies.}

We performed these evaluations on the YCB-video dataset~\cite{xiang2017posecnn}. This dataset contains 21 objects with available 3D models. It also has full annotations for detection and pose estimation on 113,198 training images and 20,531 test images. A subset of 2,949 test images (keyframes) is also available. Our results are reported for this more challenging subset, since most images in the bigger test set are video frames that are too similar and would report optimistic results.

For these experiments, we trained our network for 20 epochs (500 iterations per epoch) with a batch size of 4 images per domain. We used stochastic gradient descent with a learning rate of $0.001$ (decay of $0.1$ at epochs 10 and 16) and a Nesterov momentum of $0.9$. The ResNet-50 was pre-trained on ImageNet and the weights of the first two blocks were frozen.

\subsection{Unsupervised Domain Adaptation}\label{UnsupervisedDA}
For this experiment, we trained our model with weakly labeled synthetic images ($WS$) and unlabeled real images ($UR$). %\sout{from the training set of YCB-video dataset~\cite{xiang2017posecnn}}.
We tested three architecture configurations of domain adaptation: 1) without any domain adaptation module (WILDCAT model trained on $WS$), 2) with DANN ($WS$+$UR$) and 3) with MADA ($WS$+$UR$). We evaluated each of these configurations for both classification and detection. For classification, we used the accuracy metric to evaluate our model's capacity to discriminate which objects are in the image. We used a threshold of 0.5 on classification scores to predict the presence or absence of an object. For detection, we employed the point-wise localization metric~\cite{oquab2015object}, which is a standard metric to evaluate the ability of weakly supervised networks to localize objects. For each object in the image, the maximum value in their class-specific heatmap was used to retrieve the corresponding pixel in the original image. If this pixel is located inside the bounding box of the object of interest, it is counted as a good detection. Since the class-specific heatmap is a reduced scale of the input image due to pooling, a tolerance equal to the scale factor was added to the bounding box. In our case, a location in the class-specific heatmaps corresponds to a region of 32 pixels in the original image.
In Figure~\ref{fig_unsupervised_da_performance}, we report the average scores of the last 5 epochs over 3 independent random runs for each network variation. %\jp{Should we modify this??}
These results \emph{a)} confirm the importance of employing a domain adaptation strategy to bridge the reality gap, and \emph{b)} the necessity of having one domain discriminator $G_d^k$ for each of the X objects in the YCB database (MADA), instead of a single one (DANN). Next, we evaluate the gains obtained by employing weakly-annotated real images.

%TODO% Divide the graphs into 2 separate graphs with their own descriptions% \begin{figure}[thpb]%       \begin{center}%       \fbox{\includegraphics[width=0.6\linewidth]{images/Unsupervised_DA_performance}}%       \end{center}%       \caption{Unsupervised domain adaptation performance for classification accuracy and point-wise detection}%       \label{fig_unsupervised_da_performance}% \end{figure}\begin{figure}
 %\begin{center}
  \begin{subfigure}[b]{0.40\textwidth}
    \includegraphics[width=\textwidth]{images/Unsupervised_DA_performance}
  	\vspace{-0.5cm}
    \caption{}
      %\caption{Unsupervised domain adaptation performance for classification accuracy and point-wise detection}
      \label{fig_unsupervised_da_performance}
  \end{subfigure}
  \begin{subfigure}[b]{0.45\textwidth}
    \includegraphics[width=\textwidth]{images/class_num_image}
    \caption{}
    %\caption{Classification accuracy with respect to the number of weakly labeled real images per class}
      \label{fig_classification_num_images}
  \end{subfigure}
%\end{center}
  \caption{Performance analysis. In (a), we compare classification accuracy and point-wise detection when no label on real images are available. In (b), we compare the performance of different training processes when different numbers of real images are weakly labeled. }
  \label{fig_synthetic_images}
    	\vspace{-0.55cm}
\end{figure}%TODO (add "Weakly" to "number of labeled real images" in figure b) \begin{table*}
\begin{center}
\resizebox{0.98\textwidth}{!}{%
\begin{tabular}{|c|c|c|c|c|c|}
\hline
\textbf{Method} & \textbf{Modality} & \textbf{Supervision} & \textbf{Full Dataset} & \textbf{Accuracy (\%)}  &  \textbf{Accuracy (\%)} \\
 &  & &  & \textbf{YCB-Video}  & \textbf{Occluded Linemod} \\
\hline
PoseCNN \cite{xiang2017posecnn} & RGB & Pixelwise labels + 6D poses & Yes & 75.9  & 24.9 \\
\hline
PoseCNN+ICP \cite{xiang2017posecnn} & RGBD & Pixelwise labels + 6D poses & Yes & 93.0  & \textbf{78.0} \\
\hline
DeepHeatmaps \cite{oberweger2018making} & RGB & Pixelwise labels + 6D poses & Yes & 81.1 & 28.7  \\
\hline
FCN + Drost et. al. \cite{drost2010model} & RGBD & Pixelwise labels & Yes & 84.0  & - \\
\hline
FCN + StoCS \cite{mitash2018robust} & RGBD & Pixelwise labels & Yes & 90.1  & - \\
\hline
Brachmann et al. \cite{brachmann2014learning} & RGBD & Pixelwise labels + 6D poses & Yes & - & 56.6 \\ \hline
Michel et. al. \cite{michel2017global} & RGBD & Pixelwise labels + 6D poses & Yes & - & 76.7\\
\hline

\hline
OURS & RGBD & Object classes & No (10 weakly labeled images) & 88.7  &  68.8 \\
OURS & RGBD & Object classes & Yes & 90.2  & - \\
OURS (multiscale inference) & RGBD & Object classes & No (10 weakly labeled images) & -  & 76.6 \\
OURS (multiscale inference) & RGBD & Object classes & Yes & \textbf{93.6}  & - \\
\hline
\end{tabular}
}
\end{center}
\caption{Area under the accuracy-threshold curve for 6D Pose estimation on YCB-Video dataset and Occluded Linemod} \label{pose_ycb}
\vspace{-0.15cm}
\end{table*}% \begin{table*}% \begin{center}% %\resizebox{0.95\columnwidth}{!}{%% \begin{tabular}{|c|c|c|c|c|}% \hline % \textbf{Method} & \textbf{Modality} & \textbf{Supervision} & \textbf{Full Dataset} & \textbf{Accuracy (\%)}\\% \hline% PoseCNN \cite{xiang2017posecnn} & RGB & Pixelwise labels + 6D poses & Yes & 75.9\\% \hline% PoseCNN+ICP \cite{xiang2017posecnn} & RGBD & Pixelwise labels + 6D poses & Yes & 93.0\\% \hline% DeepHeatmaps \cite{oberweger2018making} & RGB & Pixelwise labels + 6D poses & Yes & 81.1\\% \hline% FCN + Drost et. al. \cite{drost2010model} & RGBD & Pixelwise labels & Yes & 84.0\\% \hline% FCN + StoCS \cite{mitash2018robust} & RGBD & Pixelwise labels & Yes & 90.1 \\% \hline% OURS & RGBD & Object classes & No (10 weakly labeled images) & 88.67\\% OURS & RGBD & Object classes & Yes & 90.20\\% OURS (multiscale inference) & RGBD & Object classes & Yes & 93.60\\% \hline% \end{tabular}% %}% \end{center}% \caption{Area under the accuracy-threshold curve for 6D Pose estimation on YCB-Video dataset} \label{pose_ycb}% \vspace{-0.15cm}% \end{table*}% \begin{table*}[t]% \begin{center}% %\resizebox{0.95\columnwidth}{!}{%% \begin{tabular}{|c|c|c|c|c|}% \hline % \textbf{Method} & \textbf{Modality} & \textbf{Supervision} & \textbf{Full Dataset} & \textbf{Accuracy (\%)}\\% \hline% % DeepIM \cite{li2018deepim} & RGB & 6D pose labels & 200 labels/object & 55.5\\ % % \hline% PoseCNN \cite{xiang2017posecnn} & RGB & Pixelwise labels + 6D poses & Yes & 24.9\\% \hline% PoseCNN+ICP \cite{xiang2017posecnn} & RGBD & Pixelwise labels + 6D poses & Yes & 78.0\\% \hline% DeepHeatmaps \cite{oberweger2018making} & RGB & Pixelwise labels + 6D poses & Yes & 28.7\\% % \hline% % LINEMOD & RGBD & Fully-supervised & 0 & 48.8\\% \hline% % Krull et al \cite{krull2015learning} & RGBD & Fully-supervised & 0 & 72.9\\% % \hline% Brachmann et al. \cite{brachmann2014learning} & RGBD & Pixelwise labels + 6D poses & Yes & 56.6\\% \hline% Michel et. al. \cite{michel2017global} & RGBD & Pixelwise labels + 6D poses & Yes & 76.7\\% \hline% OURS & RGBD & Object classes & No (10 weakly labeled images) & 68.8\\% OURS (multiscale inference) & RGBD & Object classes & No (10 weakly labeled images) & 76.6\\% \hline% \end{tabular}% %}% \end{center}% \caption{6D Pose estimation accuracy on Occluded-LINEMOD dataset} \label{pose_occluded}% \vspace{-0.55cm}% \end{table*}% \begin{table*}[t]% \begin{center}% %\resizebox{0.95\columnwidth}{!}{%% \begin{tabular}{|c|c|c|c|c|}% \hline % \textbf{Method} & \textbf{Modality} & \textbf{Supervision} & \textbf{Full Dataset} & \textbf{Accuracy (\%)}\\% \hline% % DeepIM \cite{li2018deepim} & RGB & 6D pose labels & 200 labels/object & 55.5\\ % % \hline% PoseCNN \cite{xiang2017posecnn} & RGB & Pixelwise labels + 6D poses & Yes & 24.9\\% \hline% PoseCNN+ICP \cite{xiang2017posecnn} & RGBD & Pixelwise labels + 6D poses & Yes & 78.0\\% \hline% DeepHeatmaps \cite{oberweger2018making} & RGB & Pixelwise labels + 6D poses & Yes & 28.7\\% % \hline% % LINEMOD & RGBD & Fully-supervised & 0 & 48.8\\% \hline% % Krull et al \cite{krull2015learning} & RGBD & Fully-supervised & 0 & 72.9\\% % \hline% Brachmann et al. \cite{brachmann2014learning} & RGBD & Pixelwise labels + 6D poses & Yes & 56.6\\% \hline% Michel et. al. \cite{michel2017global} & RGBD & Pixelwise labels + 6D poses & Yes & 76.7\\% \hline% OURS & RGBD & Object classes & No (10 weakly labeled images) & 68.8\\% OURS (after submission) & RGBD & Object classes & No (10 weakly labeled images) & 76.6\\% \hline% \end{tabular}% %}% \end{center}% \caption{6D Pose estimation accuracy on Occluded-LINEMOD dataset} \label{pose_occluded}% \end{table*}\vspace{-0.05cm}\subsection{Semi-Supervised Domain Adaptation}

A significant challenge for agile deployment of robots in industrial environments is that they ideally should be trained with limited annotated data, both in terms of numbers of images and of their extensiveness of labeling (no pose information, just class). We thus evaluated the performance of four different strategies as a function of the number of such weakly-labeled real images:
%\vspace{-0.05in}\begin{enumerate}
\setlength\itemsep{-0.02em}
\item Without domain adaptation:
	%\vspace{-0.05in}
	\begin{enumerate}
    \setlength\itemsep{-0.02em}
	\item Real Only: Trained only on weakly labeled real images,
    \item Fine-Tuning: Trained on synthetic images and then fine-tuned on weakly labeled real images,
	\end{enumerate}
\item With domain adaptation:
	%\vspace{-0.05in}
	\begin{enumerate}
    \setlength\itemsep{-0.02em}
    \setcounter{enumi}{2}
	\item Fine-Tuning: Trained on synthetic images and then fine-tuned on weakly labeled real images,
    \item Semi-Supervised: Trained with synthetic images and weakly labeled real images simultaneously.
	\end{enumerate}

% \item Real Only: Only real images are used in training
% \item No Adaptation + Fine-Tuning: Only the WILDCAT model is used in training.
% \item MADA + Fine-Tuning: Our full approach is used. We first train with synthetic images and unlabeled real images. Then, we fine-tune on weakly labeled real images.
% \item MADA (Semi-Supervised): We use our full approach with labeled synthetic and real images at the same time.
\end{enumerate}

For 1.a and 1.b, we validate that using fine-tuning on a network pre-trained with synthetic data is preferable to training directly on real images. For 2.a and 2.b, we compare the performance of our approach trained with fine-tuning, and in a semi-supervised way (using images from both domains at the same time). We are particularly interested in comparing the two approaches 2.a and 2.b, since \cite{oliver2018realistic} achieved the lowest error rate compared to any other semi-supervised approach by only using fine-tuning.

Our results are summarized in Figure~\ref{fig_classification_num_images}. From them, we conclude that training with synthetic images improves classification accuracy drastically, especially when few labels are available. Also, our approach performs slightly better when trained in a semi-supervised setting (2.b) than with a fine-tuning approach (2.a), which is contrary to~\cite{oliver2018realistic}.
%where the model is trained in unsupervised settings and fine-tuned on labeled real images .

In this Section, we justified our architecture, as well as the training technique employed, to create a network capable of performing object identification and localisation through weak learning. In the next Section, we demonstrate how the feature maps extracted by our network can be employed to perform precise 6 DoF object pose estimation via StoCS.

% \begin{figure}[thpb]%       \begin{center}%       \fbox{\includegraphics[width=0.6\linewidth]{images/class_num_image}}%       \end{center}%       \caption{Classification accuracy with respect to the number of weakly labeled real images per class}%       \label{fig_classification_num_images}% \end{figure}% In this section, we report our object detection results evaluated on 4 different datasets that are part of the 6D challenge at ECCV/ICCV (\textbf{ref?}): (1) \textit{Rutgers APC}, (2) \textit{Linemod} and \textit{Occluded Linemod}, (3) \textit{Tud Light} and (4) \textit{Toyota Light}. For each of these 4 datasets, we have generated 3 variations to train our network (for a total of 12). Subsequently, we refer to these dataset variations as (1) \textit{Real}, (2)  \textit{Synthetic + DANN} and (3) \textit{Real + Synthetic + DANN}. For the \textit{Real} datasets, we have randomly selected a certain number of real images either from the training set (if there is one) or from the test set. Our network was trained with only the multi-classifier loss \textbf{(show formula in previous section).} For the \textit{Synthetic + DANN} variation, we have generated \phil{Comment?} synthetic images combining one or multiple 3D models \phil{Commenter sur le besoin d'avoir un modle 3D connu de l'objet} at random poses with background images from LSUN dataset (\textbf{ref LSUN}) \phil{Justifier LSUN} (\textbf{show example of a synthetic and a real image for each of the datasets}. The number of generated images per class reflects the distribution of ground truth poses of each specific dataset. Also, by combining multiple objects per image, we allowed our model to learn to classify multiple objects at once and to learn about occlusions. For this variation, synthetic images were weakly annotated \phil{Spcifier ce qu'est cette weak annotation} and used to train the label classifier. The domain classifier was trained on the combination of synthetic images and images from the \textit{Real} dataset \phil{Expliciter quels sont les domaines. Si c'est rel vs. synthtique, en quoi est-ce similaire ou diffrent de GAN?.} . The difference between the last variation, \textit{Real + Synthetic + DANN}, and \textit{Synthetic + DANN}, is that the label classifier was trained with the weak annotations from the \textit{Real} and  \textit{Synthetic} datasets. Table \ref{tab_hyperparam} summarizes datasets and training details.% \begin{figure}%  \begin{center}%   \begin{subfigure}[b]{0.22\textwidth}%     \includegraphics[width=\textwidth]{images/test_apc}%     \caption{Rutgers APC}%     \label{fig_test_apc}%   \end{subfigure}%   %%   \begin{subfigure}[b]{0.22\textwidth}%     \includegraphics[width=\textwidth]{images/test_linemod}%     \caption{LineMod}%     \label{fig_test_linemod}%   \end{subfigure}%   %%   \\%   \begin{subfigure}[b]{0.22\textwidth}%     \includegraphics[width=\textwidth]{images/train_apc}%     \caption{Rutgers APC}%     \label{fig_train_apc}%   \end{subfigure}%   %%   \begin{subfigure}[b]{0.22\textwidth}%     \includegraphics[width=\textwidth]{images/train_linemod}%     \caption{LineMod}%     \label{fig_train_linemod}%   \end{subfigure}%   %%   \end{center}%   \caption{Images from test (a-d) and synthetic (e-h) datasets}% \end{figure}% \begin{table}[h!]% \begin{center}% \resizebox{0.97\columnwidth}{!}{%% \begin{tabular}{|c|c|c|c|c|}% \hline % \textbf{Dataset descriptions and } & \textbf{Rutgers APC} & \textbf{Linemod +} & \textbf{TUD Light} & \textbf{Toyota Light} \\% \textbf{Hyperparameters} & & \textbf{Occluded Linemod}  & & \\% \hline % \# of classes & 14 & 15 & 3 & 21 \\% \textit{Real} Dataset (\# image/class) & 10 & 10 & 10 & 10 \\ % \textit{Synthetic} Dataset (\# image/class) & 486 & 534 & 888 & 1 078\\ % {Synthetic} Dataset (\# objects/image) & 1 - 6 & 1 - 6 & 1 - 3 & 1 - 6 \\ % \# of test images & 5 964 & 18 273 & 23 914 & 1 680 \\% \hline % \# training epochs & 20 & 20 & 20 & 20 \\% \# iterations/epoch & 500 & 500 & 100 & 500\\% Mini-batch size (per dataset) & 4 & 4 & 4 & 4\\% Learning Rate (LC) & 1$e^{-3}$ & 1$e^{-3}$ & 1$e^{-4}$ & 1$e^{-3}$ \\% Learning Rate (DANN) & 1$e^{-3}$ & 1$e^{-3}$ & 1$e^{-3}$ & 1$e^{-3}$ \\% \hline % \end{tabular}% }% \end{center}% \caption{Dataset descriptions and training hyperparameters} \label{tab_hyperparam}% \end{table}%\subsection{Weakly Supervised Experiments}% \textbf{TODO} What is common to all datasets(like how to train the different dataset variations)% We evaluated our networks trained on different variants of 4 datasets. The first metric to evaluate our model is the classification accuracy. This metric evaluate our model capacity to discriminate which object(s) is in the image. For datasets with only one object from the dataset in test images, we retrieve the prediction with the highest score from our network. For datasets with multiple objects in test images, we used a threshold of 0.5 to predict the presence or absence of objects. % For detection, it is the point-wise localization metric \cite{oquab2015object}. To measure the ability of the network to localize objects, it takes the ground-truth object (to be independent of the classification accuracy) and measure if the detected pixel (maximum activation in the feature map) is inside the ground-truth bounding box + 32 pixels (the 32 pixel is because the feature map outputted by the ResNet is 32x smaller than the original resolution) \phil{Ou dire qu'un pixel de feature maps correspond  32 pixel en entre?}. % \subsubsection{Rutgers APC Dataset}% This is a reduced version of the dataset proposed in \textbf{???}. This version contains 14 different objects located on a warehouse shelf. The test dataset contains 5964 images for an average of 426 images per object. For the \textit{Real} dataset, we have randomly selected 10 images per class and weakly annotated them (i.e. which objects are in which image, but not its location). For the \textit{Synthetic} dataset, we have generated 486 images per object and randomly added between 0 and 5 objects in each of the generated images. For training the network, we used the same procedure for all variations of the dataset. We used a learning rate of 0.001 for the domain classifier for the whole training and 0.001 for the label classifier for epochs 1-5 and 0.0001 for epochs 6-15. Results are reported in Table \ref{tab_weakly_sup_apc}.% \phil{De ce que je comprends, les modles sont des variantes de ton architecture. Comment comptes-tu te comparer contre d'autres approches de la littrature? Quelles sont les approches les plus similaires?}% \begin{table}[h!]% \begin{center}% %\captionsetup{width=.85\columnwidth}% \resizebox{0.85\columnwidth}{!}{%% \begin{tabular}{|c|c|c|c|c|}% \hline % \textbf{Training Data} & \textbf{Labels} & \textbf{Model} & \textbf{Classification (\%)} & \textbf{Detection (\%)}\\  % \hline % Real & R & LC &  &  \\% \hline% Synthetic & S & LC &  &  \\ % Synthetic + Real & S & LC + DANN  & 67.05 & 81.44  \\% Synthetic + Real & S & LC + MADA (original) &  &  \\% Synthetic + Real & S & LC + MADA (modified) &  &  \\% \hline% Synthetic + Real & S + R & LC + MADA (mod) &   &  \\% \hline% Synthetic + Real & R (FT) & LC  & 96.34  & 93.98 \\% Synthetic + Real & R (FT) & LC + DANN &  &  \\% Synthetic + Real & R (FT) & LC + MADA (mod) &   &  \\% \hline% Synthetic + Real & R (FT) & LC + ALT-ITERS & 94.7 & 98.69  \\% Synthetic + Real & R (FT) & LC + ALT-EPOCHS & 97.18	& 98.96 \\% \hline% \end{tabular}% }% \end{center}% \caption{Performance analysis on modified Rutgers APC dataset} \label{tab_weakly_sup_apc}% \end{table}%\subsubsection{YCB-video dataset}% \begin{table}[h!]% \begin{center}% %\captionsetup{width=.85\columnwidth}% \resizebox{0.85\columnwidth}{!}{%% \begin{tabular}{|c|c|c|c|c|}% \hline % \textbf{Training Data} & \textbf{Labels} & \textbf{Model} & \textbf{Classification (\%)} & \textbf{Detection (\%)}\\  % \hline % Real & R & LC & & \\% \hline% Synthetic & S & LC & 68.84 & 93.06  \\ % Synthetic + Real & S & LC + DANN & 79.37  & 95.11 \\% Synthetic + Real & S & LC + MADA & 83.77 & 96.61 \\% \hline% Synthetic + Real & FT & LC (FT-10) & 86.89 &  96.52 \\% Synthetic + Real & FT & MADA (FT-10) & 88.16 & 96.94  \\% % Synthetic + Real & FT & LC (ALT-10) & 88.89	& 97.02 \\% % Synthetic + Real & FT & MADA (ALT-10) &  &   \\% \hline% Synthetic + Real & FT & LC (FT-100) & 86.87 & 96.82 \\% Synthetic + Real & FT & MADA (FT-100) & 89.48 & 97.76  \\% % Synthetic + Real & FT & LC (ALT-100) & 88.91 & 97.41 \\% % Synthetic + Real & FT & MADA (ALT-100) &  &   \\% \hline% Synthetic + Real & FT & LC (FT-1000) & 87.64 & 97.25 \\% Synthetic + Real & FT & MADA (FT-1000) & 89.45 & 96.77\\% %Synthetic + Real & FT & LC (ALT-1000) & 90.49 & 97.32 \\% %Synthetic + Real & FT & MADA (ALT-1000) &  &   \\% \hline% \end{tabular}% }% \end{center}% \caption{Performance analysis on YCB-video dataset} \label{tab_weakly_sup_ycb}% \end{table}% \subsubsection{Linemod and Occluded Linemod}% \begin{table}[h!]% \begin{center}% \resizebox{0.95\columnwidth}{!}{%% \begin{tabular}{|c|c|c|c|c|}% \hline % \textbf{Training Data} & \textbf{Labels} & \textbf{Model} & \textbf{Classification} & \textbf{Detection}\\%  &  &    & \textbf{(occluded)} & \textbf{(occluded)}\\% \hline % Real & R & LC  & 96.45 (0) (93.44 & 27.98 (32.34 \\ % \hline% Synthetic & S & LC  & 53.97 (46.28) & 66.65 (67.05) \\% Synthetic + Real & S & LC + DANN   & 45.33 & 65.39 \\% Synthetic + Real  & S & LC + MADA  & 59.09 & 70.29 \\% \hline% Synthetic + Real & S + R  & LC  & 86.02 (3)& 74.38 \\% Synthetic + Real & S + R & LC + DANN  & 84.01 (14)(84.75 & 75.69(76.95 \\ % \hline% Synthetic + Real  & R (FT) & LC   & 90.15 & 87.99 \\% Synthetic + Real  & R (FT) & LC + DANN  &  &  \\% \hline% Synthetic + Real  & S + R & LC + ALT-EPOCHS   & 91.32	& 91.52  \\% \hline% \end{tabular}% }% \end{center}% \caption{Performance analysis on LINEMOD and Occluded LINEMOD datasets} \label{tab_weakly_sup_linemod}% \end{table}\section{6D Pose Estimation Experiments}\label{sec:6d_exp}
We evaluated our full approach for 6D pose estimation on YCB-video~\cite{xiang2017posecnn} and Occluded Linemod~\cite{krull2015learning} datasets. We used the most common metrics to compare with similar methods. The average distance (ADD) metric \cite{hinterstoisser2012model} measures the average distance between the pairwise 3D model points transformed by the ground truth and predicted pose. For symmetric objects, the ADD-S metric measures the average distance using the closest point distance. Also, the visible surface discrepancy \cite{hodan2018bop} compares the distance maps of rendered models for estimated and ground-truth poses.

We used the same training details mentionned in section \ref{sec:w_exp}. Since the network architecture is fully convolutional, we also added an experiment for which we combined the output of the network for 3 different scales of the input image (at test time only).

\subsection{YCB-Video Dataset}% 2 568 images per object x 21 objects = 53 928 images%\phil{Please specify what is different here than the dataset described in the previous section: }
This dataset comprises several frames from 92 video sequences of cluttered scenes created with 21 YCB objects. The training for competing methods \cite{xiang2017posecnn, oberweger2018making, drost2010model} is performed using 113,199 frames from 80 video sequences with semantic (pixelwise) and pose labels. For our proposed approach, we used only 10 randomly sampled weakly annotated (class labels only) real images per object class combined with synthetic images. As in \cite{xiang2017posecnn}, we report the area under the curve (AUC) of the accuracy-threshold curve, using the ADD-S metric.
%The evaluation metric proposed in the dataset benchmark \cite{xiang2017posecnn} was used to report the accuracy. It uses the average distance (ADD) metric to plot the accuracy-threshold curve, and the area under the curve (AUC) is reported. The ADD metric measures the average distance between model points transformed by the ground truth and predicted pose.
Results are reported in Table \ref{pose_ycb}. Our proposed method achieves 88.67\% accuracy with a limited number of weakly labeled images and up to 93.60\% when using the full dataset with multiscale inference. It outperforms competing approaches, with the exception of PoseCNN+ICP, which performs similarly. However, our approach has a large computational advantage with an average runtime of 0.6 seconds per object as opposed to approximately 10 seconds per object for the modified-ICP refinement for PoseCNN. It also uses \emph{a)} nearly a hundredfold less real data, and \emph{b)} also only using the class labels. This results thus demonstrate that we can reach \emph{fast} and \emph{competitive} results without the need of 6D fully-annotated real datasets.

% For the poster (different reference numbers)% \begin{table}[h!]% \begin{center}% \resizebox{0.95\columnwidth}{!}{%% \begin{tabular}{|c|c|c|c|c|}% \hline % \textbf{Method} & \textbf{Modality} & \textbf{Supervision} & \textbf{Full Dataset} & \textbf{Accuracy}\\% \hline% PoseCNN [1] & RGB & Pixelwise + 6D pose & Yes & 75.9\\% \hline% PoseCNN+ICP [1] & RGBD & Pixelwise + 6D pose & Yes & 93.0\\% \hline% DeepHeatmaps [9] & RGB & Pixelwise + 6D pose & Yes & 81.1\\% \hline% FCN + Drost et. al. [10] & RGBD & Pixelwise & Yes & 84.0\\% \hline% OURS & RGBD & Object classes & No (10 images/object) & 86.5\\% \hline% \end{tabular}% }% \end{center}% \caption{Area under the accuracy-threshold curve for 6D Pose estimation on YCB-Video dataset} \label{pose_ycb}% \end{table}% \begin{table}[h!]% \begin{center}% \resizebox{0.95\columnwidth}{!}{%% \begin{tabular}{|c|c|c|c|c|}% \hline % \textbf{Method} & \textbf{Modality} & \textbf{Supervision} & \textbf{Number of real images} & \textbf{Accuracy}\\% \hline% PoseCNN \cite{xiang2017posecnn} & RGB & Pixelwise + 6D pose & Full Dataset & 75.9\\% \hline% PoseCNN+ICP \cite{xiang2017posecnn} & RGBD & Pixelwise + 6D pose & Full Dataset & 93.0\\% \hline% DeepHeatmaps \cite{oberweger2018making} & RGB & Pixelwise + 6D pose & Full Dataset & 81.1\\% \hline% FCN + Drost et. al. \cite{drost2010model} & RGBD & Pixelwise & Full Dataset & 84.0\\% \hline% OURS & RGBD & Object classes & 210 & 86.5\\% \hline% \end{tabular}% }% \end{center}% \caption{Area under the accuracy-threshold curve for 6D Pose estimation on YCB-Video dataset} \label{pose_ycb}% \end{table}% 534 images per object x 15 objects = 8 010 image\subsection{Occluded Linemod Dataset}
This dataset contains 1215 frames from a single video sequence with pose labels for 9 objects from the LINEMOD dataset with high level of occlusion. Competing methods are trained using the standard LINEMOD dataset, which consists in average of 1220 images per object. In our case, we used 10 real random images per object (manually labelled) on top of the generated synthetic images, using the weak (class) labels only.
%In this case, we did not use the full training dataset, since Linemod only has annotations for one object per image.
As reported in Table \ref{pose_ycb}, our method achieved scores of 68.8\% and 76.6\% (multiscale) for the ADD evaluation metric and using a threshold of $10\%$ of the 3D model diameter.
These results compare with state-of-the-art methods while using less supervision and a fraction of training data. The multiscale variant (input image at 3 different resolutions) made our approach more robust to occlusions. We did not train with the full Linemod training dataset, since the dataset only has annotations for 1 object per image and our method requires the full list of objects that are in the image. Furthermore, we evaluated our approach on the 6D pose estimation benchmark~\cite{hodan2018bop} using the visual discrepency metric. We evaluated our network with multiscale inference and we can see in Table \ref{table_vsd} that we are among the top 3 for the recall score while being the fastest. We also tested the effect of combining ICP with StoCS. At the cost of more processing time, we obtain the best performance among the methods that were evaluated on the benchmark.

% Whereas it outperforms all approaches using RGB only data, its performance is slightly inferior to some competing methods that use RGBD. We suspect the sensor noise leads to error-prone surface normal computation, which are extensively used by the StoCS approach for base sampling and in the optimization cost. \vspace{-0.35cm}\begin{minipage}[t]{8cm}
    %\begin{center}
    \centering
        \begin{adjustbox}{center, margin=0 0 0 20px}
        \begin{tabular}{|c|c|c|}
        \hline
        \textbf{Method} & \textbf{Recall Score (\%)} & \textbf{Time (s)}\\
        \hline
        Vidal-18 \cite{vidal20186d} & 59.3 & 4.7 \\
        Drost-10 \cite{drost2010model} & 55.4 & 2.3 \\
        Brachmann-16 \cite{brachmann2016uncertainty} & 52.0 & 4.4 \\
        Hodan-15 \cite{hodavn2015detection} & 51.4 & 13.5 \\
        Brachmann-14 \cite{brachmann2014learning} & 41.5 & 1.4 \\
        Buch-17-ppfh \cite{buch2017rotational} & 37.0 & 14.2 \\
        Kehl-16 \cite{kehl2016deep} & 33.9 & 1.8 \\
        \hline
        OURS (multiscale) & 55.2 & 0.6 \\
        OURS (multiscale) + ICP & \textbf{62.1} & 6.4 \\
        \hline
        \end{tabular}
        \end{adjustbox}
        \captionof{table}{Visual discrepency recall scores (\%) (correct pose estimation) for $\tau = 20$mm and $\theta = 0.3$ on Occluded Linemod, based on the 6D pose estimation benchmark \cite{hodan2018bop}.}
        \label{table_vsd}
        \vspace{-0.25cm}
    %\end{center}
\end{minipage}% \begin{minipage}[t]{0.5\textwidth}% \begin{center}%   \begin{adjustbox}{center, width=\columnwidth}%     \begin{table}  %     \begin{tabular}{|c|c|c|}%         \hline %         \textbf{Method} & \textbf{Recall Score (\%)} & \textbf{Time (s)}\\%         \hline%         Vidal-18 \cite{vidal20186d} & 59.31 & 4.7 \\%         Drost-10 \cite{drost2010model} & 55.36 & 2.3 \\%         Hodan-15 \cite{hodavn2015detection} & 51.42 & 13.5 \\%         Brachmann-16 \cite{brachmann2016uncertainty} & 52.04 & 4.4 \\%         Buch-17-ppfh \cite{buch2017rotational} & 36.96 & 14.2 \\%         Kehl-16 \cite{kehl2016deep} & 33.91 & 1.8 \\%         Brachmann-14 \cite{brachmann2014learning} & 41.52 & 1.4 \\%         \hline%         OURS & 55.20 & 0.6 \\%         OURS + ICP & 62.07 & ??? \\%         \hline%         \end{tabular}%     \end{table}%   \end{adjustbox}%   \captionof{table}{cap}                % \end{center}% \end{minipage}% \begin{table*}[t]% %\begin{center}% \resizebox{0.95\columnwidth}{!}{%% \begin{tabular}{|c|c|c|}% \hline % \textbf{Method} & \textbf{Recall Score (\%)} & \textbf{Time (s)}\\% \hline% Vidal-18 \cite{vidal20186d} & 59.31 & 4.7 \\% Drost-10 \cite{drost2010model} & 55.36 & 2.3 \\% Hodan-15 \cite{hodavn2015detection} & 51.42 & 13.5 \\% Brachmann-16 \cite{brachmann2016uncertainty} & 52.04 & 4.4 \\% Buch-17-ppfh \cite{buch2017rotational} & 36.96 & 14.2 \\% Kehl-16 \cite{kehl2016deep} & 33.91 & 1.8 \\% Brachmann-14 \cite{brachmann2014learning} & 41.52 & 1.4 \\% \hline% OURS & 55.20 & 0.6 \\% OURS + ICP & 62.07 & ??? \\% \hline% \end{tabular}%% }% %\captionsetup{width=\columnwidth, singlelinecheck=false}% \caption{Visual discrepency recall scores (\%) (correct pose estimation) for $\tau = 20$mm and $\theta = 0.3$ on Occluded Linemod, based on the 6D pose estimation benchmark \cite{hodan2018bop}.}% \label{pose_occluded}% %\end{center}% \end{table*}% For the poster (with different reference numbers)% \begin{table}[h!]% \begin{center}% \resizebox{0.95\columnwidth}{!}{%% \begin{tabular}{|c|c|c|c|c|}% \hline % \textbf{Method} & \textbf{Modality} & \textbf{Supervision} & \textbf{Full Dataset} & \textbf{Accuracy}\\% \hline% % DeepIM \cite{li2018deepim} & RGB & 6D pose labels & 200 labels/object & 55.5\\ % % \hline% PoseCNN [1] & RGB & Pixelwise + 6D pose & Yes & 24.9\\% \hline% PoseCNN+ICP [1] & RGBD & Pixelwise + 6D pose & Yes & 78.0\\% \hline% DeepHeatmaps [9] & RGB & Pixelwise + 6D pose & Yes & 28.7\\% % \hline% % LINEMOD & RGBD & Fully-supervised & 0 & 48.8\\% \hline% % Krull et al \cite{krull2015learning} & RGBD & Fully-supervised & 0 & 72.9\\% % \hline% Brachmann et al. [11] & RGBD & Pixelwise + 6D pose & Yes & 56.6\\% \hline% Michel et. al. [12] & RGBD & Pixelwise + 6D pose & Yes & 76.7\\% \hline% OURS & RGBD & Object classes & No (10 images/object) & 68.8\\% \hline% \end{tabular}% }% \end{center}% \caption{6D Pose estimation accuracy on Occluded-LINEMOD dataset} \label{pose_occluded}% \end{table}
\section{Conclusion}\label{sec:conclu}

In this paper, we explored the problem of 6D pose estimation in the context of limited annotated training datasets. To this effect, we demonstrated that the output of a weakly-trained network is sufficiently rich to perform full 6D pose estimation. Pose estimation experiments on two datasets showed that our approach is competitive with recent approaches (such as PoseCNN), despite requiring \emph{significantly less annotated images}. Most importantly, our annotation level requirement for real images is \emph{much weaker}, as we only need a class label without any spatial information (either bounding box or full 6D ground truth). %By employing the StoCS algorithm, we also circumvented the need to employ time-consuming pose estimation algorithms such as ICP.
In this end, this makes our approach compatible with an agile automated warehouse, where new objects to be manipulated are constantly introduced in a training database by non-expert employees.

\begin{comment}
In this paper, we explored the problem of 6D pose estimation in the context of limited annotated training datasets. To this effect, we first proposed a novel deep neural network architecture that exploits two state-of-the-art and orthogonal approaches to perform object detection and classification. The first one, the WILDCAT architecture, was used to leverage weakly-labeled real data, enabling the object localization without pose annotation. The second one, a domain adaptation technique called MADA, allowed the use of a mixture of synthetic and real, unannotated data. We experimentally justified the use of MADA over the vanilla version of DANN in Section~\ref{UnsupervisedDA}. By combining these two architectures in a novel way, we were able to capture the best of both worlds.

We then demonstrated that the output of our weakly-trained network is sufficiently rich to perform full 6D pose estimation, via a recent pose-search algorithm called StoCS. Pose estimation experiments on the YCB-Video and the Occluded Linemod datasets showed that our approach is competitive with recent approaches (such as PoseCNN), despite requiring \emph{significantly less annotated images}. Most importantly, our annotation level requirement for real images is \emph{much weaker}, as we only need a class label without any spatial information (either bounding box or full 6D ground truth). By employing the StoCS algorithm, we also circumvented the need to employ time-consuming pose estimation algorithms such as ICP. In this end, this makes our approach compatible with an agile automated warehouse, where new objects to be manipulated are constantly introduced in a training database by non-expert employees.

\end{comment}%In this paper, we explored the problem of object detection, classification and 6D pose estimation in the context of limited annotated training datasets. To this effect, we proposed a novel deep neural network architecture that exploits two state-of-the-art and orthogonal approaches. The first one, the WILDCAT architecture, was used to leverage weakly-labeled real data, enabling the object localization without pose annotation. The second one, a domain adaptation technique called MADA, allowed the use of a mixture of synthetic and real, unannotated data. We experimentally justified the use of MADA over the vanilla version of DANN in Section~\ref{UnsupervisedDA}. By combining these two architectures in a novel way, we were able to capture the best of both worlds. % We also demonstrated the ability of WILDCAT to allow for fine-tuning of detection and localisation modules on weakly annotated real data as well as the importance of using domain adaptation to leverage fully-annotated synthetic data in Section~\ref{UnsupervisedDA}.%Future work?%We also confirmed a known hypothesis that fine-tuning tends to perform better than semi-supervised learning.

%%%%%%%%%%%%%%%%%%%%%%%%%%%%%%%%%%%%%%%%%%%%%%%%%%%%%%%%%%%%%%%%%%%%%%%%%%%%%%%%

%    \begin{figure}[thpb]
%       \centering
%       \framebox{\parbox{3in}{We suggest that you use a text box to insert a graphic (which is ideally a 300 dpi TIFF or EPS file, with all fonts embedded) because, in an document, this method is somewhat more stable than directly inserting a picture.
% }}
%       %\includegraphics[scale=1.0]{figurefile}
%       \caption{Inductance of oscillation winding on amorphous
%        magnetic core versus DC bias magnetic field}
%       \label{figurelabel}
%    \end{figure}


%\addtolength{\textheight}{-12cm}   % This command serves to balance the column lengths
                                  % on the last page of the document manually. It shortens
                                  % the textheight of the last page by a suitable amount.
                                  % This command does not take effect until the next page
                                  % so it should come on the page before the last. Make
                                  % sure that you do not shorten the textheight too much.

%%%%%%%%%%%%%%%%%%%%%%%%%%%%%%%%%%%%%%%%%%%%%%%%%%%%%%%%%%%%%%%%%%%%%%%%%%%%%%%%



%%%%%%%%%%%%%%%%%%%%%%%%%%%%%%%%%%%%%%%%%%%%%%%%%%%%%%%%%%%%%%%%%%%%%%%%%%%%%%%%



%%%%%%%%%%%%%%%%%%%%%%%%%%%%%%%%%%%%%%%%%%%%%%%%%%%%%%%%%%%%%%%%%%%%%%%%%%%%%%%%
% \section*{APPENDIX}

% Appendixes should appear before the acknowledgment.

% \section*{ACKNOWLEDGMENT}

% The preferred spelling of the word acknowledgment in America is without an e after the g. Avoid the stilted expression, One of us (R. B. G.) thanks . . .  Instead, try R. B. G. thanks. Put sponsor acknowledgments in the unnumbered footnote on the first page.



%%%%%%%%%%%%%%%%%%%%%%%%%%%%%%%%%%%%%%%%%%%%%%%%%%%%%%%%%%%%%%%%%%%%%%%%%%%%%%%%

%\bibliographystyle{IEEEtrans}
%\bibliography{IEEEabrv,corl}
\bibliography{IEEEabrv,references}
\bibliographystyle{IEEEtran}


% \begin{thebibliography}{99}

% \bibitem{c1} G. O. Young, Synthetic structure of industrial plastics (Book style with paper title and editor), 	in Plastics, 2nd ed. vol. 3, J. Peters, Ed.  New York: McGraw-Hill, 1964, pp. 15Ð64.
% \bibitem{c2} W.-K. Chen, Linear Networks and Systems (Book style).	Belmont, CA: Wadsworth, 1993, pp. 123Ð135.
% \bibitem{c3} H. Poor, An Introduction to Signal Detection and Estimation.   New York: Springer-Verlag, 1985, ch. 4.
% \bibitem{c4} B. Smith, An approach to graphs of linear forms (Unpublished work style), unpublished.
% \bibitem{c5} E. H. Miller, A note on reflector arrays (Periodical styleÑAccepted for publication), IEEE Trans. Antennas Propagat., to be publised.
% \bibitem{c6} J. Wang, Fundamentals of erbium-doped fiber amplifiers arrays (Periodical styleÑSubmitted for publication), IEEE J. Quantum Electron., submitted for publication.
% \bibitem{c7} C. J. Kaufman, Rocky Mountain Research Lab., Boulder, CO, private communication, May 1995.
% \bibitem{c8} Y. Yorozu, M. Hirano, K. Oka, and Y. Tagawa, Electron spectroscopy studies on magneto-optical media and plastic substrate interfaces(Translation Journals style), IEEE Transl. J. Magn.Jpn., vol. 2, Aug. 1987, pp. 740Ð741 [Dig. 9th Annu. Conf. Magnetics Japan, 1982, p. 301].
% \bibitem{c9} M. Young, The Techincal Writers Handbook.  Mill Valley, CA: University Science, 1989.
% \bibitem{c10} J. U. Duncombe, Infrared navigationÑPart I: An assessment of feasibility (Periodical style), IEEE Trans. Electron Devices, vol. ED-11, pp. 34Ð39, Jan. 1959.
% \bibitem{c11} S. Chen, B. Mulgrew, and P. M. Grant, A clustering technique for digital communications channel equalization using radial basis function networks, IEEE Trans. Neural Networks, vol. 4, pp. 570Ð578, July 1993.
% \bibitem{c12} R. W. Lucky, Automatic equalization for digital communication, Bell Syst. Tech. J., vol. 44, no. 4, pp. 547Ð588, Apr. 1965.
% \bibitem{c13} S. P. Bingulac, On the compatibility of adaptive controllers (Published Conference Proceedings style), in Proc. 4th Annu. Allerton Conf. Circuits and Systems Theory, New York, 1994, pp. 8Ð16.
% \bibitem{c14} G. R. Faulhaber, Design of service systems with priority reservation, in Conf. Rec. 1995 IEEE Int. Conf. Communications, pp. 3Ð8.
% \bibitem{c15} W. D. Doyle, Magnetization reversal in films with biaxial anisotropy, in 1987 Proc. INTERMAG Conf., pp. 2.2-1Ð2.2-6.
% \bibitem{c16} G. W. Juette and L. E. Zeffanella, Radio noise currents n short sections on bundle conductors (Presented Conference Paper style), presented at the IEEE Summer power Meeting, Dallas, TX, June 22Ð27, 1990, Paper 90 SM 690-0 PWRS.
% \bibitem{c17} J. G. Kreifeldt, An analysis of surface-detected EMG as an amplitude-modulated noise, presented at the 1989 Int. Conf. Medicine and Biological Engineering, Chicago, IL.
% \bibitem{c18} J. Williams, Narrow-band analyzer (Thesis or Dissertation style), Ph.D. dissertation, Dept. Elect. Eng., Harvard Univ., Cambridge, MA, 1993.
% \bibitem{c19} N. Kawasaki, Parametric study of thermal and chemical nonequilibrium nozzle flow, M.S. thesis, Dept. Electron. Eng., Osaka Univ., Osaka, Japan, 1993.
% \bibitem{c20} J. P. Wilkinson, Nonlinear resonant circuit devices (Patent style), U.S. Patent 3 624 12, July 16, 1990.

% \end{thebibliography}




\end{document}
